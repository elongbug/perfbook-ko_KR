% appendix/questions/after.tex

\section{What Does ``After'' Mean?}
\label{sec:app:questions:What Does ``After'' Mean?}

``After'' 는 직관적이지만 놀라우리만큼 어려운 개념입니다.
한가지 중요한 반직관적 문제는 코드가 언제든 얼만큼이든 지연되어서 수행될 수
있다는 점입니다.
타임스탬프 ``t'' 와 정수 필드 ``a'', ``b'', 그리고 ``c'' 를 포함하는 글로벌
구조체를 이용해서 통신을 하는 생산자와 소비자 구조를 생각해 봅시다.
생산자는
Figure~\ref{fig:app:questions:After Producer Function} 에 보여진 것처럼
(1970 년으로부터의 현재 시각까지 지난 초를 10진수로 나타내는) 현재 시각을
기록하고 ``a'', ``b'', 그리고 ``c'' 를 업데이트 하는 루프를 돕니다.
소비자는
Figure~\ref{fig:app:questions:After Consumer Function} 에 보여진 것처럼 역시
현재 시각을 기록하지만 생성자의 타임스탬프와 ``a'', ``b'', 그리고 ``c'' 필드의
값을 복사해 옵니다.
프로그램 수행 종료 시점에서, 소비자는 이례적인 기록들을 출력하는데, 예를 들면
시간이 뒤로 돌아간 것으로 보이는 경우입니다.
\iffalse

``After'' is an intuitive, but surprisingly difficult concept.
An important non-intuitive issue is that code can be delayed at
any point for any amount of time.
Consider a producing and a consuming thread that communicate using
a global struct with a timestamp ``t'' and integer fields ``a'', ``b'',
and ``c''.
The producer loops recording the current time
(in seconds since 1970 in decimal),
then updating the values of ``a'', ``b'', and ``c'',
as shown in Figure~\ref{fig:app:questions:After Producer Function}.
The consumer code loops, also recording the current time, but also
copying the producer's timestamp along with the fields ``a'',
``b'', and ``c'', as shown in
Figure~\ref{fig:app:questions:After Consumer Function}.
At the end of the run, the consumer outputs a list of anomalous recordings,
e.g., where time has appeared to go backwards.
\fi

\begin{figure}[htbp]
{ \scriptsize
\begin{verbbox}
  1 /* WARNING: BUGGY CODE. */
  2 void *producer(void *ignored)
  3 {
  4   int i = 0;
  5
  6   producer_ready = 1;
  7   while (!goflag)
  8     sched_yield();
  9   while (goflag) {
 10     ss.t = dgettimeofday();
 11     ss.a = ss.c + 1;
 12     ss.b = ss.a + 1;
 13     ss.c = ss.b + 1;
 14     i++;
 15   }
 16   printf("producer exiting: %d samples\n", i);
 17   producer_done = 1;
 18   return (NULL);
 19 }
\end{verbbox}
}
\centering
\theverbbox
\caption{``After'' Producer Function}
\label{fig:app:questions:After Producer Function}
\end{figure}

\begin{figure}[htbp]
{ \scriptsize
\begin{verbbox}
  1 /* WARNING: BUGGY CODE. */
  2 void *consumer(void *ignored)
  3 {
  4   struct snapshot_consumer curssc;
  5   int i = 0;
  6   int j = 0;
  7
  8   consumer_ready = 1;
  9   while (ss.t == 0.0) {
 10     sched_yield();
 11   }
 12   while (goflag) {
 13     curssc.tc = dgettimeofday();
 14     curssc.t = ss.t;
 15     curssc.a = ss.a;
 16     curssc.b = ss.b;
 17     curssc.c = ss.c;
 18     curssc.sequence = curseq;
 19     curssc.iserror = 0;
 20     if ((curssc.t > curssc.tc) ||
 21         modgreater(ssc[i].a, curssc.a) ||
 22         modgreater(ssc[i].b, curssc.b) ||
 23         modgreater(ssc[i].c, curssc.c) ||
 24         modgreater(curssc.a, ssc[i].a + maxdelta) ||
 25         modgreater(curssc.b, ssc[i].b + maxdelta) ||
 26         modgreater(curssc.c, ssc[i].c + maxdelta)) {
 27       i++;
 28       curssc.iserror = 1;
 29     } else if (ssc[i].iserror)
 30       i++;
 31     ssc[i] = curssc;
 32     curseq++;
 33     if (i + 1 >= NSNAPS)
 34       break;
 35   }
 36   printf("consumer exited, collected %d items of %d\n",
 37          i, curseq);
 38   if (ssc[0].iserror)
 39     printf("0/%d: %.6f %.6f (%.3f) %d %d %d\n",
 40            ssc[0].sequence, ssc[j].t, ssc[j].tc,
 41            (ssc[j].tc - ssc[j].t) * 1000000,
 42            ssc[j].a, ssc[j].b, ssc[j].c);
 43   for (j = 0; j <= i; j++)
 44     if (ssc[j].iserror)
 45       printf("%d: %.6f (%.3f) %d %d %d\n",
 46              ssc[j].sequence,
 47              ssc[j].t, (ssc[j].tc - ssc[j].t) * 1000000,
 48              ssc[j].a - ssc[j - 1].a,
 49              ssc[j].b - ssc[j - 1].b,
 50              ssc[j].c - ssc[j - 1].c);
 51   consumer_done = 1;
 52 }
\end{verbbox}
}
\centering
\theverbbox
\caption{``After'' Consumer Function}
\label{fig:app:questions:After Consumer Function}
\end{figure}

\QuickQuiz{}
	이 예제에서 어떤 SMP 코딩 에러가 보이나요?
	전체 코드를 보기 위해선 \path{time.c} 파일을 보세요.
	\iffalse

	What SMP coding errors can you see in these examples?
	See \path{time.c} for full code.
	\fi
\QuickQuizAnswer{
	\begin{enumerate}
	\item	루프에서 barrier() 나 volatile 을 사용하지 않음.
	\item	업데이트 쪽에서 메모리 배리어를 사용하지 않음.
	\item	생성자와 소비자 사이의 동기화가 없음.
	\iffalse

	\item	Missing barrier() or volatile on tight loops.
	\item	Missing Memory barriers on update side.
	\item	Lack of synchronization between producer and consumer.
	\fi
	\end{enumerate}
} \QuickQuizEnd

생성자가 타임스탬프나 값들을 저장하는데에 그렇게 많은 시간이 걸리지 않을
것이기에 생성자와 소비자 타임스탬프 간의 차이는 상당히 작을 것이라고 예상하는
사람들이 있겠습니다.
듀얼코어 1GHz x86 에서의 출력 결과가
Table~\ref{tab:app:questions:After Program Sample Output} 에 보여져 있습니다.
여기서, ``seq'' 행은 루프를 수행한 횟수이고, ``time'' 행은 변칙적 행위가 나타난
시각을 초로 나타낸 것이고, ``delta'' 행은 소비자의 타임스탬프가 생성자의 것보다
얼마나 뒤의 것이었는지 (즉, 이 값이 음수라면 소비자가 생성자보다 타임스탬프를
먼저 받았음을 말합니다), 그리고 ``a'', ``b'', 그리고 ``c'' 로 나타내어진 행들은
이 변수들이 소비자에 의해 수집된 앞의 스냅샷에 비해 얼마나 증가되었는지를
보입니다.
\iffalse

One might intuitively expect that the difference between the producer
and consumer timestamps would be quite small, as it should not take
much time for the producer to record the timestamps or the values.
An excerpt of some sample output on a dual-core 1GHz x86 is shown in
Table~\ref{tab:app:questions:After Program Sample Output}.
Here, the ``seq'' column is the number of times through the loop,
the ``time'' column is the time of the anomaly in seconds, the ``delta''
column is the number of seconds the consumer's timestamp follows that
of the producer (where a negative value indicates that the consumer
has collected its timestamp before the producer did), and the
columns labelled ``a'', ``b'', and ``c'' show the amount that these
variables increased since the prior snapshot collected by the consumer.
\fi

\begin{table}[htbp]
\centering
\scriptsize
\begin{tabular}{rcrrrr}
seq    & time (seconds) & delta~    &  a &  b &  c \\
\hline
17563: & 1152396.251585 & ($-16.928$) & 27 & 27 & 27 \\
18004: & 1152396.252581 & ($-12.875$) & 24 & 24 & 24 \\
18163: & 1152396.252955 & ($-19.073$) & 18 & 18 & 18 \\
18765: & 1152396.254449 & ($-148.773$) & 216 & 216 & 216 \\
19863: & 1152396.256960 & ($-6.914$) & 18 & 18 & 18 \\
21644: & 1152396.260959 & ($-5.960$) & 18 & 18 & 18 \\
23408: & 1152396.264957 & ($-20.027$) & 15 & 15 & 15 \\
\end{tabular}
\caption{``After'' Program Sample Output}
\label{tab:app:questions:After Program Sample Output}
\end{table}

왜 시간이 거꾸로 가는 걸까요?
괄호 안의 숫자는 마이크로세컨드 단위의 차이로써, 큰 수는 10 마이크로세컨드를
넘기고, 한번은 100 마이크로세컨드를 넘기기조차 했습니다!
이 CPU 는 그동안 100,000 개의 인스트럭션을 수행할 수도 있음을 알아두시기
바랍니다.

한가지 가능한 이유는 다음과 같은 이벤트 시퀀스로 설명됩니다:
\iffalse

Why is time going backwards?
The number in parentheses is the difference in microseconds, with
a large number exceeding 10 microseconds, and one exceeding even
100 microseconds!
Please note that this CPU can potentially execute more than 100,000
instructions in that time.

One possible reason is given by the following sequence of events:
\fi
\begin{enumerate}
\item	소비자가 타임스탬프를 얻어옵니다
	(Figure~\ref{fig:app:questions:After Consumer Function}, line~13).
\item	소비자가 preemption 당합니다.
\item	임의의 시간이 흐릅니다.
\item	생성자가 타임스탬프를 얻어옵니다
	(Figure~\ref{fig:app:questions:After Producer Function}, line~10).
\item	소비자가 수행을 다시 재개하고, 생성자의 타임스탬프를 읽어옵니다
	(Figure~\ref{fig:app:questions:After Consumer Function}, line~14).
\iffalse

\item	Consumer obtains timestamp
	(Figure~\ref{fig:app:questions:After Consumer Function}, line~13).
\item	Consumer is preempted.
\item	An arbitrary amount of time passes.
\item	Producer obtains timestamp
	(Figure~\ref{fig:app:questions:After Producer Function}, line~10).
\item	Consumer starts running again, and picks up the producer's
	timestamp
	(Figure~\ref{fig:app:questions:After Consumer Function}, line~14).
\fi
\end{enumerate}

이 시나리오 상에서, 생성자의 타임스탬프는 소비자의 타임스탬프보다 얼만큼이든
뒤의 것일 수 있습니다.

여러분은 여러분의 SMP 코드가 ``after'' 의 의미로 고민하는 것을 어떻게
막으시나요?

그냥 SMP 기능들을 설계된 대로 사용하세요.
\iffalse

In this scenario, the producer's timestamp might be an arbitrary
amount of time after the consumer's timestamp.

How do you avoid agonizing over the meaning of ``after'' in your
SMP code?

Simply use SMP primitives as designed.
\fi

이 예제에서, 가장 간단한 수정방법은 락킹을 사용하는 것으로, 예를 들어 생성자는
Figure~\ref{fig:app:questions:After Producer Function} 의 line~10 앞에서 락을
잡고 소비자는
Figure~\ref{fig:app:questions:After Consumer Function} 의 line~13 앞에서 락을
잡도록 합니다.
또한, 이 락은
Figure~\ref{fig:app:questions:After Producer Function} 의 line~13 뒤에서
해제되고
Figure~\ref{fig:app:questions:After Consumer Function} 의 line~17 뒤에서
해제되어야만 합니다.
이 락들은
Figure~\ref{fig:app:questions:After Producer Function} 의 line~10-13 과
Figure~\ref{fig:app:questions:After Consumer Function} 의 line~13-17 의 코드
조각들이 서로를 {\em 배제} 하도록 해주는데, 달리 말하자면 서로에 대해서
어토믹하게 수행되게 된다는 말입니다.
이는
Figure~\ref{fig:app:questions:Effect of Locking on Snapshot Collection} 로
표현되어 있습니다:
이 락킹은 모든 상자 안의 코드가 시간상으로 겹쳐지는 것을 방지해 줘서, 소비자의
타임스탬프는 앞의 생성자의 타임스탬프 뒤에 수집될 수 있도록 해줍니다.
이 그림에 있는 각각의 상자 안의 코드 조각들은 ``크리티컬 섹션'' 이라
명명됩니다; 한 시점에는 오로지 하나의 크리티컬 섹션만이 수행됩니다.
\iffalse

In this example, the easiest fix is to use locking, for example,
acquire a lock in the producer before line~10 in
Figure~\ref{fig:app:questions:After Producer Function} and in
the consumer before line~13 in
Figure~\ref{fig:app:questions:After Consumer Function}.
This lock must also be released after line~13 in
Figure~\ref{fig:app:questions:After Producer Function} and
after line~17 in
Figure~\ref{fig:app:questions:After Consumer Function}.
These locks cause the code segments in lines~10-13 of
Figure~\ref{fig:app:questions:After Producer Function} and in lines~13-17 of
Figure~\ref{fig:app:questions:After Consumer Function} to {\em exclude}
each other, in other words, to run atomically with respect to each other.
This is represented in
Figure~\ref{fig:app:questions:Effect of Locking on Snapshot Collection}:
the locking prevents any of the boxes of code from overlapping in time, so
that the consumer's timestamp must be collected after the prior
producer's timestamp.
The segments of code in each box in this figure are termed
``critical sections''; only one such critical section may be executing
at a given time.
\fi

\begin{figure}[htb]
\centering
\includegraphics{appendix/questions/after-snapshot}
\caption{Effect of Locking on Snapshot Collection}
\label{fig:app:questions:Effect of Locking on Snapshot Collection}
\end{figure}

이렇게 락킹을 추가하면
Table~\ref{fig:app:questions:Locked After Program Sample Output} 에 보여진 것과
같은 출력 결과가 나옵니다.
여기선 시간이 뒤로 간 경우는 없었고, 그대신 소비자에 의해 읽어진 연속된 읽기로
읽혀진 값들 사이에는 1,000 카운트 이상의 차이들이 생겼습니다.
\iffalse

This addition of locking results in output as shown in
Table~\ref{fig:app:questions:Locked After Program Sample Output}.
Here there are no instances of time going backwards, instead,
there are only cases with more than 1,000 counts difference between
consecutive reads by the consumer.
\fi

\begin{table}[htbp]
\centering
\scriptsize
\begin{tabular}{rcrrrr}
seq    & time (seconds) & delta~    &  a &  b &  c \\
\hline
58597:  & 1156521.556296 & (3.815) & 1485 & 1485 & 1485 \\
403927: & 1156523.446636 & (2.146) & 2583 & 2583 & 2583 \\
\end{tabular}
\caption{Locked ``After'' Program Sample Output}
\label{fig:app:questions:Locked After Program Sample Output}
\end{table}

\QuickQuiz{}
	소비자의 연속적인 읽기들 사이에 어떻게 그렇게 큰 간격이 존재하게
	된걸까요?
	전체 코드를 위해선 \path{timelocked.c} 파일을 보세요.
	\iffalse

	How could there be such a large gap between successive
	consumer reads?
	See \path{timelocked.c} for full code.
	\fi
\QuickQuizAnswer{
	\begin{enumerate}
	\item	소비자는 긴 시간동안 preemption 당했을 수 있습니다.
	\item	오랫동안 수행되는 인터럽트가 소비자를 지연시켰을 수 있습니다.
	\item	생성자는 소비자가 수행되는 CPU 에 비해 더 빠른 CPU 위에서
		수행되었을 수 있습니다 (예를 들어, CPU 들 가운데 하나는 발열
		처리나 에너지 소비 제한에 의해 자신의 클락 주파수를 낮췄을 수
		있습니다).
	\iffalse

	\item	The consumer might be preempted for long time periods.
	\item	A long-running interrupt might delay the consumer.
	\item	The producer might also be running on a faster CPU than is the
		consumer (for example, one of the CPUs might have had to
		decrease its
		clock frequency due to heat-dissipation or power-consumption
		constraints).
	\fi
	\end{enumerate}
} \QuickQuizEnd

요약하자면, 여러분이 배타적 락을 획득한다면, 여러분은 그 락을 잡은채 하는 모든
일이 그 락을 앞서 잡고서 행한 모든 것보다 뒤에 행해진 것으로 나타남을 알게
됩니다.
어떤 CPU 가 메모리 배리어를 수행했는지 안했는가로 걱정할 필요가 없고, CPU 나
컴파일러가 오퍼레이션들을 재배치 했는지에 대해 걱정하지 않아도 됩니다---삶은
간단합니다.
물론, 이 락킹이 이 두개의 코드 조각을 동시적으로 수행되지 못하도록 막는 것은
프로그램이 멀티프로세서에서 성능을 높일 가능성을 막는데, 즉 ``안전하지만 느린''
상황을 초래할 수도 있는 것입니다.
Chapter~\ref{cha:Partitioning and Synchronization Design} 는 많은 상황에서
성능과 확장성을 높일 수 있는 방법들을 설명합니다.

하지만, 대부분의 경우에 있어서, 어떤 주어진 코드 조각의 전과 후에 어떤 일이
일어나는지 걱정된다면, 표준 기능들의 사용을 더 잘 하기 위해 이를 힌트로 삼아야
합니다.
이 기능들이 여러분이 걱정을 하지 않아도 되도록 하도록 해주세요.
\iffalse

In summary, if you acquire an exclusive lock, you {\em know} that
anything you do while holding that lock will appear to happen after
anything done by any prior holder of that lock.
No need to worry about which CPU did or did not execute a memory
barrier, no need to worry about the CPU or compiler reordering
operations---life is simple.
Of course, the fact that this locking prevents these two pieces of
code from running concurrently might limit the program's ability
to gain increased performance on multiprocessors, possibly resulting
in a ``safe but slow'' situation.
Chapter~\ref{cha:Partitioning and Synchronization Design} describes ways of
gaining performance and scalability in many situations.

However, in most cases, if you find yourself worrying about what happens
before or after a given piece of code, you should take this as a hint to
make better use of the standard primitives.
Let these primitives do the worrying for you.
\fi
