% appendix/rcuimpl/rcutree.tex

\section{Hierarchical RCU Overview}
\label{app:rcuimpl:rcutree:Hierarchical RCU Overview}
\OriginallyPublished{Appendix}{app:rcuimpl:rcutree:Hierarchical RCU Overview}{Hierarchical RCU Overview}{Linux Weekly News}{PaulEMcKenney2008HierarchicalRCU}

Classic RCU 의 read-side 기능들은 훌륭한 성능과 확장성을 갖긴 합니다만, 언제
이전부터 존재한 read-side 크리티컬 섹션들이 완료되었는지를 결정하는데 사용되는
update-side 기능들은 수십개의 CPU 들만을 고려한 채 설계되었습니다.
이것들의 확장성은 각 grace period 마다 최소 한번은 각 CPU 에 의해 획득되어야
하는 global lock 에 의해 제한됩니다.
Classic RCU 는 실제로는 수백개의 CPU 들까지도 확장성을 갖고, 대략 천개의 CPU
들까지도 확장성을 갖도록 (더 길어지는 grace period 를 대가로) 꼼수를 부릴 수
있지만, 발전되는 중인 멀티코어 시스템들은 더 높은 확장성을 필요로 할겁니다.
\iffalse

Although Classic RCU's read-side primitives enjoy excellent
performance and scalability, the update-side primitives, which
determine when pre-existing read-side critical sections have
finished, were designed with only a few tens of CPUs in mind.
Their scalability is limited by a global lock that must be
acquired by each CPU at least once during each grace period.
Although Classic RCU actually scales to a couple of hundred CPUs, and
can be tweaked to scale to roughly a thousand CPUs (but at the expense of
extending grace periods), emerging multicore systems will require
it to scale better.
\fi

또한, Classic RCU 는 최적화 되지 않은 dynticks 인터페이스를 가지고 있어서,
Classic RCU 는 매 grace period 마다 최소 한번씩은 깨어나게 할겁니다.
이 문제를 자세히 보려면, 4개의 CPU 만이 바쁘게 유지되고 있을 정도로 적은
부하만을 가지고 있는 16-CPU 시스템을 생각해 보세요.
완벽한 세계에서는, 나머지 열두개의 CPU 들은 에너지를 아끼기 위해 깊은 sleep
모드로 빠져있을 수 있을 겁니다.
불행히도, 네개의 바쁜 CPU 들이 자주 RCU update 동작을 수행한다면, 이 열두개의
idle CPU 들은 자주 깨어나서 상당한 전력을 소모할 겁니다.
따라서, Classic RCU 에의 주요 변경 사항들은 항상 잠들어있는 CPU 들을 놔두어야
합니다.
\iffalse

In addition, Classic RCU has a sub-optimal dynticks interface,
with the result that Classic RCU will wake up every CPU at least
once per grace period.
To see the problem with this, consider a 16-CPU system that
is sufficiently lightly loaded that it is keeping only four
CPUs busy.
In a perfect world, the remaining twelve CPUs could be put into
deep sleep mode in order to conserve energy.
Unfortunately, if the four busy CPUs are frequently performing
RCU updates, those twelve idle CPUs will be awakened frequently,
wasting significant energy.
Thus, any major change to Classic RCU should also leave sleeping CPUs lie.
\fi

Classic 과 hierarchical 구현들 둘 다 Classic RCU 의 semantic 과 동일한 API 들을
가지고 있습니다만, 기존의 구현은 ``classic RCU'' 로 불리울 것이고, 새로운
구현은 ``hierarchical RCU'' 라 불릴 겁니다.
\iffalse

Both the classic and the hierarchical implementations
have Classic RCU semantics and identical APIs, however,
the old implementation will be called ``classic RCU''
and the new implementation will be called ``hierarchical RCU''.
\fi

Section~\ref{app:rcuimpl:rcutree:Review of RCU Fundamentals}
은 RCU 의 기본에 대한 짧은 개괄을 제공하고
Section~\ref{app:rcuimpl:rcutree:Brief Overview of Classic RCU Implementation}
은 기존의 ``Classic RCU'' 구현에 대한 전반적인 개괄을 제공합니다.
Section~\ref{app:rcuimpl:rcutree:RCU Desiderata}
은 RCU 에서 일어나지 않을 것 같은 일들을 나열해보고,
Section~\ref{app:rcuimpl:rcutree:Towards a More Scalable RCU Implementation}
과~\ref{app:rcuimpl:rcutree:Towards a Greener RCU Implementation}
에서는 확장성과 전력 효율성을 위한 설계상의 고려할 점들을 각각 알아보고,
Section~\ref{app:rcuimpl:rcutree:State Machine}
에서 hierarchical RCU state machine 을 설명합니다.
Section~\ref{app:rcuimpl:rcutree:Use Cases},
Section~\ref{app:rcuimpl:rcutree:Testing}
는 테스트를 다루고, 마지막으로
Section~\ref{app:rcuimpl:rcutree:Conclusion}
에서 결론을 내립니다.
\iffalse

Section~\ref{app:rcuimpl:rcutree:Review of RCU Fundamentals}
gives a brief review of RCU fundamentals and
Section~\ref{app:rcuimpl:rcutree:Brief Overview of Classic RCU Implementation}
gives a brief overview of the old ``Classic RCU'' implementation.
Section~\ref{app:rcuimpl:rcutree:RCU Desiderata}
lists RCU desiderata,
Sections~\ref{app:rcuimpl:rcutree:Towards a More Scalable RCU Implementation}
and~\ref{app:rcuimpl:rcutree:Towards a Greener RCU Implementation}
lay out design considerations for scalability and energy efficiency,
respectively, and
Section~\ref{app:rcuimpl:rcutree:State Machine}
describes the hierarchical RCU state machine.
Section~\ref{app:rcuimpl:rcutree:Use Cases},
Section~\ref{app:rcuimpl:rcutree:Testing}
covers testing, and finally,
Section~\ref{app:rcuimpl:rcutree:Conclusion}
presents concluding remarks.
\fi

\subsection{Review of RCU Fundamentals}
\label{app:rcuimpl:rcutree:Review of RCU Fundamentals}

가장 기본적 형태에 있어, RCU 는 일이 끝나길 기다리는 방법입니다.
물론, 일이 끝나길 기다리는 방법은 레퍼런스 카운트, reader-writer 락, 이벤트,
등등 매우 많이 존재합니다.
RCU 의 커다란 장점은 (대략) 20,000 개의 서로 다른 일들을 명시적으로 하나하나 다
정보를 추적할 필요 없이, 그리고 명시적 추적에서는 피할 수 없는 성능의 하락,
확장성의 제약, 복잡한 데드락 시나리오, 메모리 누수 문제 없이 각각 기다릴 수
있다는 겁니다.
\iffalse

In its most basic form, RCU is a way of waiting for things to finish.
Of course, there are a great many other ways of waiting for things to
finish, including reference counts, reader-writer locks, events, and so on.
The great advantage of RCU is that it can wait for each of
(say) 20,000 different things without having to explicitly
track each and every one of them, and without having to worry about
the performance degradation, scalability limitations, complex deadlock
scenarios, and memory-leak hazards that are inherent in schemes
using explicit tracking.
\fi

RCU 의 경우, 기다리는 것들은 ``RCU read-side 크리티컬 섹션'' 이라 불리웁니다.
하나의 RCU read-side 크리티컬 섹션은 \co{rcu_read_lock()} 기능으로 시작하고,
연관된 \co{rcu_read_unlock()} 기능으로 종료됩니다.
RCU read-side 크리티컬 섹션들은 중첩될 수 있고, 명시적으로 블록되거나 잠들지
않는다면, 상당히 많은 양의 임의적 코드를 포함할 수 있습니다 (
Section~\ref{app:rcuimpl:Sleepable RCU Implementation} 에서 설명한 SRCU 라
불리는 특수한 형태의 RCU 는 SRCU read-side 크리티컬 섹션 내에서의 일반적인
sleeping 을 허용하긴 하지만요).
여러분이 이러한 표준을 준수한다면, 여러분은 \emph{모든} 코드 조각에 대해
완료되길 기다리는데에 RCU 를 사용할 수 있습니다.

RCU 는 언제 이것들이 끝나는지를 간접적으로 판단함으로써 이 기능을 구현하는데,
classic RCU 의 경우 다른 곳~\cite{McKenney98} 에서, preemptible RCU 의 경우
Section~\ref{app:rcuimpl:Preemptible RCU} 에서 자세히 설명합니다.
\iffalse

In RCU's case, the things waited on are called
``RCU read-side critical sections''.
An RCU read-side critical section starts with an
\co{rcu_read_lock()} primitive, and ends with a corresponding
\co{rcu_read_unlock()} primitive.
RCU read-side critical sections can be nested, and may contain pretty
much any code, as long as that code does not explicitly block or sleep
(although a special form of RCU called SRCU, described in
Section~\ref{app:rcuimpl:Sleepable RCU Implementation}
does permit general sleeping in SRCU read-side critical sections).
If you abide by these conventions, you can use RCU to wait for \emph{any}
desired piece of code to complete.

RCU accomplishes this feat by indirectly determining when these
other things have finished, as has been described
elsewhere~\cite{McKenney98}
for classic RCU and
Section~\ref{app:rcuimpl:Preemptible RCU} for preemptible RCU.
\fi

자세히 말하자면,
page~\pageref{fig:defer:Readers and RCU Grace Period} 의
Figure~\ref{fig:defer:Readers and RCU Grace Period} 에서 보인 바와 같이, RCU 는
앞서 존재하기 시작한 RCU read-side 크리티컬 섹션들과 그것들에 의해 수행된
메모리 오퍼레이션들이 완전히 종료되기를 기다리는 방법입니다.

하지만, 특정 grace period 가 시작한 후에 시작된 RCU read-side 크리티컬 섹션들은
해당 grace period 종료 뒤까지도 연장될 수 있음을 알아두시기 바랍니다.

다음 섹션은 Classic RCU 구현이 어떻게 동작하는지에 대한 높은 수준에서의 개관을
제공합니다.
\iffalse

In particular, as shown in the
Figure~\ref{fig:defer:Readers and RCU Grace Period} on
page~\pageref{fig:defer:Readers and RCU Grace Period},
RCU is a way of
waiting for pre-existing RCU read-side critical sections to completely
finish, also including the memory operations executed
by those critical sections.

However, note that RCU read-side critical sections
that begin after the beginning
of a given grace period can and will extend beyond the end of that grace
period.

The following section gives a very high-level view of how
the Classic RCU implementation operates.
\fi

\subsection{Brief Overview of Classic RCU Implementation}
\label{app:rcuimpl:rcutree:Brief Overview of Classic RCU Implementation}

Classic RCU 구현 아래의 핵심 컨셉은, Classic RCU read-side 크리티컬 섹션은 커널
코드에 국한되어 있고 블록이 허용되어있지 않다는 것입니다.
이말은 언제든 어떤 CPU 가 블록되어 있는 중으로 보이거나, idle loop 을 돌고
있거나, 커널 모드를 빠져나가 있다면, 우리는 해당 CPU 에서 앞서 수행중이던 RCU
read-side 크리티컬 섹션이 모두 완료되었어야 함을 알 수 있다는 것입니다.
그런 상태는 ``quiescent states'' 라 불리며, 각각의 CPU 가 적어도 하나의
quiscent state 를 거쳤다면, 해당 RCU grace period 는 끝납니다.
\iffalse

The key concept behind the Classic RCU implementation is that
Classic RCU read-side critical sections are confined to kernel
code and are not permitted to block.
This means that any time a given CPU is seen
either blocking, in the idle loop, or exiting the kernel, we know that all
RCU read-side critical sections that were previously running on
that CPU must have completed.
Such states are called ``quiescent states'', and
after each CPU has passed through at least one quiescent state,
the RCU grace period ends.
\fi

\begin{figure}[htb]
\centering
\resizebox{3in}{!}{\includegraphics{appendix/rcuimpl/FlatClassicRCU}}
\caption{Flat Classic RCU State}
\label{fig:app:rcuimpl:rcutree:Flat Classic RCU State}
\end{figure}

Classic RCU 의 가장 중요한 데이터 구조는 \co{rcu_ctrlblk} 구조체로,
Figure~\ref{fig:app:rcuimpl:rcutree:Flat Classic RCU State} 에 보인 것처럼 CPU
당 하나의 bit 을 담고 있는 \co{->cpumask} 필드를 가지고 있습니다.
각각의 CPU 의 bit 은 각각의 grace period 의 시작지점에서 1로 설정되고, 각각의
CPU 는 quiescent state 를 지날 때마다 자신의 bit 의 값을 해제합니다.
여러개의 CPU 들이 각자의 bit 을 동시적으로 값 해제하고자 할 수 있으며 이는
\co{->cpumask} 필드를 오염시킬 수 있기 때문에, 그런 문제를 방지하기 위해
\co{->cpumask} 필드를 보호하는데에 \co{->lock} 스핀락이 사용됩니다.
안타깝게도, 멀티코어 트렌드가 지속된다면 곧 흔한 경우가 될 수백개가 넘는 CPU 가
존재한다면 이 스핀락 또한 상당한 경쟁상태로 성능에 악영향을 줄 수 있습니다.
더 나쁜건, \emph{모든} CPU 들이 각자의 bit 을 값 해제해야 한다는 사실은 CPU
들이 grace period 동안은 잠들지 않아야 함을 말하는데, 이는 Linux 의 전력 소모
감소 능력이 제한됨을 말합니다.

다음 섹션은 새로운 real-time 이 아닌 RCU 구현에서 필요한 것들을 나열합니다.
\iffalse

Classic RCU's most important data structure is the \co{rcu_ctrlblk}
structure, which contains the \co{->cpumask} field, which contains
one bit per CPU, as shown in
Figure~\ref{fig:app:rcuimpl:rcutree:Flat Classic RCU State}.
Each CPU's bit is set to one at the beginning of each grace period,
and each CPU must clear its bit after it passes through a quiescent
state.
Because multiple CPUs might want to clear their bits concurrently,
which would corrupt the \co{->cpumask} field, a
\co{->lock}
spinlock is used to protect \co{->cpumask}, preventing any
such corruption.
Unfortunately, this spinlock can also suffer extreme contention if there
are more than a few hundred CPUs, which might soon become quite common
if multicore trends continue.
Worse yet, the fact that \emph{all} CPUs must clear their own bit means
that CPUs are not permitted to sleep through a grace period, which limits
Linux's ability to conserve power.




The next section lays out what we need from a new non-real-time
RCU implementation.
\fi

\subsection{RCU Desiderata}
\label{app:rcuimpl:rcutree:RCU Desiderata}

Real-time RCU 의 요구사항~\cite{PaulMcKenney05b} 을 나열해 보는게 좋은 시작점이
될겁니다:
\iffalse

The list of real-time RCU desiderata~\cite{PaulMcKenney05b}
is a very good start:
\fi

\begin{enumerate}
\item	하나의 RCU grace period 는 모든 이전부터 존재한 RCU read-side 크리티컬
	섹션들이 완료될 때까지 끝날 수 없도록 하는 지연된 해체.
\item	일년 365일 24시간 지속되는 동작을 지원할 수 있는 신뢰성.
\item	Irq 핸들러에서 호출할 수 있을 것.
\item	너무 많은 콜백들이 존재한다면 grace period 들을 신속히 처리할 수 있는
	메커니즘이 존재할 수 있도록 하는, 억제된 메모리 사용량 (LCA2005
	리스트로 효력이 약화되었습니다.)
\item	RCU 가 어떤 있음직한 메모리 할당자와도 동작할 수 있도록 해주는 메모리
	블락들의 독립성.
\item	CPU 나 태스크의 지역적 메모리 에서의 평범한, 어토믹하지 않은 인스트럭션
	동작이 허용될 수 있도록 하는 동기화에 자유로운 read side (이건 LCA2005
	리스트에서 강화되었습니다.).
\item	Update-side 락이 RCU read-side 크리티컬 섹션 내에서 획득되어지는,
	리눅스 커널의 여러곳에서 사용되는 무조건적인 read-to-write 업그레이드.
\item	호환성 있는 API.
\iffalse

\item	Deferred destruction, so that an RCU grace period cannot end
	until all pre-existing RCU read-side critical sections have
	completed.
\item	Reliable, so that RCU supports 24x7 operation for years at
	a time.
\item	Callable from irq handlers.
\item	Contained memory footprint, so that mechanisms exist to expedite
	grace periods if there are too many callbacks.  (This is weakened
	from the LCA2005 list.)
\item	Independent of memory blocks, so that RCU can work with any
	conceivable memory allocator.
\item	Synchronization-free read side, so that only normal non-atomic
	instructions operating on CPU- or task-local memory are permitted.
	(This is strengthened from the LCA2005 list.)
\item	Unconditional read-to-write upgrade, which is used in several
	places in the Linux kernel where the update-side lock is
	acquired within the RCU read-side critical section.
\item	Compatible API.
\fi

\item	이건 real-time RCU 가 될 것은 아니기에, preemption 가능한 RCU read-side
	크리티컬 섹션에 대한 요구사항은 빠졌습니다.
	하지만, 과거 수년동안의 변경사항들을 처리하기 위해 다음의 새로운
	요구사항들을 추가할 필요가 있습니다.

\item	극단적으로 낮은 internal-to-RCU 락 경쟁상황을 갖는 확장성.
	RCU 는 최소 1,024 개, 가능하면 최소 4,096 개의 CPU 들을 지원해야
	합니다.
\item	전력 효율성: RCU 는 저전력 상태의 dynticks-idle CPU 들을 깨우지
	않으면서도 언제 현재의 grace period 가 종료되는지를 알 수 있어야
	합니다.
	이는 real-time RCU 에서 구현되었습니다만, 상당한 단순화를 필요로
	합니다.
\item	RCU read-side 크리티컬 섹션들은 irq 핸들러는 물론이고 NMI 핸들러
	내에서도 사용이 가능해야 합니다.  Preemptible RCU 의 경우에는 별도로
	구현된 \co{synchronize_sched()} 덕분에 이 필요성이 없었음을 알아두시기
	바랍니다.
\item	RCU 는 반복되는 CPU-hotplug 오퍼레이션들의 상황에서도 잘 동작해야
	합니다.
	이는 classic RCU 와 real-time RCU 에서의 필요사항들을 가져올 뿐입니다.
\item	앞서 등록된 모든 RCU 콜백들이 완료되길 기다릴 수 있어야만 하는데, 이는
	이미 \co{rcu_barrier()} 의 형태로 제공되긴 합니다.
\iffalse

\item	Because this is not to be a real-time RCU, the requirement for
	preemptible RCU read-side critical sections can be dropped.
	However, we need to add the following new requirements to account
	for changes over the past few years.

\item	Scalability with extremely low internal-to-RCU lock contention.
	RCU must support at least 1,024 CPUs gracefully, and preferably
	at least 4,096.
\item	Energy conservation: RCU must be able to avoid awakening
	low-power-state dynticks-idle CPUs, but still determine
	when the current grace period ends.
	This has been implemented in real-time RCU, but needs serious
	simplification.
\item	RCU read-side critical sections must be permitted in NMI
	handlers as well as irq handlers.  Note that preemptible RCU
	was able to avoid this requirement due to a separately
	implemented \co{synchronize_sched()}.
\item	RCU must operate gracefully in face of repeated CPU-hotplug
	operations.
	This is simply carrying forward a requirement met by both
	classic and real-time.
\item	It must be possible to wait for all previously registered
	RCU callbacks to complete, though this is already provided
	in the form of \co{rcu_barrier()}.
\fi
\item	RCU 와 다양한 무한 루프 버그들, 그리고 RCU grace period 가 종료되기를
	막는 하드웨어 상의 문제들을 진단하는데 도움이 될 수 있도록, 응답을
	하는데 실패하는 CPU 들을 파악할 수 있다면 좋습니다.
\item	하나의 RCU grace period 가 마지막 연관된 RCU read-side 크리티컬 섹션이
	완료되는데에 수백 마이크로세컨드 만이 걸릴 수 있도록 하는 RCU grace
	period 의 극단적인 속도 촉진이 가능하면 좋습니다.
	하지만, 그런 동작은 상당한 CPU 오버헤드를 가져올 거라 예상되어지며,
	각각이 RCU grace period 를 기다려야 하는 긴 일련의 오퍼레이션들을
	처리해야 할때 유용할 겁니다.
\iffalse

\item	Detecting CPUs that are failing to respond is desirable,
	to assist diagnosis both of RCU and of various infinite
	loop bugs and hardware failures that can prevent RCU grace
	periods from ending.
\item	Extreme expediting of RCU grace periods is desirable,
	so that an RCU grace period can be forced to complete within
	a few hundred microseconds of the last relevant RCU read-side
	critical second completing.
	However, such an operation would be expected to incur
	severe CPU overhead, and would be primarily useful when
	carrying out a long sequence of operations that each needed
	to wait for an RCU grace period.
\fi
\end{enumerate}

새로운 효구사항들 가운데 가장 압박이 되는건 첫번째 항목인 확장성입니다.
따라서 다음 섹션은 RCU 의 내부 락들의 경쟁 상황을 수십배 줄여주는지 설명합니다.
\iffalse

The most pressing of the new requirements is the first one, scalability.
The next section therefore describes how to make order-of-magnitude reductions
in contention on RCU's internal locks.
\fi

\subsection{Towards a More Scalable RCU Implementation}
\label{app:rcuimpl:rcutree:Towards a More Scalable RCU Implementation}

\begin{figure}[htb]
\centering
\resizebox{3in}{!}{\includegraphics{appendix/rcuimpl/TreeClassicRCU}}
\caption{Hierarchical RCU State}
\label{fig:app:rcuimpl:rcutree:Hierarchical RCU State}
\end{figure}

락 경쟁을 줄여주는 효과적인 방법 가운데 하나는
Figure~\ref{fig:app:rcuimpl:rcutree:Hierarchical RCU State} 에 보인 것처럼
계층을 만드는 것입니다.
여기서, 각각의 네개의 \co{rcu_node} 구조체는 자신의 락을 가지고 있어서, CPU~0
과 1 만이 좌하단의 \co{rcu_node} 의 락을 획득하며, CPU~2 와 3 만이 중하단의
\co{rcu_node} 의 락을 획득하고, CPU~4 와 5 만이 우하단의 \co{rcu_node} 의 락을
획득합니다.
어떤 특정 grace period 동안에도, 아래쪽 \co{rcu_node} 구조체들에 접근하는 CPU
들 가운데 하나만이 위의 \co{rcu_node} 에 접근하게 되어서, 각각의 CPU 짝 들
가운데 마지막 하나만이 연관된 grace period 에 대한 quiescent state 를 기록하게
됩니다.

이는 락 경쟁의 상당한 감소를 초래합니다:
여섯개의 CPU 들이 하나의 락을 위해 매 grace period 마다 경쟁하는 대신, 위쪽
\co{rcu_node} 락에 세개의 CPU 만이 경쟁하고 (50\% 경쟁 감소) 아래쪽
\co{rcu_node} 의 락에 대해서는 각각 두개의 CPU 만이 경쟁을 합니다 (67\% 경쟁
감소).
\iffalse

One effective way to reduce lock contention is to create a hierarchy,
as shown in
Figure~\ref{fig:app:rcuimpl:rcutree:Hierarchical RCU State}.
Here, each of the four \co{rcu_node} structures has its own lock,
so that only CPUs~0 and 1 will acquire the lower left
\co{rcu_node}'s lock, only CPUs~2 and 3 will acquire the
lower middle \co{rcu_node}'s lock, and only CPUs~4 and 5
will acquire the lower right \co{rcu_node}'s lock.
During any given grace period,
only one of the CPUs accessing each of the lower \co{rcu_node}
structures will access the upper \co{rcu_node}, namely, the
last of each pair of CPUs to record a quiescent state for the corresponding
grace period.

This results in a significant reduction in lock contention:
instead of six CPUs contending for a single lock each grace period,
we have only three for the upper \co{rcu_node}'s lock
(a reduction of 50\%) and only
two for each of the lower \co{rcu_node}s' locks (a reduction
of 67\%).
\fi

\begin{figure}[htb]
\centering
\resizebox{3in}{!}{\includegraphics{appendix/rcuimpl/TreeMapping}}
\caption{Mapping {\tt rcu\_node} Hierarchy Into Array}
\label{fig:app:rcuimpl:rcutree:Mapping rcu-node Hierarchy Into Array}
\end{figure}

\co{rcu_node} 구조체들의 트리는 \co{rcu_state} 구조체 안의 선형적 배열로 트리의
루트를 0번 원소로 저장하는 식으로 내장되는데,
Figure~\ref{fig:app:rcuimpl:rcutree:Mapping rcu-node Hierarchy Into Array} 에
여덟개의 CPU 를 갖는 시스템에서 세 단계를 갖는 계층으로 구성했을 때를 보이고
있습니다.
각각의 화살표는 특정 \co{rcu_node} 구조체를 그 부모로 연결시키는데,
\co{rcu_node} 의 \co{->parent}  필드를 나타냅니다.
각각의 \co{rcu_node} 는 자신이 관리하는 CPU 들의 범위를 나타내는데, 루트 노드는
모든 CPU 들을 관리하고, leaf 레벨의 각각의 노드는 한쌍의 CPU 들을 관리합니다.
이 배열은 컴파일 시점에 \co{NR_CPUS} 값에 따라 정적으로 할당됩니다.
\iffalse

The tree of \co{rcu_node} structures is embedded into
a linear array in the \co{rcu_state} structure,
with the root of the tree in element zero, as shown in
Figure~\ref{fig:app:rcuimpl:rcutree:Mapping rcu-node Hierarchy Into Array}
for an eight-CPU
system with a three-level hierarchy.
Each arrow links a given \co{rcu_node} structure to its parent,
representing the \co{rcu_node}'s \co{->parent} field.
Each \co{rcu_node} indicates the range of CPUs covered,
so that the root node covers all of the CPUs, each node in the second
level covers half of the CPUs, and each node in the leaf level covering
a pair of CPUs.
This array is allocated statically at compile time based on the value
of \co{NR_CPUS}.
\fi

\begin{figure}[htbp]
\centering
\resizebox{3in}{!}{\includegraphics{appendix/rcuimpl/TreeClassicRCUGP}}
\caption{Hierarchical RCU Grace Period}
\label{fig:app:rcuimpl:rcutree:Hierarchical RCU Grace Period}
\end{figure}

Figure~\ref{fig:app:rcuimpl:rcutree:Hierarchical RCU Grace Period}
안의 다이어그램들의 시퀀스는 어떻게 grace period 들이 파악되는지를 보입니다.
첫번째 그림에서는 어떤 CPU 도 빨간 네모로 표시되었듯이 quiescent state 에
이르지 못했습니다.
모든 여섯개의 CPU 들이 동시에 RCU 에게 자신이 quiescent state 에 이르렀노라고
이야기하고자 한다고 생각해 봅시다.
각각의 쌍 안의 CPU 들 중 하나만이 연관된 아래쪽 \co{rcu_node} 의 락을 획득할 수
있을 것이고, 따라서 두번째 그림은 그런 운좋은 CPU 들이 0, 3, 그리고 5 라고
초록색 네모로 나타내고 있습니다.
이 운좋은 CPU 들이 일단 마무리 되면, 다른 CPU 들이 락을 획득할 것인데, 세번째
그림으로 보인 바와 같습니다.
이 CPU 들 각각은 자신들이 자신의 그룹 내에서는 마지막에 이르렀음을 보게 되고,
따라서 모든 세개의 CPU 들은 위쪽의 \co{rcu_node} 로 이동하려 시도하게 됩니다.
한번에 하나만이 위쪽 \co{rcu_node} 구조체의 락을 얻을 수 있고, 네번째,
다섯번째, 그리고 여섯번째 그림은 CPU~1, CPU~2, 그리고 CPU~4 가 순서대로 그 락을
잡는다는 가정 하에 상태의 흐름을 보입니다.
마지막 여섯번째 그림은 모든 CPU 들이 quiescent state 를 지난, 그래서 grace
period 가 끝난 경우를 보이고 있습니다.
\iffalse

The sequence of diagrams in
Figure~\ref{fig:app:rcuimpl:rcutree:Hierarchical RCU Grace Period}
shows how grace periods are detected.
In the first figure, no CPU has yet passed through a quiescent state,
as indicated by the red rectangles.
Suppose that all six CPUs simultaneously try to tell RCU that they have
passed through a quiescent state.
Only one of each pair will be able to acquire the lock on the
corresponding lower \co{rcu_node}, and so the second figure
shows the result if the lucky CPUs are numbers 0, 3, and 5, as indicated
by the green rectangles.
Once these lucky CPUs have finished, then the other CPUs will acquire
the lock, as shown in the third figure.
Each of these CPUs will see that they are the last in their group,
and therefore all three will attempt to move to the upper
\co{rcu_node}.
Only one at a time can acquire the upper \co{rcu_node} structure's
lock, and the fourth, fifth, and sixth figures show the sequence of
states assuming that CPU~1, CPU~2, and CPU~4 acquire
the lock in that order.
The sixth and final figure in the group shows that all CPUs have passed
through a quiescent state, so that the grace period has ended.
\fi

\begin{figure}[htb]
\centering
\resizebox{3in}{!}{\includegraphics{appendix/rcuimpl/BigTreeClassicRCU}}
\caption{Hierarchical RCU State 4,096 CPUs}
\label{fig:app:rcuimpl:rcutree:Hierarchical RCU State 4,096 CPUs}
\end{figure}

앞의 예에서, Classic RCU 에서라면 여섯개의 CPU 들이 경쟁하게 될 터인 것을 어떤
하나의 락에 대해서는 세개가 넘는 CPU 들이 경쟁하지 않았습니다.
하지만, 더 많은 수의 CPU 들에서라면 이보다도 극적인 락 경쟁의 감소가
가능합니다.
64 개의 아래쪽 구조체와 64*64=4,096 CPU 들로 구성된,
Figure~\ref{fig:app:rcuimpl:rcutree:Hierarchical RCU State 4,096 CPUs} 에 보인
것과 같은 \co{rcu_node} 구조체들의 계층을 생각해 보세요.

여기서 각각의 아래쪽 \co{rcu_node} 구조체의 락들은 64 개의 CPU 들에 의해
획득되어지는데, 이는 Classic RCU 의 하나의 글로벌 락에 4,096 개의 CPU 들이
경쟁하는 것에 비해 64배의 감소입니다.
비슷하게, 특정 grace period 동안에, 아래쪽 \co{rcu_node} 구조체의 CPU 들 가운데
하나만이 위쪽 \co{rcu_node} 구조체의 락을 잡을 것이므로, 이는 또다시 4,096 개의
CPU 로 구성된 시스템에서 동작하는 Classic RCU 가 겪게 되는 경쟁의 64x
감소입니다.
\iffalse

In the above sequence, there were never more than three CPUs
contending for any one lock, in happy contrast to Classic RCU,
where all six CPUs might contend.
However, even more dramatic reductions in lock contention are
possible with larger numbers of CPUs.
Consider a hierarchy of \co{rcu_node} structures, with
64 lower structures and 64*64=4,096 CPUs, as shown in
Figure~\ref{fig:app:rcuimpl:rcutree:Hierarchical RCU State 4,096 CPUs}.

Here each of the lower \co{rcu_node} structures' locks
are acquired by 64 CPUs, a 64-times reduction from the 4,096 CPUs
that would acquire Classic RCU's single global lock.
Similarly, during a given grace period, only one CPU from each of
the lower \co{rcu_node} structures will acquire the
upper \co{rcu_node} structure's lock, which is again
a 64x reduction from the contention level that would be experienced
by Classic RCU running on a 4,096-CPU system.
\fi

\QuickQuiz{}
	잠깐만요!
	이 새로운 락들을 가지고, 데드락은 어떻게 막나요?
	\iffalse

	Wait a minute!
	With all those new locks, how do you avoid deadlock?
	\fi
\QuickQuizAnswer{
	한번에 한개의 \co{rcu_node} 구조체의 락만을 잡도록 하는 것으로 데드락이
	막아집니다.
	이 알고리즘은 두개의 락을 더 사용하는데, 하나는 grace-period 진척과
	동시에 CPU hotplug 오퍼레이션이 수행되는 것을 막고 (\co{onofflock})
	또하나는 한번에 하나의 CPU 만이 quiescent state 가 빨리 끝나도록
	강제하는데에 사용됩니다 (\co{fqslock}).
	이것들은 락킹 계층에 반하고, 따라서 \co{fqslock} 은 \co{onofflock} 보다
	먼저 획득되어야만 하는데, 이말은 \co{rcu_node} 구조체의 락들보다 먼저
	획득되어야만 한다는 말입니다.

	또한, 실제 사용에서의 효율이 중요하듯이, 한버에 하나의 \co{rcu_node} 의
	락만을 잡는 것을 거부함은 어떤 것이 잡혀있는지를 추적할 필요가 없음을
	의미합니다.
	그런 추적은 불필요할 뿐 아니라 고통스러울 겁니다.
	\iffalse

	Deadlock is avoided by never holding more than one of the
	\co{rcu_node} structures' locks at a given time.
	This algorithm uses two more locks, one to prevent CPU hotplug
	operations from running concurrently with grace-period advancement
	(\co{onofflock}) and another
	to permit only one CPU at a time from forcing a quiescent state
	to end quickly (\co{fqslock}).
	These are subject to a locking hierarchy, so that
	\co{fqslock} must be acquired before
	\co{onofflock}, which in turn must be acquired before
	any of the \co{rcu_node} structures' locks.

	Also, as a practical matter, refusing to ever hold more than
	one of the \co{rcu_node} locks means that it is unnecessary
	to track which ones are held.
	Such tracking would be painful as well as unnecessary.
	\fi
} \QuickQuizEnd

\QuickQuiz{}
	왜 64-배 감소에서 멈추는 거죠?
	왜 그이상 가서 수백배까지 줄이지 않는거예요?
	\iffalse

	Why stop at a 64-times reduction?
	Why not go for a few orders of magnitude instead?
	\fi
\QuickQuizAnswer{
	RCU 는 수백개의 CPU 을 갖는 시스템에서도 문제없이 동작하기 때문에,
	64개의 CPU 들이 하나의 락을 경쟁하도록 하는 것으로도 충분합니다.
	이 락들은 상당히 드물게 획득되어지므로, 각각의 CPU 는 grace period 마다
	한번씩 정도만 이 일을 하게 될 것이고, grace period 는 밀리세컨드
	단위까지 길어짐을 기억해 두시기 바랍니다.
	\iffalse

	RCU works with no problems on
	systems with a few hundred CPUs, so allowing 64 CPUs to contend on
	a single lock leaves plenty of headroom.
	Keep in mind that these locks are acquired quite rarely, as each
	CPU will check in about one time per grace period, and grace periods
	extend for milliseconds.
	\fi
} \QuickQuizEnd

\QuickQuiz{}
	하지만 전 McKenney 의 Quick Quiz 2 에의 게으른 변명 따위 신경쓰지
	않아요!!!
	저는 하나의 락에 경쟁하는 CPU 들의 수를 16이나 그정도 쯤까지 더
	합리적인 수준으로 낮추고 싶어요!!!
	\iffalse

	But I don't care about McKenney's lame excuses in the answer to
	Quick Quiz 2!!!
	I want to get the number of CPUs contending on a single lock down
	to something reasonable, like sixteen or so!!!
	\fi
\QuickQuizAnswer{
	좋아요, 그럼 그렇게 해보세요!
	\co{CONFIG_RCU_FANOUT=16} 으로 설정하고 (\co{NR_CPUS=4096 에 대해})
	나면 가장 낮은 단계에서의 256 \co{rcu_node} 구조체, 중간 계층에서의 16
	개의 \co{rcu_node} 구조체, 그리고 하나의 root 레벨 \co{rcu_node} 를
	갖게 되어 세단계의 계층을 갖게 될 겁니다.
	이에 대해 지불하게 될 비용은 더 많은 \co{rcu_node} 구조체들이 어떤 CPU
	들이 자신의 quiescent state 를 완료하는데 도움이 필요한지를 체크하는데
	더 많은 작업이 걸릴 거라는 겁니다 (64 대신 256 이 되는거죠).
	\iffalse

	OK, have it your way, then!
	Set \co{CONFIG_RCU_FANOUT=16} and (for \co{NR_CPUS=4096})
	you will get a
	three-level hierarchy with with 256 \co{rcu_node} structures
	at the lowest level, 16 \co{rcu_node} structures as intermediate
	nodes, and a single root-level \co{rcu_node}.
	The penalty you will pay is that more \co{rcu_node} structures
	will need to be scanned when checking to see which CPUs need help
	completing their quiescent states (256 instead of only 64).
	\fi
} \QuickQuizEnd

\begin{figure}[htb]
\centering
\resizebox{3in}{!}{\includegraphics{appendix/rcuimpl/BigTreeClassicRCUBH}}
\caption{Hierarchical RCU State With BH}
\label{fig:app:rcuimpl:rcutree:Hierarchical RCU State With BH}
\end{figure}

실제 구현은 RCU 콜백의 리스트와 같은 per-CPU 데이터를 유지하도록 되어 있으며,
이는 \co{rcu_data} 구조체에 정리되어 있습니다.
또한, rcu (\co{call_rcu()} 에서의) 와 rcu\_bh (\co{call_rcu_bh()} 에서의) 는
각각
Figure~\ref{fig:app:rcuimpl:rcutree:Hierarchical RCU State With BH} 에 보인
것과 같이 각각의 계층을 유지합니다.
\iffalse

The implementation maintains some per-CPU data, such as lists of
RCU callbacks, organized into \co{rcu_data} structures.
In addition, rcu (as in \co{call_rcu()}) and
rcu\_bh (as in \co{call_rcu_bh()}) each maintain their own
hierarchy, as shown in
Figure~\ref{fig:app:rcuimpl:rcutree:Hierarchical RCU State With BH}.
\fi

\QuickQuiz{}
	좋아요, 근데 색깔은 뭘 의미하나요?
	\iffalse

	OK, so what is the story with the colors?
	\fi
\QuickQuizAnswer{
	(\co{rcu_ctrlblk} 을 포함해서) \co{rcu_state} 에 비슷한 데이터 구조들은
	노란색으로, 어떤 CPU 들이 들어와 있는지를 파악하는데 사용되는 bitmap
	들을 포함한 것들은 핑크로, 그리고 per-CPU \co{rcu_data} 구조체들은
	파란색으로 표시되어 있습니다.
	(\co{rcu_dynticks} 와 같이) 전력을 아끼기 위해 사용되는 데이터
	구조체들은 초록색으로 칠해져 있습니다.
	\iffalse

	Data structures analogous to \co{rcu_state} (including
	\co{rcu_ctrlblk}) are yellow,
	those containing the bitmaps used to determine when CPUs have checked
	in are pink,
	and the per-CPU \co{rcu_data} structures are blue.
	The data structures used to conserve energy
	(such as \co{rcu_dynticks}) will be colored green.
	\fi
} \QuickQuizEnd

다음 섹션은 에너지 소모 감소에 대해 이야기 해봅니다.
\iffalse

The next section discusses energy conservation.
\fi

\subsection{Towards a Greener RCU Implementation}
\label{app:rcuimpl:rcutree:Towards a Greener RCU Implementation}

앞서 언급되었듯이, 이러한 노력의 중요한 목표는 전력 소모를 방지하기 위해 잠자는
CPU 들을 깨우지 않는 것입니다.
반면에, Classic RCU 는 잠들어 있는 CPU 들을 어떤 겨웅에 있어서는 매 grace
period 마다 한번씩은 깨울 것인데, 이는 적은 수의 CPU 들만이 바쁘게 일을 하면서
RCU update 를 하고 있고 나머지 대부분의 CPU 들은 idle 상태일 때에
비효율적입니다.
이 상황은 peak load 를 위해 맞춰진 시스템들에서 자주 발생하며, 우린 이런 상황을
잘 대처할 필요가 있습니다.
더 나아가서, 우린 긴시간 동작하는 RCU read-side 크리티컬 섹션을 담고 있으며
인터럽트를 처리하는 dynticks-idle CPU 가 RCU grace period 가 끝나는 것을 막는데
실패하는, Classic RCU 의 오랫동안 있어왔던 버그를 고칠 필요가 있습니다.
\iffalse

As noted earlier, an important goal of this effort is to leave sleeping
CPUs lie in order to promote energy conservation.
In contrast, classic RCU will happily awaken each and every sleeping CPU
at least once per grace period in some cases,
which is suboptimal in the case where
a small number of CPUs are busy doing RCU updates and the majority of
the CPUs are mostly idle.
This situation occurs frequently in systems sized for peak loads, and
we need to be able to accommodate it gracefully.
Furthermore, we need to fix a long-standing bug in Classic RCU where
a dynticks-idle CPU servicing an interrupt containing a long-running
RCU read-side critical section will fail to prevent an RCU grace period
from ending.
\fi

\QuickQuiz{}
	그런 어처구니 없는 버그가 있다면, Linux 는 대체 어떻게 동작하고
	있는거죠?
	\iffalse

	Given such an egregious bug, why does Linux run at all?
	\fi
\QuickQuizAnswer{
	리눅스 커널은 (상대적으로) 얌전한 디바이스 드라이버들을 가지고 있기
	때문입니다.
	만약 그것들 가운데 일부가 RCU read-side 크리티컬 섹션 내에서 spin 을 수
	밀리세컨드까지 한다면 이는 이 버그를 일으킬 수 있습니다.
	그러나 이 버그는 고쳐져야만 하고, 이 RCU 변종은 그런 수정을 합니다.
	\iffalse

	Because the Linux kernel contains device drivers that are (relatively)
	well behaved.
	Few if any of them spin in RCU read-side critical sections for the
	many milliseconds that would be required to provoke this bug.
	The bug nevertheless does need to be fixed, and this variant of
	RCU does fix it.
	\fi
} \QuickQuizEnd

\begin{figure}[htb]
\centering
\resizebox{3in}{!}{\includegraphics{appendix/rcuimpl/BigTreeClassicRCUBHdyntick}}
\caption{Hierarchical RCU State With Dynticks}
\label{fig:app:rcuimpl:rcutree:Hierarchical RCU State With Dynticks}
\end{figure}

이는 모든 CPU 들이 per-CPU \co{rcu_dynticks} 구조체 안에 위치한 카운터들을
조정하도록 함으로써 이루어집니다.
간단히 말해서, 이 카운터들은 연관된 CPU 가 dynticks idle 모드에 있을 경우에는
짝수의 값을 가지고, 그렇지 않을 때에는 홀수의 값을 갖습니다.
따라서 RCU 는 이 \co{rcu_dynticks} 카운터들이 홀수인 CPU 들에 대해서만
quiescent state 를 기다리면 되고, 그 카운터의 값이 짝수인 잠들어있는 CPU 들은
깨울 필요가 없습니다.
Figure~\ref{fig:app:rcuimpl:rcutree:Hierarchical RCU State With Dynticks} 에
보여진 것처럼, 각각의 per-CPU \co{rcu_dynticks} 구조체는 ``rcu'' 와 ``rcu\_bh''
구현에 의해 공유됩니다.

다음 섹션은 높은 레벨에서의 RCU state machine 에 대한 개요를 보입니다.
\iffalse

This is accomplished by requiring that all CPUs manipulate counters
located in a per-CPU \co{rcu_dynticks} structure.
Loosely speaking, these counters have even-numbered values when the
corresponding CPU is in dynticks idle mode, and have odd-numbered values
otherwise.
RCU thus needs to wait for quiescent states only for those CPUs whose
\co{rcu_dynticks} counters are odd, and need not wake up sleeping
CPUs, whose counters will be even.
As shown in
Figure~\ref{fig:app:rcuimpl:rcutree:Hierarchical RCU State With Dynticks},
each per-CPU \co{rcu_dynticks} structure
is shared by the ``rcu'' and ``rcu\_bh'' implementations.

The following section presents a high-level view of the RCU state machine.
\fi

\subsection{State Machine}
\label{app:rcuimpl:rcutree:State Machine}

\begin{figure}[htbp]
\centering
\resizebox{3in}{!}{\includegraphics{appendix/rcuimpl/GenericRCUStateMachine}}
\caption{Generic RCU State Machine}
\label{fig:app:rcuimpl:rcutree:Generic RCU State Machine}
\end{figure}

충분히 높은 단계에서 보면, 리눅스 커널 RCU 구현은
Figure~\ref{fig:app:rcuimpl:rcutree:Generic RCU State Machine} 에 보인 state
machine 으로 볼 수 있습니다.
바쁘게 동작하는 시스템에서의 이 state machine 의 일반적인 경우의 수행경로는
가장 위쪽 두개의 루프로ㅡ 각각의 grace period (GP) 의 시작으로 초기화되어서,
quiescent state (QS) 를 기다리고, 각 CPU 가 grace period 동안 첫번째 quiescent
state 를 지나가는 경로입니다.
그런 시스템에서는, quiescent state 는 각 컨텍스트 스위치마다 발생할 수도 있고,
idle 이거나 user-mode 코드를 수행중인 CPU 에서는 각 scheduling-clock 인터럽트
마다 발생할 수 있습니다.
CPU-hotplug 이벤트는 state machine 을 ``CPU Offline'' 상자로 가져가게 되며,
quiescent state 로의 빠른 전환을 실패하게 하는 ``holdout'' CPU 들의 존재 시에는
``Send resched IPIs to Hodout CPUs'' 상자로의 전환을 일으킬 겁니다.
불필요하게 dyntick-idle CPU 들을 불필요하게 깨우는 것을 막는 RCU 구현은 그런
CPU 들을 연장된 quiescent state 에 있는 것으로 표시해 둘 것이고 ``CPUs in
dyntick-idle Mode?'' 결정 마름모꼴의 ``Y'' 브랜치를 취하게 될 것입니다 (하지만
dyntick-idle 모드의 CPU 들은 resched IPI 들을 보내지 \emph{않을} 것임을
알아두세요).
마지막으로, \co{CONFIG_RCU_CPU_STALL_DETECTOR} 가 켜져있다면, quiescent state
에 이르기까지의 정말로 너무 긴 딜레이는 ``Complain About Holdout CPUs'' 경로를
수행하게 만들겁니다.
\iffalse

At a sufficiently high level, Linux-kernel RCU implementations can
be thought of as high-level state machines as shown in
Figure~\ref{fig:app:rcuimpl:rcutree:Generic RCU State Machine}.
The common-case path through this state machine on a busy system
goes through the two uppermost loops, initializing at the
beginning of each grace period (GP),
waiting for quiescent states (QS), and noting when each CPU passes through
its first quiescent state for a given grace period.
On such a system, quiescent states will occur on each context switch,
or, for CPUs that are either idle or executing user-mode code, each
scheduling-clock interrupt.
CPU-hotplug events will take the state machine through the
``CPU Offline'' box, while the presence of ``holdout''
CPUs that fail to pass through quiescent states quickly enough will exercise
the path through the ``Send resched IPIs to Holdout CPUs'' box.
RCU implementations that avoid unnecessarily awakening dyntick-idle
CPUs will mark those CPUs as being in an extended quiescent state,
taking the ``Y'' branch out of the ``CPUs in dyntick-idle
Mode?'' decision diamond (but note that CPUs in dyntick-idle mode
will \emph{not} be sent resched IPIs).
Finally, if \co{CONFIG_RCU_CPU_STALL_DETECTOR} is enabled,
truly excessive delays in reaching quiescent states will exercise the
``Complain About Holdout CPUs'' path.
\fi

\QuickQuiz{}
	하지만 이 state diagram 은 dyntick-idle CPU 들이 reschedule IPI 들을
	맞게 될 것을 보이고 있지 않나요?  그게 그것들을 깨워버리지 않을까요?
	\iffalse

	But doesn't this state diagram indicate that dyntick-idle CPUs will
	get hit with reschedule IPIs?  Won't that wake them up?
	\fi
\QuickQuizAnswer{
	아니요.
	RCU 는 CPU 그룹들을 처리하고 있음을 명심하세요.
	하나의 특정한 그룹은 dyntick-idle CPU 들과 어떻게든 quiescent state 를
	지나는 것을 막도록 관리되는 평범한 모드의 CPU 들을 모두 포함하고 있을
	수 있습니다.
	뒤쪽 그룹들만이 reschedule IPI 를 받을 겁니다; dyntick-idle CPU 들은
	단지 extended quiescent state 로 표시되어질 뿐입니다.
	\iffalse

	No.
	Keep in mind that RCU is handling groups of CPUs.
	One particular group might contain both dyntick-idle CPUs and
	CPUs in normal mode that have somehow managed to avoid passing through
	a quiescent state.
	Only the latter group will be sent a reschedule IPI; the dyntick-idle
	CPUs will merely be marked as being in an extended quiescent state.
	\fi
} \QuickQuizEnd

\begin{figure}[htb]
\centering
\resizebox{3in}{!}{\includegraphics{appendix/rcuimpl/TreeRCUStateMachine}}
\caption{RCU State Machine and Hierarchical RCU Data Structures}
\label{fig:app:rcuimpl:rcutree:RCU State Machine and Hierarchical RCU Data Structures}
\end{figure}

앞의 satte machine 의 내용은 또다른 데이터 구조와 상호작용하게 되는데,
Figure~\ref{fig:app:rcuimpl:rcutree:RCU State Machine and Hierarchical RCU Data Structures}
에 보인 것과 같습니다.
하지만, 이 state machine 의 내용은 어떤 RCU 구현에 있어서도 그대로 C 코드로
변환되어지지는 않습니다.
그보다는, 이 구현들은 커널 내에 event-driven 코드로 만들어져 있습니다.
따라서, 다음 섹션은 일부 ``사용 예'', 또는 RCU 알고리즘이 앞의 state machine 의
내용을 진행하는 방법은 물론이고 관련된 데이터 구조를 알아봅니다.
\iffalse

The events in the above state schematic interact with different
data structures, as shown in
Figure~\ref{fig:app:rcuimpl:rcutree:RCU State Machine and Hierarchical RCU Data Structures}.
However, the state schematic does not directly translate into C code
for any of the RCU implementations.
Instead, these implementations are coded as an event-driven system within
the kernel.
Therefore, the following section describes some ``use cases'',
or ways in which the RCU algorithm traverses the above state schematic
as well as the relevant data structures.
\fi

\subsection{Use Cases}
\label{app:rcuimpl:rcutree:Use Cases}

이 섹션은 사용되는 데이터 구조들과 호출되는 함수들을 나열하며 RCU 구현 내에서의
몇가지 ``사용 예'' 들에 대한 개요를 제공합니다.
이 사용 예들은 다음과 같습니다:
\iffalse

This section gives an overview of several ``use cases''
within the RCU implementation, listing the data structures touched
and the functions invoked.
The use cases are as follows:
\fi

\begin{enumerate}
\item	새로운 Grace Period 의 시작
	(Section~\ref{app:rcuimpl:rcutree:Start a New Grace Period})
\item	Quiescent State 의 통과
	(Section~\ref{app:rcuimpl:rcutree:Pass Through a Quiescent State})
\item	RCU 에게 Quiescent State 를 알리기
	(Section~\ref{app:rcuimpl:rcutree:Announce a Quiescent State to RCU})
\item	Dynticks Idle Mode 의 진입과 빠져나오기
	(Section~\ref{app:rcuimpl:rcutree:Enter and Leave Dynticks Idle Mode})
\item	Dynticks Idle Mode 로부터의 인터럽트
	(Section~\ref{app:rcuimpl:rcutree:Interrupt from Dynticks Idle Mode})
\item	Dynticks Idle Mode 로부터의 NMI
	(Section~\ref{app:rcuimpl:rcutree:NMI from Dynticks Idle Mode})
\item	CPU 가 Dynticks Idle Mode 에 있음을 알기
	(Section~\ref{app:rcuimpl:rcutree:Note That a CPU is in Dynticks Idle Mode})
\item	CPU 를 끄기
	(Section~\ref{app:rcuimpl:rcutree:Offline a CPU})
\item	CPU 를 켜기
	(Section~\ref{app:rcuimpl:rcutree:Online a CPU})
\item	너무 긴 Grace Period 를 감지하기
	(Section~\ref{app:rcuimpl:rcutree:Detect a Too-Long Grace Period})
\iffalse

\item	Start a New Grace Period
	(Section~\ref{app:rcuimpl:rcutree:Start a New Grace Period})
\item	Pass Through a Quiescent State
	(Section~\ref{app:rcuimpl:rcutree:Pass Through a Quiescent State})
\item	Announce a Quiescent State to RCU
	(Section~\ref{app:rcuimpl:rcutree:Announce a Quiescent State to RCU})
\item	Enter and Leave Dynticks Idle Mode
	(Section~\ref{app:rcuimpl:rcutree:Enter and Leave Dynticks Idle Mode})
\item	Interrupt from Dynticks Idle Mode
	(Section~\ref{app:rcuimpl:rcutree:Interrupt from Dynticks Idle Mode})
\item	NMI from Dynticks Idle Mode
	(Section~\ref{app:rcuimpl:rcutree:NMI from Dynticks Idle Mode})
\item	Note That a CPU is in Dynticks Idle Mode
	(Section~\ref{app:rcuimpl:rcutree:Note That a CPU is in Dynticks Idle Mode})
\item	Offline a CPU
	(Section~\ref{app:rcuimpl:rcutree:Offline a CPU})
\item	Online a CPU
	(Section~\ref{app:rcuimpl:rcutree:Online a CPU})
\item	Detect a Too-Long Grace Period
	(Section~\ref{app:rcuimpl:rcutree:Detect a Too-Long Grace Period})
\fi
\end{enumerate}

이 사용 예들 각각은 다음 섹션들에서 설명됩니다.
\iffalse

Each of these use cases is described in the following sections.
\fi

\subsubsection{Start a New Grace Period}
\label{app:rcuimpl:rcutree:Start a New Grace Period}

\co{rcu_start_gp()} 함수는 새로운 grace period 를 시작합니다.
이 함수는 CPU 가 grace period 를 기다리는 콜백을 가지고 있는 CPU 가 어떤 grace
period 도 진행중이 아니란 것을 알릴 때에 호출됩니다.

\co{rcu_start_gp()} 함수는 새로이 시작된 grace period 를 알리기 위해
\co{rcu_state} 와 \co{rcu_data} 구조체의 상태를 업데이트 하는데, 동시의
CPU-hotplug 오퍼레이션들을 배제하기 위해 \co{->onoff} 락을 잡고 (그리고 irq
들을 불능화 시키고), (이 CPU 를 포함해) 모든 CPU 들은 quiescent state 를
지나야만 함을 알리기 위해 모든 \co{rcu_node} 구조체의 bit 들의 값을 설정하고,
마지막으로 \co{->onoff} 락을 해제합니다.
\iffalse

The \co{rcu_start_gp()} function starts a new grace period.
This function is invoked when a CPU having callbacks waiting for a
grace period notices that no grace period is in progress.

The \co{rcu_start_gp()} function updates state in
the \co{rcu_state} and \co{rcu_data} structures
to note the newly started grace period,
acquires the \co{->onoff} lock (and disables irqs) to exclude
any concurrent CPU-hotplug operations,
sets the
bits in all of the \co{rcu_node} structures to indicate
that all CPUs (including this one) must pass through a quiescent
state,
and finally
releases the \co{->onoff} lock.
\fi

이 bit 값 설정 동작은 두개의 페이즈로 진행됩니다.
먼저, non-leaf \co{rcu_node} 구조체의 bit 들은 어떤 락도 더이상 잡지 않은채
설정되고, 이후엔 각각의 leaf \co{rcu_node} 구조체들의 bit 들이 해당 구조체의
\co{->lock} 을 잡은채 설정되어집니다.
\iffalse

The bit-setting operation is carried out in two phases.
First, the non-leaf \co{rcu_node} structures' bits are set without
holding any additional locks, and then finally each leaf \co{rcu_node}
structure's bits are set in turn while holding that structure's
\co{->lock}.
\fi

\QuickQuiz{}
	하지만 한 CPU 가 bit 값 설정하는 CPU 가 종료되기 전에 (자신의 bit 의
	값을 해제하면서) quiescent state 를 지나감을 이야기 하려 하면 어떻게
	되나요?
	\iffalse

	But what happens if a CPU tries to report going through a quiescent
	state (by clearing its bit) before the bit-setting CPU has finished?
	\fi
\QuickQuizAnswer{
	여기서 고려할 세개의 경우가 존재합니다:
	\iffalse

	There are three cases to consider here:
	\fi

	\begin{enumerate}
	\item	아직 초기화 되지 않은 leaf \co{rcu_node} 구조체에 연관된 CPU 가
		quiescent state 를 보고하려 합니다.
		이 CPU 는 이미 자신의 bit 의 값을 해제했고, 따라서 자신의
		quiescent state 를 보고하는 것을 포기할 겁니다.
		나중의 quiescent state 가 새로운 grace period 를 위해 동작될
		겁니다.
	\item	현재 초기화 중인 leaf \co{rcu_node} 구조체에 연관된 CPU 가
		quiescent state 를 보고하려 합니다.
		이 CPU 는 해당 \co{rcu_node} 구조체의 \co{->lock} 이 잡혀
		있음을 보게 되고, 따라서 그 락이 해제될 때까지 spin 을 하게
		됩니다.
		하지만 일단 이 락이 해제되면, \co{rcu_node} 구조체는 초기화
		되었을 것이므로, 다음 경우로 합쳐집니다.
	\item	이미 초기화 된 leaf \co{rcu_node} 에 연관된 CPU 가 quiescent
		state 를 보고하려 합니다.
		이 CPU 는 자신의 bit 에 값이 설정된 것을 보게 되고, 따라서 이를
		해제하려 합니다.
		만약 이게 해당 leaf node 의 마지막 CPU 라면, 계층의 다음 레벨로
		올라갈 겁니다.
		하지만, 이 CPU 는 시스템에서 quiescent state 를 보고하려는
		마지막 CPU 일 수가 없는데, 초기화를 하는 CPU 는 아직
		검사되어졌을 수가 없기 때문입니다.
	\iffalse

	\item	A CPU corresponding to a non-yet-initialized leaf
		\co{rcu_node} structure tries to report a quiescent state.
		This CPU will see its bit already cleared, so will give up on
		reporting its quiescent state.
		Some later quiescent state will serve for the new grace period.
	\item	A CPU corresponding to a leaf \co{rcu_node} structure that
		is currently being initialized tries to report a quiescent
		state.
		This CPU will see that the \co{rcu_node} structure's
		\co{->lock} is held, so will spin until it is
		released.
		But once the lock is released, the \co{rcu_node}
		structure will have been initialized, reducing to the
		following case.
	\item	A CPU corresponding to a leaf \co{rcu_node} that has
		already been initialized tries to report a quiescent state.
		This CPU will find its bit set, and will therefore clear it.
		If it is the last CPU for that leaf node, it will
		move up to the next level of the hierarchy.
		However, this CPU cannot possibly be the last CPU in the
		system to report a quiescent state, given that the CPU
		doing the initialization cannot yet have checked in.
	\fi
	\end{enumerate}

	따라서, 모든 세개의 경우에 있어서, 잠재적인 경주상황은 모두 올바르게
	해결됩니다.
	\iffalse

	So, in all three cases, the potential race is resolved correctly.
	\fi
} \QuickQuizEnd

\QuickQuiz{}
	Bit 값 설정 CPU 가 종료되기 전에 \emph{모든} CPU 들이 quiescent state
	를 지나감을 보고하려고 해서 새로운 grace period 를 시작도 하기 전에
	종료해 버리면 무슨 일이 일어나나요?
	\iffalse

	And what happens if \emph{all} CPUs try to report going
	through a quiescent
	state before the bit-setting CPU has finished, thus ending the new
	grace period before it starts?
	\fi
\QuickQuizAnswer{
	Bit 의 값을 설정하는 CPU 는 초기화 중에 quiescent state 를 지나갈 수가
	없는데, irq 들을 불능화 시켰기 때문입니다.
	따라서 해당 bit 들은 0이 아닌 값으로 남아있어서, 데이터 구조가 완전히
	초기화 되기 전에 grace period 가 종료되는 것을 막습니다.
	\iffalse

	The bit-setting CPU cannot pass through a
	quiescent state during initialization, as it has irqs disabled.
	Its bits therefore remain non-zero, preventing the grace period from
	ending until the data structure has been fully initialized.
	\fi
} \QuickQuizEnd

\subsubsection{Pass Through a Quiescent State}
\label{app:rcuimpl:rcutree:Pass Through a Quiescent State}

RCU 의 rcu 와 rcu\_bh 향은 서로 다른 quiescent state 집합을 갖습니다.
Quiescent state 는 rcu 에 있어서는 컨텍스트 스위치, idle (dynticks 또는 idle
루프), 그리고 user-mode 수행인 반면, rcu\_bh 에 있어서는 interrupt 가 활성화 된
채의 softirq 바깥 모든 코드입니다.
Rcu 를 위한 quiescent state 는 곧 rcu\_bh 를 위한 quiescent state 이기도 하다는
점을 알아 두시기 바랍니다.
Rcu 를 위한 quiescent state 들은 \co{rcu_qsctr_inc()} 를 호출하는 것으로
기록되고, rcu\_bh 를 위한 quiescent state 들은 \co{rcu_bh_qsctr_inc()} 를
호출함으로써 기록되어집니다.
이 두개의 함수들은 각자의 상태를 현재 CPU 의 \co{rcu_data} 구조체 안에
기록합니다.
\iffalse

The rcu and rcu\_bh flavors of RCU have different sets of quiescent
states.
Quiescent states for rcu are context switch, idle (either dynticks or
the idle loop), and user-mode execution, while quiescent states for
rcu\_bh are any code outside of softirq with interrupts enabled.
Note that an quiescent state for rcu is also a quiescent state
for rcu\_bh.
Quiescent states for rcu are recorded by invoking \co{rcu_qsctr_inc()},
while quiescent states for rcu\_bh are recorded by invoking
\co{rcu_bh_qsctr_inc()}.
These two functions record their state in the current CPU's
\co{rcu_data} structure.
\fi

이 함수들은 스케쥴러에서 호출되어지는데, \co{__do_softirq()} 와
\co{rcu_check_callbacks()} 에서입니다.
뒤의 함수는 scheduling-clock 인터럽트에 의해서 호출되어지며, 이 인터럽트가
\co{rcu_qsctr_inc()} 또는 \co{rcu_bh_qsctr_inc()} 를 호출한 quiescent state
에서 발생한 것인지를 판단하기 위해 상태를 분석합니다.
이 함수는 또한 \co{RCU_SOFTIRQ} 에 의해서도 호출되는데, 나중에 softirq
컨텍스트에서 현재 CPU 에서 \co{rcu_process_callbacks()} 를 호출되게 합니다.
\iffalse

These functions are invoked from the scheduler, from
\co{__do_softirq()}, and from \co{rcu_check_callbacks()}.
This latter function is invoked from the scheduling-clock interrupt,
and analyzes state to determine whether this interrupt occurred within
a quiescent state, invoking \co{rcu_qsctr_inc()} and/or
\co{rcu_bh_qsctr_inc()}, as appropriate.
It also raises \co{RCU_SOFTIRQ}, which results in
\co{rcu_process_callbacks()} being invoked on the current
CPU at some later time from softirq context.
\fi

\subsubsection{Announce a Quiescent State to RCU}
\label{app:rcuimpl:rcutree:Announce a Quiescent State to RCU}

앞에서 언급된 \co{rcu_process_callbacks()} 함수는 몇가지 의무를 가지고
있습니다:
\iffalse

The afore-mentioned \co{rcu_process_callbacks()} function
has several duties:
\fi

\begin{enumerate}
\item	언제 너무 오래 유지된 grace period 를 (\co{force_quiescent_state()}를
	통해) 끝낼지 판단하는 것.
\item	다른 CPU 가 grace period 의 끝을 파악했을 때 (\co{rcu_process_gp_end()}
	를 통해) 적절한 행동을 취하는 것.
	``적절한 행동`` 은 이 CPU 의 콜백들을 수행하고 새로운 grace period 를
	기록하는 것을 포함합니다.
	이 똑같은 함수는 다른 CPU 가 새로운 grace period 를 시작하는 것에 대한
	응답으로 상태를 업데이트 합니다.
\item	현재 CPU 의 quiescent state 들을 (결국은 \co{cpu_quiet()} 를 호출하는
	\co{rcu_check_quiescent_state()} 를 통해) RCU 메커니즘의 핵심에 알리는
	것.
	이는 물론 현재 grace period 의 끝을 표시합니다.
\item	진행중인 grace period 가 없고 이 CPU 가 여전히 grace period 를 기다리고
	있는 RCU 콜백들을 가지고 있다면 (\co{cpu_needs_another_gp()} 와
	\co{rcu_start_gp()} 를 통해) 새로운 grace period 를 시작하는 것.
\item	Grace period 가 끝난 이 CPU 의 콜백들을 (\co{rcu_do_batch()} 를 통해)
	호출하는 것.
\iffalse

\item	Determining when to take measures to end an over-long grace period
	(via \co{force_quiescent_state()}).
\item	Taking appropriate action when some other CPU detected the end of
	a grace period (via \co{rcu_process_gp_end()}).
	``Appropriate action`` includes advancing this CPU's
	callbacks and recording the new grace period.
	This same function updates state in response to some other
	CPU starting a new grace period.
\item	Reporting the current CPU's quiescent states to the core RCU
	mechanism (via \co{rcu_check_quiescent_state()}, which
	in turn invokes \co{cpu_quiet()}).
	This of course might mark the end of the current grace period.
\item	Starting a new grace period if there is no grace period in progress
	and this CPU has RCU callbacks still waiting for a grace period
	(via \co{cpu_needs_another_gp()} and
	\co{rcu_start_gp()}).
\item	Invoking any of this CPU's callbacks whose grace period has ended
	(via \co{rcu_do_batch()}).
\fi
\end{enumerate}

이 동작들은 앞의 grace period 로부터의 quiescent state 를 현재의 grace period
에게 보고한다던가 하는 것과 같은 버그를 막기 위해 매우 조심스럽게 짜여져
동작합니다.
\iffalse

These interactions are carefully orchestrated in order to avoid
buggy behavior such as reporting a quiescent state from the previous
grace period against the current grace period.
\fi

\subsubsection{Enter and Leave Dynticks Idle Mode}
\label{app:rcuimpl:rcutree:Enter and Leave Dynticks Idle Mode}

스케쥴러는 dynticks-idle 모드에 들어가기 위해 \co{rcu_enter_nohz()} 를
호출하고, 거기서 나오기 위해 \co{rcu_exit_nohz()} 를 호출합니다.
\co{rcu_enter_nohz()} 함수는 per-CPU \co{dynticks_nesting} 변수와 per-CPU
\co{dynticks} 변수의 값을 증가시키는데, \co{dynticks} 는 따라서 짝수의 값을
가져야만 합니다.
\co{rcu_exit_nohz()} 함수는 같은 per-CPU \co{dynticks_nesting} 변수의 값을
감소시키고 per-CPU \co{dynticks} 카운터의 값을 증가시키는데, 따라서
\co{dynticks} 는 홀수 값을 갖게 됩니다.

\co{dynticks} 카운터는 다른 CPU 에 의해 보여질 수 있습니다.
만약 그 값이 짝수라면, 첫번째 CPU 는 연장된 quiescent state 안에 있는 것입니다.
비슷하게, 만약 그 카운터의 값이 grace period 동안 바뀌었다면, 첫번째 CPU 는
grace period 동안의 언젠가에 연장된 quiescent state 안에 있었던 셈입니다.
하지만, 또한 보여져야만 하는 \co{dynticks_nmi} per-CPU 변수가 하나 더 있는데,
이에 대해 아래에서 설명합니다.
\iffalse

The scheduler invokes \co{rcu_enter_nohz()} to
enter dynticks-idle mode, and invokes \co{rcu_exit_nohz()}
to exit it.
The \co{rcu_enter_nohz()} function increments a per-CPU
\co{dynticks_nesting} variable and
also a per-CPU \co{dynticks} counter, the latter of which must
then have an even-numbered value.
The \co{rcu_exit_nohz()} function decrements this same
per-CPU \co{dynticks_nesting} variable,
and again increments the per-CPU \co{dynticks}
counter, the latter of which must then have an odd-numbered value.

The \co{dynticks} counter can be sampled by other CPUs.
If the value is even, the first CPU is in an extended quiescent state.
Similarly, if the counter value changes during a given grace period,
the first CPU must have been in an extended quiescent state at some
point during the grace period.
However, there is another \co{dynticks_nmi} per-CPU variable
that must also be sampled, as will be discussed below.
\fi

\subsubsection{Interrupt from Dynticks Idle Mode}
\label{app:rcuimpl:rcutree:Interrupt from Dynticks Idle Mode}

Dynticks idle 모드로부터의 인터럽트는 \co{rcu_irq_enter()} 와
\co{rcu_irq_exit()} 에서 처리되어집니다.
\co{rcu_irq_enter()} 함수는 per-CPU \co{dynticks_nesting} 변수의 값을
증가시키고, 만약 기존 값이 0 이었다면, \co{dynticks} per-CPU 변수의 값 또한
증가시킵니다 (이제 홀수 값을 갖게 될겁니다).

\co{rcu_irq_exit()} 함수는 per-CPU \co{dynticks_nesting} 변수의 값을
감소시키고, 새로운 값이 0이라면, \co{dynticks} per-CPU 변수의 값을 증가시킵니다
(이제 짝수 값을 갖게 될겁니다).

Irq 핸들러에 들어가게 되면 dynticks idle 모드를 빠져나가는 것이고 그 반대도
성립됨을 알아두시기 바랍니다.
이렇게 들어가고 빠져나오는게 서로 일치하지 못하는 것은 상당한 혼란을 가져올 수
있습니다.
경고 드렸습니다.
\iffalse

Interrupts from dynticks idle mode are handled by
\co{rcu_irq_enter()} and \co{rcu_irq_exit()}.
The \co{rcu_irq_enter()} function increments the
per-CPU \co{dynticks_nesting} variable, and, if the prior
value was zero, also increments the \co{dynticks}
per-CPU variable (which must then have an odd-numbered value).

The \co{rcu_irq_exit()} function decrements the
per-CPU \co{dynticks_nesting} variable, and, if the new
value is zero, also increments the \co{dynticks}
per-CPU variable (which must then have an even-numbered value).

Note that entering an irq handler exits dynticks idle mode
and vice versa.
This enter/exit anti-correspondence can cause much confusion.
You have been warned.
\fi

\subsubsection{NMI from Dynticks Idle Mode}
\label{app:rcuimpl:rcutree:NMI from Dynticks Idle Mode}

Dynticks idle 모드로부터의 NMI 들은 \co{rcu_nmi_ente()} 와 \co{rcu_nmi_exit()}
에서 처리됩니다.
이 함수들은 둘 다 \co{dynticks_nmi} 카운터의 값을 증가시키는데, 다만 앞서
이야기한 \co{dynticks} 카운터의 값이 짝수일 때에만 그렇습니다.
달리 말하자면, NMI 가 dynticks-idle 모드가 아닐 때에 또는 인터럽트 핸들러
안에서 발생했다면 \co{dynticks_nmi} 카운터를 조정하지 않습니다.

두 함수간의 유일한 차이점은 에러 체크로, \co{rcu_nmi_enter()} 는
\co{dynticks_nmi} 카운터를 그 값을 홀수로 한 채 끝나야 하고,
\co{rcu_nmi_exit()} 는 이 카운터의 값이 짝수가 되게 하고 끝나야 하기
때문입니다.
\iffalse

NMIs from dynticks idle mode are handled by \co{rcu_nmi_enter()}
and \co{rcu_nmi_exit()}.
These functions both increment the \co{dynticks_nmi} counter,
but only if the aforementioned \co{dynticks} counter is even.
In other words, NMI's refrain from manipulating the
\co{dynticks_nmi} counter if the NMI occurred in non-dynticks-idle
mode or within an interrupt handler.

The only difference between these two functions is the error checks,
as \co{rcu_nmi_enter()} must leave the \co{dynticks_nmi}
counter with an odd value, and \co{rcu_nmi_exit()} must leave
this counter with an even value.
\fi

\subsubsection{Note That a CPU is in Dynticks Idle Mode}
\label{app:rcuimpl:rcutree:Note That a CPU is in Dynticks Idle Mode}

\co{force_quiescent_state()} 함수는 세 단계의 state machine 을 구현합니다.
첫번째 단계 (\co{RCU_INITIALIZING}) 은 \co{rcu_start_gp()} 가 grace-period
초기화를 완료하기를 기다립니다.
이 상태는 \co{force_quiescent_state()} 에 의해서가 아니라 \co{rcu_start_gp()}
에 의해 종료됩니다.

두번째 단계 (\co{RCU_SAVE_DYNTICK}) 에서는 \co{dyntick_save_progress_counter()}
함수가 아직 quiescent state 를 보고하지 않은 CPU 들을 스캔하면서 각각의 per-CPU
\co{dynticks} 와 \co{dynticks_nmi} 카운터들의 값을 기록합니다.
만약 이 카운터들이 모두 짝수 값을 갖는다면, 연관된 CPU 는 dynticks-idle 상태에
있는 것으로, 따라서 extended quiescent state 로 알려지는 겁니다
(\co{cpu_quiet_msk()} 로 보고됩니다).
\iffalse

The \co{force_quiescent_state()} function implements a
three-phase state machine.
The first phase (\co{RCU_INITIALIZING}) waits for \co{rcu_start_gp()}
to complete grace-period initialization.
This state is not exited by \co{force_quiescent_state()}, but rather
by \co{rcu_start_gp()}.

In the second phase (\co{RCU_SAVE_DYNTICK}), the
\co{dyntick_save_progress_counter()} function scans the CPUs that
have not yet reported a quiescent state, recording their per-CPU
\co{dynticks} and \co{dynticks_nmi} counters.
If these counters both have even-numbered values, then the corresponding
CPU is in dynticks-idle state, which is therefore noted as an extended
quiescent state (reported via \co{cpu_quiet_msk()}).
\fi

세번째 단계 (\co{RCU_FORCE_QS}) 에서는 \co{rcu_implicit_dynticks_qs()} 함수가
다시 (명시적으로든 \co{RCU_SAVE_DYNTICK}) 단계를 통해 묵시적으로든) 아직
quiescent state 를 보고하지 않은 CPU 들을 스캔하면서, per-CPU \co{dynticks} 와
\co{dynticks_nmi} 카운터들을 스캔합니다.
이것들 각각이 그 값을 바꿨거나 이제 짝수라면, 거기에 연관된 CPU 는 dynticks
idle 을 지나왔거나 아직 그 안에 있는 것으로, 앞에서와 같이 extended quiescent
state 에 있는 것을 알리는 겁니다.

\co{rcu_implicit_dynticks_qs()} 가 특정 CPU 가 dynticks idle 모드에 들어가
있지도 않고 quiescent state 를 보고하지도 않았음을 알게 되면, 이 함수는
\co{rcu_implicit_offline_qs()} 를 호출하는데, 이 함수는 해당 CPU 가 offline
인지 확인하는데, offline 임은 또한 extended quiescent state 로 보고되어
있습니다.
만약 해당 CPU 가 online 상태라면, \co{rcu_implicit_offline_qs()} 는 여기에 RCU
로 quiescent state 를 보고할 의무를 다시 한번 알리기 위해 reschedule IPI 를
날립니다.

\co{force_quiescent_state()} 는 지겁적으로 \co{dyntick_save_progress_counter()}
도 \co{rcu_implicit_dynticks_qs()} 도 호출하지 않고, 대신 이 함수들을 CPU 들을
스캔하고 extended quiescent state 를 보고하는데에 연관된 공통된 코드들을 추상화
시킨 함수인, \co{rcu_process_dyntick()} 함수에게 넘김을 알아두시기 바랍니다.
\iffalse

In the third phase (\co{RCU_FORCE_QS}), the
\co{rcu_implicit_dynticks_qs()} function again scans the CPUs
that have not yet reported a quiescent state (either explicitly or
implicitly during the \co{RCU_SAVE_DYNTICK} phase), again checking the
per-CPU \co{dynticks} and \co{dynticks_nmi} counters.
If each of these has either changed in value or is now even, then
the corresponding CPU has either passed through or is now in dynticks
idle, which as before is noted as an extended quiescent state.

If \co{rcu_implicit_dynticks_qs()} finds that a given CPU
has neither been in dynticks idle mode nor reported a quiescent state,
it invokes \co{rcu_implicit_offline_qs()}, which checks to see
if that CPU is offline, which is also reported as an extended quiescent
state.
If the CPU is online, then \co{rcu_implicit_offline_qs()} sends
it a reschedule IPI in an attempt to remind it of its duty to report
a quiescent state to RCU.

Note that \co{force_quiescent_state()} does not directly
invoke either \co{dyntick_save_progress_counter()} or
\co{rcu_implicit_dynticks_qs()}, instead passing these functions
to an intervening \co{rcu_process_dyntick()} function that
abstracts out the common code involved in scanning the CPUs and reporting
extended quiescent states.
\fi

\QuickQuiz{}
	한 CPU 가 dyntick-idle 모드에서 빠져나오고선 다른 CPU 가 첫번째 CPU 가
	dyntick-idle 모드에 있었음을 알게 되자마자 quiescent state 를
	지나가버리면 어떻게 되나요?
	두 CPU 모두 동시에 quiescent state 를 보고하려 해서, 혼란이 생기지
	않을까요?
	\iffalse

	And what happens if one CPU comes out of dyntick-idle mode and then
	passed through a quiescent state just as another CPU notices that the
	first CPU was in dyntick-idle mode?
	Couldn't they both attempt to report a quiescent state at the same
	time, resulting in confusion?
	\fi
\QuickQuizAnswer{
	두 CPU 모두 같은 leaf \co{rcu_node} 구조체의 락을 잡으려 할겁니다.
	락을 잡으려 하는 첫번째 것이 quiescent state 를 보고하고 연관된 bit 의
	값을 지울 거고, 두번째 것은 락을 잡고나서 이 bit 이 이미 지워져 있음을
	보게 될겁니다.
	\iffalse

	They will both attempt to acquire the lock on the same leaf
	\co{rcu_node} structure.
	The first one to acquire the lock will report the quiescent state
	and clear the appropriate bit, and the second one to acquire the
	lock will see that this bit has already been cleared.
	\fi
} \QuickQuizEnd

\QuickQuiz{}
	하지만 \emph{모든} CPU 들이 dyntick-idle 모드에 있으면 어떻게 되죠?
	이게 현재의 RCU grace period 가 영원히 끝나지 못하게 하지는 않을까요?
	\iffalse

	But what if \emph{all} the CPUs end up in dyntick-idle mode?
	Wouldn't that prevent the current RCU grace period from ever ending?
	\fi
\QuickQuizAnswer{
	실제로 그럴겁니다!
	하지만, RCU 콜백들을 가지고 있는 CPU 들은 dyntick-idle 모드에 빠질 수
	없게 되어 있어서, \emph{모든} CPU 들이 dyntick-idle 모드에 빠질 수 있는
	유일한 방법은 시스템 내에 RCU 콜백이 정말 하나도 없을 때입니다.
	그리고 시스템에 RCU 콜백이 하나도 없다면, RCU grace period 가 끝나길
	기다려야 할 이유도 없죠.
	사실, RCU grace period 가 \emph{시작}될 필요도 없는 겁니다.

	만약 어떤 irq 핸들러가 RCU 콜백이 연관된 CPU 에 보이게 해서 해당 CPU 가
	dyntick-idle 모드의 바깥으로 나오도록 강제할 거고, 결국 현재의 RCU
	grace period 가 종료되도록 하는 \co{call_rcu()} 를 한다면 RCU 가 재시작
	될겁니다.
	\iffalse

	Indeed it will!
	However, CPUs that have RCU callbacks are not permitted to enter
	dyntick-idle mode, so the only way that \emph{all} the CPUs could
	possibly end up in dyntick-idle mode would be if there were
	absolutely no RCU callbacks in the system.
	And if there are no RCU callbacks in the system, then there is no
	need for the RCU grace period to end.
	In fact, there is no need for the RCU grace period to even
	\emph{start}.

	RCU will restart if some irq handler does a \co{call_rcu()},
	which will cause an RCU callback to appear on the corresponding CPU,
	which will force that CPU out of dyntick-idle mode, which will in turn
	permit the current RCU grace period to come to an end.
	\fi
} \QuickQuizEnd

\QuickQuiz{}
	\co{force_quiescent_state()} 가 세단계 state machine 이란 것을 놓고
	보면, 모든 CPU 들을 스캐닝 하는 것 때문에 스케쥴링 응답시간을 세배로
	늘리는 것 아닌가요?
	\iffalse

	Given that \co{force_quiescent_state()} is a three-phase state
	machine, don't we have triple the scheduling latency due to scanning
	all the CPUs?
	\fi
\QuickQuizAnswer{
	아, 하지만 이 세단계는 같은 CPU 위에서 back-to-back 으로 수행되진 않을
	거고, 더 나아가서 첫번째 (초기화) 단계는 스캐닝을 아예 하지 않아요.
	따라서, 이 세단계 알고리즘에서의 스케쥴링 응답시간은 한단계 알고리즘의
	것과 크게 다르지 않습니다.
	스케쥴링 응답시간이 문제가 된다면, state machine 이 CPU 들을 점진적으로
	스캔하도록 상태를 per-leaf-\co{rcu_node} 에 저장하는 방식으로 코드를
	고치는게 한가지 방법이 될 수 있습니다.
	하지만 일단은 실제 세계에서 발생하는 문제를 제게 보여주세요,
	\emph{그러면} 제가 그걸 고치겠습니다!
	\iffalse

	Ah, but the three phases will not execute back-to-back on the same CPU,
	and, furthermore, the first (initialization) phase doesn't do any
	scanning.
	Therefore, the scheduling-latency hit of the three-phase algorithm
	is no different than that of a single-phase algorithm.
	If the scheduling latency becomes a problem, one approach would be to
	recode the state machine to scan the CPUs incrementally, most likely
	by keeping state on a per-leaf-\co{rcu_node} basis.
	But first show me a problem in the real world, \emph{then}
	I will consider fixing it!
	\fi
} \QuickQuizEnd

\subsubsection{Offline a CPU}
\label{app:rcuimpl:rcutree:Offline a CPU}

CPU-offline 이벤트는 \co{rcu_cpu_notify()} 가 \co{rcu_offline_cpu()} 를
호출하게 하며, 이 함수는 결국 해당 데이터 구조의 rcu 와 rcu\_bh 인스턴스
모두에서 \co{__rcu_offline_cpu()} 를 호출하게 됩니다.
이 함수는 꺼지는 CPU 의 bit 들의 값을 지워서 나중의 grace period 가 이 CPU 가
quiescent state 를 발표할 것이라 기대하지 않게끔 하고, 나아가서는 offline 으로
발생한 extended quiescent state 를 알리기 위해 \co{cpu_quiet()} 를 호출합니다.
이 작업은 동시의 grace-period 초기화와의 상호간섭을 막기 위해 잡혀지는,
전역적인 \co{->onofflock} 과 함께 수행됩니다.
\iffalse

CPU-offline events cause \co{rcu_cpu_notify()} to invoke
\co{rcu_offline_cpu()}, which in turn invokes
\co{__rcu_offline_cpu()} on both the rcu and the rcu\_bh
instances of the data structures.
This function clears the outgoing CPU's bits so that future grace
periods will not expect this CPU to announce quiescent states,
and further invokes \co{cpu_quiet()} in order to announce
the offline-induced extended quiescent state.
This work is performed with the global \co{->onofflock}
held in order to prevent interference with concurrent grace-period
initialization.
\fi

\QuickQuiz{}
	하지만 \co{->onofflock} 을 잡는 또다른 이유는 여러개의 동시적인
	online/offline 오퍼레이션들을 방지하기 위함이 맞지요?
	\iffalse

	But the other reason to hold \co{->onofflock} is to prevent
	multiple concurrent online/offline operations, right?
	\fi
\QuickQuizAnswer{
	분명, 아닙니다!
	CPU-hotplug 코드의 동기화 설계는 복수의 동시적인 CPU online/offline
	오퍼레이션들을 방지하므로, 한번에 하나의 CPU 의 online/offline
	오퍼레이션만이 수행될 수 있습니다.
	따라서, \co{->onofflock} 의 유일한 목적은 CPU online 또는 offline
	오퍼레이션이 grace-period 초기화와 동시에 수행되는 것을 막는 것입니다.
	\iffalse

	Actually, no!
	The CPU-hotplug code's synchronization design prevents multiple
	concurrent CPU online/offline operations, so only one CPU
	online/offline operation can be executing at any given time.
	Therefore, the only purpose of \co{->onofflock} is to prevent a CPU
	online or offline operation from running concurrently with grace-period
	initialization.
	\fi
} \QuickQuizEnd

\subsubsection{Online a CPU}
\label{app:rcuimpl:rcutree:Online a CPU}

CPU-online events cause \co{rcu_cpu_notify()} to invoke
\co{rcu_online_cpu()}, which initializes the incoming CPU's
dynticks state, and then invokes \co{rcu_init_percpu_data()}
to initialize the incoming CPU's \co{rcu_data} structure,
and also to set this CPU's bits (again protected by
the global \co{->onofflock}) so that future grace periods
will wait for a quiescent state from this CPU.
Finally, \co{rcu_online_cpu()}
sets up the RCU softirq vector for this CPU.

\QuickQuiz{}
	Given all these acquisitions of the global \co{->onofflock},
	won't there
	be horrible lock contention when running with thousands of CPUs?
\QuickQuizAnswer{
	Actually, there can be only three acquisitions of this lock per grace
	period, and each grace period lasts many milliseconds.
	One of the acquisitions is by the CPU initializing for the current
	grace period, and the other two onlining and offlining some CPU.
	These latter two cannot run concurrently due to the CPU-hotplug
	locking, so at most two CPUs can be contending for this lock at any
	given time.

	Lock contention on \co{->onofflock} should therefore
	be no problem, even on systems with thousands of CPUs.
} \QuickQuizEnd

\QuickQuiz{}
	Why not simplify the code by merging the detection of dyntick-idle
	CPUs with that of offline CPUs?
\QuickQuizAnswer{
	It might well be that such merging may eventually be the right
	thing to do.
	In the meantime, however, there are some challenges:

	\begin{enumerate}
	\item	CPUs are not allowed to go into dyntick-idle mode while they
		have RCU callbacks pending, but CPUs \emph{are} allowed to go
		offline with callbacks pending.
		This means that CPUs going offline need to have their callbacks
		migrated to some other CPU, thus, we cannot allow CPUs to
		simply go quietly offline.
	\item	Present-day Linux systems run with \co{NR_CPUS}
		much larger than the actual number of CPUs.
		A unified approach could thus end up uselessly waiting on
		CPUs that are not just offline, but which never existed in
		the first place.
	\item	RCU is already operational when CPUs get onlined one
		at a time during boot, and therefore must handle the online
		process.
		This onlining must exclude grace-period initialization, so
		the \co{->onofflock} must still be used.
	\item	CPUs often switch into and out of dyntick-idle mode
		extremely frequently, so it is not reasonable to use the
		heavyweight online/offline code path for entering and exiting
		dyntick-idle mode.
	\end{enumerate}
} \QuickQuizEnd

\subsubsection{Detect a Too-Long Grace Period}
\label{app:rcuimpl:rcutree:Detect a Too-Long Grace Period}

When the \co{CONFIG_RCU_CPU_STALL_DETECTOR} kernel parameter
is specified, the \co{record_gp_stall_check_time()} function
records the time and also a timestamp set three seconds into the future.
If the current grace period still has not ended by that time, the
\co{check_cpu_stall()} function will check for the culprit,
invoking \co{print_cpu_stall()} if the current CPU is the
holdout, or \co{print_other_cpu_stall()} if it is some other CPU.
A two-jiffies offset helps ensure that CPUs report on themselves
when possible, taking advantage of the fact that a CPU can normally
do a better job of tracing its own stack than it can tracing some other
CPU's stack.

\subsection{Testing}
\label{app:rcuimpl:rcutree:Testing}

RCU is fundamental synchronization code, so any failure of RCU
results in random, difficult-to-debug memory corruption.
It is therefore extremely important that RCU be \emph{highly} reliable.
Some of this reliability stems from careful design, but at the
end of the day we must also rely on heavy stress testing, otherwise
known as torture.

Fortunately, although there has been some debate as to exactly
what populations are covered by the provisions of the Geneva Convention
% <a href="http://www.unhchr.ch/html/menu3/b/91.htm">Geneva Convention</a>,
it is still the case that it does not apply to software.
Therefore, it is still legal to torture your software.
In fact, it is strongly encouraged, because if you don't torture your
software, it will end up torturing \emph{you} by crashing at the most
inconvenient times imaginable.

Therefore, we torture RCU quite vigorously using the rcutorture module.

However, it is not sufficient to torture the common-case uses of RCU.
It is also necessary to torture it in unusual situations, for example,
when concurrently onlining and offlining CPUs and when CPUs are concurrently
entering and exiting dynticks idle mode.
I use a simple scripts to online and offline CPUs that runs concurently
with the rcutorture module.
This module is given the \co{test_no_idle_hz} module parameter in order
to stress-test dynticks idle mode.
Just to be fully paranoid, I sometimes run a kernbench workload in parallel
as well.
Ten hours of this sort of torture on a 128-way machine seems sufficient
to shake out most bugs.

Even this is not the complete story.
As Alexey Dobriyan and Nick Piggin demonstrated in early 2008, it is
also necessary to torture RCU with all relevant combinations of kernel
parameters.
The relevant kernel parameters may be identified using yet another
simple script, and are as follows:

\begin{enumerate}
\item    \co{CONFIG_CLASSIC_RCU}: Classic RCU.
\item    \co{CONFIG_PREEMPT_RCU}: Preemptible (real-time) RCU.
\item    \co{CONFIG_TREE_RCU}: Classic RCU for huge SMP systems.
\item    \co{CONFIG_RCU_FANOUT}: Number of children for each
		\co{rcu_node}.
\item    \co{CONFIG_RCU_FANOUT_EXACT}: Balance the
		\co{rcu_node} tree.
\item    \co{CONFIG_HOTPLUG_CPU}: Allow CPUs to be offlined
		and onlined.
\item    \co{CONFIG_NO_HZ}: Enable dyntick-idle mode.
\item    \co{CONFIG_SMP}: Enable multi-CPU operation.
\item    \co{CONFIG_RCU_CPU_STALL_DETECTOR}: Enable RCU to detect
		when CPUs go on extended quiescent-state vacations.
\item    \co{CONFIG_RCU_TRACE}: Generate RCU trace files in debugfs.
\end{enumerate}

We ignore the \co{CONFIG_DEBUG_LOCK_ALLOC} configuration
variable under the perhaps-naive assumption that hierarchical RCU
could not have broken lockdep.
There are still 10 configuration variables, which would result in
1,024 combinations if they were independent boolean variables.
Fortunately the first three are mutually exclusive, which reduces
the number of combinations down to 384, but \co{CONFIG_RCU_FANOUT}
can take on values from 2 to 64, increasing the number of combinations
to 12,096.
This is an infeasible number of combinations.

One key observation is that only \co{CONFIG_NO_HZ}
and \co{CONFIG_PREEMPT} can be expected to have changed behavior
if either \co{CONFIG_CLASSIC_RCU} or
\co{CONFIG_PREEMPT_RCU} are in effect, as only these portions
of the two pre-existing RCU implementations were changed during this effort.
This cuts out almost two thirds of the possible combinations.

Furthermore, not all of the possible values of
\co{CONFIG_RCU_FANOUT} produce significantly different results,
in fact only a few cases really need to be tested separately:

\begin{enumerate}
\item	Single-node ``tree''.
\item	Two-level balanced tree.
\item	Three-level balanced tree.
\item	Autobalanced tree, where \co{CONFIG_RCU_FANOUT}
	specifies an unbalanced tree, but such that it is auto-balanced
	in absence of \co{CONFIG_RCU_FANOUT_EXACT}.
\item	Unbalanced tree.
\end{enumerate}

Looking further, \co{CONFIG_HOTPLUG_CPU} makes sense only
given \co{CONFIG_SMP}, and \co{CONFIG_RCU_CPU_STALL_DETECTOR}
is independent, and really only needs to be tested once (though someone
even more paranoid than am I might decide to test it both with
and without \co{CONFIG_SMP}).
Similarly, \co{CONFIG_RCU_TRACE} need only be tested once,
but the truly paranoid (such as myself) will choose to run it both with
and without \co{CONFIG_NO_HZ}.

This allows us to obtain excellent coverage of RCU with only 15
test cases.
All test cases specify the following configuration parameters in order
to run rcutorture and so that \co{CONFIG_HOTPLUG_CPU=n} actually
takes effect:

\vspace{5pt}
\begin{minipage}[t]{\columnwidth}
\scriptsize
\begin{verbatim}
CONFIG_RCU_TORTURE_TEST=m
CONFIG_MODULE_UNLOAD=y
CONFIG_SUSPEND=n
CONFIG_HIBERNATION=n
\end{verbatim}
\end{minipage}
\vspace{5pt}

The 15 test cases are as follows:

\begin{enumerate}
\item	Force single-node ``tree'' for small systems:

\vspace{5pt}
\begin{minipage}[t]{\columnwidth}
\scriptsize
\begin{verbatim}
	CONFIG_NR_CPUS=8
	CONFIG_RCU_FANOUT=8
	CONFIG_RCU_FANOUT_EXACT=n
	CONFIG_RCU_TRACE=y
	CONFIG_PREEMPT_RCU=n
	CONFIG_CLASSIC_RCU=n
	CONFIG_TREE_RCU=y
\end{verbatim}
\end{minipage}
\vspace{5pt}

\item	Force two-level tree for large systems:

\vspace{5pt}
\begin{minipage}[t]{\columnwidth}
\scriptsize
\begin{verbatim}
	CONFIG_NR_CPUS=8
	CONFIG_RCU_FANOUT=4
	CONFIG_RCU_FANOUT_EXACT=n
	CONFIG_RCU_TRACE=n
	CONFIG_PREEMPT_RCU=n
	CONFIG_CLASSIC_RCU=n
	CONFIG_TREE_RCU=y
\end{verbatim}
\end{minipage}
\vspace{5pt}

\item	Force three-level tree for huge systems:

\vspace{5pt}
\begin{minipage}[t]{\columnwidth}
\scriptsize
\begin{verbatim}
	CONFIG_NR_CPUS=8
	CONFIG_RCU_FANOUT=2
	CONFIG_RCU_FANOUT_EXACT=n
	CONFIG_RCU_TRACE=y
	CONFIG_PREEMPT_RCU=n
	CONFIG_CLASSIC_RCU=n
	CONFIG_TREE_RCU=y
\end{verbatim}
\end{minipage}
\vspace{5pt}

\item	Test autobalancing to a balanced tree:

\vspace{5pt}
\begin{minipage}[t]{\columnwidth}
\scriptsize
\begin{verbatim}
	CONFIG_NR_CPUS=8
	CONFIG_RCU_FANOUT=6
	CONFIG_RCU_FANOUT_EXACT=n
	CONFIG_RCU_TRACE=y
	CONFIG_PREEMPT_RCU=n
	CONFIG_CLASSIC_RCU=n
	CONFIG_TREE_RCU=y
\end{verbatim}
\end{minipage}
\vspace{5pt}

\item	Test unbalanced tree:

\vspace{5pt}
\begin{minipage}[t]{\columnwidth}
\scriptsize
\begin{verbatim}
	CONFIG_NR_CPUS=8
	CONFIG_RCU_FANOUT=6
	CONFIG_RCU_FANOUT_EXACT=y
	CONFIG_RCU_CPU_STALL_DETECTOR=y
	CONFIG_RCU_TRACE=y
	CONFIG_PREEMPT_RCU=n
	CONFIG_CLASSIC_RCU=n
	CONFIG_TREE_RCU=y
\end{verbatim}
\end{minipage}
\vspace{5pt}

\item	Disable CPU-stall detection:

\vspace{5pt}
\begin{minipage}[t]{\columnwidth}
\scriptsize
\begin{verbatim}
	CONFIG_SMP=y
	CONFIG_NO_HZ=y
	CONFIG_RCU_CPU_STALL_DETECTOR=n
	CONFIG_HOTPLUG_CPU=y
	CONFIG_RCU_TRACE=y
	CONFIG_PREEMPT_RCU=n
	CONFIG_CLASSIC_RCU=n
	CONFIG_TREE_RCU=y
\end{verbatim}
\end{minipage}
\vspace{5pt}

\item	Disable CPU-stall detection and dyntick idle mode:

\vspace{5pt}
\begin{minipage}[t]{\columnwidth}
\scriptsize
\begin{verbatim}
	CONFIG_SMP=y
	CONFIG_NO_HZ=n
	CONFIG_RCU_CPU_STALL_DETECTOR=n
	CONFIG_HOTPLUG_CPU=y
	CONFIG_RCU_TRACE=y
	CONFIG_PREEMPT_RCU=n
	CONFIG_CLASSIC_RCU=n
	CONFIG_TREE_RCU=y
\end{verbatim}
\end{minipage}
\vspace{5pt}

\item	Disable CPU-stall detection and CPU hotplug:

\vspace{5pt}
\begin{minipage}[t]{\columnwidth}
\scriptsize
\begin{verbatim}
	CONFIG_SMP=y
	CONFIG_NO_HZ=y
	CONFIG_RCU_CPU_STALL_DETECTOR=n
	CONFIG_HOTPLUG_CPU=n
	CONFIG_RCU_TRACE=y
	CONFIG_PREEMPT_RCU=n
	CONFIG_CLASSIC_RCU=n
	CONFIG_TREE_RCU=y
\end{verbatim}
\end{minipage}
\vspace{5pt}

\item	Disable CPU-stall detection, dyntick idle mode, and CPU hotplug:

\vspace{5pt}
\begin{minipage}[t]{\columnwidth}
\scriptsize
\begin{verbatim}
	CONFIG_SMP=y
	CONFIG_NO_HZ=n
	CONFIG_RCU_CPU_STALL_DETECTOR=n
	CONFIG_HOTPLUG_CPU=n
	CONFIG_RCU_TRACE=y
	CONFIG_PREEMPT_RCU=n
	CONFIG_CLASSIC_RCU=n
	CONFIG_TREE_RCU=y
\end{verbatim}
\end{minipage}
\vspace{5pt}

\item	Disable SMP, CPU-stall detection, dyntick idle mode, and CPU hotplug:

\vspace{5pt}
\begin{minipage}[t]{\columnwidth}
\scriptsize
\begin{verbatim}
	CONFIG_SMP=n
	CONFIG_NO_HZ=n
	CONFIG_RCU_CPU_STALL_DETECTOR=n
	CONFIG_HOTPLUG_CPU=n
	CONFIG_RCU_TRACE=y
	CONFIG_PREEMPT_RCU=n
	CONFIG_CLASSIC_RCU=n
	CONFIG_TREE_RCU=y
\end{verbatim}
\end{minipage}
\vspace{5pt}

	This combination located a number of compiler warnings.

\item	Disable SMP and CPU hotplug:

\vspace{5pt}
\begin{minipage}[t]{\columnwidth}
\scriptsize
\begin{verbatim}
	CONFIG_SMP=n
	CONFIG_NO_HZ=y
	CONFIG_RCU_CPU_STALL_DETECTOR=y
	CONFIG_HOTPLUG_CPU=n
	CONFIG_RCU_TRACE=y
	CONFIG_PREEMPT_RCU=n
	CONFIG_CLASSIC_RCU=n
	CONFIG_TREE_RCU=y
\end{verbatim}
\end{minipage}
\vspace{5pt}

\item	Test Classic RCU with dynticks idle but without preemption:

\vspace{5pt}
\begin{minipage}[t]{\columnwidth}
\scriptsize
\begin{verbatim}
	CONFIG_NO_HZ=y
	CONFIG_PREEMPT=n
	CONFIG_RCU_TRACE=y
	CONFIG_PREEMPT_RCU=n
	CONFIG_CLASSIC_RCU=y
	CONFIG_TREE_RCU=n
\end{verbatim}
\end{minipage}
\vspace{5pt}

\item	Test Classic RCU with preemption but without dynticks idle:

\vspace{5pt}
\begin{minipage}[t]{\columnwidth}
\scriptsize
\begin{verbatim}
	CONFIG_NO_HZ=n
	CONFIG_PREEMPT=y
	CONFIG_RCU_TRACE=y
	CONFIG_PREEMPT_RCU=n
	CONFIG_CLASSIC_RCU=y
	CONFIG_TREE_RCU=n
\end{verbatim}
\end{minipage}
\vspace{5pt}

\item	Test Preemptible RCU with dynticks idle:

\vspace{5pt}
\begin{minipage}[t]{\columnwidth}
\scriptsize
\begin{verbatim}
	CONFIG_NO_HZ=y
	CONFIG_PREEMPT=y
	CONFIG_RCU_TRACE=y
	CONFIG_PREEMPT_RCU=y
	CONFIG_CLASSIC_RCU=n
	CONFIG_TREE_RCU=n
\end{verbatim}
\end{minipage}
\vspace{5pt}
\item	Test Preemptible RCU without dynticks idle:

\vspace{5pt}

\begin{minipage}[t]{\columnwidth}
\scriptsize
\begin{verbatim}
	CONFIG_NO_HZ=n
	CONFIG_PREEMPT=y
	CONFIG_RCU_TRACE=y
	CONFIG_PREEMPT_RCU=y
	CONFIG_CLASSIC_RCU=n
	CONFIG_TREE_RCU=n
\end{verbatim}
\end{minipage}
\vspace{5pt}
\end{enumerate}

For a large change that affects RCU core code, one should run
rcutorture for each of the above combinations, and concurrently
with CPU offlining and onlining for cases with
\co{CONFIG_HOTPLUG_CPU}.
For small changes, it may suffice to run kernbench in each case.
Of course, if the change is confined to a particular subset of
the configuration parameters, it may be possible to reduce the
number of test cases.

Torturing software: the Geneva Convention does not (yet) prohibit
it, and I strongly recommend it!

\subsection{Conclusion}
\label{app:rcuimpl:rcutree:Conclusion}

This hierarchical implementation of RCU reduces lock contention,
avoids unnecessarily awakening dyntick-idle sleeping CPUs, while
helping to debug Linux's hotplug-CPU code paths.
This implementation is designed to handle single systems with
thousands of CPUs, and on 64-bit systems has an architectural
limitation of a quarter million CPUs, a limit I expect to be
sufficient for at least the next few years.

This RCU implementation of course has some limitations:

\begin{enumerate}
\item	The \co{force_quiescent_state()} can scan the full
	set of CPUs with irqs disabled.
	This would be fatal in a real-time implementation of RCU,
	so if hierarchy ever needs to be introduced to preemptible
	RCU, some other approach will be required.
	It is possible that it will be problematic on 4,096-CPU
	systems, but actual testing on such systems is required
	to prove this one way or the other.
	
	On busy systems, the \co{force_quiescent_state()} scan
	would not be expected to happen,
	as CPUs should pass through quiescent states within three
	jiffies of the start of a quiescent state.  On semi-busy
	systems, only the CPUs in dynticks-idle mode throughout would
	need to be scanned.
	In some cases, for example when a dynticks-idle CPU is handling
	an interrupt during a scan, subsequent scans are required.
	However, each such scan is performed separately, so scheduling
	latency is degraded by the overhead of only one such scan.
	
	If this scan proves problematic, one straightforward solution
	would be to do the scan incrementally.
	This would increase code complexity slightly and would also
	increase the time required to end a grace period, but would
	nonetheless be a likely solution.
	
\item	The \co{rcu_node} hierarchy is created at compile
	time, and is therefore sized for the worst-case \co{NR_CPUS}
	number of CPUs.
	However, even for 4,096 CPUs, the \co{rcu_node}
	hierarchy consumes only 65 cache lines on a 64-bit machine
	(and just you try accommodating 4,096 CPUs on a 32-bit machine!).
	Of course, a kernel built with \co{NR_CPUS=4096}
	running on a 16-CPU machine would use a two-level tree when
	a single-node tree would work just fine.
	Although this configuration would incur added locking overhead,
	this does not affect hot-path read-side code, so should not be a
	problem in practice.
	
\item	This patch does increase kernel text and data somewhat:
	the old Classic RCU implementation consumes 1,757 bytes of
	kernel text and 456 bytes of kernel data for a total of 2,213 bytes,
	while the new hierarchical RCU implementation consumes 4,006
	bytes of kernel text and 624 bytes of kernel data for a total
	of 4,630 bytes on a \co{NR_CPUS=4} system.
	This is a non-problem even for most embedded systems, which
	often come with hundreds of megabytes of main memory.
	However, if this is a problem for tiny embedded systems, it may
	be necessary to provide both ``scale up'' and
	``scale down'' implementations of RCU.
\end{enumerate}

This hierarchical RCU implementation should nevertheless be a vast
improvement over Classic RCU for machines with hundreds of CPUs.
After all, Classic RCU was designed for systems with only 16-32 CPUs.

At some point, it may be necessary to also apply hierarchy to the
preemptible RCU implementation.
This will be challenging due to the modular arithmetic used on the
per-CPU counter pairs, but should be doable.
