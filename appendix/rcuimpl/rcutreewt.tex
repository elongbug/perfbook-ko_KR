% appendix/rcuimpl/rcutreewt.tex

\section{Hierarchical RCU Code Walkthrough}
\label{app:rcuimpl:rcutreewt:Hierarchical RCU Code Walkthrough}

이 섹션은 리눅스 커널 hierarchical RCU 코드의 일부 부분들을 분석합니다.
이 섹션은 hierarchical RCU 를 매우 낮은 단계에서 이해하기를 바라는 hard-core
해커들을 위해 지어졌으며, 그런 해커들이라면 먼저
Section~\ref{app:rcuimpl:rcutree:Hierarchical RCU Overview} 를 읽어야 합니다.
하드코어 매저키스트들 역시 이 섹션을 읽어볼 흥미가 생길겁니다.
물론 \emph{진짜} 하드코어 매저키스트들은
Section~\ref{app:rcuimpl:rcutree:Hierarchical RCU Overview} 를 읽기 전에 이
섹션을 읽을 겁니다.
\iffalse

This section walks through selected sections of the Linux-kernel
hierarchical RCU code.
As such, this section is intended for hard-core hackers who wish
to understand hierarchical RCU at a very low level, and such hackers
should first read
Section~\ref{app:rcuimpl:rcutree:Hierarchical RCU Overview}.
Hard-core masochists might also be interested in reading this section.
Of course \emph{really} hard-core masochists will read this section
before reading
Section~\ref{app:rcuimpl:rcutree:Hierarchical RCU Overview}.
\fi

Section~\ref{app:rcuimpl:rcutreewt:Data Structures and Kernel Parameters}
은 데이터 구조와 커널 패러미터들을 설명하고,
Section~\ref{app:rcuimpl:rcutreewt:External Interfaces}
은 외부로의 함수 인터페이스를 다루며,
Section~\ref{app:rcuimpl:rcutreewt:Initialization}
는 초기화 프로세스를 서술하고,
Section~\ref{app:rcuimpl:rcutreewt:CPU Hotplug}
는 CPU-hotplug 인터페이스를 설명하고,
Section~\ref{app:rcuimpl:rcutreewt:Miscellaneous Functions}
는 그외의 유틸리티 함수들을 다루고,
Section~\ref{app:rcuimpl:rcutreewt:Grace-Period-Detection Functions}
는 grace-period 탐지 메커니즘을 설명하고,
Section~\ref{app:rcuimpl:rcutreewt:Dyntick-Idle Functions}
는 dynticks-idle 인터페이스를 보고,
Section~\ref{app:rcuimpl:rcutreewt:Forcing Quiescent States}
는 (offline 과 dynticks-idle CPU 들을 포함해) holdout CPU 들을 다루는 함수들을
알아보고,
Section~\ref{app:rcuimpl:rcutreewt:CPU-Stall Detection}
는 stall 된 CPU, 즉 커널 모드에서 수초 동안을 spinning 하는 CPU 들을 알리는
함수를 보입니다.
마지막으로,
Section~\ref{app:rcuimpl:rcutreewt:Possible Flaws and Changes}
는 있을법한 설계상의 흠결과 수정사항들을 이야기합니다.
\iffalse

Section~\ref{app:rcuimpl:rcutreewt:Data Structures and Kernel Parameters}
describes data structures and kernel parameters,
Section~\ref{app:rcuimpl:rcutreewt:External Interfaces}
covers external function interfaces,
Section~\ref{app:rcuimpl:rcutreewt:Initialization}
presents the initialization process,
Section~\ref{app:rcuimpl:rcutreewt:CPU Hotplug}
explains the CPU-hotplug interface,
Section~\ref{app:rcuimpl:rcutreewt:Miscellaneous Functions}
covers miscellaneous utility functions,
Section~\ref{app:rcuimpl:rcutreewt:Grace-Period-Detection Functions}
describes the mechanics of grace-period detection,
Section~\ref{app:rcuimpl:rcutreewt:Dyntick-Idle Functions}
presents the dynticks-idle interface,
Section~\ref{app:rcuimpl:rcutreewt:Forcing Quiescent States}
covers the functions that handle holdout CPUs (including offline and
dynticks-idle CPUs), and
Section~\ref{app:rcuimpl:rcutreewt:CPU-Stall Detection}
presents functions that report on stalled CPUs, namely those spinning
in kernel mode for many seconds.
Finally,
Section~\ref{app:rcuimpl:rcutreewt:Possible Flaws and Changes}
reports on possible design flaws and fixes.
\fi

\subsection{Data Structures and Kernel Parameters}
\label{app:rcuimpl:rcutreewt:Data Structures and Kernel Parameters}

Hierarchical RCU 데이터 구조의 완전한 이해는 이 알고리즘의 이해에 매우
중요합니다.
따라서,
Section~\ref{app:rcuimpl:rcutreewt:Tracking Dyntick State}
은 각 CPU 의 dyntick-idle state 를 추적하는데 사용되는 데이터 구조들을
설명하고,
Section~\ref{app:rcuimpl:rcutreewt:Nodes in the Hierarchy}
는 \co{rcu_node} 계층을 만드는데 사용되는 per-node 데이터 구조의 필드들을
설명하며,
Section~\ref{app:rcuimpl:rcutreewt:Per-CPU Data}
은 per-CPU \co{rcu_data} 구조체를 설명하고,
Section~\ref{app:rcuimpl:rcutreewt:RCU Global State}
에서는 global \co{rcu_state} 구조체의 필드를 설명하며
Section~\ref{app:rcuimpl:rcutreewt:Kernel Parameters}
에서는 Hierarchical RCU 의 동작을 조절하는 커널 패러미터들을 설명합니다.
\iffalse

A full understanding of the Hierarchical RCU data structures is
critically important to understanding the algorithms.
To this end,
Section~\ref{app:rcuimpl:rcutreewt:Tracking Dyntick State}
describes the data structures used to track each CPU's dyntick-idle state,
Section~\ref{app:rcuimpl:rcutreewt:Nodes in the Hierarchy}
describes the fields in the per-node data structure making up the
\co{rcu_node} hierarchy,
Section~\ref{app:rcuimpl:rcutreewt:Per-CPU Data}
describes per-CPU \co{rcu_data} structure,
Section~\ref{app:rcuimpl:rcutreewt:RCU Global State}
describes the field in the global \co{rcu_state} structure,
and
Section~\ref{app:rcuimpl:rcutreewt:Kernel Parameters}
describes the kernel parameters that control Hierarchical RCU's
operation.
\fi

Page~\pageref{fig:app:rcuimpl:rcutree:Hierarchical RCU State With Dynticks}
의
Figure~\ref{fig:app:rcuimpl:rcutree:Hierarchical RCU State With Dynticks}
와
Page~\pageref{fig:app:rcuimpl:rcutree:Initialized RCU Data Layout}
의
Figure~\ref{fig:app:rcuimpl:rcutree:Initialized RCU Data Layout}
가 다음의 데이터 구조 설명들을 따라가는데에 많은 도움이 될 수 있습니다.
\iffalse

Figure~\ref{fig:app:rcuimpl:rcutree:Hierarchical RCU State With Dynticks}
on
Page~\pageref{fig:app:rcuimpl:rcutree:Hierarchical RCU State With Dynticks}
and
Figure~\ref{fig:app:rcuimpl:rcutree:Initialized RCU Data Layout}
on
Page~\pageref{fig:app:rcuimpl:rcutree:Initialized RCU Data Layout}
can be very helpful in keeping one's place through the following detailed
data-structure descriptions.
\fi

\subsubsection{Tracking Dyntick State}
\label{app:rcuimpl:rcutreewt:Tracking Dyntick State}

Per-CPU \co{rcu_dynticks} 구조체는 다음 필드들을 이용해 dynticks state 를
추적합니다:
\iffalse

The per-CPU \co{rcu_dynticks} structure tracks dynticks state using the
following fields:
\fi

\begin{itemize}
\item	\co{dynticks_nesting}:
	이 \co{int} 는 연관된 CPU 가 RCU read-side 크리티컬 섹션들을 모니터
	되어야 하는 이유의 수를 셉니다.
	만약 이 CPU 가 dynticks-idle 모드에 있다면, 이 필드는 irq 가 감싸진
	횟수를 세고, 그렇지 않다면 irq 감싸진 횟수보다 하나 큰 값을 갖습니다.
\item	\co{dynticks}:
	이 \co{int} 카운터의 값은 연관된 CPU 가 dynticks-idle 모드에 있고 현재
	해당 CPU 에서 수행 중인 irq 핸들러가 없다면 짝수의 값을 갖고, 그렇지
	않다면 홀수의 값을 갖습니다.
	달리 말하자면, 이 카운터의 값이 홀수라면, 연관된 CPU 는 RCU read-side
	크리티컬 섹션에 있을 수 있습니다.
\item	\co{dynticks_nmi}:
	이 \co{int} 카운터의 값은 연관된 CPU 가 NMI 핸들러 수행중이라면
	홀수인데, 이 CPU 가 irq 핸들러 수행중이지 않은 채로 dyntick-idle 모드에
	있는 동안 NMI 를 받았을 때에만 입니다.
	그렇지 않다면, 이 카운터의 값은 짝수입니다.
\iffalse

\item	\co{dynticks_nesting}:
	This \co{int} counts the number of reasons that the corresponding
	CPU should be monitored for RCU read-side critical sections.
	If the CPU is in dynticks-idle mode, then this counts the
	irq nesting level, otherwise it is one greater than the
	irq nesting level.
\item	\co{dynticks}:
	This \co{int} counter's value is even if the corresponding CPU is
	in dynticks-idle mode and there are no irq handlers currently
	running on that CPU, otherwise the counter's value is odd.
	In other words, if this counter's value is odd, then the
	corresponding CPU might be in an RCU read-side critical section.
\item	\co{dynticks_nmi}:
	This \co{int} counter's value is odd if the corresponding CPU is
	in an NMI handler, but only if the NMI arrived while this
	CPU was in dyntick-idle mode with no irq handlers running.
	Otherwise, the counter's value will be even.
\fi
\end{itemize}

이 상태는 rcu 와 rcu\_bh 구현 사이에서 공유됩니다.
\iffalse

This state is shared between the rcu and rcu\_bh implementations.
\fi

\subsubsection{Nodes in the Hierarchy}
\label{app:rcuimpl:rcutreewt:Nodes in the Hierarchy}

앞서 언급했듯이, \co{rcu_node} 계층은
page~\pageref{fig:app:rcuimpl:rcutree:Mapping rcu-node Hierarchy Into Array}
의
Figure~\ref{fig:app:rcuimpl:rcutree:Mapping rcu-node Hierarchy Into Array}
에 보인 것처럼 \co{rcu_state} 구조체 안에 평면화 되어 존재합니다.
이 계층 안의 \co{rcu_node} 각각은 다음과 같은 필드들을 갖습니다:
\iffalse

As noted earlier, the \co{rcu_node} hierarchy is flattened into
the \co{rcu_state} structure as shown in
Figure~\ref{fig:app:rcuimpl:rcutree:Mapping rcu-node Hierarchy Into Array}
on
page~\pageref{fig:app:rcuimpl:rcutree:Mapping rcu-node Hierarchy Into Array}.
Each \co{rcu_node} in this hierarchy has fields as follows:
\fi

\begin{itemize}
\item	\co{lock}:
	이 스핀락은 이 구조체의 상수가 아닌 필드들을 보호합니다.
	이 락은 softirq 컨텍스트에서 얻어지고, 따라서 irq 들을 비활성화 시켜야
	합니다.
\iffalse

\item	\co{lock}:
	This spinlock guards the non-constant fields in this structure.
	This lock is acquired from softirq context, so must disable
	irqs.
\fi

\QuickQuiz{}
	\co{rcu_data} 구조체의 \co{lock} 을 획득할 때 그냥 bottom halves
	(softirq) 를 비활성화시키는게 어떤가요?
	그게 더 빠르지 않겠어요?
	\iffalse

	Why not simply disable bottom halves (softirq) when acquiring
	the \co{rcu_data} structure's \co{lock}?
	Wouldn't this be faster?
	\fi
\QuickQuizAnswer{
	이 락은 irq 핸들러들에서 수행될 수 있는 \co{call_rcu()} 에서 호출된
	함수에서 획득될 수 있기 때문입니다.
	따라서, 이 락을 획득할 때 irq 들은 \emph{반드시} 비활성화 되어야만
	합니다.
	\iffalse

	Because this lock can be acquired from functions
	called by \co{call_rcu()}, which in turn can be
	invoked from irq handlers.
	Therefore, irqs \emph{must} be disabled when
	holding this lock.
	\fi
} \QuickQuizEnd

	루트 \co{rcu_node} 의 \co{lock} 필드는 추가적인 의무를 갖습니다:
	\begin{enumerate}
	\item	CPU-stall 검사를 직렬화 시켜서, 하나의 stall 이 하나의 CPU 에
		의해서만 보고될 수 있도록 합니다.
		이는 수천개의 CPU 들이 존재하는 시스템에서는 중요합니다!
	\item	새로운 grace period 를 시작하는 것을 직렬화 시켜서, 여러 CPU
		들이 grace period 들을 동시에 시작하지 않도록 합니다.
	\item	새로운 grace period 들이 하나의 grace period 에 국한되어서
		시작되어야 할 필요를 없앱니다.
	\item	Reschedule IPI 들의 갯수를 너무 지나치지 않도록 유지하기 위해
		(\co{force_quiescent_state()} 의) state machine forcing
		quiescent state 들을 직렬화 시킵니다.
	\iffalse

	The \co{lock} field of the root \co{rcu_node} has additional
	responsibilities:
	\begin{enumerate}
	\item	Serializes CPU-stall checking, so that a given stall
		is reported by only one CPU.
		This can be important on systems with thousands of
		CPUs!
	\item	Serializes starting a new grace period, so that
		multiple CPUs don't start conflicting grace periods
		concurrently.
	\item	Prevents new grace periods from starting in code that
		needs to run within the confines of a single grace period.
	\item	Serializes the state machine forcing quiescent states
		(in \co{force_quiescent_state()}) in order to
		keep the number of reschedule IPIs down to a dull
		roar.
	\fi
	\end{enumerate}
\item	\co{qsmask}:
	이 bitmask 는 현재의 grace period 가 끝나기 위해 지나가야 하는
	quiescent state 를 여전히 통과해야 하는 CPU 들 (leaf \co{rcu_node}
	구조체들을 위해) 이나 CPU 들의 그룹들 (leaf 가 아닌 \co{rcu_node}
	구조체들을 위해) 을 추적합니다.
\item	\co{qsmaskinit}:
	이 bitmask 는 어떤 CPU 들이나 CPU 들의 그룹들이 다음 grace period 들이
	끝나기 위해 quiescent state 를 지나가야 하는지를 추적합니다.
	Online/offline 코드는 \co{qsmaskinit} 필드들을 조정하는데, 각 grace
	period 의 시작 때 \co{qsmask} 필드들이 복사되어집니다.
	이 복사 작업은 왜 grace period 초기화가 online/offline 오퍼레이션들과
	배타적으로 수행되어야 하는지에 대한 이유 중 하나입니다.
\item	\co{grpmask}:
	이 bitmask 는 하나의 bit set 을 가지며, 이는 부모 \co{rcu_node}
	구조체의 \co{qsmask} 와 \co{qsmaskinit} 필드들에서의 이 \co{rcu_node}
	구조체의 위치에 연관되어 있습니다.
	이 필드의 사용은 Manfred Spraul 이 제안한대로 quiescent-state 처리를
	단순화 시킵니다.
\iffalse

\item	\co{qsmask}:
	This bitmask tracks which CPUs (for leaf \co{rcu_node} structures)
	or groups of CPUs (for non-leaf \co{rcu_node} structures)
	still need to pass through a quiescent state in order for the
	current grace period to end.
\item	\co{qsmaskinit}:
	This bitmask tracks which CPUs or groups of CPUs will need to
	pass through a quiescent state for subsequent grace periods
	to end.
	The online/offline code manipulates the \co{qsmaskinit} fields,
	which are copied to the corresponding \co{qsmask} fields at
	the beginning of each grace period.
	This copy operation is one reason why grace period initialization
	must exclude online/offline operations.
\item	\co{grpmask}:
	This bitmask has a single bit set, and that is the bit corresponding
	to the this \co{rcu_node} structure's position in the parent
	\co{rcu_node} structure's \co{qsmask} and \co{qsmaskinit}
	fields.
	Use of this field simplifies quiescent-state processing,
	as suggested by Manfred Spraul.
\fi

\QuickQuiz{}
	Leaf \co{rcu_node} 구조체의 \co{qsmask} 와 \co{qsmaskinit} 필드들은
	어떤가요?
	이 필드들의 어떤 bit 들이 이 \co{rcu_node} 구조체에 의해 처리되는 각
	CPU 들에 연관되는지를 알기 위한 어떤 방법이 필요하지 않나요?
	\iffalse

	How about the \co{qsmask} and \co{qsmaskinit}
	fields for the leaf \co{rcu_node} structures?
	Doesn't there have to be some way to work out
	which of the bits in these fields corresponds
	to each CPU covered by the \co{rcu_node} structure
	in question?
	\fi
\QuickQuizAnswer{
	실제로 그렇습니다!
	각 CPU 의 \co{rcu_data} 구조체의 \co{grpmask} 필드가 이 일을 합니다.
	\iffalse

	Indeed there does!
	The \co{grpmask} field in each CPU's \co{rcu_data}
	structure does this job.
	\fi
} \QuickQuizEnd

\item	\co{grplo}:
	이 필드는 이 \co{rcu_node} 구조체에 의해 처리되는 CPU 들 중 가장 낮은
	숫자의 CPU 의 숫자를 담고 있습니다.
\item	\co{grphi}:
	이 필드는 이 \co{rcu_node} 구조체에 의해 처리되는 CPU 들 중 가장 높은
	숫자의 CPU 의 숫자를 담고 있습니다.
\item	\co{grpnum}:
	이 필드는 부모 \co{rcu_node} 구조체의, 이 \co{rcu_node} 구조체가
	연관되어 있는 \co{qsmask} 와 \co{qsmaskinit} 필드들의 bit 수를 담고
	있습니다.
	달리 말하자면, 특정 \co{rcu_node} 구조체로의 포인터 \co{rnp} 가 있다면,
	\co{1UL << rnp->grpnum == rnp->grpmask} 가 될 겁니다.
	이 \co{grpnum} 필드는 tracing 출력을 위해서만 사용됩니다.
\item	\co{level}:
	이 필드는 루트 \co{rcu_node} 구조체에서는 0이 되고, 루트의 자식인
	\co{rcu_node} 구조체에서는 1의 값을, 그리고 계층의 아래로 가면서
	이런식으로 값을 갖게 됩니다.
\item	\co{parent}:
	이 필드는 부모 \co{rcu_node} 구조체로의 포인터이며, 루트 \co{rcu_node}
	구조체에 있어서는 NULL 입니다.
\iffalse

\item	\co{grplo}:
	This field contains the number of the lowest-numbered CPU covered
	by this \co{rcu_node} structure.
\item	\co{grphi}:
	This field contains the number of the highest-numbered CPU covered
	by this \co{rcu_node} structure.
\item	\co{grpnum}:
	This field contains the bit number in the parent \co{rcu_node}
	structure's \co{qsmask} and \co{qsmaskinit} fields that this
	\co{rcu_node} structure corresponds to.
	In other words, given a pointer \co{rnp} to a given
	\co{rcu_node} structure, it will always be the case that
	\co{1UL << rnp->grpnum == rnp->grpmask}.
	The \co{grpnum} field is used only for tracing output.
\item	\co{level}:
	This field contains zero for the root \co{rcu_node} structure,
	one for the \co{rcu_node} structures that are children of
	the root, and so on down the hierarchy.
\item	\co{parent}:
	This field is a pointer to the parent \co{rcu_node} structure,
	or NULL for the root \co{rcu_node} structure.
\fi
\end{itemize}

\subsubsection{Per-CPU Data}
\label{app:rcuimpl:rcutreewt:Per-CPU Data}

\co{rcu_data} 구조체는 RCU 의 per-CPU 상태를 담고 있습니다.
이 상태는 grace period 와 quiescent state 들을 관리하는 조정을 위한 변수들을
담고 있습니다 ( \co{completed}, \co{gpnum}, \co{passed_quiesc_completed},
\co{passed_quiesc}, \co{qs_pending}, \co{beenonline}, \co{mynode}, 그리고
\co{grpmask}).
\co{rcu_data} 구조체는 또한 RCU 콜백들로 관련되는 조정을 위한 변수들 또한 담고
있습니다 (\co{nxtlist}, \co{nxttail}, \co{qlen}, 그리고 \co{blimit}).
Dynticks 가 활성화된 커널들은 관련된 조정을 위한 변수들을 \co{rcu_data} 구조체
안에 가지고 있습니다
(\co{dynticks}, \co{dynticks_snap}, 그리고 \co{dynticks_nmi_snap}).
\co{rcu_data} 구조체는 tracing 에서 사용되는 이벤트 카운터들을 담고 있습니다
(\co{dynticks_fqs} given dynticks, \co{offline_fqs}, 그리고 \co{resched_ipi}).
마지막으로, 한 쌍의 필드들은 언제 quiescent state 를 강제해야 하는지 판단하기
위해 \co{rcu_pending()} 호출 횟수를 세고 있고 (\co{n_rcu_pending} 그리고
\co{n_rcu_pending_force_qs}), \co{cpu} 필드는 해당 \co{rcu_data} 구조체가 어떤
CPU 에 연관되어 있는지를 알립니다.

이 필드들 각각에 대해 아래에서 설명합니다.
\iffalse

The \co{rcu_data} structure contains RCU's per-CPU state.
It contains control variables governing grace periods and
quiescent states (\co{completed}, \co{gpnum}, \co{passed_quiesc_completed},
\co{passed_quiesc}, \co{qs_pending}, \co{beenonline}, \co{mynode},
and \co{grpmask}).
The \co{rcu_data} structure also contains control variables pertaining
to RCU callbacks
(\co{nxtlist}, \co{nxttail}, \co{qlen}, and \co{blimit}).
Kernels with dynticks enabled will have relevant control variables in
the \co{rcu_data} structure
(\co{dynticks}, \co{dynticks_snap}, and \co{dynticks_nmi_snap}).
The \co{rcu_data} structure contains event counters used by tracing
(\co{dynticks_fqs} given dynticks, \co{offline_fqs}, and \co{resched_ipi}).
Finally, a pair of fields count calls to \co{rcu_pending()} in order
to determine when to force quiescent states (\co{n_rcu_pending} and
\co{n_rcu_pending_force_qs}), and a \co{cpu} field indicates which
CPU to which a given \co{rcu_data} structure corresponds.

Each of these fields is described below.
\fi

\begin{itemize}
\item	\co{completed}:
	이 필드는 이 CPU 가 끝나길 기다리고 있는 최근의 grace period 들의 수를
	담습니다.
\item	\co{gpnum}:
	이 필드는 이 CPU 가 시작되길 신경쓰고 있는 최근의 grace period 들의
	수를 담습니다.
\item	\co{passed_quiesc_completed}:
	이 필드는 이 CPU 가 마지막으로 quiescent state 를 지났을 때 가장 최근에
	완료된 grace period 의 수를 담습니다.
	``가장 최근에 완료된'' 이란 말은 quiescent state 를 지나는 CPU 의
	관점에서입니다: 만약 이 CPU 가 아직 (말하자면) 42 번 grace period 가
	완료되길 신경쓰고 있지 않다면, 여전히 값 41 을 기록하고 있게 될겁니다.
	이 CPU 가 이미 quiescent state 를 지나갔는가 여부가 grace period 가
	완료될 수 있는 유일한 방법이기 때문에 이건 괜찮습니다.
	이 필드는 부팅할 때와 CPU 를 online 화 시킬 때의 race condition 을
	없애기 위해 (미신적일 수 있는) 과거의 grace period 수로 초기화 됩니다.
\iffalse

\item	\co{completed}:
	This field contains the number of the most recent grace period
	that this CPU is aware of having completed.
\item	\co{gpnum}:
	This field contains the number of the most recent grace period
	that this CPU is aware of having started.
\item	\co{passed_quiesc_completed}:
	This field contains the number of the grace period that had most
	recently completed when this
	CPU last passed through a quiescent state.
	The ``most recently completed'' will be from the viewpoint of
	the CPU passing through the quiescent state: if the CPU is
	not yet aware that grace period (say) 42 has completed, it
	will still record the old value of 41.
	This is OK, because the only way that the grace period can
	complete is if this CPU has already passed through a
	quiescent state.
	This field is initialized to a (possibly mythical) past
	grace period number to avoid race conditions when booting
	and when onlining a CPU.
\fi
\item	\co{passed_quiesc}:
	이 필드는 이 CPU 가 \co{passed_quiesc_completed} 에 저장된 수의 grace
	period 가 완료된 이래로 quiescent state 를 지난적이 있는지를
	가리킵니다.
	이 필드는 연관된 CPU 가 새로운 grace period 의 시작을 신경쓰게 될때마다
	지워집니다.
\item	\co{qs_pending}:
	이 필드는 이 CPU 가 코어 RCU 메커니즘이 자신이 quiescent state 를
	지나길 기다리고 있음을 알고 있음을 알립니다.
	이 필드는 이 CPU 가 새로운 grace period 를 파악하거나 한 CPU 가 online
	이 될 때 값이 채워집니다.
\iffalse

\item	\co{passed_quiesc}:
	This field indicates whether this CPU has passed
	through a quiescent state since the grace period number
	stored in \co{passed_quiesc_completed} completed.
	This field is cleared each time the corresponding CPU
	becomes aware of the start of a new grace period.
\item	\co{qs_pending}:
	This field indicates that this CPU is aware that the core
	RCU mechanism is waiting for it to pass through a quiescent state.
	This field is set to one when the CPU detects a new grace
	period or when a CPU is coming online.
\fi

\QuickQuiz{}
	Offline 이 되는 건 모든 진행중인 grace period 를 처리해야할 extended
	quiescent state 인데도 왜 CPU 가 online 이 될 때 \co{qs_pending} 의
	값을 1 로 설정하는 거죠?
	\iffalse

	But why bother setting \co{qs_pending} to one when a CPU
	is coming online, given that being offline is an extended
	quiescent state that should cover any ongoing grace period?
	\fi
\QuickQuizAnswer{
	이게 CPU 가 online 이 되는 것과 새로운 grace period 가 시작하는 사이의
	race 를 해결하는 걸 돕기 때문입니다.
	\iffalse

	Because this helps to resolve a race between a CPU coming online
	just as a new grace period is starting.
	\fi
} \QuickQuizEnd

\QuickQuiz{}
	왜 \co{passed_quiesc_completed} 에 마지막으로 완료된 grace period 의
	수를 기록하는거죠?
	그건 어떤 grace period 도 진행중이지 않을 때에 보여진 quiescent state
	들이 올바르지 않게 다음에 시작되는 grace period 에 적용되는 것에 이 RCU
	구현이 취약해 지게 하지 않나요?
	\iffalse

	Why record the last completed grace period number in
	\co{passed_quiesc_completed}?
	Doesn't that cause this RCU implementation to be vulnerable
	to quiescent states seen while no grace period was in progress
	being incorrectly applied to the next grace period that starts?
	\fi
\QuickQuizAnswer{
	한 grace period 의 종료 근처에서 알려진 quiescent state 가 올바르지
	않게 다음 grace period 에 적용되는 race 를 막기 위해, 특히 dyntick 과
	CPU-offline grace period 들을 위해 우린 마지막으로 완료된 grace period
	수를 기록합니다.
	따라서, \co{force_quiescent_state()} 와 그 부류들은 모두 그런 race 들을
	막기 위해 마지막으로 완료된 grace period 수를 체크합니다.

	이제 이 dyntick 과 CPU-offline grace period 들은 어떤 grace period 가
	실제로 살아있을 때에만 체크됩니다.
	어떤 grace period 도 진행중이지 않을 때에 기록될 수 있는 유일한
	quiescent state 들은 스스로 파악되는 quiescent state 들로,
	\co{passed_quiesc_completed}, \co{passed_quiesc}, 그리고
	\co{qs_pending} 에 기록되는 것들입니다.
	이 변수들은 연관된 CPU 가 새로운 grace period 가 시작했음을 알게 됐을
	때마다 초기화 되어서, 시대에 뒤져버린 quiescent state 가 새로운 grace
	period 에 적용되는 것을 막습니다.

	그렇지만, grace-period 응답시간을 최적화 하는 건 \co{completed} 에
	더해서 \co{gpnum} 이 추적될 것을 필요로 할수 있습니다.
	\iffalse

	We record the last completed grace period number in order
	to avoid races where a quiescent state noted near the end of
	one grace period is incorrectly applied to the next grace
	period, especially for dyntick and CPU-offline grace periods.
	Therefore, \co{force_quiescent_state()} and friends all
	check the last completed grace period number to avoid such races.

	Now these dyntick and CPU-offline grace periods are only checked
	for when a grace period is actually active.
	The only quiescent states that can be recorded when no grace
	period is in progress are self-detected quiescent states,
	which are recorded in the \co{passed_quiesc_completed},
	\co{passed_quiesc}, and \co{qs_pending}.
	These variables are initialized every time the corresponding
	CPU notices that a new grace period has started, preventing
	any obsolete quiescent states from being applied to the
	new grace period.

	All that said, optimizing grace-period latency may require that
	\co{gpnum} be tracked in addition to \co{completed}.
	\fi
} \QuickQuizEnd

\item	\co{beenonline}:
	초기값 0을 갖는 이 필드는 연관된 CPU 가 online 이 될 때마다 값이
	설정됩니다.
	이 필드는 online 이 된 적 없는 CPU 들을 위한 불필요한 tracing 출력이
	생성되는 것을 막기 위해 사용되며, 실제 CPU 들의 수보다 과하게 많은
	\co{NR_CPUS} 를 가는 커널들에서 유용합니다.
	\iffalse

	This field, initially zero, is set to one whenever the corresponding
	CPU comes online.
	This is used to avoid producing useless tracing output for CPUs
	that never have been online, which is useful in kernels where
	\co{NR_CPUS} greatly exceeds the actual number of CPUs.
	\fi

\QuickQuiz{}
	실제 CPU 들의 수보다 많은 \co{NR_CPUS} 를 갖는 시스템을 운영하는 이유가
	뭐죠?
	\iffalse

	What is the point of running a system with \co{NR_CPUS}
	way bigger than the actual number of CPUs?
	\fi
\QuickQuizAnswer{
	이는 다양한 시스템에서 돌아가는 하나의 리눅스 커널 바이너리를 만들 수
	있게 해서 운영과 검증을 간단하게 해주기 때문입니다.
	\iffalse

	Because this allows producing a single binary of the Linux kernel
	that runs on a wide variety of systems, greatly easing administration
	and validation.
	\fi
} \QuickQuizEnd

\item	\co{mynode}:
	이 필드는 이 연관된 CPU 를 처리하는 leaf \co{rcu_node} 구조체로의
	포인터입니다.
\item	\co{grpmask}:
	이 필드는 \co{mynode->qsmask} 의 어떤 bit 이 연관된 CPU 를 나타내는지를
	알리는 하나의 bit 이 set 되어있는 bitmask 입니다.
\item	\co{nxtlist}:
	이 필드는 이 CPU 에 위치한 가장 오래된 RCU 콜백 (\co{rcu_head} 구조체)
	으로의 포인터이고, 이 CPU 가 현재 콜백을 가지고 있지 않다면 \co{NULL}
	입니다.
	추가적인 콜백들은 \co{next} 포인터들을 통해 연결될 수 있습니다.
\item	\co{nxttail}:
	이 필드는 \co{nxtlist} 콜백 리스트로의 double-indirect tail 포인터들의
	배열입니다.
	만약 \co{nxtlist} 가 비어있다면, 모든 \co{nxttail} 포인터들은 이
	\co{nxtlist} 필드를 가리키게 됩니다.
	\co{nxttail} 배열의 각 원소들은 다음과 같은 의미를 갖습니다:
	\iffalse

\item	\co{mynode}:
	This field is a pointer to the leaf \co{rcu_node} structure that
	handles the corresponding CPU.
\item	\co{grpmask}:
	This field is a bitmask that has the single bit set that indicates
	which bit in \co{mynode->qsmask} signifies the corresponding CPU.
\item	\co{nxtlist}:
	This field is a pointer to the oldest RCU callback (\co{rcu_head}
	structure) residing on this CPU, or \co{NULL} if this CPU currently
	has no such callbacks.
	Additional callbacks may be chained via their \co{next} pointers.
\item	\co{nxttail}:
	This field is an array of double-indirect tail pointers
	into the \co{nxtlist} callback list.
	If \co{nxtlist} is empty, then all of the \co{nxttail} pointers
	directly reference the \co{nxtlist} field.
	Each element of the \co{nxttail} array has meaning as follows:
	\fi
	\begin{itemize}
	\item	\co{RCU_DONE_TAIL=0}:
		이 원소는 자신의 grace period 를 지나와서 호출될 준비가 된
		마지막 콜백의 \co{->next} 필드를 가리키거나 그런 콜백이
		없다면 \co{nxtlist} 필드를 가리킵니다.
	\item	\co{RCU_WAIT_TAIL=1}:
		이 원소는 현재의 grace period 가 끝나길 기다리고 있는 마지막
		콜백의 \co{next} 필드를 가리키고 있지만, 그런 콜백이 존재하지
		않는다면 \co{RCU_DONE_TAIL} 원소와 같습니다.
	\item	\co{RCU_WAIT_TAIL=2}:
		이 원소는 다음 grace period 를 기다릴 준비가 된 마지막 콜백의
		\co{next} 필드를 가리키고 있으며, 그런 콜백이 없다면
		\co{RCU_WAIT_TAIL} 원소와 같습니다.
	\item	\co{RCU_WAIT_TAIL=3}:
		이 원소는 이 리스트의 마지막 콜백의 \co{next} 필드를 가리키고
		있으며, 이 리스트가 비어있다면 \co{nxtlist} 필드를 가리킵니다.
	\iffalse

	\item	\co{RCU_DONE_TAIL=0}:
		This element references the \co{->next} field of
		the last callback that has passed through its grace
		period and is ready to invoke, or references the \co{nxtlist}
		field if there is no such callback.
	\item	\co{RCU_WAIT_TAIL=1}:
		This element references the \co{next} field of the
		last callback that is waiting for the current grace
		period to end, or is equal to the \co{RCU_DONE_TAIL}
		element if there is no such callback.
	\item	\co{RCU_NEXT_READY_TAIL=2}:
		This element references the \co{next} field of the
		last callback that is ready to wait for the next
		grace period, or is equal to the \co{RCU_WAIT_TAIL}
		element if there is no such callback.
	\item	\co{RCU_NEXT_TAIL=3}:
		This element references the \co{next} field of the
		last callback in the list, or references the \co{nxtlist}
		field if the list is empty.
	\fi
	\end{itemize}

\QuickQuiz{}
	이 웃기는 여러 꼬리를 가진 리스트 보다는 여러개의 리스트를 그냥
	유지하는게 어때요?
	\iffalse

	Why not simply have multiple lists rather than this funny
	multi-tailed list?
	\fi
\QuickQuizAnswer{
	Lai Jiangshan 덕에, 이 여러 꼬리를 가진 리스트 방법은 콜백 처리를
	단순화 시킵니다.
	\iffalse

	Because this multi-tailed approach, due to Lai Jiangshan,
	simplifies callback processing.
	\fi
} \QuickQuizEnd

\item	\co{qlen}:
	이 필드는 \co{nxtlist} 에 들어가 있는 콜백들의 수를 담습니다.
\item	\co{blimit}:
	이 필드는 한번에 호출될 수 있는 콜백들의 최대 갓수를 담습니다.
	이 한계는 많은 부하 아래에서의 시스템의 반응성을 개선합니다.
\item	\co{dynticks}:
	이 필드는
	Section~\ref{app:rcuimpl:rcutreewt:Tracking Dyntick State} 에서
	이야기한대로 연관된 CPU 의 \co{rcu_dynticks} 구조체를 참조합니다.
\item	\co{dynticks_snap}:
	이 필드는 \co{dynticks->dynticks} 의 예전 값을 담고 있는데,
	\co{force_quiescent_state()} 가 체크하는, 이 CPU 가 irq 핸들러에 있을
	때마다 언제 CPU 가 dynticks idle state 를 지나가는지 판단하는데에
	사용됩니다.
\item	\co{dynticks_nmi_snap}:
	이 필드는 \co{dynticks->dynticks_nmi} 의 예전 값을 갖는데,
	\co{force_quiescent_state()} 가 체크하는, 이 CPU 가 NMI 핸들러에 있을
	때마다 언제 CPU 가 dynticks idle state 를 지나가는지 판단하는데에
	사용됩니다.
\item	\co{dynticks_fqs}:
	이 필드는 dynticks-idle 모드로 인해 이 \co{rcu_data} 구조체에 연관된
	CPU 의 아래에서 다른 CPU 가 몇번이나 quiescent state 를 알렸는지 그
	횟수를 셉니다.
\item	\co{offline_fqs}:
	이 필드는 offline 이 되는 일로 인해 이 \co{rcu_data} 구조체에 연관된
	CPU 의 아래에서 다른 CPU 가 몇번이나 quiescent state 를 알렸는지 그
	횟수를 셉니다.
\iffalse

\item	\co{qlen}:
	This field contains the number of callbacks queued on
	\co{nxtlist}.
\item	\co{blimit}:
	This field contains the maximum number of callbacks that may
	be invoked at a time.
	This limitation improves system responsiveness under heavy load.
\item	\co{dynticks}:
	This field references the \co{rcu_dynticks} structure for
	the corresponding CPU, which is described in
	Section~\ref{app:rcuimpl:rcutreewt:Tracking Dyntick State}.
\item	\co{dynticks_snap}:
	This field contains a past value of \co{dynticks->dynticks},
	which is used to detect when a CPU passes through a dynticks
	idle state when this CPU happens to be in an irq
	handler each time that \co{force_quiescent_state()} checks it.
\item	\co{dynticks_nmi_snap}:
	This field contains a past value of \co{dynticks->dynticks_nmi},
	which is used to detect when a CPU passes through a dynticks
	idle state when this CPU happens to be in an NMI
	handler each time that \co{force_quiescent_state()} checks it.
\item	\co{dynticks_fqs}:
	This field counts the number of times that some other CPU noted
	a quiescent state on behalf of
	the CPU corresponding to this \co{rcu_data} structure due to
	its being in dynticks-idle mode.
\item	\co{offline_fqs}:
	This field counts the number of times that some other CPU noted
	a quiescent state on behalf of
	the CPU corresponding to this \co{rcu_data} structure due to
	its being offline.
\fi

\QuickQuiz{}
	그래서 일부 불쌍한 CPU 는 모든 각각의 offline CPU 아래에서도 quiescent
	state 들을 알려야 한다는 건가요?
	으얽웕!
	그건 적은 수의 CPU 를 가지고 있지만 큰 값의 \co{NR_CPUS} 를 가진,
	흔하지 않은 시스템에서는 지나친 오버헤드가 되지 않을까요?
	\iffalse

	So some poor CPU has to note quiescent states on behalf of
	each and every offline CPU?
	Yecch!
	Won't that result in excessive overheads in the not-uncommon
	case of a system with a small number of CPUs but a large value
	for \co{NR_CPUS}?
	\fi
\QuickQuizAnswer{
	실제로는, 그러지 않을 겁니다!

	Offline CPU 들은 \co{qsmask} 와 \co{qsmaskinit} bit mask 들에서
	빠져있고, 따라서 RCU 는 일반적으로 이것들을 무시합니다.
	하지만, offline CPU 가 \co{qsmask} bit 을 설정되고 있는 결과가 나오게
	하는 online/offline 오퍼레이션들에서의 race 가 존재합니다.
	이 race 들은 물론 올바르게 처리되어야 하며, 그게 처리되게 하는 방법은
	다른 CPU 들에게 RCU 가 offline CPU 로부터의 quiescent state 를 기다리고
	있다고 알려주는 겁니다.
	\iffalse

	Actually, no it will not!

	Offline CPUs are excluded from both the \co{qsmask} and
	\co{qsmaskinit} bit masks, so RCU normally ignores them.
	However, there are races with online/offline operations that
	can result in an offline CPU having its \co{qsmask} bit set.
	These races must of course be handled correctly, and the way
	they are handled is to permit other CPUs to note that RCU
	is waiting on a quiescent state from an offline CPU.
	\fi
} \QuickQuizEnd

\item	\co{resched_ipi}:
	이 필드는 reschedule IPI 가 연관된 CPU 로 보내어진 횟수를 셉니다.
	이런 IPI 들은 offline 인것도 dynticks idle state 에 있는 것도 아니면서
	시간 내에 quiescent state 를 지나지 못한 CPU 들에게 보내어집니다.
\item	\co{n_rcu_pending}:
	이 필드는 non-dynticks-idle CPU 들에게 jiffy 마다 호출되는
	\co{rcu_pending()} 을 호출한 횟수를 셉니다.
\item	\co{n_rcu_pending_force_qs}:
	이 필드는 \co{n_rcu_pending} 의 값의 threshold 를 갖습니다.
	만약 \co{n_rcu_pending} 이 이 threshold 에 닿는다면, 이는 현재의 grace
	period 가 너무 길게 이어졌음을 의미하며, 따라서 이걸 신속화 시키기 위해
	\co{force_quiescent_state()} 가 호출됩니다.
\iffalse

\item	\co{resched_ipi}:
	This field counts the number of times that a reschedule IPI
	is sent to the corresponding CPU.
	Such IPIs are sent to CPUs that fail to report passing through
	a quiescent states in a timely manner, but are neither offline
	nor in dynticks idle state.
\item	\co{n_rcu_pending}:
	This field counts the number of calls to \co{rcu_pending()},
	which is called once per jiffy on non-dynticks-idle CPUs.
\item	\co{n_rcu_pending_force_qs}:
	This field holds a threshold value for \co{n_rcu_pending}.
	If \co{n_rcu_pending} reaches this threshold, that indicates
	that the current grace period has extended too long, so
	\co{force_quiescent_state()} is invoked to expedite it.
\fi
\end{itemize}

\subsubsection{RCU Global State}
\label{app:rcuimpl:rcutreewt:RCU Global State}

\co{rcu_state} 구조체는 RCU (rcu 와 rcu\_bh) 인스턴스 각각을 위한 RCU 의 global
state 를 담고 있습니다.  이는 \co{rcu_node} 구조체의 계층과 관련된 필드들도
있고, \co{node} 배열 자체, 계층의 각 단계들을 향한 포인터들을 담는 \co{level}
배열, 이 계층 상의 각 단계의 노드의 갯수를 담는 \co{levelcnt} 배열, 이 계층의
각 단계의 노드별 자식의 갯수를 담는 \co{levelspread} 배열, 그리고 각 CPU 의
\co{rcu_data} 구조체로의 포인터의 \co{rda} 배열을 포함합니다.  \co{rcu_state}
구조체 또한 다양한 현재 grace period 의 자세한 내용들과 다른 메커니즘들
(\co{signaled}, \co{gpnum}, \co{completed}, \co{onofflock}, \co{fqslock},
\co{jiffies_force_qs}, \co{n_force_qs}, \co{n_force_qs_lh},
\co{n_force_qs_ngp}, \co{gp_start}, \co{jiffies_stall}, 그리고
\co{dynticks_completed}) 과의 상호작용을 조정하는 다양한 필드들을 갖고
있습니다.

이 필드들 각각에 대해 아래에서 설명합니다.
\iffalse

The \co{rcu_state} structure contains RCU's global state for
each instance of RCU (rcu and rcu\_bh).
It includes fields relating to
the hierarchy of \co{rcu_node} structures, including
the \co{node} array itself,
the \co{level} array that contains
pointers to the levels of the hierarchy,
the \co{levelcnt} array that contains the count of nodes at each level
of the hierarchy,
the \co{levelspread} array that contains the number of children
per node for each level of the hierarchy,
and the \co{rda} array of pointer to each of the CPU's
\co{rcu_data} structures.
The \co{rcu_state} structure also contains a number of fields
coordinating various details of the current grace period and its
interaction with other mechanisms (\co{signaled},
\co{gpnum}, \co{completed}, \co{onofflock}, \co{fqslock},
\co{jiffies_force_qs}, \co{n_force_qs}, \co{n_force_qs_lh},
\co{n_force_qs_ngp}, \co{gp_start}, \co{jiffies_stall},
and \co{dynticks_completed}).

Each of these fields are described below.
\fi

\begin{itemize}
\item	\co{node}:
	이 필드는 \co{rcu_node} 구조체의 배열로, 계층 상의 루트 노드는
	\co{->node[0]} 에 위치합니다.
	이 배열의 크기는 \co{NUM_RCU_NODES} C 전처리기 매크로로 명시되는데,
	Section~\ref{app:rcuimpl:rcutreewt:Kernel Parameters} 에서 설명되듯이
	\co{NR_CPUS} 와 \co{CONFIG_RCU_FANOUT} 을 가지고 계산됩니다.
	첫번째 원소부터 시작해서 \co{->node} 배열을 순회하는 것은 \co{rcu_node}
	계층을 breadth-first 탐색하는 것과 같은 효과를 냅니다.
	\iffalse

	This field is the array of \co{rcu_node} structures,
	with the root node of the hierarchy being located at
	\co{->node[0]}.
	The size of this array is specified by the
	\co{NUM_RCU_NODES} C-preprocessor macro, which is computed
	from \co{NR_CPUS} and \co{CONFIG_RCU_FANOUT}
	as described in
	Section~\ref{app:rcuimpl:rcutreewt:Kernel Parameters}.
	Note that traversing the \co{->node} array starting at
	element zero has the effect of doing a breadth-first search
	of the \co{rcu_node} hierarchy.
	\fi
\item	\co{level}:
	이 필드는 \co{node} 뱅렬로의 포인터들의 배열입니다.
	해당 계층의 루트 노드는 \co{->level[0]} 를 통해 참조되며, 계층상 두번째
	단계의 (존재한다면) 첫번째 노드는 \co{->level[1]} 로 참조되며, 나머지도
	그렇게 구성됩니다.
	첫번째 leaf 노드는 \co{->level[NUM_RCU_LVLS-1]} 로 참조되며, 따라서
	\co{level} 배열의 크기는 \co{NUM_RCU_LVLS} 로 명시되는데,
	Section~\ref{app:rcuimpl:rcutreewt:Kernel Parameters} 에서 설명된대로
	계산됩니다.
	\co{->level} 필드는 \co{rcu_node} 계층의 한 단계를, 예를 들면 leaf
	노드들을 스캔하기 위해 \co{->node} 와 함께 조합되어 사용됩니다.
	\co{->level} 의 원소들은 부팅 시점에 \co{rcu_init_one()} 함수에 의해
	채워집니다.
	\iffalse

	This field is an array of pointers into the \co{node} array.
	The root node of the hierarchy is referenced by
	\co{->level[0]}, the first node of the second level of
	the hierarchy (if there is one) by \co{->level[1]}, and so on.
	The first leaf node is referenced by
	\co{->level[NUM_RCU_LVLS-1]}, and the size of the \co{level}
	array is thus specified by \co{NUM_RCU_LVLS}, which is
	computed as described in
	Section~\ref{app:rcuimpl:rcutreewt:Kernel Parameters}.
	The \co{->level} field is often used in combination with
	\co{->node} to scan a level of the \co{rcu_node} hierarchy,
	for example, all of the leaf nodes.
	The elements of \co{->level} are filled in by the
	boot-time \co{rcu_init_one()} function.
	\fi
\item	\co{levelcnt}:
	이 필드는 \co{rcu_node} 계층의 각 단계의 노드의 갯수들을 담고 있는
	배열로, leaf \co{rcu_node} 구조체를 참조하는 \co{rcu_data} 구조체의
	갯수를 포함해서, 이 배열은 \co{->level} 배열보다 원소를 하나 더 많이
	갖게 됩니다.
	\co{->levelcnt[0]} 는 이 계층의 꼭대기에 있는 하나의 루트 \co{rcu_node}
	에 연관되어서 1의 값을 항상 가질 겁니다.
	이 배열은 C 전처리키 매크로로
	Section~\ref{app:rcuimpl:rcutreewt:Kernel Parameters} 에서 설명된 대로
	계산되는
	\co{NUM_RCU_LVL_0},
	\co{NUM_RCU_LVL_1},
	\co{NUM_RCU_LVL_2}, 그리고
	\co{NUM_RCU_LVL_3},
	의 값으로 초기화 됩니다.
	\co{->levelcnt} 필드는 디버깅 목적으로 이 계층의 다른 부분들을 초기화
	하는데에 사용됩니다.
	\iffalse

	This field is an array containing the number of \co{rcu_node}
	structures in each level of the hierarchy, including the
	number of \co{rcu_data} structures referencing the leaf
	\co{rcu_node} structures, so that this array has one more
	element than does the \co{->level} array.
	Note that \co{->levelcnt[0]} will always contain a value of
	one, corresponding to the single root \co{rcu_node} at the
	top of the hierarchy.
	This array is initialized with the values
	\co{NUM_RCU_LVL_0},
	\co{NUM_RCU_LVL_1},
	\co{NUM_RCU_LVL_2}, and
	\co{NUM_RCU_LVL_3},
	which are C-preprocessor macros computed as described in
	Section~\ref{app:rcuimpl:rcutreewt:Kernel Parameters}.
	The \co{->levelcnt} field is used to initialize
	other parts of the hierarchy and for debugging purposes.
	\fi
\item	\co{levelspread}:
	이 필드의 각 원소는 \co{rcu_node} 계층의 연관된 각 단계의 희망되는
	자식들의 갯수를 담습니다.
	이 배열의 원소들의 값들은 실행시점에 \co{CONFIG_RCU_FANOUT_EXACT} 커널
	패러미터로 선택되는 두개의 \co{rcu_init_levelspread()} 함수들 중 하나로
	계산됩니다.
	\iffalse

	Each element of this field contains the desired number of children
	for the corresponding level of the \co{rcu_node} hierarchy.
	This array's element's values are computed at runtime
	by one of the two \co{rcu_init_levelspread()} functions,
	selected by the \co{CONFIG_RCU_FANOUT_EXACT} kernel parameter.
	\fi
\item	\co{rda}:
	이 필드의 각 원소는 연관된 CPU 의 \co{rcu_data} 구조체로의 포인터를
	담고 있습니다.
	이 배열은 부팅 시점에 \co{RCU_DATA_PTR_INIT()} 매크로로 초기화 됩니다.
	\iffalse

	Each element of this field contains a pointer to the
	corresponding CPU's \co{rcu_data} structure.
	This array is initialized at boot time by the
	\co{RCU_DATA_PTR_INIT()} macro.
	\fi
\item	\co{signaled}:
	이 필드는
	Section~\ref{app:rcuimpl:rcutreewt:Forcing Quiescent States} 에서
	설명된 것처럼 \co{force_quiescent_state()} 함수에 의해 사용되는 상태를
	유지하기 위해 사용됩니다.
	이 필드는 다음의 값들 중 하나를 취합니다:
	\iffalse

	This field is used to maintain state used by the
	\co{force_quiescent_state()} function, as described in
	Section~\ref{app:rcuimpl:rcutreewt:Forcing Quiescent States}.
	This field takes on values as follows:
	\fi
	\begin{itemize}
	\item	\co{RCU_GP_INIT}:
		이 값은 현재의 grace period 가 초기화 되어가는 과정이 진행
		중이어서 \co{force_quiescent_state()} 는 아무 동작도 하지
		말아야 함을 알립니다.
		물론, 이런 상황이 발생하기 위해선 grace-period 초기화가 세개의
		jiffiy 만큼이나 길게 수행될 필요가 있겠지만, 여러분이 매우 많은
		수의 CPU 들을 사용한다면 이 race 는 실제로 일어날 수 있습니다.
		일단 grace-period 초기화가 완료된다면, 이 값은
		(\co{CONFIG_NO_HZ} 라면) \co{RCU_SAVE_DYNTICK} 이나
		\co{RCU_FORCE_QS} 로 설정됩니다.
		\iffalse

		This value indicates that the current grace period
		is still in the process of being initialized,
		so that \co{force_quiescent_state()} should take
		no action.
		Of course, grace-period initialization would need
		to stretch out for three jiffies before this race
		could arise, but if you have a very large number
		of CPUs, this race could in fact occur.
		Once grace-period initialization is complete,
		this value is set to either \co{RCU_SAVE_DYNTICK}
		(if \co{CONFIG_NO_HZ}) or \co{RCU_FORCE_QS} otherwise.
		\fi
	\item	\co{RCU_SAVE_DYNTICK}:
		이 값은 현재의 grace period 를 위한 quiescent state 를 아직
		보고하지 않은 모든 CPU 에 대해 \co{force_quiescent_state()} 가
		dynticks state 를 체크해야 함을 알립니다.
		Quiescent state 는 dyntick-idle 모드에 있는 모든 CPU 들
		아래에서 보고될 겁니다.
		\iffalse

		This value indicates that \co{force_quiescent_state()}
		should check the dynticks state of any CPUs that have
		not yet reported quiescent states for the current
		grace period.
		Quiescent states will be reported on behalf of any
		CPUs that are in dyntick-idle mode.
		\fi
	\item	\co{RCU_FORCE_QS}:
		이 값은 현재의 grace period 를 위한 quiescent state 를 아직
		보고하지 않은 모든 CPU 에 대해 \co{force_quiescent_state()} 가
		dynticks state 와 online/offline 상태를 다시 체크해야 함을
		알립니다.
		다시 dynticks state 를 체크하는 것은 이 구현이 dynticks-idle
		state 에 있지만, 그게 체크될 때 irq 나 NMI 핸들러에 있었던 CPU
		의 경우에 대해서도 처리할 수 있게 해줍니다.
		이외에 모두 실패한다면, 이 꾸물거리는 CPU 에게 reschedule IPI
		가 날아갈 겁니다.
		\iffalse

		This value indicates that \co{force_quiescent_state()}
		should recheck dynticks state along with the online/offline
		state of any CPUs that have
		not yet reported quiescent states for the current
		grace period.
		The rechecking of dynticks states allows the implementation
		to handle cases where a given CPU might be in dynticks-idle
		state, but have been in an irq or NMI handler both
		times it was checked.
		If all else fails, a reschedule IPI will be sent to
		the laggard CPU.
		\fi
	\end{itemize}
	이 필드는 루트 \co{rcu_node} 구조체의 락으로 보호됩니다.
	\iffalse

	This field is guarded by the root \co{rcu_node} structure's lock.
	\fi

\QuickQuiz{}
	그럼 이 앞에서 이야기된 이 구조체의 필드들은 뭘로 보호되나요?
	\iffalse

	So what guards the earlier fields in this structure?
	\fi
\QuickQuizAnswer{
	아무것도 보호하지 않는데, 그것들은 컴파일 시점이나 부팅 시점에 그 값이
	설정되는 상수들이기 때문입니다.
	물론, \co{->node} 배열 안의 \co{rcu_node} 안의 필드들은 바뀔 수도
	있겠습니다만, 그것들은 별도로 보호됩니다.
	\iffalse

	Nothing does, as they are constants set at compile time
	or boot time.
	Of course, the fields internal to each \co{rcu_node}
	in the \co{->node} array may change, but they are
	guarded separately.
	\fi
} \QuickQuizEnd

\item	\co{gpnum}:
	이 필드는 현재 grace period 의 수를, 또는 현재 grace period 가 없다면
	마지막 grace period 의 수를 갖습니다.
	이 필드는 루트 \co{rcu_node} 구조체의 락으로 보호됩니다만, 이 락을 잡지
	않은채로도 빈번히 액세스 됩니다 (하지만 수정되진 않습니다).
	\iffalse

	This field contains the number of the current grace period,
	or that of the last grace period if no grace period is currently
	in effect.
	This field is guarded by the root \co{rcu_node} structure's lock,
	but is frequently accessed (but never modified) without holding
	this lock.
	\fi
\item	\co{completed}:
	이 필드는 마지막으로 완료된 grace period 의 수를 갖습니다.
	현재 진행중인 grace period 가 없다면 이는 \co{->gpnum} 과 동일하고,
	현재 진행중인 grace period 가 있다면 \co{->gpnum} 보다 1 작습니다.
	원론적으로, 이 두개의 필드들을 하나의 boolean 으로 바꿀 수도 있을 텐데,
	일부 버전의 Linux 에서의 Classic RCU 가 그러합니다만, 실제로는 두개의
	숫자를 갖는 편이 race 처리에 훨씬 간단합니다.
	이 필드는 루트 \co{rcu_node} 구조체의 락으로 보호됩니다만, 이 락을 잡지
	않은채로도 빈번히 액세스 됩니다 (하지만 수정되진 않습니다).
	\iffalse

	This field contains the number of the last completed grace period.
	As such, it is equal to \co{->gpnum} when there is no grace period
	in progress, or one less than \co{->gpnum} when there is a
	grace period in progress.
	In principle, one could replace this pair of fields with a single
	boolean, as is done in Classic RCU in some versions of Linux,
	but in practice race resolution is much simpler given the pair
	of numbers.
	This field is guarded by the root \co{rcu_node} structure's lock,
	but is frequently accessed (but never modified) without holding
	this lock.
	\fi
\item	\co{onofflock}:
	이 필드는 grace-period 초기화와 동시에 online/offline 처리가 수행되는
	것을 막습니다.
	여기엔 하나의 예외가 있습니다: 만약 \co{rcu_node} 계층이 하나의
	구조체로만 이루어져 있다면, 이 하나의 구조체의 \co{->lock} 필드가 이
	일을 대신 처리할 겁니다.
	\iffalse

	This field prevents online/offline processing from running
	concurrently with grace-period initialization.
	There is one exception to this: if the \co{rcu_node}
	hierarchy consists of but a single structure, then
	that single structure's \co{->lock} field will instead take on
	this job.
	\fi
\item	\co{fqslock}:
	이 필드는 두개 이상의 task 가 \co{force_quiescent_state()} 로 quiescent
	state 를 강제하는 것을 막습니다.
	\iffalse

	This field prevents more than one task from forcing quiescent
	states with \co{force_quiescent_state()}.
	\fi
\item	\co{jiffies_force_qs}:
	이 필드는 CPU 들을 quiescent state 로 강제하기 위해, 그리고/또는
	extended quiescent state 를 보고하기 위해 언제
	\co{force_quiescent_state()} 가 호출되어야 하는지를 jiffy 단위로
	기록하고 있습니다.
	이 필드는 루트 \co{rcu_node} 구조체의 락으로 보호됩니다만, 이 락 없이도
	자주 액세스 됩니다 (하지만 수정은 되지 않습니다).
	\iffalse

	This field contains the time, in jiffies, when
	\co{force_quiescent_state()} should be invoked in order to
	force CPUs into quiescent states and/or report extended
	quiescent states.
	This field is guarded by the root \co{rcu_node} structure's lock,
	but is frequently accessed (but never modified) without holding
	this lock.
	\fi
\item	\co{n_force_qs}:
	이 필드는 \co{force_quiescent_state()} 가 grace period 가 이미
	완료되었거나, 동시적으로 \co{force_quiescent_state()} 를 수행중인 다른
	CPU 가 있거나, \co{force_quiescent_state()} 가 너무 최근에 수행된 적이
	있거나 하지 않고 실제로 일을 수행한 횟수를 셉니다.
	이 필드는 tracing 과 디버깅에 사용되며, \co{->fqslock} 으로 보호됩니다.
	\iffalse

	This field counts the number of calls to \co{force_quiescent_state()}
	that actually do work, as opposed to leaving early due to
	the grace period having already completed, some other
	CPU currently running \co{force_quiescent_state()},
	or \co{force_quiescent_state()} having run too recently.
	This field is used for tracing and debugging, and
	is guarded by \co{->fqslock}.
	\fi
\item	\co{n_force_qs_lh}:
	이 필드는 \co{force_quiescent_state()} 가 \co{->fqslock} 이 다른 CPU 에
	의해 잡혀 있었던 관계로 곧바로 리턴한 대략적 횟수를 셉니다.
	이 필드는 tracing 과 debugging 에 사용되며 대략적인 수인 만큼 어떤
	락으로도 보호되지 않습니다.
	\iffalse

	This field holds an approximate count of the number of times that
	\co{force_quiescent_state()} returned early due to the
	\co{->fqslock} being held by some other CPU.
	This field is used for tracing and debugging, and is not
	guarded by any lock, hence its approximate nature.
	\fi
\item	\co{n_force_qs_ngp}:
	이 필드는 성공적으로 \co{->fqslock} 을 잡았지만 진행중인 grace period
	가 없다는 것을 알아차리게 된 \co{force_quiescent_state()} 수행 횟수를
	셉니다.
	이 필드는 tracing 과 debugging 을 위해 사용되며, \co{->fqslock} 으로
	보호됩니다.
	\iffalse

	This field counts the number of times that
	\co{force_quiescent_state()} that successfully acquire
	\co{->fqslock}, but then find that there is no grace period
	in progress.
	This field is used for tracing and debugging, and
	is guarded by \co{->fqslock}.
	\fi
\item	\co{gp_start}:
	이 필드는 가장 최근의 grace period 가 시작한 시간을 jiffy 단위로
	기록합니다.
	이는 stalled CPU 들의 파악에 사용됩니다만,
	\co{CONFIG_RCU_CPU_STALL_DETECTOR} 커널 패러미터가 선택되었을 때만의
	이야기입니다.
	이 필드는 루트 \co{rcu_node} 의 \co{->lock} 으로 보호됩니다만, 가끔은
	이 락을 잡지 않고도 액세스 됩니다 (하지만 수정되진 않습니다).
	\iffalse

	This field records the time at which the most recent grace period
	began, in jiffies.
	This is used to detect stalled CPUs, but only when the
	\co{CONFIG_RCU_CPU_STALL_DETECTOR} kernel parameter is selected.
	This field is guarded by the root \co{rcu_node}'s \co{->lock},
	but is sometimes accessed (but not modified) outside of this
	lock.
	\fi
\item	\co{jiffies_stall}:
	이 필드는 현재의 grace period 가 연장될 시간을 jiffy 단위로 기록하며,
	CPU stall 을 체크하기 위해 사용되기 적합할 겁니다.
	\co{->gp_start} 와 같이, 이 필드는 \co{CONFIG_RCU_CPU_STALL_DETECTOR}
	커널 패러미터가 선택되었을 때에만 존재합니다.
	이 필드는 루트 \co{rcu_node} 의 \co{->lock} 으로 보호됩니다만, 가끔은
	이 락을 잡지 않은 채 액세스 됩니다 (하지만 수정되지는 않습니다).
	\iffalse

	This field holds the time, in jiffies, at which the current
	grace period will have extended for so long that it will
	be appropriate to check for CPU stalls.
	As with \co{->gp_start}, this field exists only when the
	\co{CONFIG_RCU_CPU_STALL_DETECTOR} kernel parameter is selected.
	This field is guarded by the root \co{rcu_node}'s \co{->lock},
	but is sometimes accessed (but not modified) outside of this
	lock.
	\fi
\item	\co{dynticks_completed}:
	이 필드는 \co{force_quiescent_state()} 가 dyntick state 의 스냅샷을
	찍을 때의 \co{->completed} 의 값을 기록합니다만, 또한 매 grace period
	의 시작 때마다 기존 grace period 로 초기화 됩니다.
	이 필드는 기존 grace period 로부터의 dyntick-idle quiescent state 가
	현재 grace period 에 적용되는 것을 막는데에 사용됩니다.
	이 필드는 \co{CONFIG_NO_HZ} 커널 패러미터가 선택되었을 때에만
	존재합니다.
	이 필드는 루트 \co{rcu_node} 의 \co{->lock} 으로 보호됩니다만, 가끔은
	이 락을 잡지 않은 채로 액세스 됩니다 (하지만 수정되지는 않습니다).
	\iffalse

	This field records the value of \co{->completed} at the time when
	\co{force_quiescent_state()} snapshots dyntick state, but
	is also initialized to an earlier grace period at the beginning
	of each grace period.
	This field is used to prevent dyntick-idle quiescent states
	from a prior grace period from being applied to the current
	grace period.
	As such, this field exists only when the \co{CONFIG_NO_HZ}
	kernel parameter is selected.
	This field is guarded by the root \co{rcu_node}'s \co{->lock},
	but is sometimes accessed (but not modified) outside of this
	lock.
	\fi
\end{itemize}

\subsubsection{Kernel Parameters}
\label{app:rcuimpl:rcutreewt:Kernel Parameters}

다음의 커널 패러미터들이 이 RCU 변종에 영향을 끼칩니다:
\iffalse

The following kernel parameters affect this variant of RCU:
\fi

\begin{itemize}
\item	\co{NR_CPUS}, 시스템의 CPU 의 최대 갯수.
\item	\co{CONFIG_RCU_FANOUT}, \co{rcu_node} 계층의 각 노드의 바람직한 자식
	노드의 수.
\item	\co{CONFIG_RCU_FANOUT_EXACT}, \co{rcu_node} 계층의 rebalancing 을 막는
	boolean 값.
\item	\co{CONFIG_HOTPLUG_CPU}, CPU 들이 online 이 되고 offline 이 될 수
	있도록 허용.
\item	\co{CONFIG_NO_HZ}, dynticks-idle 모드가 지원되는지 여부를 알림.
\item	\co{CONFIG_SMP}, 복수개의 CPU 들이 존재할 수 있음을 알림.
\item	\co{CONFIG_RCU_CPU_STALL_DETECTOR}, RCU grace period 가 너무 길게
	연장된다면 RCU 는 stalled CPU 를 체크해야 함을 알림.
\item	\co{CONFIG_RCU_TRACE}, RCU 가 \co{debugfs} 에 tracing 정보를 제공해야
	함을 알림.
\iffalse

\item	\co{NR_CPUS}, the maximum number of CPUs in the system.
\item	\co{CONFIG_RCU_FANOUT}, the desired number of children for
	each node in the \co{rcu_node} hierarchy.
\item	\co{CONFIG_RCU_FANOUT_EXACT}, a boolean preventing rebalancing
	of the \co{rcu_node} hierarchy.
\item	\co{CONFIG_HOTPLUG_CPU}, permitting CPUs to come online and go
	offline.
\item	\co{CONFIG_NO_HZ}, indicating that dynticks-idle mode is supported.
\item	\co{CONFIG_SMP}, indicating that multiple CPUs may be present.
\item	\co{CONFIG_RCU_CPU_STALL_DETECTOR}, indicating that RCU should
	check for stalled CPUs when RCU grace periods extend too long.
\item	\co{CONFIG_RCU_TRACE}, indicating that RCU should provide
	tracing information in \co{debugfs}.
\fi
\end{itemize}

\begin{figure*}[tbp]
{
\scriptsize
\begin{verbatim}
  1 #define MAX_RCU_LVLS    3
  2 #define RCU_FANOUT      (CONFIG_RCU_FANOUT)
  3 #define RCU_FANOUT_SQ   (RCU_FANOUT * RCU_FANOUT)
  4 #define RCU_FANOUT_CUBE (RCU_FANOUT_SQ * RCU_FANOUT)
  5
  6 #if NR_CPUS <= RCU_FANOUT
  7 #  define NUM_RCU_LVLS  1
  8 #  define NUM_RCU_LVL_0 1
  9 #  define NUM_RCU_LVL_1 (NR_CPUS)
 10 #  define NUM_RCU_LVL_2 0
 11 #  define NUM_RCU_LVL_3 0
 12 #elif NR_CPUS <= RCU_FANOUT_SQ
 13 #  define NUM_RCU_LVLS  2
 14 #  define NUM_RCU_LVL_0 1
 15 #  define NUM_RCU_LVL_1 (((NR_CPUS) + RCU_FANOUT - 1) / RCU_FANOUT)
 16 #  define NUM_RCU_LVL_2 (NR_CPUS)
 17 #  define NUM_RCU_LVL_3 0
 18 #elif NR_CPUS <= RCU_FANOUT_CUBE
 19 #  define NUM_RCU_LVLS  3
 20 #  define NUM_RCU_LVL_0 1
 21 #  define NUM_RCU_LVL_1 (((NR_CPUS) + RCU_FANOUT_SQ - 1) / RCU_FANOUT_SQ)
 22 #  define NUM_RCU_LVL_2 (((NR_CPUS) + (RCU_FANOUT) - 1) / (RCU_FANOUT))
 23 #  define NUM_RCU_LVL_3 NR_CPUS
 24 #else
 25 # error "CONFIG_RCU_FANOUT insufficient for NR_CPUS"
 26 #endif /* #if (NR_CPUS) <= RCU_FANOUT */
 27
 28 #define RCU_SUM (NUM_RCU_LVL_0 + NUM_RCU_LVL_1 + NUM_RCU_LVL_2 + NUM_RCU_LVL_3)
 29 #define NUM_RCU_NODES (RCU_SUM - NR_CPUS)
\end{verbatim}
}
\caption{Determining Shape of RCU Hierarchy}
\label{fig:app:rcuimpl:rcutreewt:Determining Shape of RCU Hierarchy}
\end{figure*}

\co{CONFIG_RCU_FANOUT} 과 \co{NR_CPUS} 패러미터들은
Figure~\ref{fig:app:rcuimpl:rcutreewt:Determining Shape of RCU Hierarchy} 에
보인 것처럼 컴파일 시점에 \co{rcu_node} 계층의 모습을 결정하는데에 사용됩니다.
Line~1 은 \co{rcu_node} 계층의 최대 depth 를 정하는데, 여기선 3 입니다.
최대 허용된 depth 를 늘리는 행위는 여기저기 변경을 필요로 하는데, 예를 들면
line~6-26 의 \co{#if} 문에 또다른 경우를 추가합니다.
Line~2-4 는 각각 fanout, fanout 의 제곱, 그리고 fanout 의 세제곱을 계산합니다.
\iffalse

The \co{CONFIG_RCU_FANOUT} and \co{NR_CPUS} parameters are used to
determine the shape of the \co{rcu_node} hierarchy at compile time,
as shown in
Figure~\ref{fig:app:rcuimpl:rcutreewt:Determining Shape of RCU Hierarchy}.
Line~1 defines the maximum depth of the \co{rcu_node} hierarchy,
currently three.
Note that increasing the maximum permitted depth requires changes
elsewhere, for example, adding another leg to the \co{#if}
statement running from lines~6-26.
Lines~2-4 compute the fanout, the square of the fanout, and the cube
of the fanout, respectively.
\fi

이후에 이 값들은 \co{rcu_node} 계층의 요구되는 depth 를 결정하기 위해
\co{NR_CPUS} 와 비교되어지며, 이 depth 는 \co{NUM_RCU_LVLS} 에 저장되는데 이는
\co{rcu_state} 구조체의 여러 배열들의 크기를 정하는데 사용되어집니다.
루트 단계에는 항상 하나의 노드가 존재하며, leaf 단계 아래에는 \co{NUM_CPUS}
갯수의 \co{rcu_data} 구조체가 존재합니다.
루트 단계 이상의 단계가 존재한다면, leaf 단계에서의 노드의 갯수는 \co{NR_CPUS}
를 \co{RCU_FANOUT_SQ} 로 나누고 반올림 한 값으로 계산됩니다.
다른 단계에서의 노드들의 갯수는 (예를 들면) \co{RCU_FANOUT_SQ} 대신
\co{RCU_FANOUT} 을 사용하는 식으로 비슷하지만 약간 다르게 계산됩니다.
\iffalse

Then these values are compared to \co{NR_CPUS} to determine the required
depth of the \co{rcu_node} hierarchy, which is placed into
\co{NUM_RCU_LVLS}, which is used to size a number of arrays
in the \co{rcu_state} structure.
There is always one node at the root level, and there are always
\co{NUM_CPUS} number of \co{rcu_data} structures below the leaf
level.
If there is more than just the root level, the number of nodes at
the leaf level is computed
by dividing \co{NR_CPUS} by \co{RCU_FANOUT}, rounding up.
The number of nodes at other levels is computed in a similar manner,
but using (for example) \co{RCU_FANOUT_SQ} instead of \co{RCU_FANOUT}.
\fi

이어서 Line~28 은 이 모든 단계의 수들을 더해서 \co{rcu_node} 구조체들의 수와
\co{rcu_data} 구조체의 수를 더한 값을 구합니다.
마지막으로, line~29 는 이 합으로부터 (\co{rcu_data} 구조체들의 갯수인)
\co{NR_CPUS} 를 빼서 \co{rcu_node} 구조체의 갯수를 구해서 \co{NUM_RCU_NODES}
안에 저장합니다.
이 값은 \co{rcu_state} 구조체의 \co{->nodes} 배열의 크기를 정하는데에
사용됩니다.
\iffalse

Line~28 then sums up all of the levels, resulting in the number of
\co{rcu_node} structures plus the number of \co{rcu_data} structures.
Finally, line~29 subtracts \co{NR_CPUS} (which is the number of
\co{rcu_data} structures) from the sum, resulting in the number
of \co{rcu_node} structures, which is retained in
\co{NUM_RCU_NODES}.
This value is then used to size the \co{->nodes} array in the
\co{rcu_state} structure.
\fi

\subsection{External Interfaces}
\label{app:rcuimpl:rcutreewt:External Interfaces}

RCU 의 외부로의 인터페이스는 표준 RCU API 만이 아니라 RCU 구현 자체에 필요한
커널의 나머지 부분들로의 내부적 인터페이스도 포함됩니다.
이 인터페이스들은
\co{rcu_read_lock()},
\co{rcu_read_unlock()},
\co{rcu_read_lock_bh()},
\co{rcu_read_unlock_bh()},
(\co{__call_rcu()} 의 wrapper 인) \co{call_rcu()},
(마찬가지인) \co{call_rcu_bh()},
\co{rcu_check_callbacks()},
(\co{__rcu_process_callbacks()} 의 wrapper 인)
\co{rcu_process_callbacks()},
( \co{__rcu_pending()} 의 wrapper 인)
\co{rcu_pending()},
\co{rcu_needs_cpu()},
\co{rcu_cpu_notify()}, 그리고
\co{__rcu_init()} 입니다.
\co{synchronize_rcu()} 와 \co{rcu_barrier()} 는 모든 RCU 구현에 공통되며,
\co{call_rcu()} 규모에서 정의되어 있음을 알아두시기 바랍니다.
비슷하게, \co{rcu_barrier_bh()} 는 모든 RCU 구현에 공통되며, \co{call_rcu_bh()}
규모에서 정의되어 있습니다.

이런 외부로의 API 들 각각을 다음 섹션들에서 설명합니다.
\iffalse

RCU's external interfaces include not just the standard RCU API,
but also the internal interfaces to the rest of the kernel that
are required for the RCU implementation itself.
The interfaces are
\co{rcu_read_lock()},
\co{rcu_read_unlock()},
\co{rcu_read_lock_bh()},
\co{rcu_read_unlock_bh()},
\co{call_rcu()} (which is a wrapper around
\co{__call_rcu()}),
\co{call_rcu_bh()} (ditto),
\co{rcu_check_callbacks()},
\co{rcu_process_callbacks()} (which is a wrapper around
\co{__rcu_process_callbacks()}),
\co{rcu_pending()} (which is a wrapper around
\co{__rcu_pending()}),
\co{rcu_needs_cpu()},
\co{rcu_cpu_notify()}, and
\co{__rcu_init()}.
Note that \co{synchronize_rcu()} and \co{rcu_barrier()} are
common to all RCU implementations, and are defined in terms of
\co{call_rcu()}.
Similarly, \co{rcu_barrier_bh()} is common to all RCU implementations
and is defined in terms of \co{call_rcu_bh()}.

These external APIs are each described in the following sections.
\fi

\subsubsection{Read-Side Critical Sections}
\label{app:rcuimpl:rcutreewt:Read-Side Critical Sections}

\begin{figure}[tbp]
{ \scriptsize
\begin{verbatim}
  1 void __rcu_read_lock(void)
  2 {
  3   preempt_disable();
  4   __acquire(RCU);
  5   rcu_read_acquire();
  6 }
  7
  8 void __rcu_read_unlock(void)
  9 {
 10   rcu_read_release();
 11   __release(RCU);
 12   preempt_enable();
 13 }
 14
 15 void __rcu_read_lock_bh(void)
 16 {
 17   local_bh_disable();
 18   __acquire(RCU_BH);
 19   rcu_read_acquire();
 20 }
 21
 22 void __rcu_read_unlock_bh(void)
 23 {
 24   rcu_read_release();
 25   __release(RCU_BH);
 26   local_bh_enable();
 27 }
\end{verbatim}
}
\caption{RCU Read-Side Critical Sections}
\label{fig:app:rcuimpl:rcutreewt:RCU Read-Side Critical Sections}
\end{figure}

Figure~\ref{fig:app:rcuimpl:rcutreewt:RCU Read-Side Critical Sections}
는 RCU read-side 크리티컬 섹션들을 표시하는 함수들을 보입니다.
Line~1-6 은 ``rcu'' read-side 크리티컬 섹션을 시작하는 \co{__rcu_read_lock()}
을 보입니다.
Line~3 은 preemption 을 불능화 하고,
line~4 는 RCU read-side 크리티컬 섹션의 시작을 알리는 가벼운 표시이고,
line~5 는 lockdep 상태를 업데이트 합니다.
Line~8-13 은 \co{__rcu_read_lock()} 의 반대 역할인 \co{__rcu_read_unlock()} 을
보입니다.
Line~15-20 은 \co{__rcu_read_lock_bh()} 를 보이고 line~22-27 은
\co{__rcu_read_unlock_bh()} 을 보이는데, 앞의 두개와 비슷한 함수입니다만,
preemption 이 아니라 bottom-half 처리를 불능화하고 활성화 합니다.
\iffalse

Figure~\ref{fig:app:rcuimpl:rcutreewt:RCU Read-Side Critical Sections}
shows the functions that demark RCU read-side critical sections.
Lines~1-6 show \co{__rcu_read_lock()}, which begins an ``rcu''
read-side critical section.
line~3 disables preemption,
line~4 is a sparse marker noting the beginning of an RCU read-side critical
section,
and
line~5 updates lockdep state.
Lines~8-13 show \co{__rcu_read_unlock()}, which is the inverse of
\co{__rcu_read_lock()}.
Lines~15-20 show \co{__rcu_read_lock_bh()} and lines~22-27 show
\co{__rcu_read_unlock_bh()}, which are analogous to the previous
two functions, but disable and enable bottom-half processing rather
than preemption.
\fi

\QuickQuiz{}
	전 RCU read-side 처리가 \emph{빠를} 거라고 생각했는데요!
	Figure~\ref{fig:app:rcuimpl:rcutreewt:RCU Read-Side Critical Sections}
	에 보인 함수들은 너무 쓰레기가 많이 들어있어서 이건 느릴 \emph{수밖에}
	없겠는걸요!
	이게 뭡니까?
	\iffalse

	I thought that RCU read-side processing was supposed to
	be \emph{fast}!
	The functions shown in
	Figure~\ref{fig:app:rcuimpl:rcutreewt:RCU Read-Side Critical Sections}
	have so much junk in them that they just \emph{have} to be slow!
	What gives here?
	\fi
\QuickQuizAnswer{
	외양은 착각을 일으킬 수 있습니다.
	\co{preempt_disable()}, \co{preempt_enable()},
	\co{local_bh_disable()}, 그리고 \co{local_bh_enable()} 는 각각 로컬
	데이터에 대한 하나의 atomic 하지 않은 변경을 가합니다.
	그것조차도 \co{CONFIG_PREEMPT} 를 가정한 것이고, 그렇지 않다면
	\co{preempt_disable()} 과 \co{preempt_enable()} 함수는 아무 코드도
	수행하지 않으며 컴파일러 지시어도 내놓지 않습니다.
	\co{__acquire()} 와 \co{__release()} 함수들은 아무 코드도 (아무
	컴파일러 지시어도) 만들지 않고 문법 분석 버그 탐색 프로그램인
	\co{sparse} 에 사용될 뿐입니다.
	마지막으로, \co{rcu_read_acquire()} 와 \co{rcu_read_release()} 는
	실제로 그 비용이 비쌀 수 있는 경우인, 락 순서 디버깅 기능인 ``lockdep''
	이 활성화 되어 있는 경우가 아니라면 아무 코드도 (아무 컴파일러
	지시어도) 내놓지 않습니다.

	짧게 말해서, 여러분이 디버깅 옵션을 켜놓은 커널 해커가 아니라면, 이
	함수들은 매우 가볍고, 일부 경우에는 완전히 오버헤드가 없습니다.
	그리고, 포틀랜드 지역 가구 판매사에서 말하는 걸 인용하자면, ``free is a
	\co{very} good price''.
	\iffalse

	Appearances can be deceiving.
	The \co{preempt_disable()}, \co{preempt_enable()},
	\co{local_bh_disable()}, and \co{local_bh_enable()} each
	do a single non-atomic manipulation of local data.
	Even that assumes \co{CONFIG_PREEMPT}, otherwise,
	the \co{preempt_disable()} and \co{preempt_enable()}
	functions emit no code, not even compiler directives.
	The \co{__acquire()} and \co{__release()} functions
	emit no code (not even compiler directives), but are instead
	used by the \co{sparse} semantic-parsing bug-finding program.
	Finally, \co{rcu_read_acquire()} and \co{rcu_read_release()}
	emit no code (not even compiler directives) unless the
	``lockdep'' lock-order debugging facility is enabled, in
	which case they can indeed be somewhat expensive.

	In short, unless you are a kernel hacker who has enabled
	debugging options, these functions are extremely cheap,
	and in some cases, absolutely free of overhead.
	And, in the words of a Portland-area furniture retailer,
	``free is a \emph{very} good price''.
	\fi
} \QuickQuizEnd

\subsubsection{\tt call\_rcu()}
\label{app:rcuimpl:rcutreewt:call-rcu}

\begin{figure}[tbp]
{ \scriptsize
\begin{verbatim}
  1 static void
  2 __call_rcu(struct rcu_head *head,
  3            void (*func)(struct rcu_head *rcu),
  4            struct rcu_state *rsp)
  5 {
  6   unsigned long flags;
  7   struct rcu_data *rdp;
  8
  9   head->func = func;
 10   head->next = NULL;
 11   smp_mb();
 12   local_irq_save(flags);
 13   rdp = rsp->rda[smp_processor_id()];
 14   rcu_process_gp_end(rsp, rdp);
 15   check_for_new_grace_period(rsp, rdp);
 16   *rdp->nxttail[RCU_NEXT_TAIL] = head;
 17   rdp->nxttail[RCU_NEXT_TAIL] = &head->next;
 18   if (ACCESS_ONCE(rsp->completed) ==
 19       ACCESS_ONCE(rsp->gpnum)) {
 20     unsigned long nestflag;
 21     struct rcu_node *rnp_root = rcu_get_root(rsp);
 22
 23     spin_lock_irqsave(&rnp_root->lock, nestflag);
 24     rcu_start_gp(rsp, nestflag);
 25   }
 26   if (unlikely(++rdp->qlen > qhimark)) {
 27     rdp->blimit = LONG_MAX;
 28     force_quiescent_state(rsp, 0);
 29   } else if ((long)(ACCESS_ONCE(rsp->jiffies_force_qs) -
 30                     jiffies) < 0 ||
 31              (rdp->n_rcu_pending_force_qs -
 32               rdp->n_rcu_pending) < 0)
 33     force_quiescent_state(rsp, 1);
 34   local_irq_restore(flags);
 35 }
 36
 37 void call_rcu(struct rcu_head *head,
 38               void (*func)(struct rcu_head *rcu))
 39 {
 40   __call_rcu(head, func, &rcu_state);
 41 }
 42
 43 void call_rcu_bh(struct rcu_head *head,
 44                  void (*func)(struct rcu_head *rcu))
 45 {
 46   __call_rcu(head, func, &rcu_bh_state);
 47 }
\end{verbatim}
}
\caption{{\tt call\_rcu()} Code}
\label{fig:app:rcuimpl:rcutreewt:Code for rcutree call-rcu}
\end{figure}

Figure~\ref{fig:app:rcuimpl:rcutreewt:Code for rcutree call-rcu}
는 \co{__call_rcu()}, \co{call_rcu()}, 그리고
\co{call_rcu_bh()} 의 코드를 보입니다.
\co{call_rcu()} 와 \co{call_rcu_bh()} 는 \co{__call_rcu()} 의 wrapper 일 뿐임을
알아주시기 바라며, 따라서 여기서는 더이상 다루지 않습니다.

관심을 \co{__call_rcu()} 로 돌려보면, line~9-10 은 명시된 \co{rcu_head} 를
초기화하고, line~11 은 \co{__call_rcu()} 가 호출되기 전에 수행된 RCU 로
보호되는 데이터 구조로의 업데이트가 콜백 등록 전에 된것으로 보여질 것을
보장합니다.
Line~12 와 34 는 인터럽트 핸들러로부터의 \co{__call_rcu()} 호출에 의한 파괴적
간섭을 막기 위해 인터럽트를 비활성화하고 재활성화 합니다.
Line~13 은 현재 CPU 의 \co{rcu_data} 구조체로의 레퍼런스를 얻어오고, line~14 는
현재 grace period 가 끝나지 않았다면 콜백들을 진행시키기 위해
\co{rcu_process_gp_end()} 를 수행하고, line~15 는 새로운 grace period 가
시작되면 상태를 기록하기 위해 \co{check_for_new_grace_period()} 를 호출합니다.
\iffalse

Figure~\ref{fig:app:rcuimpl:rcutreewt:Code for rcutree call-rcu}
shows the code for \co{__call_rcu()}, \co{call_rcu()}, and
\co{call_rcu_bh()}.
Note that \co{call_rcu()} and \co{call_rcu_bh()} are simple wrappers
for \co{__call_rcu()}, and thus will not be considered further here.

Turning attention to \co{__call_rcu()}, lines~9-10 initialize the
specified \co{rcu_head}, and line~11 ensures that updates to
RCU-protected data structures carried out prior to invoking
\co{__call_rcu()} are seen prior to callback registry.
Lines~12 and 34 disable and re-enable interrupts to prevent destructive
interference by any calls to \co{__call_rcu()} from an interrupt
handler.
Line 13 obtains a reference to the current CPU's \co{rcu_data}
structure, line~14 invokes \co{rcu_process_gp_end()} in order
to advance callbacks if the current grace period has now ended,
while line~15 invokes \co{check_for_new_grace_period()} to
record state if a new grace period has started.
\fi

\QuickQuiz{}
	Figure~\ref{fig:app:rcuimpl:rcutreewt:Code for rcutree call-rcu}
	의 line~13 에서 현재 CPU 의 \co{rcu_data} 구조체로의 레퍼런스를
	가져오는데에 간단히 \co{__get_cpu_var()} 를 사용하는게 어떤가요?
	\iffalse

	Why not simply use \co{__get_cpu_var()} to pick up a
	reference to the
	current CPU's \co{rcu_data} structure on line~13 in
	Figure~\ref{fig:app:rcuimpl:rcutreewt:Code for rcutree call-rcu}?
	\fi
\QuickQuizAnswer{
	우리는 \co{call_rcu()} 에서 호출되었거나 (이 경우라면 우린
	\co{__get_cpu_var(rcu_data)} 를 사용해야 할겁니다) \co{call_rcu_bh()}
	에서 호출되었을 수 있습니다 (이 경우라면
	\co{__get_cpu_var(rcu_bh_data)} 를 사용해야 합니다).
	우리가 넘긴 \co{rcu_state} 구조체가 뭔가에 관계없이 \co{->rda[]} 배열을
	사용하는 것은 어떤 API 의 \co{__call_rcu()} 가 호출되었는가에 관계없이
	올바르게 동작합니다 (Lai Jiangshan 에 의해
	제안되었습니다~\cite{LaiJiangshan2008NewClassicAlgorithm}).
	\iffalse

	Because we might be called either from \co{call_rcu()}
	(in which case we would need \co{__get_cpu_var(rcu_data)})
	or from \co{call_rcu_bh()} (in which case we would need
	\co{__get_cpu_var(rcu_bh_data)}).
	Using the \co{->rda[]} array of whichever
	\co{rcu_state} structure we were passed works correctly
	regardless of which API \co{__call_rcu()} was invoked from
	(suggested by Lai Jiangshan~\cite{LaiJiangshan2008NewClassicAlgorithm}).
	\fi
} \QuickQuizEnd

Line~16 과 17 은 새로운 콜백을 집어넣습니다.
Line~18 과 19 는 진행중인 grace period 가 있는지 체크해보며, 그렇지 않다면
line~23 에서 루트 \co{rcu_node} 구조체의 락을 얻어오고 line~24 에서 새로운
grace period 를 시작하기 위해 (그리고 얻어온 락을 내려놓기 위해)
\co{rcu_start_gp()} 를 실행합니다.

Line~26 은 이 CPU 를 너무 많은 RCU 콜백들이 기다리고 있지 않은지 확인해보고,
만약 그렇다면 line~27 에서 콜백들이 수행되는 비율을 증가시키기 위해
\co{->blimit} 을 증가시키며, line~28 에서는 제공된 CPU 들이 quiescent state 를
지나도록 시도하기 위해 급히 \co{force_quiescent_state()} 를 호출합니다.
그렇지 않다면, line~29-32 에서 grace period 가 시작된 이래로 (또는 지난번의
\co{force_quiescent_state()} 호출 이래로) 너무 오래되었는지 확인하고, 만약
그렇다면 line~33 에서 다시한번 제공된 CPU 들이 quiescent state 를 지나도록
시도하기 위해 \co{force_quiescent_state()} 를 급하지 않게 호출합니다.
\iffalse

Lines~16 and 17 enqueue the new callback.
Lines 18 and 19 check to see there is a grace period in progress,
and, if not, line~23 acquires the root \co{rcu_node} structure's
lock and line~24 invokes \co{rcu_start_gp()} to start a new grace
period (and also to release the lock).

Line~26 checks to see if too many RCU callbacks are waiting on
this CPU, and, if so, line~27 increases \co{->blimit} in order
to increase the rate at which callbacks are processed, while
line~28 invokes \co{force_quiescent_state()} urgently in order to
try to convince holdout CPUs to pass through quiescent states.
Otherwise, lines~29-32 check to see if it has been too long since
the grace period started (or since the last call to
\co{force_quiescent_state()}, as the case may be), and, if so,
line~33 invokes \co{force_quiescent_state()} non-urgently, again
to convince holdout CPUs to pass through quiescent states.
\fi

\subsubsection{\tt rcu\_check\_callbacks()}
\label{app:rcuimpl:rcutreewt:rcu-check-callbacks}

\begin{figure}[tbp]
{ \scriptsize
\begin{verbatim}
  1 static int __rcu_pending(struct rcu_state *rsp,
  2                          struct rcu_data *rdp)
  3 {
  4   rdp->n_rcu_pending++;
  5
  6   check_cpu_stall(rsp, rdp);
  7   if (rdp->qs_pending)
  8     return 1;
  9   if (cpu_has_callbacks_ready_to_invoke(rdp))
 10     return 1;
 11   if (cpu_needs_another_gp(rsp, rdp))
 12     return 1;
 13   if (ACCESS_ONCE(rsp->completed) != rdp->completed)
 14     return 1;
 15   if (ACCESS_ONCE(rsp->gpnum) != rdp->gpnum)
 16     return 1;
 17   if (ACCESS_ONCE(rsp->completed) !=
 18       ACCESS_ONCE(rsp->gpnum) &&
 19       ((long)(ACCESS_ONCE(rsp->jiffies_force_qs) -
 20               jiffies) < 0 ||
 21        (rdp->n_rcu_pending_force_qs -
 22         rdp->n_rcu_pending) < 0))
 23     return 1;
 24   return 0;
 25 }
 26
 27 int rcu_pending(int cpu)
 28 {
 29   return __rcu_pending(&rcu_state,
 30                        &per_cpu(rcu_data, cpu)) ||
 31          __rcu_pending(&rcu_bh_state,
 32                        &per_cpu(rcu_bh_data, cpu));
 33 }
 34
 35 void rcu_check_callbacks(int cpu, int user)
 36 {
 37   if (user ||
 38       (idle_cpu(cpu) && !in_softirq() &&
 39        hardirq_count() <= (1 << HARDIRQ_SHIFT))) {
 40     rcu_qsctr_inc(cpu);
 41     rcu_bh_qsctr_inc(cpu);
 42   } else if (!in_softirq()) {
 43     rcu_bh_qsctr_inc(cpu);
 44   }
 45   raise_softirq(RCU_SOFTIRQ);
 46 }
\end{verbatim}
}
\caption{{\tt rcu\_check\_callbacks()} Code}
\label{fig:app:rcuimpl:rcutreewt:Code for rcutree rcu-check-callbacks}
\end{figure}

Figure~\ref{fig:app:rcuimpl:rcutreewt:Code for rcutree rcu-check-callbacks}
는 각 CPU 에서 jiffy 마다 한번씩 scheduling-clock 인터럽트 핸들러에서 호출되는
코드를 보입니다.
\co{rcu_pending()} 함수 (\co{__rcu_pending()} 의 wrapper 입니다) 가 실행되고,
이게 0이 아닌 값을 리턴하면, \co{rcu_check_callbacks()} 가 호출됩니다.
(\co{rcu_pending()} 을 \co{rcu_check_callbacks()} 안으로 합칠 생각도 있었음을
알아두시기 바랍니다.)
\iffalse

Figure~\ref{fig:app:rcuimpl:rcutreewt:Code for rcutree rcu-check-callbacks}
shows the code that is called from the scheduling-clock interrupt
handler once per jiffy from each CPU.
The \co{rcu_pending()} function (which is a wrapper for \co{__rcu_pending()})
is invoked, and if it returns non-zero, then \co{rcu_check_callbacks()}
is invoked.
(Note that there is some thought being given to merging \co{rcu_pending()}
into \co{rcu_check_callbacks()}.)
\fi

\co{__rcu_opending()} 의 시작과 함께, line~4 는 이 \co{rcu_pending()} 의 호출
횟수를 세는데, 언제 quiescent state 를 강제할지 결정하는데 사용하기
위해서입니다.
Line~6 은 \co{CONFIG_RCU_CPU_STALL_DETECTOR} 가 선택되어 있다면 커널 내에서
spinning 하고 있거나, 또는 하드웨어 문제를 겪고 있는 CPU 들을 보고하기 위해
\co{check_cpu_stall()} 을 호출합니다.
Line~7-23 은 여러 검새를 하고, RCU 가 현재 CPU 가 다른 일을 하도록 해야 한다면
0 이 아닌 값을 리턴합니다.
Line~7 은 현재 CPU 가 현재의 grace period 를 위해 quiescent state 를 빚지고
있는지 검사하고,
line~9 는 현재 CPU 가 끝난 grace period 를 위한, 따라서 호출될 준비가 된 콜백을
가지고 있는지 검사하며,
line~11 은 현재 CPU 가 다른 RCU grace period 가 지나가길 기다려야 하는 콜백을
가지고 있는지 보기 위해 \co{cpu_needs_another_gp()} 를 호출하며,
line~13 은 현재 grace period 가 끝났는지 체크하고,
line~15 는 새로운 grace period 가 시작되었는지 검사하며,
마지막으로 line~17-22 에서는 제공된 CPU 들이 quiescent state 를 지나도록
강제해야 할 시간인지 체크합니다.
이 마지막 검사는 다음과 같이 나뉘어집니다: (1) line~17-18 은 진행중인 grace
period 가 있는지 체크하고, 만약 그렇다면 line~19-22 에서
\co{force_quiescent_state()} 가 호출되기에 충분할 정도로 jiffy 들 (line~19-20)
또는 \co{rcu_pending()} 호출 (line~21-22) 이 지나갔는지 체크합니다.
이 검사들 중 어느것도 해당하지 않는다면, line~24 에서 0을 리턴하여
\co{rcu_check_callbacks()} 가 호출될 필요가 없음을 알립니다.

Line~27-33 은 단순히 \co{__rcu_pending()} 을 두번, 한번은 ``rcu'' 를 위해
그다음은 ``rcu\_bh'' 를 위해 호출하는 \co{rcu_pending()} 을 보입니다.
\iffalse

Starting with \co{__rcu_pending()}, line~4 counts this call to
\co{rcu_pending()} for use in deciding when to force quiescent states.
Line~6 invokes \co{check_cpu_stall()} in order to report on CPUs
that are spinning in the kernel, or perhaps that have hardware problems,
if \co{CONFIG_RCU_CPU_STALL_DETECTOR} is selected.
Lines~7-23 perform a series of checks, returning non-zero if RCU
needs the current CPU to do something.
Line~7 checks to see if the current CPU owes RCU a quiescent state for the
current grace period,
line~9 invokes \co{cpu_has_callbacks_ready_to_invoke()} to see if
the current CPU has callbacks whose grace period has ended, thus being
ready to invoke,
line~11 invokes \co{cpu_needs_another_gp()} to see if the current
CPU has callbacks that need another RCU grace period to elapse,
line~13 checks to see if the current grace period has ended,
line~15 checks to see if a new grace period has started,
and, finally, lines~17-22 check to see if it is time to attempt
to force holdout CPUs to pass through a quiescent state.
This latter check breaks down as follows: (1) lines~17-18 check to see
if there is a grace period in progress, and, if so, lines~19-22
check to see if sufficient jiffies (lines~19-20) or calls to
\co{rcu_pending()} (lines~21-22) have elapsed that
\co{force_quiescent_state()} should be invoked.
If none of the checks in the series triggers, then line~24 returns
zero, indicating that \co{rcu_check_callbacks()} need not be invoked.

Lines~27-33 show \co{rcu_pending()}, which simply invokes
\co{__rcu_pending()} twice, once for ``rcu'' and again for
``rcu\_bh''.
\fi

\QuickQuiz{}
	Figure~\ref{fig:app:rcuimpl:rcutreewt:Code for rcutree rcu-check-callbacks}
	의 line~29-32 에서 \co{rcu_pending()} 이 항상 두번 호출된다는 점을
	생각해보면 이 두개의 구조체들에 대한 검사들을 하나로 합칠 방법이 있어야
	하는거 아닌가요?
	\iffalse

	Given that \co{rcu_pending()} is always called twice
	on lines~29-32 of
	Figure~\ref{fig:app:rcuimpl:rcutreewt:Code for rcutree rcu-check-callbacks},
	shouldn't there be some way to combine the checks of the
	two structures?
	\fi
\QuickQuizAnswer{
	미안해요, 이건 훼이크 질문이었어요.
	C 언어의 단락 boolean 표현에 대한 평가 규칙은 \co{rcu_state} 에 대한
	\co{__rcupending()} 호출이 0을 리턴했을 때에만 \co{rcu_bh_state} 를
	위한 두번째 호출을 수행할 것을 보장합니다.

	이런 순서로 두개의 함수 호출이 이루어지는 이유는 ``rcu\_bh'' 보다는
	``rcu'' 가 더 자주 사용되기 때문으로, 따라서 첫번째 호출이 0이 아닌
	값을 리턴할 확률이 두번째 것보다 높습니다.
	\iffalse

	Sorry, but this was a trick question.
	The C language's short-circuit boolean expression evaluation
	means that \co{__rcu_pending()} is invoked on
	\co{rcu_bh_state} only if the prior invocation on
	\co{rcu_state} returns zero.

	The reason the two calls are in this order is that
	``rcu'' is used more heavily than is ``rcu\_bh'', so
	the first call is more likely to return non-zero than
	is the second.
	\fi
} \QuickQuizEnd

Line~35-48 은 \co{rcu_check_callbacks()} 함수를 보이는데, 이 함수는
scheduling-clock 인터럽트가 연장된 quiescent state 를 인터럽트 했는지 검사하고,
만약 그렇다면 RCU 의 softirq 처리를 시작 (\co{rcu_process_callbacks()} 합니다.
Line~37-41 은 ``rcu'' 를 위한 이 검사를 수행하고 line~42-43 은 ``rcu\_bh'' 를
위한 이 검사를 수행합니다.
\iffalse

Lines~35-48 show \co{rcu_check_callbacks()}, which checks to see
if the scheduling-clock interrupt interrupted an extended quiescent
state, and then initiates RCU's softirq processing
(\co{rcu_process_callbacks()}).
Lines~37-41 perform this check for ``rcu'', while lines~42-43
perform the check for ``rcu\_bh''.
\fi

Line~37-39 는 이 scheduling clock 인터럽트가 사용자 모드 수행으로부터
들어왔는지 (line~37) 또는 idle 루프로부터 곧바로 들어왔는지 (line~38 의
\co{idle_cpu()} 수행) 를 인터럽트의 단계를 건드리지 않고 e들어왔는지 (line~38
의 뒷부분과 line~39) 검사합니다.
이 검사에 걸린다면, 즉 scheduling clock 인터럽트가 연장된 quiescent state
로부터 왔다면, ``rcu'' 를 위한 모든 quiescent state 는 또한 ``rcu\_bh'' 를 위한
quiescent state 이기도 하므로, line~40 과 41 은 이 quiescent state 를 두 종류의
RCU 모두를 위한 것으로 보고합니다.

비슷하게 ``rcu\_bh'' 를 위해서도 line~42 에서 이 scheduling-clock 인터럽트가
softirq 가 활성화된 코드 영역에서 온 것인지 검사하고 만약 그렇다면 line~43 에서
이 quiescent state 를 ``rcu\_bh'' 만을 위한 것으로 보고합니다.
\iffalse

Lines~37-39 check to see if the scheduling clock interrupt came
from user-mode execution (line~37) or directly from the idle
loop (line~38's \co{idle_cpu()} invocation) with no intervening
levels of interrupt (the remainder of line~38 and all of line~39).
If this check succeeds, so that the scheduling clock interrupt
did come from an extended quiescent state, then
because any quiescent state for ``rcu'' is also a quiescent state
for ``rcu\_bh'', lines~40 and 41 report the quiescent state for
both flavors of RCU.

Similarly for ``rcu\_bh'', line~42 checks to see if the scheduling-clock
interrupt came from a region of code with softirqs enabled, and, if so
line~43 reports the quiescent state for ``rcu\_bh'' only.
\fi

\QuickQuiz{}
	Figure~\ref{fig:app:rcuimpl:rcutreewt:Code for rcutree rcu-check-callbacks}
	의 line~42 는 \co{in_hardirq()} 검사도 해야하지 않습니까?
	\iffalse

	Shouldn't line~42 of
	Figure~\ref{fig:app:rcuimpl:rcutreewt:Code for rcutree rcu-check-callbacks}
	also check for \co{in_hardirq()}?
	\fi
\QuickQuizAnswer{
	아니요.
	\co{rcu_read_lock_bh()} 기능은 softirq 를 비활성화 하지, hardirq 를
	비활성화 하지는 않습니다.
	\co{call_rcu_bh()} 는 앞서 존재한 ``rcu\_bh'' read-side 크리티컬 섹션이
	완료되기만 기다리면 되므로, 우린 \co{in_softirq()} 만을 검사합니다.
	\iffalse

	No.
	The \co{rcu_read_lock_bh()} primitive disables
	softirq, not hardirq.
	Because \co{call_rcu_bh()} need only wait for pre-existing
	``rcu\_bh'' read-side critical sections to complete,
	we need only check \co{in_softirq()}.
	\fi
} \QuickQuizEnd

어느 경우든, line~45 는 RCU softirq 를 수행하는데, 이는 이 CPU 에서 (인터럽트가
현재의 scheduler-clock 인터럽트 후에 재활성화 되거나 하는) 미래에
\co{rcu_process_callbacks()} 가 호출되게 할겁니다.
\iffalse

In either case, line~45 invokes an RCU softirq, which will result in
\co{rcu_process_callbacks()} being called on this CPU at some future
time (like when interrupts are re-enabled after exiting the
scheduler-clock interrupt).
\fi

\subsubsection{\tt rcu\_process\_callbacks()}
\label{app:rcuimpl:rcutreewt:rcu-process-callbacks}

\begin{figure}[tbp]
{ \scriptsize
\begin{verbatim}
  1 static void
  2 __rcu_process_callbacks(struct rcu_state *rsp,
  3                         struct rcu_data *rdp)
  4 {
  5   unsigned long flags;
  6
  7   if ((long)(ACCESS_ONCE(rsp->jiffies_force_qs) -
  8              jiffies) < 0 ||
  9       (rdp->n_rcu_pending_force_qs -
 10        rdp->n_rcu_pending) < 0)
 11     force_quiescent_state(rsp, 1);
 12   rcu_process_gp_end(rsp, rdp);
 13   rcu_check_quiescent_state(rsp, rdp);
 14   if (cpu_needs_another_gp(rsp, rdp)) {
 15     spin_lock_irqsave(&rcu_get_root(rsp)->lock, flags);
 16     rcu_start_gp(rsp, flags);
 17   }
 18   rcu_do_batch(rdp);
 19 }
 20
 21 static void
 22 rcu_process_callbacks(struct softirq_action *unused)
 23 {
 24   smp_mb();
 25   __rcu_process_callbacks(&rcu_state,
 26                           &__get_cpu_var(rcu_data));
 27   __rcu_process_callbacks(&rcu_bh_state,
 28                           &__get_cpu_var(rcu_bh_data));
 29   smp_mb();
 30 }
\end{verbatim}
}
\caption{{\tt rcu\_process\_callbacks()} Code}
\label{fig:app:rcuimpl:rcutreewt:Code for rcutree rcu-process-callbacks}
\end{figure}

Figure~\ref{fig:app:rcuimpl:rcutreewt:Code for rcutree rcu-process-callbacks}
는 \co{__rcu_process_callbacks()} 의 wrapper 인 \co{rcu_process_callbacks()} 의
코드를 보입니다.
이 함수들은 예를 들면 RCU 코어는 이 CPU 가 뭔가 할 필요가 있다고 생각할 이유가
있다면 일반적으로 행해지는,
Figure~\ref{fig:app:rcuimpl:rcutreewt:Code for rcutree rcu-check-callbacks} 의
line~47 과 같이 \co{raise_softirq(RCU_SOFTIRQ)} 호출의 결과로 실행됩니다.

Line~7-10 은 현재의 grace period 가 시작된지 한참이 지났는지 체크하고, 만약
그렇다면 line~11 에서 제공된 CPU 들이 이 grace period 를 위해 quiescent state
를 지나가도록 설득하기 위해 \co{force_quiescent_state()} 를 호출합니다.
\iffalse

Figure~\ref{fig:app:rcuimpl:rcutreewt:Code for rcutree rcu-process-callbacks}
shows the code for \co{rcu_process_callbacks()}, which is a wrapper around
\co{__rcu_process_callbacks()}.
These functions are invoked as a result of a call to
\co{raise_softirq(RCU_SOFTIRQ)}, for example, line~47 of
Figure~\ref{fig:app:rcuimpl:rcutreewt:Code for rcutree rcu-check-callbacks},
which is normally done if there is reason to believe that the RCU core
needs this CPU to do something.

Lines~7-10 check to see if it has been awhile since the current grace
period started, and, if so, line~11 invokes \co{force_quiescent_state()}
in order to try to convince holdout CPUs to pass through a quiescent
state for this grace period.
\fi

\QuickQuiz{}
	하지만
	Figure~\ref{fig:app:rcuimpl:rcutreewt:Code for rcutree rcu-process-callbacks}
	의 \co{__rcu_process_callbacks} 에서 grace period 가 실제로
	진행중인지도 체크해 봐야 하지 않을까요?
	\iffalse

	But don't we also need to check that a grace period is
	actually in progress in \co{__rcu_process_callbacks} in
	Figure~\ref{fig:app:rcuimpl:rcutreewt:Code for rcutree rcu-process-callbacks}?
	\fi
\QuickQuizAnswer{
	실제로 우린 그렇게 합니다!
	그리고 \co{force_quiescent_state()} 가 하는 첫번째 일이 바로 그
	검사입니다.
	\iffalse

	Indeed we do!
	And the first thing that \co{force_quiescent_state()} does
	is to perform exactly that check.
	\fi
} \QuickQuizEnd

어떤 경우든, line~12 는 \co{rcu_process_gp_end()} 를 호출하는데, 이 함수는 어떤
다른 CPU 가 이 CPU 가 신경쓰고 있던 grace period 를 끝내버리진 않았는지
검사하고, 만약 그렇다면 해당 grace period 의 종료를 알아채고 그에 맞춰 이 CPU
의 RCU 콜백들을 진행시킵니다.
Line~13 은 \co{rcu_check_quiescent_state()} 를 호출하는데, 이 함수는 어떤 다른
CPU 가 새로운 grace period 를 시작했는지, 또한 현재의 CPU 가 현재의 grace
period 를 위한 quiescent state 를 지났는지 검사하고, 만약 그렇다면 상태를
올바르게 업데이트 합니다.
Line~14 는 진행중인 grace period 가 존재치 않는지 검사하고 현재 CPU 가 다른
grace period 를 필요로 하는 콜백들을 가지고 있지 않은지 검사합니다.
만약 그렇다면, line~15 는 루트 \co{rcu_node} 구조체의 락을 획득하고, line~17
에서 새로운 grace period ㄹ르 시작하는 \co{rcu_start_gp()} 를 호출합니다
(그리고 루트 \co{rcu_node} 구조체의 락을 해제합니다).
어떤 경우든, line~18 은 이 CPU 의 콜백들 가운데 grace period 가 완료된 것들을
모두 실행합니다.
\iffalse

In any case, line~12 invokes \co{rcu_process_gp_end()}, which checks to
see if some other CPU ended the
last grace period that this CPU was aware of, and, if so, notes the
end of the grace period and advances this CPU's RCU callbacks
accordingly.
Line~13 invokes \co{rcu_check_quiescent_state()}, which checks to
see if some other CPU has started a new grace period, and also whether
the current CPU has passed through a quiescent state for the current
grace period, updating state appropriately if so.
Line~14 checks to see if there is no grace period in progress and whether
the current CPU has callbacks that need another grace period.
If so, line~15 acquires the root \co{rcu_node} structure's lock,
and line~17 invokes \co{rcu_start_gp()}, which starts a new grace
period (and also releases the root \co{rcu_node} structure's lock).
In either case, line~18 invokes \co{rcu_do_batch()}, which
invokes any of this CPU's callbacks whose grace period has completed.
\fi

\QuickQuiz{}
	Figure~\ref{fig:app:rcuimpl:rcutreewt:Code for rcutree rcu-process-callbacks}
	에서 두개의 CPU 들이 새로운 grace period 를 동시적으로 시작하려
	시도하면 어떻게 되나요?
	\iffalse

	What happens if two CPUs attempt to start a new grace
	period concurrently in
	Figure~\ref{fig:app:rcuimpl:rcutreewt:Code for rcutree rcu-process-callbacks}?
	\fi
\QuickQuizAnswer{
	해당 CPU 들 가운데 하나는 루트 \co{rcu_node} 구조체의 락을 획득할
	것이고, 해당 CPU 가 grace period 를 시작할 겁니다.
	다른 CPU 는 이후에 이 락을 획득하고 \co{rcu_start_gp()} 를 호출할텐데,
	이 함수는 grace period 가 이미 진행 중임을 보게 될 것이고, 곧바로 락을
	해제하고 리턴할 겁니다.
	\iffalse

	One of the CPUs will be the first to acquire the root
	\co{rcu_node} structure's lock, and that CPU will start
	the grace period.
	The other CPU will then acquire the lock and invoke
	\co{rcu_start_gp()}, which, seeing that a grace period
	is already in progress, will immediately release the
	lock and return.
	\fi
} \QuickQuizEnd

Line~21-30 은 \co{rcu_process_callbacks()} 로, \co{__rcu_process_callbacks()}
의 wrapper 입니다.
Line~24 는 모든 앞의 RCU read-side 크리티컬 섹션들이 뒤따르는 RCU 처리에 앞서
완료된 것으로 보일 것을 보장하기 위해 메모리 배리어를 실행합니다.
Line~25-26 과 27-28 은 각각 ``rcu'' 와 ``rcu\_bh'' 를 위해
\co{__rcu_process_callbacks()} 를 호출하고, 마지막으로 line~29 에서
\co{__rcu_process_callbacks()} 에 의해 진행된 모든 RCU 처리 과정이 뒤따르는 RCU
read-side 크리티컬 섹션들 전에 보일 수 있도록 보장하기 위해 메모리 배리어를
수행합니다.
\iffalse

Lines~21-30 are \co{rcu_process_callbacks()}, which is again a
wrapper for \co{__rcu_process_callbacks()}.
Line~24 executes a memory barrier to ensure that any prior RCU
read-side critical sections are seen to have ended before any
subsequent RCU processing.
Lines~25-26 and 27-28 invoke \co{__rcu_process_callbacks()} for
``rcu'' and ``rcu\_bh'', respectively, and, finally,
line~29 executes a memory barrier to ensure that any RCU
processing carried out by \co{__rcu_process_callbacks()}
is seen prior to any subsequent RCU read-side critical sections.
\fi

\subsubsection{{\tt rcu\_needs\_cpu()} and {\tt rcu\_cpu\_notify()}}
\label{app:rcuimpl:rcutreewt:rcu-needs-cpu and rcu-cpu-notify}

\begin{figure}[tbp]
{ \scriptsize
\begin{verbatim}
  1 int rcu_needs_cpu(int cpu)
  2 {
  3   return per_cpu(rcu_data, cpu).nxtlist ||
  4          per_cpu(rcu_bh_data, cpu).nxtlist;
  5 }
  6
  7 static int __cpuinit
  8 rcu_cpu_notify(struct notifier_block *self,
  9                unsigned long action, void *hcpu)
 10 {
 11   long cpu = (long)hcpu;
 12
 13   switch (action) {
 14   case CPU_UP_PREPARE:
 15   case CPU_UP_PREPARE_FROZEN:
 16     rcu_online_cpu(cpu);
 17     break;
 18   case CPU_DEAD:
 19   case CPU_DEAD_FROZEN:
 20   case CPU_UP_CANCELED:
 21   case CPU_UP_CANCELED_FROZEN:
 22     rcu_offline_cpu(cpu);
 23     break;
 24   default:
 25     break;
 26   }
 27   return NOTIFY_OK;
 28 }
\end{verbatim}
}
\caption{{\tt rcu\_needs\_cpu()} and {\tt rcu\_cpu\_notify}  Code}
\label{fig:app:rcuimpl:rcutreewt:Code for rcu-needs-cpu and rcu-cpu-notify}
\end{figure}

Figure~\ref{fig:app:rcuimpl:rcutreewt:Code for rcu-needs-cpu and rcu-cpu-notify}
는 \co{rcu_needs_cpu()} 와 \co{rcu_cpu_notify()} 함수의 코드를 보이는데,
이것들은 리눅스 커널에서 각각 dynticks-idle 모드로의 전환을 체크하고 CPU
hotplug 를 처리하기 위해 호출됩니다.

Line~1-5 는 \co{rcu_needs_cpu()} 를 보이는데, 이 함수는 단순히 명시된 CPU 가
``rcu'' (line~3) 또는 ``rcu\_bh'' (line~4) 콜백을 가지고 있는지 체크합니다.

Line~7-28 은 \co{rcu_cpu_notify()} 를 보이는데, 이 함수는 일반적인 \co{switch}
구문을 사용하는 매우 일반적인 CPU-hotplug notifier 함수입니다.
Line~16 은 명시된 CPU 가 online 이 되려 한다면 \co{rcu_online_cpu()} 를
호출하고, line~22 는 명시된 CPU 가 offline 으로 가려 한다면
\co{rcu_offline_cpu()} 를 호출합니다.
CPU-hotplug 오퍼레이션은 atomic 하지 않으며, 여러 grace period 를 거칠 정도로
길어질 수 있음을 알아두어야 합니다.
따라서 RCU 는 들어오거나 나가는 과정 상의 CPU 들을 우아하게 처리해야만 합니다.
\iffalse

Figure~\ref{fig:app:rcuimpl:rcutreewt:Code for rcu-needs-cpu and rcu-cpu-notify}
shows the code for \co{rcu_needs_cpu()} and \co{rcu_cpu_notify()},
which are invoked by the Linux kernel to check on switching to
dynticks-idle mode and to handle CPU hotplug, respectively.

Lines~1-5 show \co{rcu_needs_cpu()}, which simply checks if the specified
CPU has either ``rcu'' (line~3) or ``rcu\_bh'' (line~4) callbacks.

Lines~7-28 show \co{rcu_cpu_notify()}, which is a very typical
CPU-hotplug notifier function with the typical \co{switch} statement.
Line~16 invokes \co{rcu_online_cpu()} if the specified CPU is going
to be coming online, and line~22 invokes \co{rcu_offline_cpu()} if
the specified CPU has gone to be going offline.
It is important to note that CPU-hotplug operations are not atomic,
but rather happen in stages that can extend for multiple grace periods.
RCU must therefore gracefully handle CPUs that are in the process
of coming or going.
\fi

\subsection{Initialization}
\label{app:rcuimpl:rcutreewt:Initialization}

\begin{figure*}[tb]
\centering
\resizebox{6in}{!}{\includegraphics{appendix/rcuimpl/RCUTreeInit}}
\caption{Initialized RCU Data Layout}
\label{fig:app:rcuimpl:rcutree:Initialized RCU Data Layout}
\end{figure*}

이 섹션은 주요한 데이터 구조체들을
Figure~\ref{fig:app:rcuimpl:rcutree:Initialized RCU Data Layout} 에 보인 것처럼
연결시키는 초기화 코드를 살펴보겠습니다.
노란 영역은 \co{rcu_state} 데이터 구조체의 필드들을 가리키는데, \co{->node}
배열, 핑크색으로 보여진 각각의 원소들을 포함하며,
Section~\ref{app:rcuimpl:rcutree:Hierarchical RCU Overview} 에서 사용된 규칙을
지킵니다.
파란 네모들은 각각 하나의 \co{rcu_data} 구조체를 가리키며, 파란 네모들의 그룹은
per-CPU \co{rcu_data} 구조체의 집합을 구성합니다.
\iffalse

This section walks through the initialization code, which links the
main data structures together as shown in
Figure~\ref{fig:app:rcuimpl:rcutree:Initialized RCU Data Layout}.
The yellow region represents fields in the \co{rcu_state} data
structure, including the \co{->node} array, individual elements
of which are shown in pink, matching the convention used in
Section~\ref{app:rcuimpl:rcutree:Hierarchical RCU Overview}.
The blue boxes each represent one \co{rcu_data} structure,
and the group of blue boxes makes up a set of per-CPU \co{rcu_data}
structures.
\fi

\co{->levelcnt[]} 배열은 \co{->level[0]} 가 그렇듯이 컴파일 타임에 초기화
됩니다만 나머지 값들과 포인터들은 뒤의 섹션들에서 설명할 함수들에 의해 내용이
채워집니다.
이 그림은 두단계의 계층을 보이고 있습니다만, 한개의 단계도 세개의 단계로 구성된
계층도 만들어질 수 있습니다.
\co{->levelspread[]} 배열의 각 원소는 이 계층상의 연관된 단계에서의 노드당
자식의 갯수를 갖습니다.
따라서, 이 그림상에서 root 노드는 두개의 자식을 갖고 leaf 단계의 노드들은 각각
세개의 자식을 갖습니다.
\co{levelcnt[]} 배열의 각 원소는 이 계층의 연관된 단계에 얼마나 많은 노드들이
존재하는지 알립니다: root 단계에선 1, leaf 단계에선 2, 그리고 \co{rcu_data}
단계에선 6입니다---그리고 그외의 모든 원소들은 사용되지 않으므로 0의 값을
갖습니다.
\co{->level[]} 배열의 각 원소는 \co{rcu_node} 계층상의 연관된 단계의 첫번째
노드를 가리키며, \co{->rda[]} 배열의 각 원소는 연관된 CPU 의 \co{rcu_data}
구조체를 가리킵니다.
각 \co{rcu_node} 구조체의 \co{->parent} 필드는 그 부모를 가리키는데, \co{NULL}
\co{->parent} 포인터를 갖는 root \co{rcu_node} 구조체는 예외입니다.
마지막으로, \co{rcu_data} 구조체의 \co{->mynode} 필드는 부모 \co{rcu_node}
구조체를 가리킵니다.
\iffalse

The \co{->levelcnt[]} array is initialized at compile time, as is
\co{->level[0]}, but the rest of the values and pointers are filled
in by the functions described in the following sections.
The figure shows a two-level hierarchy, but one-level and three-level
hierarchies are possible as well.
Each element of the \co{->levelspread[]} array gives the number of
children per node at the corresponding level of the hierarchy.
In the figure, therefore, the root node has two children and the
nodes at the leaf level each have three children.
Each element of the \co{levelcnt[]} array indicates how many nodes
there are on the corresponding level of the hierarchy: 1 at the root
level, 2 at the leaf level, and 6 at the \co{rcu_data} level---and any
extra elements are unused and left as zero.
Each element of the \co{->level[]} array references the first
node of the corresponding level of the \co{rcu_node} hierarchy,
and each element of the \co{->rda[]} array references the corresponding
CPU's \co{rcu_data} structure.
The \co{->parent} field of each \co{rcu_node} structure references
its parent, except for the root \co{rcu_node} structure, which
has a \co{NULL} \co{->parent} pointer.
Finally, the \co{->mynode} field of each \co{rcu_data} structure
references its parent \co{rcu_node} structure.
\fi

\QuickQuiz{}
	해당 코드는 \co{rcu_node} 계층의 root 부터 leaf 까지를 어떻게
	횡단하나요?
	\iffalse

	How does the code traverse a given path through
	the \co{rcu_node} hierarchy from root to leaves?
	\fi
\QuickQuizAnswer{
	해당 코드는 그런 횡단을 할 필요가 없고, 따라서 이를 처리하는 특수한
	일이 필요 없습니다.
	\iffalse

	It turns out that the code never needs to do such a traversal,
	so there is nothing special in place to handle this.
	\fi
} \QuickQuizEnd

다시 말하건대, 다음 섹션들은 이 구조체를 만드는 코드를 들여다 봅니다.
\iffalse

Again, the following sections walk through the code that builds this
structure.
\fi

\subsubsection{\tt rcu\_init\_levelspread()}
\label{app:rcuimpl:rcutreewt:rcu-init-levelspread}

\begin{figure}[tbp]
{ \scriptsize
\begin{verbatim}
  1 #ifdef CONFIG_RCU_FANOUT_EXACT
  2 static void __init
  3 rcu_init_levelspread(struct rcu_state *rsp)
  4 {
  5   int i;
  6
  7   for (i = NUM_RCU_LVLS - 1; i >= 0; i--)
  8     rsp->levelspread[i] = CONFIG_RCU_FANOUT;
  9 }
 10 #else
 11 static void __init
 12 rcu_init_levelspread(struct rcu_state *rsp)
 13 {
 14   int ccur;
 15   int cprv;
 16   int i;
 17
 18   cprv = NR_CPUS;
 19   for (i = NUM_RCU_LVLS - 1; i >= 0; i--) {
 20     ccur = rsp->levelcnt[i];
 21     rsp->levelspread[i] = (cprv + ccur - 1) / ccur;
 22     cprv = ccur;
 23   }
 24 }
 25 #endif
\end{verbatim}
}
\caption{{\tt rcu\_init\_levelspread()} Code}
\label{fig:app:rcuimpl:rcutreewt:Code for rcu-init-levelspread}
\end{figure}

Figure~\ref{fig:app:rcuimpl:rcutreewt:Code for rcu-init-levelspread}
는 \co{rcu_node} 계층 안의 fanout, 즉 부모당 자식의 갯수를 조절하는
\co{rcu_init_levelspread()} 함수의 코드를 보입니다.
이 함수의 두가지 버전이 있는데, 하나는 (\co{CONFIG_RCU_FANOUT} 으로 명시된)
정확한 fanout 을 강제하는 line~2-9 에 보여져 있고, 다른 하나는 자식 노드의
갯수를 명시된 fanout 에 간접적으로 기반해서 정하지만 이후 트리를 균형잡는
형태의 line~11-25 에 보여져 있습니다.
\co{CONFIG_RCU_FANOUT_EXACT} 커널 패러미터는 해당 커널 빌드에서 어떤 버전을
사용할지 선택합니다.

Exact-fanout 버전은 line~7 과 9 의 루프에 보인 것처럼 단순히 명시된
\co{rcu_state} 구조체의 \co{->levelspread} 배열의 모든 원소들을
\co{CONFIG_RCU_FANOUT} 커널 패러미터로 값 할당합니다.
\iffalse

Figure~\ref{fig:app:rcuimpl:rcutreewt:Code for rcu-init-levelspread}
shows the code for the \co{rcu_init_levelspread()} function, which controls
the fanout, or the number of children per parent,
in the \co{rcu_node} hierarchy.
There are two versions of this function, one shown on lines~2-9 that
enforces the exact fanout (specified by \co{CONFIG_RCU_FANOUT}),
and the other on lines~11-25 that determines the number of child nodes
based indirectly on the specified fanout, but then balances the tree.
The \co{CONFIG_RCU_FANOUT_EXACT} kernel parameter selects which version
to use for a given kernel build.

The exact-fanout version simply assigns all of the elements of the
specified \co{rcu_state} structure's \co{->levelspread} array to
the \co{CONFIG_RCU_FANOUT} kernel parameter, as shown by the loop
on lines~7 and 8.
\fi

Line~11-24 의 계층의 균형을 잡는 버전은 한쌍의 지역 변수 \co{ccur} 와 \co{cprv}
를 사용하는데 이것들은 현재와 앞 단계의 \co{rcu_node} 구조체의 갯수를 각각
추적합니다.
이 함수는 leaf 단계부터 계층의 위쪽으로 동작하는데, 따라서 \co{cprv} 의 값은
line~18 에서 \co{NR_CPUS} 로 초기화 되며, 이는 leaf 단계에 들어가는
\co{rcu_data} 구조체의 갯수에 연관됩니다.
Line~19-23 은 leaf 부터 루트까지 반복합니다.
이 루프 안에서, line~20 에서는 현재 단계의 \co{rcu_node} 구조체의 갯수를
\co{ccur} 에 넣습니다.
이어서 line~21 은 앞의 (아래의) 단계의 노드의 갯수의 현재 단계의 \co{rcu_node}
구조체의 갯수에 대한 비율을 반올림해서 명시된 \co{rcu_state} 구조체의
\co{->levelspread} 배열에 집어넣습니다.
이후 line~22 에서는 이 루프의 다음 반복을 위한 환경 구성을 합니다.

어떤 쪽이든 이 함수들에 대한 호출 후에는 \co{->levelspread} 배열은
\co{rcu_node} 계층의 각 단계별 자식 노드의 갯수를 담고 있게 됩니다.
\iffalse

The hierarchy-balancing version on lines~11-24
uses a pair of local variables \co{ccur} and \co{cprv} which track
the number of \co{rcu_node} structures on the current and previous
levels, respectively.
This function works from the leaf level up the hierarchy, so \co{cprv}
is initialized by line~18 to \co{NR_CPUS}, which corresponds
to the number of \co{rcu_data} structures that feed into the leaf level.
Lines~19-23 iterate from the leaf to the root.
Within this loop, line~20 picking up
the number of \co{rcu_node} structures for the current level into
\co{ccur}.
Line~21 then rounds up the ratio of the number of nodes on the previous
(lower) level (be they \co{rcu_node} or \co{rcu_data})
to the number of \co{rcu_node} structures on the current
level, placing the result in the specified \co{rcu_state} structure's
\co{->levelspread} array.
Line~22 then sets up for the next pass through the loop.

After a call to either function, the \co{->levelspread} array contains
the number of children for each level of the \co{rcu_node} hierarchy.
\fi

\subsubsection{\tt rcu\_init\_one()}
\label{app:rcuimpl:rcutreewt:rcu-init-one}

\begin{figure}[tbp]
{ \scriptsize
\begin{verbatim}
  1 static void __init rcu_init_one(struct rcu_state *rsp)
  2 {
  3   int cpustride = 1;
  4   int i;
  5   int j;
  6   struct rcu_node *rnp;
  7
  8   for (i = 1; i < NUM_RCU_LVLS; i++)
  9     rsp->level[i] = rsp->level[i - 1] +
 10                     rsp->levelcnt[i - 1];
 11   rcu_init_levelspread(rsp);
 12   for (i = NUM_RCU_LVLS - 1; i >= 0; i--) {
 13     cpustride *= rsp->levelspread[i];
 14     rnp = rsp->level[i];
 15     for (j = 0; j < rsp->levelcnt[i]; j++, rnp++) {
 16       spin_lock_init(&rnp->lock);
 17       rnp->qsmask = 0;
 18       rnp->qsmaskinit = 0;
 19       rnp->grplo = j * cpustride;
 20       rnp->grphi = (j + 1) * cpustride - 1;
 21       if (rnp->grphi >= NR_CPUS)
 22         rnp->grphi = NR_CPUS - 1;
 23       if (i == 0) {
 24         rnp->grpnum = 0;
 25         rnp->grpmask = 0;
 26         rnp->parent = NULL;
 27       } else {
 28         rnp->grpnum = j % rsp->levelspread[i - 1];
 29         rnp->grpmask = 1UL << rnp->grpnum;
 30         rnp->parent = rsp->level[i - 1] +
 31                 j / rsp->levelspread[i - 1];
 32       }
 33       rnp->level = i;
 34     }
 35   }
 36 }
\end{verbatim}
}
\caption{{\tt rcu\_init\_one()} Code}
\label{fig:app:rcuimpl:rcutreewt:Code for rcu-init-one}
\end{figure}

Figure~\ref{fig:app:rcuimpl:rcutreewt:Code for rcu-init-one}
는 명시된 \co{rcu_state} 구조체에 대해 부트 타임 초기화를 행하는
\co{rcu_init_one()} 코드를 보입니다.

Section~\ref{app:rcuimpl:rcutreewt:RCU Global State}
에서 이야기했듯 \co{rcu_state} 구조체의 \co{->levelcnt[]} 배열은 root 부터
시작하는 계층의 각 단계의 노드들의 수로 컴파일 시점에 초기화 되며, 이 배열의
하나의 추가적인 원소는 있을 수 있는 최대한의 CPU 의 갯수인 \co{NR_CPUS} 로
초기화 됩니다.
여기에 더해서, \co{->level[]} 배열의 첫번째 원소는 root \co{rcu_node}
구조체로의 레퍼런스로 초기화 되는데, 이는 결국 \co{rcu_state} 구조체의
\co{->node[]} 배열의 첫번째 원소입니다.
이 뱅려은 더 나아가서 breadth-first 순서로 위치됩니다.
이 모든 것을 명심하고, line~8-10 의 루프는 \co{->level[]} 배열의 나머지들을
\co{rcu_node} 구조체의 각 단계의 첫번째 \co{rcu_node} 구조체를 레퍼런스 하도록
초기화 합니다.
\iffalse

Figure~\ref{fig:app:rcuimpl:rcutreewt:Code for rcu-init-one}
shows the code for \co{rcu_init_one()}, which does boot-time initialization
for the specified
\co{rcu_state} structure.

Recall from
Section~\ref{app:rcuimpl:rcutreewt:RCU Global State}
that the \co{->levelcnt[]} array in the \co{rcu_state} structure
is compile-time initialized to the number of nodes at each level of
the hierarchy starting from the root,
with an additional element in the array initialized
to the maximum possible number of CPUs, \co{NR_CPUS}.
In addition, the first element of the \co{->level[]} array is compile-time
initialized to reference to the root \co{rcu_node} structure, which is
in turn
the first element of the \co{->node[]} array in the \co{rcu_state} structure.
This array is further laid out in breadth-first order.
Keeping all of this in mind, the loop at lines~8-10 initializes the rest
of the \co{->level[]} array to reference the first \co{rcu_node} structure
of each level of the \co{rcu_node} hierarchy.
\fi

이어서 line~11 은 \co{rcu_init_levelspread()} 를 호출하는데, 이 함수는
Section~\ref{app:rcuimpl:rcutreewt:rcu-init-levelspread}
에서 설명한대로 \co{->levelspread[]} 배열을 초기화 합니다.
이어서 예비의 배열들이 완전히 초기화되고, 따라서 line~15-35 의 루프가 돌아갈
준비가 되는데, 이 루프의 각 단계는 \co{rcu_node} 구조체의 각 단계를 leaf 부터
시작해서 초기화 해나갑니다.

Line~13 은 이 계층의 현재 단계를 위한 \co{rcu_node} 구조체 당 CPU 의 갯수를
계산하고, line~14 는 이 계층의 현재 단계의 첫번째 \co{rcu_node} 구조체로의
포인터를 얻어오는데, line~15-34 의, 하나의 \co{rcu_node} 구조체의 초기화를 매번
해나가는 루프를 위한 준비단계입니다.
\iffalse

Line~11 then invokes \co{rcu_init_levelspread()}, which fills in the
\co{->levelspread[]} array, as was described in
Section~\ref{app:rcuimpl:rcutreewt:rcu-init-levelspread}.
The auxiliary arrays are then fully initialized, and thus ready for
the loop from lines~15-35, each pass through which initializes
one level of the \co{rcu_node} hierarchy, starting from the leaves.

Line~13 computes the number of CPUs per \co{rcu_node} structure for
the current level of the hierarchy, and line~14 obtains a pointer
to the first \co{rcu_node} structure on the current level of the
hierarchy, in preparation for the loop from lines~15-34, each pass
through which initializes one \co{rcu_node} structure.
\fi

Line~16-18 은 \co{rcu_node} 구조체의 스핀락과 그 CPU mask 들을 초기화합니다.
\co{qsmaskinit} 필드는 나중에 부팅 시점 중에 CPU 들이 online 이 될 때 설정되고,
\co{qsmask} 필드는 첫번째 grace period 가 시작될 때 설정됩니다.
Line~19 는 \co{->grplo} 필드를 이 \co{rcu_node} 구조체의 첫번째 CPU 의 숫자로
설정하고 line~20 은 \co{->grphi} 를 이 \co{rcu_node} 구조체의 마지막 CPU 의
숫자로 설정합니다.
만약 이 계층의 주어진 단계의 마지막 \co{rcu_node} 구조체가 부분적으로만 차
있다면, line~21 과 22 는 그것의 \co{->grphi} 필드가 시스템의 마지막 존재할 수
있는 CPU 의 숫자가 되도록 만듭니다.
\iffalse

Lines~16-18 initialize the \co{rcu_node} structure's spinlock and
its CPU masks.
The \co{qsmaskinit} field will have bits set as CPUs come online
later in boot, and the \co{qsmask} field will have bits set
when the first grace period starts.
Line~19 sets the \co{->grplo} field to the number of the this
\co{rcu_node} structure's first CPU and line~20 sets the
\co{->grphi} to the number of this \co{rcu_node} structure's
last CPU.
If the last \co{rcu_node} structure on a given level of the
hierarchy is only partially full, lines~21 and 22 set its
\co{->grphi} field to the number of the last possible CPU in the system.
\fi

Line~24-26 은 root \co{rcu_node} 구조체의 \co{->grpnum}, \co{->grpmask}, 그리고
\co{->parent} 필드를 초기화 하는데, 이 구조체는 부모가 없으므로, 모두 zero 와
NULL 이 됩니다.
Line~28-31 은 이 계층의 나머지 \co{rcu_node} 구조체들의 같은 필드들을 초기화
합니다.
Line~28 은 \co{->grpnum} 필드를 같은 부모를 가진 집합 내에서의 이 \co{rcu_node}
구조체로의 인덱스로 계산하고, line~29 는 \co{->grpmask} 필드의 연관된 bit 의
값을 1로 설정합니다.
마지막으로, line~30-31 은 \co{->parent} 필드에 부모 노드로의 포인터를 넣습니다.
이 세개의 필드는 quiescent state 를 계층의 윗단계로 넘기는데에 사용됩니다.

마지막으로, line~33 은 이 계층의 단계를 \co{->level} 에 기록하는데, 이는 전체
계층을 횡단할 때 tracing 에 사용됩니다.
\iffalse

Lines~24-26 initialize the \co{->grpnum}, \co{->grpmask}, and
\co{->parent} fields for the root \co{rcu_node} structure, which
has no parent, hence the zeroes and NULL.
Lines~28-31 initialize these same fields for the rest of the
\co{rcu_node} structures in the hierarchy.
Line~28 computes the \co{->grpnum} field as the index of this
\co{rcu_node} structure within
the set having the same parent, and
line~29 sets the corresponding bit in the \co{->grpmask} field.
Finally, lines~30-31 places a pointer to the parent node into the
\co{->parent} field.
These three fields will used to propagate quiescent states up the
hierarchy.

Finally, line~33 records the hierarchy level in \co{->level},
which is used for tracing when traversing the full hierarchy.
\fi

\subsubsection{\tt \_\_rcu\_init()}
\label{app:rcuimpl:rcutreewt:rcu-init}

\begin{figure}[tbp]
{ \scriptsize
\begin{verbatim}
  1 #define RCU_DATA_PTR_INIT(rsp, rcu_data) \
  2 do { \
  3   rnp = (rsp)->level[NUM_RCU_LVLS - 1]; \
  4   j = 0; \
  5   for_each_possible_cpu(i) { \
  6     if (i > rnp[j].grphi) \
  7       j++; \
  8     per_cpu(rcu_data, i).mynode = &rnp[j]; \
  9     (rsp)->rda[i] = &per_cpu(rcu_data, i); \
 10   } \
 11 } while (0)
 12
 13 void __init __rcu_init(void)
 14 {
 15   int i;
 16   int j;
 17   struct rcu_node *rnp;
 18
 19   rcu_init_one(&rcu_state);
 20   RCU_DATA_PTR_INIT(&rcu_state, rcu_data);
 21   rcu_init_one(&rcu_bh_state);
 22   RCU_DATA_PTR_INIT(&rcu_bh_state, rcu_bh_data);
 23
 24   for_each_online_cpu(i)
 25     rcu_cpu_notify(&rcu_nb, CPU_UP_PREPARE,
 26                    (void *)(long)i);
 27   register_cpu_notifier(&rcu_nb);
 28 }
\end{verbatim}
}
\caption{{\tt \_\_rcu\_init()} Code}
\label{fig:app:rcuimpl:rcutreewt:Code for rcu-init}
\end{figure}

Figure~\ref{fig:app:rcuimpl:rcutreewt:Code for rcu-init}
는 \co{__rcu_init()} 함수와 그것의 helper macro 인 \co{RCU_DATA_PTR_INIT()} 를
보입니다.
\co{__rcu_init()} 함수는 부팅 초기에 스케쥴러가 초기화 되기 전에, 그리고 하나
이상의 CPU 가 동작하기 전에 호출됩니다.
\iffalse

Figure~\ref{fig:app:rcuimpl:rcutreewt:Code for rcu-init}
shows the \co{__rcu_init()} function and its \co{RCU_DATA_PTR_INIT()}
helper macro.
The \co{__rcu_init()} function is invoked during early boot,
before the scheduler has initialized, and before more than one
CPU is running.
\fi

\co{RCU_DATA_PTR_INIT()} 매크로는 \co{rcu_state} 구조체로의 포인터와
\co{rcu_data} per-CPU 변수 의 집합의 이름을 인자로 받습니다.
이 매크로는 per-CPU \co{rcu_data} 구조체들을 스캔하고, 각 \co{rcu_data}
구조체의 \co{->mynode} 포인터가 연관된 leaf \co{rcu_node} 구조체를 가리키도록
값을 할당합니다.
이 매크로는 또한 명시된 \co{rcu_state} 구조체의 \co{->rda[]} 배열을 그 원소들이
각각 연관된 \co{rcu_data} 구조체를 가리키도록 값을 채웁니다.
Line~3 은 첫번째 leaf \co{rcu_node} 구조체로의 포인터를 (이 매크로의 호출자에
의해 선언되어야만 하는) 지역 변수 \co{rnp} 에 넣고, line~4 는 연관된 leaf-node
숫자인 지역 변수 \co{j} 를 0으로 할당합니다.
Line~5-10 으로 구성되는 루프의 각 단계에서는 연관된 (\co{NR_CPUS} 로 명시된)
잠재적 CPU 들의 초기화를 수행합니다.
이 루프 안에서, line~6 는 우리가 현재 leaf \co{rcu_node} 구조체의 한계를 넘어
움직였는지 체크하고, 만약 그렇다면 line~7 에서 다음 구조체로 진행합니다.
그러고 나서는, 여전히 루프 안에서, line~8 은 현재 CPU 의 \co{rcu_data} 구조체의
\co{->mynode} 포인터가 현재 leaf \co{rcu_node} 구조체를 가리키도록 만들고,
line~9 는 현재 CPU 의 (\co{rcu_state} 구조체 안의) \co{->rda[]} 원소가 현재 CPU
의 \co{rcu_data} 구조체를 가리키도록 만듭니다.
\iffalse

The \co{RCU_DATA_PTR_INIT()} macro takes as arguments a pointer to
an \co{rcu_state} structure and the name of a set of \co{rcu_data}
per-CPU variables.
This macro scans the per-CPU \co{rcu_data}
structures, assigning the \co{->mynode} pointer of each \co{rcu_data}
structure to point to the corresponding leaf \co{rcu_node} structure.
It also fills out the specified \co{rcu_state} structure's
\co{->rda[]} array entries to each point to the corresponding
\co{rcu_data} structure.
Line~3 picks up a pointer to the first leaf \co{rcu_node} structure
in local variable \co{rnp} (which must be declared by the invoker of
this macro),
and line~4 sets local variable \co{j} to the corresponding leaf-node
number of zero.
Each pass through the loop spanning lines~5-10 performs initialization
for the corresponding potential CPU (as specified by \co{NR_CPUS}).
Within this loop, line~6 checks to see if we have moved beyond the
bounds of the current leaf \co{rcu_node} structure, and, if so,
line~7 advances to the next structure.
Then, still within the loop, line~8 sets the \co{->mynode} pointer
of the current CPU's \co{rcu_data} structure to reference the current
leaf \co{rcu_node} structure, and line~9 sets the current CPU's \co{->rda[]}
element (within the \co{rcu_state} structure) to reference the
current CPU's \co{rcu_data} structure.
\fi

\QuickQuiz{}
	C-preprocessor 매크로는 \emph{너무} 1990년대 스러워요!
	왜 요즘 시대에 걸맞게
	Figure~\ref{fig:app:rcuimpl:rcutreewt:Code for rcu-init} 의
	\co{RCU_DATA_PTR_INIT()} 을 함수로 바꾸지 않는거죠?
	\iffalse

	C-preprocessor macros are \emph{so} 1990s!
	Why not get with the times and convert \co{RCU_DATA_PTR_INIT()}
	in Figure~\ref{fig:app:rcuimpl:rcutreewt:Code for rcu-init}
	to be a function?
	\fi
\QuickQuizAnswer{
	왜냐하면, 특정 CPU 의 per-CPU 변수 인스턴스로의 레퍼런스를 함수에 넘길
	수는 있겠지만, 특정 per-CPU 변수의 인스턴스의 전체 집합으로의
	레퍼런스를 함수에 넘길 좋은 방법이 없기 때문입니다.
	물론 누군가는 포인터들의 배열을 만들고 그 배열로의 레퍼런스를 넘길 수도
	있겠습니다만, 그건 \co{RCU_DATA_PTR_INIT()} 매크로가 가장 먼저 하는 일
	중 하나입니다.
	\iffalse

	Because, although it is possible to pass a reference to
	a particular CPU's instance of a per-CPU variable to a function,
	there does not appear to be a good way pass a reference to
	the full set of instances of a given per-CPU variable to
	a function.
	One could of course build an array of pointers, then pass a
	reference to the array in, but that is part of what
	the \co{RCU_DATA_PTR_INIT()} macro is doing in the first place.
	\fi
} \QuickQuizEnd

\co{__rcu_init()} 함수는 먼저 line~19 에서 \co{rcu_state} 구조체를 가지고
\co{rcu_init_one()} 을 호출하고, 이어서 이 \co{rcu_state} 구조체와 per-CPU
변수들의 \co{rcu_data} 집합을 가지고 \co{RCU_DATA_PTR_INIT()} 을 호출합니다.
Line~21-22 에서는 이를 \co{rcu_bh_state} 와 \co{rcu_bh_data} 에 대해
반복합니다.
Line~24-26 에 걸쳐진 루프는 현재 online 인 각 CPU 에 대해 \co{rcu_cpu_notify()}
를 호출하고, line~27 은 CPU 가 online 이 될 때마다, RCU 에게 그 존재를 알리기
위해 \co{rcu_cpu_notify()} 가 호출되도록 알림을 설정합니다.
\iffalse

The \co{__rcu_init()} function first invokes \co{rcu_init_one()}
on the \co{rcu_state} structure on line~19, then invokes
\co{RCU_DATA_PTR_INIT()} on the \co{rcu_state} structure and
the \co{rcu_data} set of per-CPU variables.
It then repeats this for \co{rcu_bh_state} and \co{rcu_bh_data}
on lines~21-22.
The loop spanning lines~24-26 invokes \co{rcu_cpu_notify()} for
each CPU that is currently online (which should be only the boot
CPU), and line~27 registers a notifier so that \co{rcu_cpu_notify()}
will be invoked each time a CPU comes online, in order to inform
RCU of its presence.
\fi

\QuickQuiz{}
	Figure~\ref{fig:app:rcuimpl:rcutreewt:Code for rcu-init}
	의 line~25-26 에서 online CPU 가 마지막으로 알려졌을 때와 line~27 에서
	\co{register_cpu_notifier()} 가 호출된 시점 사이에서 어떤 CPU 가 online
	이 되면 어떤 일이 벌어지나요?
	\iffalse

	What happens if a CPU comes online between the time
	that the last online CPU is notified on lines~25-26 of
	Figure~\ref{fig:app:rcuimpl:rcutreewt:Code for rcu-init}
	and the time that \co{register_cpu_notifier()} is invoked
	on line~27?
	\fi
\QuickQuizAnswer{
	이 시점에서는 하나의 CPU 만이 online 이므로, 또다른 CPU 가 online 이
	되는 유일한 방법은 이 CPU 가 해당 CPU 를 online 으로 두는 것으로,
	성립되지 않는 이야기입니다.
	\iffalse

	Only one CPU is online at this point, so the only way another
	CPU can come online is if this CPU puts it online, which it
	is not doing.
	\fi
} \QuickQuizEnd

\co{rcu_cpu_notify()} 과 연관된 함수들은 뒤의
Section~\ref{app:rcuimpl:rcutreewt:CPU Hotplug}
에서 더 이야기 하겠습니다.
\iffalse

The \co{rcu_cpu_notify()} and related functions are discussed in
Section~\ref{app:rcuimpl:rcutreewt:CPU Hotplug}
below.
\fi

\subsection{CPU Hotplug}
\label{app:rcuimpl:rcutreewt:CPU Hotplug}

다음 섹션들에서 설명하는 CPU-hotplug 함수들은 어떤 CPU 들이 현재 존재하고 어떤
CPU 들은 존재하지 않는지를 RCU 가 추적할 수 있게 해주면서, 또한 각 CPU 가
online 이 될때 각 CPU 의 \co{rcu_data} 구조체를 초기화 합니다.
\iffalse

The CPU-hotplug functions described in the following sections
allow RCU to track which CPUs are and are not present, but also
complete initialization of each CPU's \co{rcu_data} structure
as that CPU comes online.
\fi

\subsubsection{\tt rcu\_init\_percpu\_data()}
\label{app:rcuimpl:rcutreewt:rcu-init-percpu-data}

\begin{figure}[tbp]
{ \scriptsize
\begin{verbatim}
  1 static void
  2 rcu_init_percpu_data(int cpu, struct rcu_state *rsp)
  3 {
  4   unsigned long flags;
  5   int i;
  6   long lastcomp;
  7   unsigned long mask;
  8   struct rcu_data *rdp = rsp->rda[cpu];
  9   struct rcu_node *rnp = rcu_get_root(rsp);
 10
 11   spin_lock_irqsave(&rnp->lock, flags);
 12   lastcomp = rsp->completed;
 13   rdp->completed = lastcomp;
 14   rdp->gpnum = lastcomp;
 15   rdp->passed_quiesc = 0;
 16   rdp->qs_pending = 1;
 17   rdp->beenonline = 1;
 18   rdp->passed_quiesc_completed = lastcomp - 1;
 19   rdp->grpmask = 1UL << (cpu - rdp->mynode->grplo);
 20   rdp->nxtlist = NULL;
 21   for (i = 0; i < RCU_NEXT_SIZE; i++)
 22     rdp->nxttail[i] = &rdp->nxtlist;
 23   rdp->qlen = 0;
 24   rdp->blimit = blimit;
 25 #ifdef CONFIG_NO_HZ
 26   rdp->dynticks = &per_cpu(rcu_dynticks, cpu);
 27 #endif /* #ifdef CONFIG_NO_HZ */
 28   rdp->cpu = cpu;
 29   spin_unlock(&rnp->lock);
 30   spin_lock(&rsp->onofflock);
 31   rnp = rdp->mynode;
 32   mask = rdp->grpmask;
 33   do {
 34     spin_lock(&rnp->lock);
 35     rnp->qsmaskinit |= mask;
 36     mask = rnp->grpmask;
 37     spin_unlock(&rnp->lock);
 38     rnp = rnp->parent;
 39   } while (rnp != NULL && !(rnp->qsmaskinit & mask));
 40   spin_unlock(&rsp->onofflock);
 41   cpu_quiet(cpu, rsp, rdp, lastcomp);
 42   local_irq_restore(flags);
 43 }
\end{verbatim}
}
\caption{{\tt rcu\_init\_percpu\_data()} Code}
\label{fig:app:rcuimpl:rcutreewt:Code for rcu-init-percpu-data}
\end{figure}

Figure~\ref{fig:app:rcuimpl:rcutreewt:Code for rcu-init-percpu-data}
는 \co{rcu_init_percpu_data()} 의 코드를 보이는데, 이 함수는 부팅때 또는 특정
CPU 가 online 이 될 때 특정 CPU 의 \co{rcu_data} 구조체를 초기화 합니다.
이 함수는 또한 이 CPU 가 다음 grace period 에 참여할 수 있도록 \co{rcu_node}
계층을 설정합니다.

Line~8 은 명시된 \co{rcu_state} 구조체를 기반으로 이 CPU 의 \co{rcu_data}
구조체로의 포인터를 가져오고, 이 포인터를 로컬 변수 \co{rdp} 에 넣습니다.
Line~9 는 명시된 \co{rcu_state} 구조체를 위한 루트 \co{rcu_node} 구조체로의
포인터를 가져와서 로컬 변수 \co{rnp} 에 넣습니다.
\iffalse

Figure~\ref{fig:app:rcuimpl:rcutreewt:Code for rcu-init-percpu-data}
shows the code for \co{rcu_init_percpu_data()}, which initializes
the specified CPU's \co{rcu_data} structure in response to booting
up or to that CPU coming online.
It also sets up the \co{rcu_node} hierarchy so that this CPU will
participate in future grace periods.

Line~8 gets a pointer to this CPU's \co{rcu_data} structure, based
on the specified \co{rcu_state} structure, and places this pointer
into the local variable \co{rdp}.
Line~9 gets a pointer to the root \co{rcu_node} structure for the
specified \co{rcu_state} structure, placing it in local variable
\co{rnp}.
\fi

Line~11-29 는 \co{rcu_data} 구조체의 필드들을 일관성 있는 값을 보장하기 위해
루트 \co{rcu_node} 구조체의 락으로 보호하면서 초기화 합니다.
Line~17 은 tracing 을 위해 중요한데, 많은 리눅스 배포판들이 \co{NR_CPUS} 를
매우 큰 수로 설정해서 \co{rcu_data} 구조체를 trace 할 때 지나치게 많은 출력을
초래할 수 있기 때문입니다.
이 문제를 해결하기 위해 \co{->beenonline} 필드가 사용되는데, 이 필드는 이미
online 이 된 CPU 에 \co{rcu_data} 구조체가 연관될 때 값을 1로 설정하고, 그 외의
모든 다른 \co{rcu_data} 구조체들을 위한 부분은 0으로 설정됩니다.
이는 tracing 코드가 불필요한 CPU 들을 쉽게 무시할 수 있도록 해줍니다.
\iffalse

Lines~11-29 initialize the fields of the \co{rcu_data} structure
under the protection of the root \co{rcu_node} structure's lock
in order to ensure consistent values.
Line~17 is important for tracing, due to the fact that many Linux
distributions set \co{NR_CPUS} to a very large number, which could
result in excessive output when tracing \co{rcu_data} structures.
The \co{->beenonline} field is used to solve this problem, as
it will be set to the value one on any \co{rcu_data} structure
corresponding to a CPU that has ever been online, and set to zero
for all other \co{rcu_data} structures.
This allows the tracing code to easily ignore irrelevant CPUs.
\fi

Line~30-40 은 online 이 되는 CPU 의 bit 을 \co{rcu_node} 계층으로 전파하는데,
루트 \co{rcu_node} 에 도달하거나 연관된 bit 이 이미 값 설정되어 있거나 할때까지
전파됩니다.
이 bit-setting 은 새로운 grace period 의 초기화를 배제하기 위해
\co{->onofflock} 의 보호 아래 이루어지고, 추가적으로, 각 \co{rcu_node} 구조체도
이 락의 보호 아래 초기화 됩니다.
이어서 line~41 은 RCU 에게 이 CPU 가 extended quiescent state 에 있었음을
알리는 신호를 주기 위해 \co{cpu_quiet()} 를 호출하고 마지막으로 line~42 에서
irq 들을 다시 활성화 시킵니다.
\iffalse

Lines~30-40 propagate the onlining CPU's bit up the \co{rcu_node}
hierarchy, proceeding until either the root \co{rcu_node} is
reached or until the corresponding bit is already set, whichever
comes first.
This bit-setting is done under the protection of \co{->onofflock}
in order to exclude initialization of a new grace period, and, in addition,
each \co{rcu_node} structure is initialized under the protection
of its lock.
Line~41 then invokes \co{cpu_quiet()} to signal RCU that this
CPU has been in an extended quiescent state, and finally, line~42
re-enables irqs.
\fi

\QuickQuiz{}
	Figure~\ref{fig:app:rcuimpl:rcutreewt:Code for rcu-init-percpu-data}
	의 line~41 에서는, 우리가 다양한 락들을 사용해 grace period 들을
	배제시켰음에도, 그리고 모든 앞서 존재한 grace period 들은 이 기존에
	offline 이 된 CPU 들을 기다리지 않을 것임에도, 왜 \co{cpu_quiet()} 를
	호출하는거죠?
	\iffalse

	Why call \co{cpu_quiet()} on line~41 of
	Figure~\ref{fig:app:rcuimpl:rcutreewt:Code for rcu-init-percpu-data},
	given that we are excluding grace periods with various
	locks, and given that any earlier grace periods would not have
	been waiting on this previously-offlined CPU?
	\fi
\QuickQuizAnswer{
	새로운 grace period 는 line~40 에서 \co{->onofflock} 이 해제된 직후에
	시작되었을 수 있습니다.
	\co{cpu_quiet()} 는 그런 grace period 를 진척시키는 도움을 줄겁니다.
	\iffalse

	A new grace period might have started just after the
	\co{->onofflock} was released on line~40.
	The \co{cpu_quiet()} will help expedite such a grace period.
	\fi
} \QuickQuizEnd

\co{rcu_init_percpu_data()} 가 부팅될 때만 호출되는게 아니라, 특정 CPU 가
online 이 될 때마다 매번 호출된다는 점을 기억해 두시기 바랍니다.
\iffalse

It is important to note that \co{rcu_init_percpu_data()} is invoked
not only at boot time, but also every time that a given CPU is brought
online.
\fi

\subsubsection{\tt rcu\_online\_cpu()}
\label{app:rcuimpl:rcutreewt:rcu-online-cpu}

\begin{figure}[tbp]
{ \scriptsize
\begin{verbatim}
  1 static void __cpuinit rcu_online_cpu(int cpu)
  2 {
  3 #ifdef CONFIG_NO_HZ
  4   struct rcu_dynticks *rdtp;
  5
  6   rdtp = &per_cpu(rcu_dynticks, cpu);
  7   rdtp->dynticks_nesting = 1;
  8   rdtp->dynticks |= 1;
  9   rdtp->dynticks_nmi = (rdtp->dynticks_nmi + 1) & ~0x1;
 10 #endif /* #ifdef CONFIG_NO_HZ */
 11   rcu_init_percpu_data(cpu, &rcu_state);
 12   rcu_init_percpu_data(cpu, &rcu_bh_state);
 13   open_softirq(RCU_SOFTIRQ, rcu_process_callbacks);
 14 }
\end{verbatim}
}
\caption{{\tt rcu\_online\_cpu()} Code}
\label{fig:app:rcuimpl:rcutreewt:Code for rcu-online-cpu}
\end{figure}

Figure~\ref{fig:app:rcuimpl:rcutreewt:Code for rcu-online-cpu}
는 RCU 에게 특정 CPU 가 online 이 되는 것을 알리는 \co{rcu_online_cpu()} 의
코드를 보이고 있습니다.

Dynticks (\co{CONFIG_NO_HZ}) 가 활성화 되어 있으면, line~6 는 명시된 CPU 의
\co{rcu_dynticks} 구조체로의 레퍼런스를 얻어오는데, 이 구조체는 RCU 의 ``rcu''
와 ``rcu\_bh'' 구현 사이에 공유됩니다.
Line~7 은 \co{->dynticks_nesting} 필드의 값을 1로 설정해서 새로 online 이 되는
CPU 가 dynticks-idle 모드에 있지 않다는 사실을 반영시킵니다 (다시 말하건대
\co{->dynticks_nesting} 필드는 연관된 CPU 가 RCU read-side 크리티컬 섹션들을
추적해야 하는 필요의 이유들의 수를 추적하며, 이 경우에는 process-level 코드를
수행할 수 있기 때문입니다).
Line~8 은 \co{->dynticks} 필드는 홀수의 값으로 강제하는데, 이 값은 최소한
기존에 online 이었을 때의 마지막 값만큼은 커야 하는데, 역시 새로 online 이 된
CPU 들이 dynticks-idle 모드에 있지 않다는 사실을 반영하는 것이며, line~9 는
\co{->dynticks_nmi} 필드를 짝수의 값으로 강제하는데, 이 값은 기존에 online
이었을 때 가졌던 마지막 값만큼은 큰 값으로, 이 CPU 가 현재 NMI 핸들러 안에서
실행되고 있다는 사실을 반영합니다.
\iffalse

Figure~\ref{fig:app:rcuimpl:rcutreewt:Code for rcu-online-cpu}
shows the code for \co{rcu_online_cpu()}, which informs RCU that the
specified CPU is coming online.

When dynticks (\co{CONFIG_NO_HZ}) is enabled, line~6 obtains a
reference to the specified CPU's \co{rcu_dynticks} structure, which
is shared between the ``rcu'' and ``rcu\_bh'' implementations of RCU.
Line~7 sets the \co{->dynticks_nesting} field to the value one,
reflecting the fact that a newly onlined CPU is not in dynticks-idle
mode (recall that the \co{->dynticks_nesting} field tracks the
number of reasons that the corresponding CPU needs to be tracked for
RCU read-side critical sections, in this case because it can run
process-level code).
Line~8 forces the \co{->dynticks} field to an odd value that is
at least as large as the last value it had when previously online,
again reflecting the fact that newly onlined CPUs are not in dynticks-idle
mode, and line~9 forces the \co{->dynticks_nmi} field to an even value
that is at least as large as the last value it had when previously
online, reflecting the fact that this CPU is not currently executing
in an NMI handler.
\fi

Line~11-13 은 \co{CONFIG_NO_HZ} 커널 패러미터의 값과 관계 없이 수행됩니다.
Line~11 은 명시된 CPU 의 ``rcu'' 를 위한 \co{rcu_data} 구조체를 초기화 하며,
line~12 는 ``rcu\_bh'' 를 위해 같은 일을 행합니다.
마지막으로, line~13 은 이 CPU 의 뒤따르는 \co{raise_softirq()} 호출에 의해
\co{rcu_process_callbacks()} 가 호출되도록 등록합니다.
\iffalse

Lines~11-13 are executed regardless of the value of the
\co{CONFIG_NO_HZ} kernel parameter.
Line~11 initializes the specified CPU's \co{rcu_data} structure
for ``rcu'', and line~12 does so for ``rcu\_bh''.
Finally, line~13 registers the \co{rcu_process_callbacks()} to be
invoked by subsequent \co{raise_softirq()} invocations on this CPU.
\fi

\subsubsection{\tt rcu\_offline\_cpu()}
\label{app:rcuimpl:rcutreewt:rcu-offline-cpu}

\begin{figure}[tbp]
{ \scriptsize
\begin{verbatim}
  1 static void
  2 __rcu_offline_cpu(int cpu, struct rcu_state *rsp)
  3 {
  4   int i;
  5   unsigned long flags;
  6   long lastcomp;
  7   unsigned long mask;
  8   struct rcu_data *rdp = rsp->rda[cpu];
  9   struct rcu_data *rdp_me;
 10   struct rcu_node *rnp;
 11
 12   spin_lock_irqsave(&rsp->onofflock, flags);
 13   rnp = rdp->mynode;
 14   mask = rdp->grpmask;
 15   do {
 16     spin_lock(&rnp->lock);
 17     rnp->qsmaskinit &= ~mask;
 18     if (rnp->qsmaskinit != 0) {
 19       spin_unlock(&rnp->lock);
 20       break;
 21     }
 22     mask = rnp->grpmask;
 23     spin_unlock(&rnp->lock);
 24     rnp = rnp->parent;
 25   } while (rnp != NULL);
 26   lastcomp = rsp->completed;
 27   spin_unlock(&rsp->onofflock);
 28   cpu_quiet(cpu, rsp, rdp, lastcomp);
 29   rdp_me = rsp->rda[smp_processor_id()];
 30   if (rdp->nxtlist != NULL) {
 31     *rdp_me->nxttail[RCU_NEXT_TAIL] = rdp->nxtlist;
 32     rdp_me->nxttail[RCU_NEXT_TAIL] =
 33         rdp->nxttail[RCU_NEXT_TAIL];
 34     rdp->nxtlist = NULL;
 35     for (i = 0; i < RCU_NEXT_SIZE; i++)
 36       rdp->nxttail[i] = &rdp->nxtlist;
 37     rdp_me->qlen += rdp->qlen;
 38     rdp->qlen = 0;
 39   }
 40   local_irq_restore(flags);
 41 }
 42
 43 static void rcu_offline_cpu(int cpu)
 44 {
 45   __rcu_offline_cpu(cpu, &rcu_state);
 46   __rcu_offline_cpu(cpu, &rcu_bh_state);
 47 }
\end{verbatim}
}
\caption{{\tt rcu\_offline\_cpu()} Code}
\label{fig:app:rcuimpl:rcutreewt:Code for rcu-offline-cpu}
\end{figure}

Figure~\ref{fig:app:rcuimpl:rcutreewt:Code for rcu-offline-cpu}
는 \co{__rcu_offline_cpu()} 와 그 wrapper 함수인 \co{rcu_offline_cpu()} 의
코드를 보입니다.
이 (그림의 line~43-47 에 보인) wrapper 함수의 목적은 단순히
\co{__rcu_offline_cpu()} 를 두번, 한번은 ``rcu'' 를 위해 그리고 ``rcu\_bh'' 를
위해 호출하는 것 뿐입니다.
\co{__rcu_offline_cpu()} 함수의 목적은 미래의 grace period 가 offline 이 된 CPU
를 기다리는 것을 방지하는 것, extended quiescent state 를 알리는 것, 그리고 이
CPU 에서 처리중인 모든 RCU callback 이 새로 위치해야 할 곳을 찾는 것입니다.

그림의 line~1-41 에 보여진 \co{__rcu_offline_cpu()} 로 돌아가서, line~12 는
명시된 \co{rcu_state} 의 \co{->onofflock} 을 획득해서 여러 \co{rcu_node} 계층을
위한 grace-period 초기화를 배제시킵니다.
\iffalse

Figure~\ref{fig:app:rcuimpl:rcutreewt:Code for rcu-offline-cpu}
shows the code for \co{__rcu_offline_cpu()} and its wrapper
function, \co{rcu_offline_cpu()}.
The purpose of this wrapper function (shown in lines~43-47 of the figure)
is simply to invoke \co{__rcu_offline_cpu()} twice, once for ``rcu'' and
again for ``rcu\_bh''.
The purpose of the \co{__rcu_offline_cpu()} function is to
prevent future grace periods from waiting on the CPU being offlined,
to note the extended quiescent state, and to find a new home for
any RCU callbacks in process on this CPU.

Turning to \co{__rcu_offline_cpu()}, shown on lines~1-41 of the figure,
line~12 acquires the specified \co{rcu_state} structure's
\co{->onofflock}, excluding grace-period initialization for
multi-\co{rcu_node} hierarchies.
\fi

\QuickQuiz{}
	하지만 작은 시스템이어서 \co{rcu_node} 계층이 하나의 구조체만을
	갖는다면 어떻게 되나요?
	Figure~\ref{fig:app:rcuimpl:rcutreewt:Code for rcu-offline-cpu} 의
	코드에서, 이 경우에 무엇이 동시의 grace-period 초기화를 방지하나요?
	\iffalse

	But what if the \co{rcu_node} hierarchy has only a single
	structure, as it would on a small system?
	What prevents concurrent grace-period initialization in that
	case, given the code in
	Figure~\ref{fig:app:rcuimpl:rcutreewt:Code for rcu-offline-cpu}?
	\fi
\QuickQuizAnswer{
	Line~16 에서의 \co{rcu_node} 구조체의 \co{->lock} 획득이 grace-period
	초기화를 배제시키는데, 초기화 역시 새로운 grace period 를 위한
	\co{rcu_node} 구조체의 초기화를 위해 이 똑같은 락을 획득해야 하기
	때문입니다.

	이 \co{->onofflock} 은 multi-node 계층들을 위해서만 필요하며, 그런
	경우에 커다란 시스템에서는 놀랍도록 큰 비용이 될, \emph{모든}
	\co{rcu_node} 구조체의 \co{->lock} 필드들을 획득하고 잡는 것에 대한
	대체로 사용됩니다.
	\iffalse

	The later acquisition of the sole \co{rcu_node} structure's
	\co{->lock} on line~16 excludes grace-period initialization,
	which must acquire this same lock in order to initialize this
	sole \co{rcu_node} structure for the new grace period.

	The \co{->onofflock} is needed only for multi-node hierarchies,
	and is used in that case as an alternative to acquiring and
	holding \emph{all} of the \co{rcu_node} structures'
	\co{->lock} fields, which would be incredibly painful on
	large systems.
	\fi
} \QuickQuizEnd

Line~13 은 이 CPU 의 \co{rcu_Data} 구조체의 \co{->mynode} 포인터를 사용해서 이
CPU 에 연관된 leaf \co{rcu_node} 구조체로의 포인터를 가져옵니다
(Figure~\ref{fig:app:rcuimpl:rcutree:Initialized RCU Data Layout} 를
참고하세요).
Line~14 는 이 CPU 의 leaf \co{rcu_node} 구조체의 \co{qsmask} 필드에 사용할 mask
를 가져옵니다.
\iffalse

Line~13 picks up a pointer to the leaf \co{rcu_node} structure corresponding
to this CPU, using the \co{->mynode} pointer in this CPU's \co{rcu_data}
structure
(see Figure~\ref{fig:app:rcuimpl:rcutree:Initialized RCU Data Layout}).
Line~14 picks up a mask with this CPU's bit set for use on
the leaf \co{rcu_node} structure's \co{qsmask} field.
\fi

이어서 line~15-25 의 루프는 이 CPU 의 bit 들을 이 CPU 의 leaf \co{rcu_node}
구조체부터 시작해서 \co{rcu_node} 계층을 위로 따라가며 지워갑니다.
Line~16 은 현재 \co{rcu_node} 구조체의 \co{->lock} 필드를 획득하고, line~17
에서는 이 CPU 에 (또는, 계층의 위로 가면 그룹에) 연관된 bit 을
\co{->qsmaskinit} 필드로부터 지워서 미래의 grace period 를 이 CPU 로부터의
quiescent state 를 기다리지 않도록 만듭니다.
그 결과 \co{->qsmaskinit} 의 값이 line~18 에서 체크되는 것처럼 0이 아니라면,
현재 \co{rcu_node} 구조체는 추적해야만 하는 online CPU 들을 갖고 있다는 것이고,
따라서 line~19 에서 현재 \co{rcu_node} 구조체의 \co{->lock} 을 해제하고 line~20
에서 루프를 빠져나갑니다.
그렇지 않다면, 우리는 \co{rcu_node} 계층을 따라 진행해야 합니다.
이 경우, line~22 는 다음 단계에 적용할 mask 를 가져오고 line~23 에서 현재
\co{rcu_node} 구조체의 \co{->lock} 을 해제한 후, line~24 에서 이 계층의 다음
레벨로 넘어갑니다.
Line~25 는 루프를 나가는데, 이 계층의 꼭대기에서일 겁니다.
\iffalse

The loop spanning lines~15-25 then clears this CPU's bits up the
\co{rcu_node} hierarchy, starting with this CPU's leaf \co{rcu_node}
structure.
Line~16 acquires the current \co{rcu_node} structure's \co{->lock}
field, and line~17 clears the bit corresponding to this CPU
(or group, higher up in the hierarchy) from the \co{->qsmaskinit}
field, so that future grace periods will not wait on quiescent states
from this CPU.
If the resulting \co{->qsmaskinit} value is non-zero, as checked by
line~18, then the
current \co{rcu_node} structure has other online CPUs that it
must track, so line~19 releases the current \co{rcu_node} structure's
\co{->lock} and line~20 exits the loop.
Otherwise, we need to continue walking up the \co{rcu_node} hierarchy.
In this case, line~22 picks up the mask to apply to the next level up,
line~23 releases the current \co{rcu_node} structure's \co{->lock},
and line~24 advances up to the next level of the hierarchy.
Line~25 exits the loop should we exit out the top of the hierarchy.
\fi

\QuickQuiz{}
	하지만 Figure~\ref{fig:app:rcuimpl:rcutreewt:Code for rcu-offline-cpu}
	의 line~25 가 정말로 이 루프를 끝낼까요?
	그렇다면, 또는 그렇지 않다면 왜죠?
	\iffalse

	But does line~25 of
	Figure~\ref{fig:app:rcuimpl:rcutreewt:Code for rcu-offline-cpu}
	ever really exit the loop?
	Why or why not?
	\fi
\QuickQuizAnswer{
	Line~25 가 이 루프를 끝내는 유일한 방법은 \emph{모든} CPU 들이 offline
	이 되려 할 때입니다.
	이는 2.6.28 버전까지의 리눅스 커널에서는 일어날 수 없는 일입니다만,
	shutdown 과정 중에 모든 CPU 들이 오프라인이 되도록 설계되는 다른 환경이
	있을 수도 있습니다.
	\iffalse

	The only way that line~25 could exit the loop is if \emph{all}
	CPUs were to be put offline.
	This cannot happen in the Linux kernel as of 2.6.28, though
	other environments have been designed to offline all CPUs
	during the normal shutdown procedure.
	\fi
} \QuickQuizEnd

Line~26 은 명시된 \co{rcu_state} 구조체의 \co{->completed} 필드를 로컬 변수
\co{lastcomp} 로 가져오고, line~27 은 \co{->onofflock} 을 해제하고 (하지만 irq
는 비활성화 된 채로 둡니다), line~28 은 offline 이 되는 CPU 가 이제 extended
quiescent state 에 있음을 알리고, 이 quiescent state 를 자신이 시작된 곳이
아니라 다른 grace period 에게 보고하는 것을 막기 위해 \co{lastcomp} 를 넘기는
\co{cpu_quiet()} 함수를 호출합니다.
\iffalse

Line~26 picks up the specified \co{rcu_state} structure's
\co{->completed} field into the local variable \co{lastcomp},
line~27 releases \co{->onofflock} (but leaves irqs disabled),
and line~28 invokes \co{cpu_quiet()} in order to note that the
CPU being offlined is now in an extended quiescent state, passing
in \co{lastcomp} to avoid reporting this quiescent state against
a different grace period than it occurred in.
\fi

\QuickQuiz{}
	Figure~\ref{fig:app:rcuimpl:rcutreewt:Code for rcu-offline-cpu} 의
	line~26 이 상당히 비순차적으로 수행되어서 \co{lastcomp} 는 어떤 앞의
	grace period 로 설정되어 있지만 현재 grace period 는 여전히 이제
	offline 이 된 CPU 를 기다리고 있다고 생각해 봅시다.
	이런 경우, \co{cpu_quiet()} 호출은 이 quiescent state 를 보고하는데
	실패하고, 따라서 이 grace period 가 이 이제는 offline 이 된 CPU 를
	영원히 기다리게 되지 않을까요?
	\iffalse

	Suppose that line~26 got executed seriously out of order in
	Figure~\ref{fig:app:rcuimpl:rcutreewt:Code for rcu-offline-cpu},
	so that \co{lastcomp} is set to some prior grace period, but
	so that the current grace period is still waiting on the
	now-offline CPU?
	In this case, won't the call to \co{cpu_quiet()} fail to
	report the quiescent state, thus causing the grace period
	to wait forever for this now-offline CPU?
	\fi
\QuickQuizAnswer{
	먼저, line~16 과 12 에서의 락 획득은 line~26 의 수행이 그렇게까지
	비순차적으로 되지 못하게 합니다.
	그러나, line~26 이 그렇게까지 극적으로 잘못 순서지어지는 일이 발생한다
	하더라도, \co{force_quiescent_state()} 가 결국은 호출되어서, 현재 grace
	period 가 offline CPU 로부터의 quiescent state 를 기다리고 있었다는 걸
	알리게 될 겁니다.
	그렇게 되면 \co{force_quiescent_state()} 는 이 offline 이 된 CPU 에
	extended quiescetn state 가 있는 것으로 보고할 겁니다.
	\iffalse

	First, the lock acquisitions on lines~16 and 12 would prevent
	the execution of line~26 from being pushed that far out of
	order.
	Nevertheless, even if line~26 managed to be misordered that
	dramatically, what would happen is that \co{force_quiescent_state()}
	would eventually be invoked, and would notice that the current
	grace period was waiting for a quiescent state from an offline
	CPU.
	Then \co{force_quiescent_state()} would report the extended
	quiescent state on behalf of the offlined CPU.
	\fi
} \QuickQuizEnd

\QuickQuiz{}
	Offline CPU 가 extended quiescent state 에 있음에도, 왜
	Figure~\ref{fig:app:rcuimpl:rcutreewt:Code for rcu-offline-cpu}
	의 line~28 은 자신이 어떤 grace period 를 처리하고 있는지 신경써야
	하는거죠?
	\iffalse

	Given that an offline CPU is in an extended quiescent state,
	why does line~28 of
	Figure~\ref{fig:app:rcuimpl:rcutreewt:Code for rcu-offline-cpu}
	need to care which grace period it is
	dealing with?
	\fi
\QuickQuizAnswer{
	이 경우엔 실제로 그런 신경을 쓸 필요가 없습니다.
	하지만, 많은 다른 경우에는 \emph{실제로} 신경을 써야 하므로,
	\co{cpu_quiet()} 함수는 grace-peiod 숫자를 인자로 받으며, 따라서 어떤
	값을 넘기긴 해야 합니다.
	\iffalse

	It really does not need to care in this case.
	However, because it \emph{does} need to care in many other
	cases, the \co{cpu_quiet()} function does take the
	grace-period number as an argument, so some value must be
	supplied.
	\fi
} \QuickQuizEnd

Line~29-39 는 offline 이 되는 CPU 의 모든 RCU 콜백들을 현재 돌아가는 CPU 로
옮깁니다.
이 작업은 옮겨지는 콜백들의 순서를 재배치 하는걸 막아야 하는데, 그러지 않는다면
\co{rcu_barrier()} 가 올바르게 동작하지 못할 것이기 때문입니다.
Line~29 는 현재 동작 중인 CPU 의 \co{rcu_data} 구조체로의 포인터를 로컬 변수
\co{rdp_me} 로 가져옵니다.
Line~30 은 이어서 offline 이 되는 CPU 가 어떤 RCU 콜백들을 가지고 있는지 체크해
봅니다.
만약 그렇다면, line~31-38 에서 그것들을 옮깁니다.
Line~31 은 콜백들의 리스트를 수행중인 CPU 의 리스트의 끝에 이어붙입니다.
Line~32-33 은 수행중인 CPU 의 콜백 tail 포인터를 offline 이 되는 CPU 의
그것으로 바꿔치기하고, 이어서 line~34-36 에서 offline 이 되는 CPU 의 리스트를
텅비도록 초기화 합니다.
Line~37 은 offline 이 되는 CPU 의 콜백 리스트의 길이를 현재 수행중인 CPU 의
그것에 더하고, 마지막으로 line~38 에서 offline 이 되는 CPU 의 리스트의 길이를
0으로 설정합니다.
\iffalse

Lines~29-39 move any RCU callbacks from the CPU going offline to the
currently running CPU.
This operation must avoid reordering the callbacks being moved, as
otherwise \co{rcu_barrier()} will not work correctly.
Line~29 puts a pointer to the currently running CPU's \co{rcu_data}
structure into local variable \co{rdp_me}.
Line~30 then checks to see if the CPU going offline has any RCU callbacks.
If so, lines~31-38 move them.
Line~31 splices the list of callbacks onto the end of the running CPU's
list.
Lines~32-33 sets the running CPU's callback tail pointer to that of
the CPU going offline, and then lines~34-36 initialize the going-offline
CPU's list to be empty.
Line~37 adds the length of the going-offline CPU's callback list to
that of the currently running CPU, and, finally, line~38 zeroes the
going-offline CPU's list length.
\fi

\QuickQuiz{}
	하지만
	Figure~\ref{fig:app:rcuimpl:rcutreewt:Code for rcu-offline-cpu}
	의 이 리스트 이동은 모든 offline 이 되는 CPU 의 콜백들이 설령 호출될
	준비가 되어 있는 상태였을지라도 다른 grace period 로 가도록 만듭니다.
	이건 비효율적이지 않나요?
	뿐만 아니라, offline 이 되고는 online 으로 돌아오는 CPU 들의 불행한
	패턴은 특정 콜백이 영원히 호출되지 않게 할 수 있지 않나요?
	\iffalse

	But this list movement in
	Figure~\ref{fig:app:rcuimpl:rcutreewt:Code for rcu-offline-cpu}
	makes all of the going-offline CPU's callbacks go through
	another grace period, even if they were ready to invoke.
	Isn't that inefficient?
	Furthermore, couldn't an unfortunate pattern of CPUs going
	offline then coming back online prevent a given callback from
	ever being invoked?
	\fi
\QuickQuizAnswer{
	비효율적이지만, 간단합니다.
	이게 흔히 수행되는 코드 경로가 아니란 점을 놓고 보면, 이건 괜찮은
	트레이드오프입니다.
	Starvation 문제는 문제가 될 수 있겠지만, online 과 offline 과정이 여러
	grace period 에 관여되는 경우엔 예외일 겁니다.
	\iffalse

	It is inefficient, but it is simple.
	Given that this is not a commonly executed code path, this
	is the right tradeoff.
	The starvation case would be a concern, except that the
	online and offline process involves multiple grace periods.
	\fi
} \QuickQuizEnd

마지막으로, line~40 은 irq 들을 재활성화 시킵니다.
\iffalse

Finally, line~40 re-enables irqs.
\fi

\subsection{Miscellaneous Functions}
\label{app:rcuimpl:rcutreewt:Miscellaneous Functions}

이 섹션은 다음의 기타 기능을 제공하는 함수들을 설명합니다:
\iffalse

This section describes the miscellaneous utility functions:
\fi
\begin{enumerate}
\item	\co{rcu_batches_completed}
\item	\co{rcu_batches_completed_bh}
\item	\co{cpu_has_callbacks_ready_to_invoke}
\item	\co{cpu_needs_another_gp}
\item	\co{rcu_get_root}
\end{enumerate}

\begin{figure}[tbp]
{ \scriptsize
\begin{verbatim}
  1 long rcu_batches_completed(void)
  2 {
  3   return rcu_state.completed;
  4 }
  5
  6 long rcu_batches_completed_bh(void)
  7 {
  8   return rcu_bh_state.completed;
  9 }
 10
 11 static int
 12 cpu_has_callbacks_ready_to_invoke(struct rcu_data *rdp)
 13 {
 14   return &rdp->nxtlist != rdp->nxttail[RCU_DONE_TAIL];
 15 }
 16
 17 static int
 18 cpu_needs_another_gp(struct rcu_state *rsp,
 19                      struct rcu_data *rdp)
 20 {
 21   return *rdp->nxttail[RCU_DONE_TAIL] &&
 22          ACCESS_ONCE(rsp->completed) ==
 23          ACCESS_ONCE(rsp->gpnum);
 24 }
 25
 26 static struct rcu_node
 27 *rcu_get_root(struct rcu_state *rsp)
 28 {
 29   return &rsp->node[0];
 30 }
\end{verbatim}
}
\caption{Miscellaneous Functions}
\label{fig:app:rcuimpl:rcutreewt:Miscellaneous Functions}
\end{figure}

Figure~\ref{fig:app:rcuimpl:rcutreewt:Miscellaneous Functions}
는 몇가지 기타 함수들을 보이고 있습니다.
Line~1-9 는 \co{rcu_batches_completed()} 와 \co{rcu_batches_completed_bh()} 를
보이는데, 이 함수들은 rcutorture test suite 에 의해 사용됩니다.
Line~11-15 는 \co{cpu_has_callbacks_ready_to_invoke()} 함수를 보이는데, 이
함수는 명시된 \co{rcu_data} 구조체가 grace period 가 지난 RCU 콜백들을 가지고
있는지를 알리는데, 이는 리스트의 헤드를 더이상 향하지 않는 ``done'' tail
포인터에 의해 알려집니다.
Line~17-24 는 명시된 \co{rcu_data} 구조체에 연관된 CPU 가 어떤 grace period 도
진행중이지 않은 시간 동안 추가적인 grace period 를 필요로 하는지 알리는
\co{cpu_needs_another_gp()} 함수를 보이고 있습니다.
이 명시된 \co{rcu_data} 구조체는 명시된 \co{rcu_state} 구조체와 연관되어 있어야
한다는 점을 알아두시기 바랍니다.
마지막으로, line~26-30 은 명시된 \co{rcu_state} 구조체에 연관된 root
\co{rcu_node} 구조체를 리턴하는 \co{rcu_get_root()} 함수를 보입니다.
\iffalse

Figure~\ref{fig:app:rcuimpl:rcutreewt:Miscellaneous Functions}
shows a number of miscellaneous functions.
Lines~1-9 show \co{rcu_batches_completed()} and
\co{rcu_batches_completed_bh()}, which are used by the rcutorture
test suite.
Lines~11-15 show \co{cpu_has_callbacks_ready_to_invoke()}, which
indicates whether the specified \co{rcu_data} structure has RCU
callbacks that have passed through their grace period, which
is indicated by the ``done'' tail pointer no longer pointing
to the head of the list.
Lines~17-24 show \co{cpu_needs_another_gp()}, which indicates
whether the CPU corresponding to the specified \co{rcu_data}
structure requires an additional grace period during a time when
no grace period is in progress.
Note that the specified \co{rcu_data} structure is required
to be associated with the specified \co{rcu_state} structure.
Finally, lines~26-30 show \co{rcu_get_root()}, which returns
the root \co{rcu_node} structure associated with the specified
\co{rcu_state} structure.
\fi

\subsection{Grace-Period-Detection Functions}
\label{app:rcuimpl:rcutreewt:Grace-Period-Detection Functions}

이 섹션은 grace period 들의 시작과 끝을 탐지하는데 직접적으로 관계된 함수들을
다룹니다.
물론 이는 grace period 들 을 실제로 시작하고 끝내는 것이 포함되며, 다른 CPU
들이 grace period 들을 시작하고 끝낸 시점을 알리는 것도 포함됩니다.
\iffalse

This section covers functions that are directly involved in detecting
beginnings and ends of grace periods.
This of course includes actually starting and ending grace periods,
but also includes noting when other CPUs have started or ended
grace periods.
\fi

\subsubsection{Noting New Grace Periods}
\label{app:rcuimpl:rcutreewt:Noting New Grace Periods}

Hierarchical RCU 의 주요 목적은 grace period 들을 탐지하는 것이고, 이 작업에 더
직접적으로 관련된 함수들이 이 섹션에서 설명되어집니다.
Section~\ref{app:rcuimpl:rcutreewt:Noting New Grace Periods}
은 CPU 들이 새로운 grace period 가 시작되었음을 CPU 들이 알 수 있게 해주는
함수를 다루고,
Section~\ref{app:rcuimpl:rcutreewt:Noting End of Old Grace Periods}
은 새로운 grace period 를 시작하는 \co{rcu_start_gp()} 를 다루며,
Section~\ref{app:rcuimpl:rcutreewt:Reporting Quiescent States}
에서는 CPU 의 quiescent state 를 RCU core 에 알리는데 관련된 함수들을 다룹니다.
\iffalse

The main purpose of Hierarchical RCU is to detect grace periods,
and the functions more directly involved in this task are described
in this section.
Section~\ref{app:rcuimpl:rcutreewt:Noting New Grace Periods}
covers functions that allow CPUs to note that a new grace period has
begun,
Section~\ref{app:rcuimpl:rcutreewt:Noting End of Old Grace Periods}
covers functions that allow CPUs to note that an existing grace period
has ended,
Section~\ref{app:rcuimpl:rcutreewt:Starting a Grace Period}
covers \co{rcu_start_gp()}, which starts a new grace period, and
Section~\ref{app:rcuimpl:rcutreewt:Reporting Quiescent States}
covers functions involved in reporting CPUs' quiescent states to
the RCU core.
\fi

\begin{figure}[tbp]
{ \scriptsize
\begin{verbatim}
  1 static void note_new_gpnum(struct rcu_state *rsp,
  2                            struct rcu_data *rdp)
  3 {
  4   rdp->qs_pending = 1;
  5   rdp->passed_quiesc = 0;
  6   rdp->gpnum = rsp->gpnum;
  7   rdp->n_rcu_pending_force_qs = rdp->n_rcu_pending +
  8               RCU_JIFFIES_TILL_FORCE_QS;
  9 }
 10
 11 static int
 12 check_for_new_grace_period(struct rcu_state *rsp,
 13                            struct rcu_data *rdp)
 14 {
 15   unsigned long flags;
 16   int ret = 0;
 17
 18   local_irq_save(flags);
 19   if (rdp->gpnum != rsp->gpnum) {
 20     note_new_gpnum(rsp, rdp);
 21     ret = 1;
 22   }
 23   local_irq_restore(flags);
 24   return ret;
 25 }
\end{verbatim}
}
\caption{Noting New Grace Periods}
\label{fig:app:rcuimpl:rcutreewt:Noting New Grace Periods}
\end{figure}

Figure~\ref{fig:app:rcuimpl:rcutreewt:Noting New Grace Periods}
는 \co{note_new_gpnum()} 함수를 보이고 있는데, 이 함수는 새로운 grace period 를
반영하기 위한 상태 업데이트를 행합니다.  또한 \co{check_for_new_grace_period()}
함수를 보이는데, 이 함수는 언제 다른 CPU 들이 새로운 grace period 를 시작했는지
파악하는데에 사용됩니다.

\co{note_new_gpnum()} 의 Line~4 는 RCU 가 이 CPU 의 quiescent state 를 필요로
함을 알리기 위해 현재 CPU 의 \co{rcu_data} 구조체의 \co{->qs_pending} 플래그의
값을 1로 세우고, line~5 는 이 CPU 가 그런 quiescent state 를 아직 지나지
않았다는 것을 알리기 위해 \co{->passed_quiesc} 플래그의 값을 0으로 설정하며,
line~6 는 grace-period 숫자를 global \co{rcu_state} 구조체로부터 이 CPU 의
\co{rcu_data} 구조체로 복사해서 이 CPU 가 이 새로운 grace period 의 시작을
알렸음을 기억할 수 있게 합니다.
마지막으로, line~7-8 은 이 CPU 가 지연되어있는 CPU 들이 quiescent state 들을
지나도록 강제하려 시도할 시간을 jiffies 로 기록하는데 (이 시간에 또는 이 시간
후에 \co{force_quiescent_state()} 를 호출함으로써 강제할겁니다) 이 전에 grace
period 가 끝나지 않는 가정 하에 그렇습니다.
\iffalse

Figure~\ref{fig:app:rcuimpl:rcutreewt:Noting New Grace Periods}
shows the code for \co{note_new_gpnum()}, which updates state to reflect
a new grace period, as well as \co{check_for_new_grace_period()}, which
is used by CPUs to detect when other CPUs have started a new grace period.

Line~4 of \co{note_new_gpnum()} sets the \co{->qs_pending} flag is
the current CPU's \co{rcu_data} structure to indicate that RCU needs
a quiescent state from this CPU, line~5 clears the \co{->passed_quiesc}
flag to indicate that this CPU has not yet passed through such a
quiescent state,
line~6 copies the grace-period number from the global \co{rcu_state}
structure to this CPU's \co{rcu_data} structure so that this CPU will
remember that it has already noted the beginning of this new grace
period.
Finally, lines~7-8 record the time in jiffies at which this CPU
will attempt to force holdout CPUs to pass through quiescent states
(by invoking \co{force_quiescent_state()} on or after that future time),
assuming that the grace period does not end beforehand.
\fi

\co{check_for_new_grace_period()} 의 line~18 과 23 은 각각 인터럽트를 비활성화
하고 재활성화 합니다.
Line~19 는 현재 CPU 가 알지 못한 새로운 grace period 가 존재하는지 보고, 만약
그렇다면 line~20 은 이 새로운 grace period 를 알리기 위해 \co{note_new_gpnum()}
을 호출하고, line~21 은 적절하게 리턴 값을 설정합니다.
어느 쪽이든, line~24 는 상태를 리턴합니다: 새로운 grace period 가 시작했다면
0이 아닌 값을, 그렇지 않으면 0입니다.
\iffalse

Lines~18 and 23 of \co{check_for_new_grace_period()} disable and
re-enable interrupts, respectively.
Line~19 checks to see if there is a new grace period that the current
CPU has not yet noted, and, if so, line~20 invokes \co{note_new_gpnum()}
in order to note the new grace period, and line~21 sets the return value
accordingly.
Either way, line~24 returns status: non-zero if a new grace period has
started, and zero otherwise.
\fi

\QuickQuiz{}
	Figure~\ref{fig:app:rcuimpl:rcutreewt:Noting New Grace Periods}
	의 \co{check_for_new_grace_period()} 안에 \co{note_new_gpnum()} 을
	그대로 집어넣지 않나요?
	\iffalse

	Why not just expand \co{note_new_gpnum()} inline into
	\co{check_for_new_grace_period()} in
	Figure~\ref{fig:app:rcuimpl:rcutreewt:Noting New Grace Periods}?
	\fi
\QuickQuizAnswer{
	\co{note_new_gpnum()} 은 각각의 새로운 grace period 를 위해
	호출되어야만 하며, 이는 이 CPU 에 시작된 것들과 다른 CPU 들에 의해
	시작된 것들을 포함하기 때문입니다.
	반면에, \co{check_for_new_grace_period()} 는 일부 다른 CPU 들이 grace
	period 를 시작한 경우에 대해서만 호출됩니다.
	\iffalse

	Because \co{note_new_gpnum()} must be called for each new
	grace period, including both those started by this CPU and
	those started by other CPUs.
	In contrast, \co{check_for_new_grace_period()} is called only
	for the case where some other CPU started the grace period.
	\fi
} \QuickQuizEnd

\subsubsection{Noting End of Old Grace Periods}
\label{app:rcuimpl:rcutreewt:Noting End of Old Grace Periods}

\begin{figure}[tbp]
{ \scriptsize
\begin{verbatim}
  1 static void
  2 rcu_process_gp_end(struct rcu_state *rsp,
  3                    struct rcu_data *rdp)
  4 {
  5   long completed_snap;
  6   unsigned long flags;
  7
  8   local_irq_save(flags);
  9   completed_snap = ACCESS_ONCE(rsp->completed);
 10   if (rdp->completed != completed_snap) {
 11     rdp->nxttail[RCU_DONE_TAIL] =
 12         rdp->nxttail[RCU_WAIT_TAIL];
 13     rdp->nxttail[RCU_WAIT_TAIL] =
 14         rdp->nxttail[RCU_NEXT_READY_TAIL];
 15     rdp->nxttail[RCU_NEXT_READY_TAIL] =
 16         rdp->nxttail[RCU_NEXT_TAIL];
 17     rdp->completed = completed_snap;
 18   }
 19   local_irq_restore(flags);
 20 }
\end{verbatim}
}
\caption{Noting End of Old Grace Periods}
\label{fig:app:rcuimpl:rcutreewt:Noting End of Old Grace Periods}
\end{figure}

\begin{figure}[tb]
\centering
\resizebox{3in}{!}{\includegraphics{appendix/rcuimpl/AdvanceRCUCallbacks}}
\caption{RCU Callback List}
\label{fig:app:rcuimpl:rcutree:RCU Callback List}
\end{figure}

Figure~\ref{fig:app:rcuimpl:rcutreewt:Noting End of Old Grace Periods}
는 CPU 가 하나의 grace period 가 끝났을 거라 의심할 때에 (해당 CPU 가 실제로
grace period 를 끝냈거나 해서) 호출되는 \co{rcu_process_gp_end()} 함수가 보여져
있습니다.
만약 하나의 grace period 가 정말로 끝났다면, 이 함수는 현재 CPU 의 RCU 콜백들을
진행시키는데, 이 콜백들은
Figure~\ref{fig:app:rcuimpl:rcutree:RCU Callback List} 에 보여진 것처럼
여러개의 tail 포인터들을 가진 singly linked list 로 관리되어집니다.
이 여러개의 tail pointer 형태는 Lai Jiangshan 에 의해 이루어졌으며 리스트
관리를 단순화 시킵니다~\cite{LaiJiangshan2008NewClassicAlgorithm}.
이 그림에서, 파란 네모는 한 CPU 의 \co{rcu_data} 구조체를 나타내며, 다이어그램
바닥의 여섯개의 하얀 네모는 여섯개의 RCU 콜백들 (\co{rcu_head} 구조체) 의
리스트를 나타냅니다.
이 리스트에서, 앞의 세개의 콜백들은 각자의 grace period 를 이미 지났고 따라서
수행되어지길 기다리고 있으며, 네번째 콜백 (두번째 줄의 첫번째 것) 은 현재 grace
period 가 완료되길 기다리고 있으며, 마지막 두개의 콜백은 다음 grace period 를
기다리고 있습니다.
마지막 두개의 tail pointer 들은 마지막 원소를 가리키며, 따라서 아직 특정 grace
period 에 관계되지 않은 콜백들을 가지고 있는 마지막 sublist 는 비어있습니다.
\iffalse

Figure~\ref{fig:app:rcuimpl:rcutreewt:Noting End of Old Grace Periods}
shows \co{rcu_process_gp_end()}, which is invoked when a CPU suspects
that a grace period might have ended (possibly because the CPU in question
in fact ended the grace period).
If a grace period really has ended, then this function advances the
current CPU's RCU callbacks, which are managed as a singly linked
list with multiple tail pointers, as shown in
Figure~\ref{fig:app:rcuimpl:rcutree:RCU Callback List}.
This multiple tail pointer layout, spearheaded by
Lai Jiangshan, simplifies list
handling~\cite{LaiJiangshan2008NewClassicAlgorithm}.
In this figure, the blue box represents one CPU's \co{rcu_data}
structure, with the six white boxes at the bottom of the diagram
representing a list of six RCU callbacks (\co{rcu_head} structures).
In this list, the first three callbacks have passed through their
grace period and are thus waiting to be invoked, the fourth
callback (the first on the second line) is waiting for the current
grace period to complete, and the last two are waiting for the
next grace period.
The last two tail pointers reference the last element, so that the
final sublist, which would comprise callbacks that had not yet been
associated with a specific grace period, is empty.
\fi

Figure~\ref{fig:app:rcuimpl:rcutreewt:Noting End of Old Grace Periods}
의 line~8 과 19 는 각각 인터럽트를 비활성화 / 재활성화 시킵니다.
Line~9 는 \co{rcu_state} 구조체의 \co{->completed} 필드를 가져오고, 그걸 로컬
변수 \co{completed_snap} 에 저장해 둡니다.
Line~10 은 현재 CPU 가 grace period 의 끝을 알고 있지 못한지 체크해보고, 만약
알고 있지 못하다면, line~11-16 은 이 CPU 의 RCU 콜백들을 tail 포인터들을
조정해서 진행시킵니다.
이어서 Line~17 은 가장 최근에 완료된 grace period 의 숫자를 \co{->completed}
필드의 이 CPU 의 \co{rcu_data} 구조체에 기록합니다.
\iffalse

Lines~8 and 19 of
Figure~\ref{fig:app:rcuimpl:rcutreewt:Noting End of Old Grace Periods}
suppress and re-enable interrupts, respectively.
Line~9 picks up a snapshot of the \co{rcu_state} structure's
\co{->completed} field, storing it in the local variable
\co{completed_snap}.
Line~10 checks to see if the current CPU is not yet aware of the
end of a grace period, and if it is not aware,
lines~11-16 advance this CPU's RCU callbacks by manipulating the
tail pointers.
Line~17 then records the most recently completed grace period number
in this CPU's \co{rcu_data} structure in the \co{->completed}
field.
\fi

\subsubsection{Starting a Grace Period}
\label{app:rcuimpl:rcutreewt:Starting a Grace Period}

\begin{figure}[tbp]
{ \scriptsize
\begin{verbatim}
  1 static void
  2 rcu_start_gp(struct rcu_state *rsp, unsigned long flags)
  3   __releases(rcu_get_root(rsp)->lock)
  4 {
  5   struct rcu_data *rdp = rsp->rda[smp_processor_id()];
  6   struct rcu_node *rnp = rcu_get_root(rsp);
  7   struct rcu_node *rnp_cur;
  8   struct rcu_node *rnp_end;
  9
 10   if (!cpu_needs_another_gp(rsp, rdp)) {
 11     spin_unlock_irqrestore(&rnp->lock, flags);
 12     return;
 13   }
 14   rsp->gpnum++;
 15   rsp->signaled = RCU_GP_INIT;
 16   rsp->jiffies_force_qs = jiffies +
 17       RCU_JIFFIES_TILL_FORCE_QS;
 18   rdp->n_rcu_pending_force_qs = rdp->n_rcu_pending +
 19       RCU_JIFFIES_TILL_FORCE_QS;
 20   record_gp_stall_check_time(rsp);
 21   dyntick_record_completed(rsp, rsp->completed - 1);
 22   note_new_gpnum(rsp, rdp);
 23   rdp->nxttail[RCU_NEXT_READY_TAIL] =
 24       rdp->nxttail[RCU_NEXT_TAIL];
 25   rdp->nxttail[RCU_WAIT_TAIL] =
 26       rdp->nxttail[RCU_NEXT_TAIL];
 27   if (NUM_RCU_NODES == 1) {
 28     rnp->qsmask = rnp->qsmaskinit;
 29     spin_unlock_irqrestore(&rnp->lock, flags);
 30     return;
 31   }
 32   spin_unlock(&rnp->lock);
 33   spin_lock(&rsp->onofflock);
 34   rnp_end = rsp->level[NUM_RCU_LVLS - 1];
 35   rnp_cur = &rsp->node[0];
 36   for (; rnp_cur < rnp_end; rnp_cur++)
 37     rnp_cur->qsmask = rnp_cur->qsmaskinit;
 38   rnp_end = &rsp->node[NUM_RCU_NODES];
 39   rnp_cur = rsp->level[NUM_RCU_LVLS - 1];
 40   for (; rnp_cur < rnp_end; rnp_cur++) {
 41     spin_lock(&rnp_cur->lock);
 42     rnp_cur->qsmask = rnp_cur->qsmaskinit;
 43     spin_unlock(&rnp_cur->lock);
 44   }
 45   rsp->signaled = RCU_SIGNAL_INIT;
 46   spin_unlock_irqrestore(&rsp->onofflock, flags);
 47 }
\end{verbatim}
}
\caption{Starting a Grace Period}
\label{fig:app:rcuimpl:rcutreewt:Starting a Grace Period}
\end{figure}

Figure~\ref{fig:app:rcuimpl:rcutreewt:Starting a Grace Period}
는 새로운 grace period 를 시작하고, 호출자에 의해 획득되어져 있어야만 하는 root
\co{rcu_node} 구조체의 락을 해제하는 \co{rcu_start_gp()} 함수를 보입니다.

Line~3 는 \co{sparse} 도구를 위한 주석으로, \co{rcu_start_gp()} 가 root
\co{rcu_node} 구조체의 락을 해제한다는 것을 알립니다.
지역변수 \co{rdp} 는 수행중인 CPU 의 \co{rcu_data} 구조체를 가리키고, \co{rnp}
는 root \co{rcu_node} 구조체를 가리키며, \co{rnp_cur} 와 \co{rnp_end} 는
\co{rcu_node} 계층을 횡단하느데에 커서로 사용됩니다.
\iffalse

Figure~\ref{fig:app:rcuimpl:rcutreewt:Starting a Grace Period}
shows \co{rcu_start_gp()}, which starts a new grace period,
also releasing the root \co{rcu_node} structure's lock, which
must be acquired by the caller.

Line~3 is annotation for the \co{sparse} utility, indicating
that \co{rcu_start_gp()} releases the root \co{rcu_node}
structure's lock.
Local variable \co{rdp} references the running CPU's \co{rcu_data}
structure, \co{rnp} references the root \co{rcu_node} structure,
and \co{rnp_cur} and \co{rnp_end} are used as cursors in traversing
the \co{rcu_node} hierarchy.
\fi

Line~10 은 이 CPU 가 정말로 또다른 grace period 를 시작할 필요가 있는지
알아보고, 만약 그렇지 않다면 line~11 에서 root \co{rcu_node} 구조체의 락을
해제하고 line~12 에서 리턴합니다.
이 코드 경로는 여러 CPU 들이 동시에 grace period 를 시작하려 시도해서 실행될
수도 있습니다.
이 경우에, 승리자인 하나의 CPU 만이 grace period 를 시작하고, 나머지들은 이
코드 경로를 통해 빠져나갑니다.

그렇지 않다면, line~14 에서 명시된 \co{rcu_state} 구조체의 \co{->gpnum} 필드의
값을 증가시켜서 새로운 grace period 의 시작을 공식적으로 표시합니다.
\iffalse

Line~10 invokes \co{cpu_needs_another_gp()} to see if this CPU really
needs another grace period to be started, and if not, line~11
releases the root \co{rcu_node} structure's lock and line~12 returns.
This code path can be executed due to multiple CPUs concurrently
attempting to start a grace period.
In this case, the winner will start the grace period, and the losers
will exit out via this code path.

Otherwise, line~14 increments the specified \co{rcu_state} structure's
\co{->gpnum} field, officially marking the start of a new grace
period.
\fi

\QuickQuiz{}
	하지만
	Figure~\ref{fig:app:rcuimpl:rcutreewt:Starting a Grace Period}
	의 line~15 까지는 초기화가 없었어요!
	만약 CPU 가 새로운 grace period 를 알게 되었지만 곧바로 quiescent state
	를 보고하려 하면 어떻게 되죠?
	이러면 좀 혼란스러워지지 않을까요?
	\iffalse

	But there has been no initialization yet at line~15 of
	Figure~\ref{fig:app:rcuimpl:rcutreewt:Starting a Grace Period}!
	What happens if a CPU notices the new grace period and
	immediately attempts to report a quiescent state?
	Won't it get confused?
	\fi
\QuickQuizAnswer{
	두가지 흥미로운 경우가 있습니다.

	첫번째 경우는, 계층상에 하나의 \co{rcu_node} 구조체만 있는 경우입니다.
	\co{rcu_start_gp()} 에서 수행중인 CPU 는 현재 \co{rcu_node} 구조체의
	락을 잡고 있으므로, quiescent state 를 보고하려 하는 CPU 는 초기화가
	완료되기 전까지는 이 락을 잡지 못할 것이어서 평범한 시점에 quiescent
	state 를 보고하게 될 겁니다.
	\iffalse

	There are two cases of interest.

	In the first case, there is only a single \co{rcu_node}
	structure in the hierarchy.
	Since the CPU executing in \co{rcu_start_gp()} is currently
	holding that \co{rcu_node} structure's lock, the CPU
	attempting to report the quiescent state will not be able
	to acquire this lock until initialization is complete,
	at which point the quiescent state will be reported
	normally.
	\fi

	두번째 경우는 여러개의 \co{rcu_node} 구조체가 존재하고 quiescent state
	를 보고하려고 하는 CPU 에 연관된 leaf \co{rcu_node} 구조체는 이미 해당
	CPU 의 \co{->qsmask} bit 의 값이 0으로 설정되어 있는 경우입니다.
	따라서, 이 quiescent state 를 보고하려 하는 CPU 는 포기를 할 것이고,
	해당 CPU 의 나중의 quiescent state 가 새로운 grace period 에 적용될
	겁니다.
	\iffalse

	In the second case, there are multiple \co{rcu_node} structures,
	and the leaf \co{rcu_node} structure corresponding to the
	CPU that is attempting to report the quiescent state already
	has that CPU's \co{->qsmask} bit cleared.
	Therefore, the CPU attempting to report the quiescent state
	will give up, and some later quiescent state for that CPU
	will be applied to the new grace period.
	\fi
} \QuickQuizEnd

Line~15 는 다른 CPU 가 이 새로운 grace period 에 그 초기화가 완료되기 전에
종룔르 강제하는 것을 막기 위해 \co{->signaled} 필드를 \co{RCU_GP_INIT} 으로
설정합니다.
Line~16-18 은 이 새로운 grace period 에 종료를 강제할 다음 시도를 위한 스케쥴을
잡는데, 첫번째는 jiffies 단위로 그리고 두번째는 \co{rcu_pending} 호출의 횟수
단위로 입니다.
물론, 만약 그 시간 전에 이 grace period 가 자연스럽게 종료된다면, 그런 종료를
강제할 시도를 할 필요가 없어집니다.
Line~20 은 더 긴 단위의 진행 체크를 위한 스케쥴을 위해
\co{record_gp_stall_check_time()} 을 호출합니다---만약 이 grace period 가 이
시간 뒤로까지 연장된다면, 이는 에러로 취급될 겁니다.
Line~22 는 이 CPU 의 \co{rcu_data} 구조체가 새로운 grace period 를 관리하도록
초기화 하기 위해 \co{note_new_gpnum()} 을 호출합니다.
\iffalse

Line~15 sets the \co{->signaled} field to \co{RCU_GP_INIT} in order
to prevent any other CPU from attempting to force an end to the new
grace period before its initialization completes.
Lines~16-18 schedule the next attempt to force an end to the new
grace period, first in terms of jiffies and second in terms of the
number of calls to \co{rcu_pending}.
Of course, if the grace period ends naturally before that time,
there will be no need to attempt to force it.
Line~20 invokes \co{record_gp_stall_check_time()} to schedule a
longer-term progress check---if the grace period extends beyond this
time, it should be considered to be an error.
Line~22 invokes \co{note_new_gpnum()} in order to initialize this
CPU's \co{rcu_data} structure to account for the new grace period.
\fi

Line~23-26 은 이 CPU 의 모든 콜백들을 진행시켜서 이 새로운 grace period 의 종료
시에 모두 호출될 수 있게 만듭니다.
이는 콜백들의 가속화를 표시하는데, 다른 CPU 들은 \co{RCU_NEXT_READY_TAIL}
배치만을 옮겨서 현재 grace period 에 의해 처리될 수 있게 합니다;
\co{RCU_NEXT_TAIL} 은 그대신에 \co{RCU_NEXT_READY_TAIL} 배치로 진척되어야 할
필요가 있을 겁니다.
이 CPU 가 \co{RCU_NEXT_TAIL} 배치를 진척시킬 수 있는 이유는 이 CPU 가 정확히
언제 이 새로운 grace period 가 시작되는지 알기 때문입니다.
반면에, 다른 CPU 들은 새로운 grace period 의 시작과 새로운 RCU z콜백의 도착
사이의 race 를 올바르게 해결할 수 없을 겁니다.
\iffalse

Lines~23-26 advance all of this CPU's callbacks so that they will
be eligible to be invoked at the end of this new grace period.
This represents an acceleration of callbacks, as other CPUs would only
be able to move the \co{RCU_NEXT_READY_TAIL} batch to be serviced
by the current grace period; the \co{RCU_NEXT_TAIL} would instead
need to be advanced to the \co{RCU_NEXT_READY_TAIL} batch.
The reason that this CPU can accelerate the \co{RCU_NEXT_TAIL} batch
is that it knows exactly when this new grace period started.
In contrast, other CPUs would be unable to correctly resolve the
race between the start of a new grace period and the arrival of
a new RCU callback.
\fi

Line~27 은 계층 상에 하나의 \co{rcu_node} 구조체만이 존재하는지 체크해보고,
만약 그렇다면 line~28 에서 \co{->qsmask} bit 들을 연관된 모든 online CPU 들,
달리 말하자면 새로운 grace period 가 종료되기 위해서는 quiescent state 를
지나야 하는 모든 CPU 들로 설정합니다.
Line~29 는 root \co{rcu_node} 구조체의 락을 해제하고 line~30 에서 리턴합니다.
이 경우, gcc 의 dead-code elimination 이 line~32-46 에 제공될 것입니다.

그렇지 않다면, \co{rcu_node} 계층은 여러 구조체를 가지고 있어서 더 관여된
초기화 계획을 필요로 합니다.
Line~32 는 root \co{rcu_node} 구조체의 락을 해제하지만, 인터럽트가 비활성화
된채로 유지하고, 이어서 line~33 에서 명시된 \co{rcu_state} 구조체의
\co{->onofflock} 을 획득하여서 다른 동시에 수행중인 CPU-hotplug 오퍼레이션들이
RCU 에 관계된 state 를 조정하는 것을 막습니다.
\iffalse

Line~27 checks to see if there is but one \co{rcu_node} structure in
the hierarchy, and if so, line~28 sets the \co{->qsmask}
bits corresponding to all online CPUs, in other words, corresponding
to those CPUs that must pass through a quiescent state for the new
grace period to end.
Line~29 releases the root \co{rcu_node} structure's lock and line~30
returns.
In this case, gcc's dead-code elimination is expected to dispense with
lines~32-46.

Otherwise, the \co{rcu_node} hierarchy has multiple structures, requiring
a more involved initialization scheme.
Line~32 releases the root \co{rcu_node} structure's lock, but keeps
interrupts disabled, and then line~33 acquires the specified
\co{rcu_state} structure's \co{->onofflock}, preventing any
concurrent CPU-hotplug operations from manipulating RCU-specific state.
\fi

Line~34 는 \co{rnp_end} 지역변수가 첫번째 leaf \co{rcu_node} 구조체를
가리키도록 설정하는데, 이 구조체는 \co{->node} 배열에서 마지막 non-leaf
\co{rcu_node} 구조체를 곧바로 뒤따르는 \co{rcu_node} 구조체가 될수도 있습니다.
Line~35 는 \co{rnp_cur} 지역 변수가 root \co{rcu_node} 구조체를 가리키도록
만드는데, 이 구조체는 \co{->node} 배열의 첫번째 구조체가 될수도 있습니다.
Line~36 과 37 d은 이어서 모든 non-leaf \co{rcu_node} 구조체들을 횡단하며, 이
새로운 grace period 가 종료되기 위해서는 quiescent state 들을 지나가야만 하는
CPU 들을 갖는 더 낮은 단계의 \co{rcu_node} 구조체들에 연관된 bit 들을
설정합니다.
\iffalse

Line~34 sets the \co{rnp_end} local variable to reference the first
leaf \co{rcu_node} structure, which also happens to be the
\co{rcu_node} structure immediately following the last non-leaf
\co{rcu_node} structure in the \co{->node} array.
Line~35 sets the \co{rnp_cur} local variable to reference the root
\co{rcu_node} structure, which also happens to be first such structure
in the \co{->node} array.
Lines~36 and 37 then traverse all of the non-leaf \co{rcu_node} structures,
setting the bits corresponding to lower-level \co{rcu_node} structures
that have CPUs that must pass through quiescent states in order for
the new grace period to end.
\fi

\QuickQuiz{}
	이봐요!
	Figure~\ref{fig:app:rcuimpl:rcutreewt:Starting a Grace Period}
	의 line~37 의 non-leaf \co{rcu_node} 구조체의 상태 분배 시에 그 락들을
	잡아야 하지 않나요???
	\iffalse

	Hey!
	Shouldn't we hold the non-leaf \co{rcu_node} structures'
	locks when munging their state in line~37 of
	Figure~\ref{fig:app:rcuimpl:rcutreewt:Starting a Grace Period}???
	\fi
\QuickQuizAnswer{
	그 락들을 잡을 필요가 없습니다.
	그 이유는 다음과 같습니다:
	\iffalse

	There is no need to hold their locks.
	The reasoning is as follows:
	\fi
	\begin{enumerate}
	\item	이 새로운 grace period 는 끝날 수 없는데, 이 수행중인 CPU
		(초기화를 하고 있는) 는 quiescent state 를 지나지 않을 것이기
		때문입니다.
		따라서, 또다른 \co{rcu_start_gp()} 호출과의 race 는 없습니다.
	\item	이 수행중인 CPU 는 \co{->onofflock} 을 잡으므로, CPU-hotplug
		오퍼레이션과의 race 는 없습니다.
	\item	이 leaf \co{rcu_node} 구조체들은 아직 초기화 되지 않았고,
		따라서 이것들은 각자의 \co{->qsmask} bit 들이 값 해제되어
		있습니다.
		이 말은 quiescent state 를 보고하려 하는 모든 다른 CPU 들은
		leaf 단계에서 멈춰서 있게 될 것이고, 따라서 현재의 CPU 와
		non-leaf \co{rcu_node} 구조체들에서 race 를 겪지 않게 될것을
		의미합니다.
	\item	\co{rcu_node} 구조체들에 RCU tracing 함수들이 접근하긴 하지만,
		수정하진 않습니다.
		이 함수들과의 race 들은 따라서 무해합니다.
	\iffalse

	\item	The new grace period cannot end, because the running CPU
		(which is initializing it) won't pass through a
		quiescent state.
		Therefore, there is no race with another invocation
		of \co{rcu_start_gp()}.
	\item	The running CPU holds \co{->onofflock}, so there
		is no race with CPU-hotplug operations.
	\item	The leaf \co{rcu_node} structures are not yet initialized,
		so they have all of their \co{->qsmask} bits cleared.
		This means that any other CPU attempting to report
		a quiescent state will stop at the leaf level,
		and thus cannot race with the current CPU for non-leaf
		\co{rcu_node} structures.
	\item	The RCU tracing functions access, but do not modify,
		the \co{rcu_node} structures' fields.
		Races with these functions is therefore harmless.
	\fi
	\end{enumerate}
} \QuickQuizEnd

Line~38 은 지역 변수 \co{rnp_end} 를 마지막 leaf \co{rcu_node} 구조체로
설정하고, line~39 는 지역 변수 \co{rnp_cur} 가 첫번째 leaf \co{rcu_node}
구조체를 가리키도록 해서 line~40-44 의 루프가 \co{rcu_node} 계층의 모든 leaf
들을 횡단하도록 만듭니다.
이 루프의 각 단계마다 line~41 은 현재 leaf \co{rcu_node} 구조체의 락을 잡고,
line~42 는 (이 새로운 grace period 가 종료될 수 있기 전에 quiescent state 를
지나가야 만 하는) online CPU 들에 연관된 bit 들을 설정하고, line~43 에서 이
락을 놓습니다.
\iffalse

Line~38 sets local variable \co{rnp_end} to one past the last leaf
\co{rcu_node} structure, and line~39 sets local variable \co{rnp_cur}
to the first leaf \co{rcu_node} structure, so that the loop spanning
lines~40-44 traverses all leaves of the \co{rcu_node} hierarchy.
During each pass through this loop, line~41 acquires the current
leaf \co{rcu_node} structure's lock, line~42 sets the bits corresponding
to online CPUs (each of which must pass through a quiescent state
before the new grace period can end), and line~43 releases the lock.
\fi

\QuickQuiz{}
	Figure~\ref{fig:app:rcuimpl:rcutreewt:Starting a Grace Period}
	의 line~36-37 의 루프를 line~40-44 에 합치는 건 어떤가요?
	\iffalse

	Why can't we merge the loop spanning lines~36-37 with
	the loop spanning lines~40-44 in
	Figure~\ref{fig:app:rcuimpl:rcutreewt:Starting a Grace Period}?
	\fi
\QuickQuizAnswer{
	그렇게 할 거였다면, non-leaf \co{rcu_node} 구조체의 락을 불필요하게
	잡게 되거나 이 루프의 매 패스마다 leaf node 인지에 대한 보기 싫은
	체크를 해야 할 겁니다.
	(Quiescent state 를 보고하려 시도할 CPU 들과의 race 때문에 leaf
	\co{rcu_node} 구조체들의 락을 잡아야만 했음을 다시 말합니다.)

	그러나, 매우 큰 시스템에서의 경험은 그런 병합 작업이 실제로는 하는게
	나은 것으로 드러나게 할수도 있습니다.
	\iffalse

	If we were to do so, we would either be needlessly acquiring locks
	for the non-leaf \co{rcu_node} structures or would need
	ugly checks for a given node being a leaf node on each pass
	through the loop.
	(Recall that we must acquire the locks for the leaf
	\co{rcu_node} structures due to races with CPUs attempting
	to report quiescent states.)

	Nevertheless, it is quite possible that experience on very large
	systems will show that such merging is in fact the right thing
	to do.
	\fi
} \QuickQuizEnd

이어서 line~45 는 명시된 \co{rcu_state} 구조체의 \co{->signaled} 필드를
quiescent state 강제를 허용할 수 있도록 설정하고, line~46 에서 CPU-hotplug
오퍼레이션들이 RCU 상태를 조정할 수 있도록 하기 위해 \co{->onofflock} 을
해제합니다.
\iffalse

Line~45 then sets the specified \co{rcu_state} structure's \co{->signaled}
field to permit forcing of quiescent states, and
line~46 releases the \co{->onofflock} to permit CPU-hotplug
operations to manipulate RCU state.
\fi

\subsubsection{Reporting Quiescent States}
\label{app:rcuimpl:rcutreewt:Reporting Quiescent States}

이 계층적 RCU 구현은 quiescent state 들을 보고하는데에 다음의 함수들을 사용해서
계층화된 방법을 구현합니다:
\iffalse

This hierarchical RCU implementation implements a layered approach
to reporting quiescent states, using the following functions:
\fi
\begin{enumerate}
\item	특정 CPU 가 quiescent state 를 지날 때마다 ``rcu'' 와 ``rcu\_bh'' 를
	위해 각각 \co{rcu_qsctr_inc()} 와 \co{rcu_bh_qsctr_inc()} 가
	호출됩니다.
	Dynticks-idle 과 CPU-offline quiescent state 는 특별하게 처리되는데,
	그런 CPU 는 수행 중이지 않고, 따라서 스스로가 quiescent state 에 있다고
	보고할 수 없기 때문이란 걸 알아두시기 바랍니다.
\iffalse

\item	\co{rcu_qsctr_inc()} and \co{rcu_bh_qsctr_inc()}
	are invoked when a given CPU passes through a
	quiescent state for ``rcu'' and ``rcu\_bh'', respectively.
	Note that the dynticks-idle and CPU-offline quiescent states
	are handled specially, due to the fact that such a CPU
	is not executing, and thus is unable to report itself as
	being in a quiescent state.
\fi
\item	\co{rcu_check_quiescent_state()} 는 현재의 CPU 가 quiescent state 를
	지났는지 알아보고, 만약 그렇다면 \co{cpu_quiet()} 를 호출합니다.
\item	\co{cpu_quiet()} 는 명시된 CPU 가 quiescent state 를 지났음을 보고하기
	위해 \co{cpu_quiet_msk()} 를 호출합니다.
	이 명시된 CPU 는 현재의 CPU 이거나 offline CPU 여야만 합니다.
\item	\co{cpu_quiet_msk()} 는 명시된 CPU 들이 quiescent state 를 지났음을
	보고합니다.  이 CPU 들은 현재의 CPU 여야만 하지도, offline 이어야만
	하지도 않습니다.
\iffalse

\item	\co{rcu_check_quiescent_state()} checks to see if the current
	CPU has passed through a quiescent state, invoking \co{cpu_quiet()}
	if so.
\item	\co{cpu_quiet()} reports the specified CPU as having passed
	through a quiescent state by invoking \co{cpu_quiet_msk()}.
	The specified CPU must either be the current CPU or an offline CPU.
\item	\co{cpu_quiet_msk()} reports the specified vector of CPUs as
	having passed through a quiescent state.  The CPUs in the
	vector need not be the current CPU, nor must they be offline.
\fi
\end{enumerate}

이 함수들은 아래에 설명됩니다.
\iffalse

Each of these functions is described below.
\fi

\begin{figure}[tbp]
{ \scriptsize
\begin{verbatim}
  1 void rcu_qsctr_inc(int cpu)
  2 {
  3   struct rcu_data *rdp = &per_cpu(rcu_data, cpu);
  4   rdp->passed_quiesc = 1;
  5   rdp->passed_quiesc_completed = rdp->completed;
  6 }
  7
  8 void rcu_bh_qsctr_inc(int cpu)
  9 {
 10   struct rcu_data *rdp = &per_cpu(rcu_bh_data, cpu);
 11   rdp->passed_quiesc = 1;
 12   rdp->passed_quiesc_completed = rdp->completed;
 13 }
\end{verbatim}
}
\caption{Code for Recording Quiescent States}
\label{fig:app:rcuimpl:rcutreewt:Code for Recording Quiescent States}
\end{figure}

Figure~\ref{fig:app:rcuimpl:rcutreewt:Code for Recording Quiescent States}
는 \co{rcu_qsctr_inc()} 와 \co{rcu_bh_qsctr_inc()} 함수의 코드를 보이는데, 이
함수들은 현재 CPU 의 quiescent state 통과를 알립니다.

\co{rcu_qsctr_inc()} 의 line~3 는 명시된 CPU 의 \co{rcu_data} 구조체
(``rcu\_bh'' 에 반대되는 ``rcu'' 에 연관된) 로의 포인터를 얻어옵니다.
Line~4 는 \co{->passed_quiesc} 필드에 값을 할당해서 이 quiescent state 를
기록합니다.
Line~5 는 \co{->passed_quiesc_completed} 필드의 값을 이 CPU 가 알고 있는
마지막으로 완료된 grace period (\co{rcu_data} 구조체의 \co{->completed} 필드에
저장되어 있는) 의 숫자로 할당합니다.
\iffalse

Figure~\ref{fig:app:rcuimpl:rcutreewt:Code for Recording Quiescent States}
shows the code for \co{rcu_qsctr_inc()} and \co{rcu_bh_qsctr_inc()},
which note the current CPU's passage through a quiescent state.

Line~3 of \co{rcu_qsctr_inc()} obtains a pointer to the specified
CPU's \co{rcu_data} structure (which corresponds to ``rcu'' as opposed
to ``rcu\_bh'').
Line~4 sets the \co{->passed_quiesc} field, recording the
quiescent state.
Line~5 sets the \co{->passed_quiesc_completed} field to the number
of the last completed grace period that this CPU knows of (which is
stored in the \co{->completed} field of the \co{rcu_data}
structure).
\fi

\co{rcu_bh_qsctr_inc()} 함수는 같은 방식으로 동작하는데, 유일한 차이점은
line~10 에서 \co{rcu_data} 포인터를 \co{rcu_data} per-CPU 변수가 아니라
\co{rcu_bh_data} per-CPU 변수로부터 가져온다는 점입니다.
\iffalse

The \co{rcu_bh_qsctr_inc()} function operates in the same manner,
the only difference being that line~10 obtains the \co{rcu_data}
pointer from the \co{rcu_bh_data} per-CPU variable rather than
the \co{rcu_data} per-CPU variable.
\fi

\begin{figure}[tbp]
{ \scriptsize
\begin{verbatim}
  1 static void
  2 rcu_check_quiescent_state(struct rcu_state *rsp,
  3                           struct rcu_data *rdp)
  4 {
  5   if (check_for_new_grace_period(rsp, rdp))
  6     return;
  7   if (!rdp->qs_pending)
  8     return;
  9   if (!rdp->passed_quiesc)
 10     return;
 11   cpu_quiet(rdp->cpu, rsp, rdp,
 12             rdp->passed_quiesc_completed);
 13 }
\end{verbatim}
}
\caption{Code for {\tt rcu\_check\_quiescent\_state()}}
\label{fig:app:rcuimpl:rcutreewt:Code for rcu-check-quiescent-state}
\end{figure}

Figure~\ref{fig:app:rcuimpl:rcutreewt:Code for rcu-check-quiescent-state}
는 언제 다른 CPU 들이 새로운 grace period 를 시작했는지 파악하고 RCU 에게 이
CPU 를 위한 최근의 quiescent state 들을 알리기 위해
(Section~\ref{app:rcuimpl:rcutreewt:rcu-process-callbacks} 에서 설명된)
\co{rcu_process_callbacks()} 에서 호출되는 \co{rcu_check_quiescent_state()}
함수의 코드를 보이고 있습니다.

Line~5 는 새로운 grace period 가 어떤 다른 CPU 에 의해 시작된 것을 체크하기
위해 \co{check_for_new_grace_period()} 를 호출하고, 또한 이 CPU 의 지역 상태를
그 새로운 grace period 를 처리할 수 있도록 업데이트 합니다.
만약 새로운 grace period 가 막 시작되었다면, line~6 는 리턴합니다.
Line~7 은 RCU rk duwjsgl guswo CPU 로부터의 queiscent state 를 기대하고 있는지
검사하고, 만약 그렇지 않다면 line~8 에서 리턴합니다.
Line~9 는 이 CPU 가 현재 grace period 의 시작 이래로 quiescent state 를
지났는지 (달리 말하면, \co{rcu_qsctr_inc()} 또는 \co{rcu_bh_qsctr_inc()} 가
각각 ``rcu'' 와 ``rcu\_bh'' 를 위해 호출되었는지) 체크하고, 만약 그렇지 않다면
line~10 에서 리턴합니다.
\iffalse

Figure~\ref{fig:app:rcuimpl:rcutreewt:Code for rcu-check-quiescent-state}
shows the code for \co{rcu_check_quiescent_state()}, which is invoked
from \co{rcu_process_callbacks()}
(described in Section~\ref{app:rcuimpl:rcutreewt:rcu-process-callbacks})
in order to determine when other CPUs have started a new grace period
and to inform RCU of recent quiescent states for this CPU.

Line~5 invokes \co{check_for_new_grace_period()} to check for
a new grace period having been started by some other CPU, and also
updating this CPU's local state to account for that new grace period.
If a new grace period has just started, line~6 returns.
Line~7 checks to see if RCU is still expecting a quiescent state from
the current CPU, and line~8 returns if not.
Line~9 checks to see if this CPU has passed through a quiescent state
since the start of the current grace period (in other words, if
\co{rcu_qsctr_inc()} or \co{rcu_bh_qsctr_inc()} have been invoked
for ``rcu'' and ``rcu\_bh'', respectively), and line~10 returns if not.
\fi

따라서, 앞서 알려진 grace period 가 여전히 효과를 발휘하고 있고, 이 CPU 가 이
grace period 를 끝나게 하기 위해서는 quiescent state 를 지나야 하며, 이 CPU 가
그런 quiescent state 를 지났을 경우에만 line~11 로 실행 흐름이 넘어옵니다.
이 경우, line~11-12 는 이 quiescent state 를 RCU 에 보고하기 위해
\co{cpu_quiet()} 를 호출합니다.
\iffalse

Therefore, execution reaches line~11 only if a previously noted grace
period is still in effect, if this CPU needs to pass through a
quiescent state in order to allow this grace period to end, and
if this CPU has passed through such a quiescent state.
In this case, lines~11-12 invoke \co{cpu_quiet()} in order to report
this quiescent state to RCU.
\fi

\QuickQuiz{}
	Figure~\ref{fig:app:rcuimpl:rcutreewt:Code for rcu-check-quiescent-state}
	의 line~11-12 가 앞의 grace period 로부터의 quiescent state 를 현재의
	grace period 에게 보고하는 것을 방지하는 건 무엇인가요?
	\iffalse

	What prevents lines~11-12 of
	Figure~\ref{fig:app:rcuimpl:rcutreewt:Code for rcu-check-quiescent-state}
	from reporting a quiescent state from a prior
	grace period against the current grace period?
	\fi
\QuickQuizAnswer{
	만약 그런 일이 일어날 수 있다면, 그건 심각한 버그인데, 해당 CPU 는 현재
	grace period 의 시작보다 먼저 시작한 RCU read-side 크리티컬 섹션에 있을
	수 있기 때문입니다.

	이런 경우를 위해 CPU 에 대해 신경써야 할 경우들이 몇가지 있습니다:
	\iffalse

	If this could occur, it would be a serious bug, since the
	CPU in question might be in an RCU read-side critical section
	that started before the beginning of the current grace period.

	There are several cases to consider for the CPU in question:
	\fi
	\begin{enumerate}
	\item	Online 으로 남아있고 계속해서 활성 상태.
	\item	최소한 현재 grace period 의 한 부분동안은 dynticks-idle 모드에
		있었음.
	\item	최소한 현재 grace period 의 한 부분동안은 offline 이었음.
	\iffalse

	\item	It remained online and active throughout.
	\item	It was in dynticks-idle mode for at least part of the current
		grace period.
	\item	It was offline for at least part of the current grace period.
	\fi
	\end{enumerate}

	첫번째 경우에는, 앞의 grace period 는 이 CPU 가 명시적으로 quiescent
	state 를 보고하지 않았다면, 즉 \co{->qs_pending} 이 0이 아니라면 종료될
	수가 없습니다.
	이는 다시 말해 line~7-8 이 리턴할 것을 의미하며, 따라서 제어 흐름은
	\co{check_for_new_grace_period()} 가 새로운 grace period 를 알리지
	않았다면 \co{cpu_quiet()} 까지 도달하지 않음을 의미합니다.
	하지만, 만약 현재 grace period 가 알려졌다면, \co{->passed_quiesc} 를 0
	으로 만들었을 것이고, 이 경우 line~9-10 이 리턴하게 되므로, 이 경우에도
	\co{cpu_quiet()} 는 호출되지 않습니다.
	마지막으로, \co{->passed_quiesc} 가 호출될 수 있는 유일한 방법은
	\co{rcu_check_callbacks()} 가
	Figure~\ref{fig:app:rcuimpl:rcutreewt:Code for rcu-check-quiescent-state}
	의 \co{rcu_check_quiescent_state()} 의 line~5 와 9 사이에서
	scheduling-clock 인터럽트가 발생해서 \co{rcu_check_callbacks()} 가
	호출된 경우일 겁니다.
	하지만, 이는 \emph{현재의} grace period 중에 quiescent state 가
	발생하는 경우가 될 것으로, 이 경우는 현재 grace period 에 대해
	보고되는게 완전히 합법적입니다.
	따라서 이 경우는 완전히 해결됩니다.
	\iffalse

	In the first case, the prior grace period could not have
	ended without this CPU explicitly reporting a quiescent
	state, which would leave \co{->qs_pending} zero.
	This in turn would mean that lines~7-8 would return, so
	that control would not reach \co{cpu_quiet()} unless
	\co{check_for_new_grace_period()} had noted the new grace
	period.
	However, if the current grace period had been noted, it would
	also have set \co{->passed_quiesc} to zero, in which case
	lines~9-10 would have returned, again meaning that \co{cpu_quiet()}
	would not be invoked.
	Finally, the only way that \co{->passed_quiesc} could be invoked
	would be if \co{rcu_check_callbacks()} was invoked by
	a scheduling-clock interrupt that occurred somewhere between
	lines~5 and 9 of \co{rcu_check_quiescent_state()} in
	Figure~\ref{fig:app:rcuimpl:rcutreewt:Code for rcu-check-quiescent-state}.
	However, this would be a case of a quiescent state occurring
	in the \emph{current} grace period, which would be totally
	legitimate to report against the current grace period.
	So this case is correctly covered.
	\fi

	문제시 되는 CPU 가 새로운 quiescent state 의 일부를 dynticks-idle
	모드로 보낸 경우인 두번째 경우에 있어서는, dynticks-idle 모드는
	extended quiescent state 이며, 따라서 이 quiescent state 를 현재의
	grace period 에게 보고하는 것은 허용 가능한 일임을 알아두시기 바랍니다.
	
	문제의 CPU 가 새로운 quiescent state 의 일부를 offline 으로 보낸 경우인
	세번째 경우에 대해서는, offline CPU 들은 extended quiescent state 에
	있는 것이어서 이번에도 현재의 grace period 에게 보고되는 것은 허용
	가능하 일입니다.

	따라서 앞의 grace period 로부터의 quiescent state 는 현재의 grace
	period 에게 보고되지 않습니다.
	\iffalse

	In the second case, where the CPU in question spent part of
	the new quiescent state in dynticks-idle mode, note that
	dynticks-idle mode is an extended quiescent state, hence
	it is again permissible to report this quiescent state against
	the current grace period.

	In the third case, where the CPU in question spent part of the
	new quiescent state offline, note that offline CPUs are in
	an extended quiescent state, which is again permissible to
	report against the current grace period.

	So quiescent states from prior grace periods are never reported
	against the current grace period.
	\fi
} \QuickQuizEnd

\begin{figure}[tbp]
{ \scriptsize
\begin{verbatim}
  1 static void
  2 cpu_quiet(int cpu, struct rcu_state *rsp,
  3           struct rcu_data *rdp, long lastcomp)
  4 {
  5   unsigned long flags;
  6   unsigned long mask;
  7   struct rcu_node *rnp;
  8
  9   rnp = rdp->mynode;
 10   spin_lock_irqsave(&rnp->lock, flags);
 11   if (lastcomp != ACCESS_ONCE(rsp->completed)) {
 12     rdp->passed_quiesc = 0;
 13     spin_unlock_irqrestore(&rnp->lock, flags);
 14     return;
 15   }
 16   mask = rdp->grpmask;
 17   if ((rnp->qsmask & mask) == 0) {
 18     spin_unlock_irqrestore(&rnp->lock, flags);
 19   } else {
 20     rdp->qs_pending = 0;
 21     rdp = rsp->rda[smp_processor_id()];
 22     rdp->nxttail[RCU_NEXT_READY_TAIL] =
 23         rdp->nxttail[RCU_NEXT_TAIL];
 24     cpu_quiet_msk(mask, rsp, rnp, flags);
 25   }
 26 }
\end{verbatim}
}
\caption{Code for {\tt cpu\_quiet()}}
\label{fig:app:rcuimpl:rcutreewt:Code for cpu-quiet}
\end{figure}

Figure~\ref{fig:app:rcuimpl:rcutreewt:Code for cpu-quiet}
는 특정 CPU 를 위한 quiescent state 를 보고하는 데에 사용되는 \co{cpu_quiet()}
를 보입니다.
앞에서도 이야기 되었듯이, 이는 현재 수행중인 CPU 또는 계속해서 offline 으로
상태가 유지될 것이 보장되는 CPU 여야만 합니다.

Line~9 는 이 CPU 에 책임을 갖는 leaf \co{rcu_node} 구조체로의 포인터를
가져옵니다.
Line~10 은 이 leaf \co{rcu_node} 구조체의 락을 획득하고 인터럽트를 비활성화
시킵니다.
Line~11 은 이 명시된 grace period 가 여전히 효력을 발휘하고 있음을 분명히 하고,
만약 그렇지 않다면 line~11 은 이 CPU 가 quiescent state 를 지나왔다는 알림을
제거해 버리고 (이미 효력이 없는 grace period 에 속해 있기 때문이죠), line~13 은
이 락을 해제하고 인터럽트를 재활성화 시킨 후, line~14 에서 호출자에게
리턴합니다.
\iffalse

Figure~\ref{fig:app:rcuimpl:rcutreewt:Code for cpu-quiet}
shows \co{cpu_quiet}, which is used to report a quiescent state
for the specified CPU.
As noted earlier, this must either be the currently running CPU
or a CPU that is guaranteed to remain offline throughout.

Line~9 picks up a pointer to the leaf \co{rcu_node} structure
responsible for this CPU.
Line~10 acquires this leaf \co{rcu_node} structure's lock and
disables interrupts.
Line~11 checks to make sure that the specified grace period is
still in effect, and, if not, line~11 clears the indication that
this CPU passed through a quiescent state (since it belongs to
a defunct grace period), line~13 releases the lock and re-enables
interrupts, and line~14 returns to the caller.
\fi

그렇지 않다면, line~16 에서 명시된 CPU 의 bit set 을 가지고 mask 를 만듭니다.
Line~17 에서 이 bit 이 여전히 leaf \co{rcu_node} 구조체에 있는지 확인하고, 만약
그렇지 않다면 line~18 에서 락을 해제하고 인터럽트를 재활성화 시킵니다.

반면에, 만약 이 CPU 의 bit 이 여전히 존재한다면, line~20 에서 \co{->qs_pending}
의 bit 들을 제거함으로써 이 CPU 가 이 grace period 를 위한 quiescent state 를
지났음을 반영시킵니다.
Line~21 에서는 지역 변수 \co{rdp} 를 수행중인 CPU 의 \co{rcu_data} 구조체로의
포인터로 덮어쓰고, line~22-23 에서는 이 수행중인 CPU 의 RCU 콜백들을
업데이트해서 모든 콜백들이 다음 grace period 에 연관되지 않게 합니다.
마지막으로, line~24 에서는 \co{rcu_node} 계층의 bit 들을 제거해서, 적절하다면
심지어 새로 시작하는 것이라도 현재의 grace period 를 종료시킵니다.
\co{cpu_quiet()} 는 락을 해제하고 인터럽트를 재활성화 시킴을 알아두시기
바랍니다.
\iffalse

Otherwise, line~16 forms a mask with the specified CPU's bit set.
Line~17 checks to see if this bit is still set in the leaf
\co{rcu_node} structure, and, if not, line~18 releases the lock
and re-enables interrupts.

On the other hand, if the CPU's bit is still set, line~20 clears
\co{->qs_pending}, reflecting that this CPU has passed through
its quiescent state for this grace period.
Line~21 then overwrites local variable \co{rdp} with a pointer to
the running CPU's \co{rcu_data} structure, and lines~22-23
updates the running CPU's RCU callbacks so that all those not yet
associated with a specific grace period
be serviced by the next grace period.
Finally, line~24 clears bits up the \co{rcu_node} hierarchy,
ending the current grace period if appropriate and perhaps even
starting a new one.
Note that \co{cpu_quiet()} releases the lock and re-enables interrupts.
\fi

\QuickQuiz{}
	Figure~\ref{fig:app:rcuimpl:rcutreewt:Code for cpu-quiet}
	의 line~22-23 은 수행중인 CPU 의 RCU 콜백들을 촉진시켜도 안전함을 아는
	거죠?
	\iffalse

	How do lines~22-23 of
	Figure~\ref{fig:app:rcuimpl:rcutreewt:Code for cpu-quiet}
	know that it is safe to promote the running CPU's RCU
	callbacks?
	\fi
\QuickQuizAnswer{
	명시된 CPU 는 아직 quiescent state 를 지나지 않았으므로, 그리고 우리가
	연관된 leaf node 의 락을 잡고 있으므로, 우리는 현재의 grace period 가
	아직 끝나지 못했을 것임을 알고 있습니다.
	따라서, 현재 대기열에 들어있는 콜백들은 모두 다음 grace period 가
	시작된 후에 등록되었더라도 그것들은 이미 대기열에 들어와 있고 다음
	grace period 는 아직 시작되지 않았으므로 안전함을 알 수 있습니다.
	\iffalse

	Because the specified CPU has not yet passed through a quiescent
	state, and because we hold the corresponding leaf node's lock,
	we know that the current grace period cannot possibly have
	ended yet.
	Therefore, there is no danger that any of the callbacks currently
	queued were registered after the next grace period started, given
	that they have already been queued and the next grace period
	has not yet started.
	\fi
} \QuickQuizEnd

\begin{figure}[tbp]
{ \scriptsize
\begin{verbatim}
  1 static void
  2 cpu_quiet_msk(unsigned long mask, struct rcu_state *rsp,
  3               struct rcu_node *rnp, unsigned long flags)
  4   __releases(rnp->lock)
  5 {
  6   for (;;) {
  7     if (!(rnp->qsmask & mask)) {
  8       spin_unlock_irqrestore(&rnp->lock, flags);
  9       return;
 10     }
 11     rnp->qsmask &= ~mask;
 12     if (rnp->qsmask != 0) {
 13       spin_unlock_irqrestore(&rnp->lock, flags);
 14       return;
 15     }
 16     mask = rnp->grpmask;
 17     if (rnp->parent == NULL) {
 18       break;
 19     }
 20     spin_unlock_irqrestore(&rnp->lock, flags);
 21     rnp = rnp->parent;
 22     spin_lock_irqsave(&rnp->lock, flags);
 23   }
 24   rsp->completed = rsp->gpnum;
 25   rcu_process_gp_end(rsp, rsp->rda[smp_processor_id()]);
 26   rcu_start_gp(rsp, flags);
 27 }
\end{verbatim}
}
\caption{Code for {\tt cpu\_quiet\_msk()}}
\label{fig:app:rcuimpl:rcutreewt:Code for cpu-quiet-msk}
\end{figure}

Figure~\ref{fig:app:rcuimpl:rcutreewt:Code for cpu-quiet-msk}
는 CPU 들의 quiescent state 동안의, 인자 \co{mask} 에 의해 알려지는 통과를
\co{rcu_node} 계층에 반영시키는 \co{cpu_quiet_msk()} 를 보이고 있습니다.
인자 \co{rnp} 는 명시된 CPU 들에 연관된 leaf \co{rcu_node} 구조체 임을 알아
두시기 바랍니다.
\iffalse

Figure~\ref{fig:app:rcuimpl:rcutreewt:Code for cpu-quiet-msk}
shows \co{cpu_quiet_msk()}, which updates the \co{rcu_node}
hierarchy to reflect the passage of the CPUs indicated by
argument \co{mask} through their respective quiescent states.
Note that argument \co{rnp} is the leaf \co{rcu_node} structure
corresponding to the specified CPUs.
\fi

\QuickQuiz{}
	Figure~\ref{fig:app:rcuimpl:rcutreewt:Code for cpu-quiet-msk}
	의 line~2 의 인자 \co{mask} 는 unisgned long 인데, 이게 어떻게 64 개
	이상의 CPU 들을 갖는 시스템들을 다룰 수 있겠습니까?
	\iffalse

	Given that argument \co{mask} on line~2 of
	Figure~\ref{fig:app:rcuimpl:rcutreewt:Code for cpu-quiet-msk}
	is an unsigned long, how can it possibly deal with systems
	with more than 64 CPUs?
	\fi
\QuickQuizAnswer{
	\co{mask} 는 명시된 leaf \co{rcu_node} 구조체에만 상관있으므로, 해당
	\co{rcu_node} 구조체에 연관된 CPU 들을 표현할 수 있을만큼만 크면
	충분합니다.
	최대 64 개의 CPU 들이 (32-bit 시스템이라면 32 CPU 들) 하나의
	\co{rcu_node} 구조체에 연관될 수 있으므로, unsigned long \co{mask}
	인자로도 충분합니다.
	\iffalse

	Because \co{mask} is specific to the specified leaf \co{rcu_node}
	structure, it need only be large enough to represent the
	CPUs corresponding to that particular \co{rcu_node} structure.
	Since at most 64 CPUs may be associated with a given
	\co{rcu_node} structure (32 CPUs on 32-bit systems),
	the unsigned long \co{mask} argument suffices.
	\fi
} \QuickQuizEnd

Line~4 는 \co{sparse} 를 위한 주석으로, \co{cpu_quiet_msk()} 가 leaf
\co{rcu_node} 구조체의 락을 해제함을 알립니다.
\iffalse

Line~4 is annotation for the \co{sparse} utility, indicating
that \co{cpu_quiet_msk()} releases the leaf \co{rcu_node}
structure's lock.
\fi

\begin{figure*}[tb]
\centering
\resizebox{6in}{!}{\includegraphics{appendix/rcuimpl/RCUTreeQSScan}}
\caption{Scanning {\tt rcu\_node} Structures When Applying Quiescent States}
\label{fig:app:rcuimpl:rcutree:Scanning rcu-node Structures When Applying Quiescent States}
\end{figure*}

Line~6-23 의 루프의 각 단계는 \co{rcu_node} 계층의 한 단계씩을 처리하는데,
Figure~\ref{fig:app:rcuimpl:rcutree:Scanning rcu-node Structures When Applying Quiescent States}
의 파란 화살표로 보여진 것처럼 데이터 구조체들을 횡단합니다.

Line~7 은 \co{mask} 의 모든 bit 들이 현재 \co{rcu_node} 구조체의 \co{->qsmask}
필드에 지워져 있는지 확인하고, 만약 그렇다면, line~8 에서 이 락을 해제하고
인터럽트를 재활성화 시킨 후, line~9 에서 호출자에게 리턴합니다.
만약 그렇지 않다면, line~11 은 \co{mask} 로 명시된 bit 들을 현재의
\co{rcu_node} 구조체의 \co{qsmask} 필드로부터 지웁니다.
Line~12 는 이어서 \co{->qsmask} 에 더 남아 있는 bit 이 있는지 보고, 만약
그렇다면 line~13 에서 그 락을 해제하고 인터럽트를 재활성화 한 후, line~14 에서
호출자에게 리턴합니다.
\iffalse

Each pass through the loop spanning lines~6-23 does the required
processing for one level of the \co{rcu_node} hierarchy, traversing
the data structures as shown by the blue arrow in
Figure~\ref{fig:app:rcuimpl:rcutree:Scanning rcu-node Structures When Applying Quiescent States}.

Line~7 checks to see if all of the bits in \co{mask} have already
been cleared in the current \co{rcu_node} structure's \co{->qsmask}
field, and, if so, line~8 releases the lock and re-enables interrupts,
and line~9 returns to the caller.
If not, line~11 clears the bits specified by \co{mask} from the current
\co{rcu_node} structure's \co{qsmask} field.
Line~12 then checks to see if there are more bits remaining
in \co{->qsmask}, and, if so, line~13 releases the lock and re-enables
interrupts, and line~14 returns to the caller.
\fi

그렇지 않다면, \co{rcu_node} 계층의 다음 단계로 넘어갈 필요가 있습니다.
이 다음 단계로 가기 위한 준비 단계로, line~16 에서는 현재의 \co{rcu_node}
구조체에 그 부모로부터 연관되는 하나의 bit 을 설정한 mask 를 준비합니다.
Line~17 은 실제로 현재 \co{rcu_node} 구조체의 부모가 존재하는지 알아보고, 만약
그렇지 않다면, line~18 은 이 루프를 깨고 나갑니다.
반대로, 부모 \co{rcu_node} 구조체가 존재한다면 line~20 에서는 현재
\co{rcu_node} 구조체의 락을 해제하고 line~21 에서 \co{rnp} 지역 변수를 부모로
진행시키고, line~22 에서 부모의 락을 잡습니다.
실행 흐름은 이어서 line~7 의 루프의 시작점으로 돌아갑니다.
\iffalse

Otherwise, it is necessary to advance up to the next level of the
\co{rcu_node} hierarchy.
In preparation for this next level, line~16 places a mask with the
single bit set corresponding to the current \co{rcu_node} structure within
its parent.
Line~17 checks to see if there in fact is a parent for the current
\co{rcu_node} structure, and, if not, line~18 breaks from the
loop.
On the other hand, if there is a parent \co{rcu_node} structure,
line~20 releases the current \co{rcu_node} structure's lock,
line~21 advances the \co{rnp} local variable to the parent,
and line~22 acquires the parent's lock.
Execution then continues at the beginning of the loop on line~7.
\fi

Line~18 이 루프를 깬다면, 우린 현재의 grace period 가 끝났음을 알게 되는데,
root \co{rcu_node} 구조체의 모든 bit 들이 지워질 수 있는 유일한 방법은 모든 CPU
들이 quiescent state 를 지났을 때이기 때문입니다.
이 경우에, line~24 는 \co{rcu_state} 구조체의 \co{->complted} 필드를 새로이
끝난 grace period 의 숫자에 맞도록 업데이트 해서, 해당 grace period 가 실제로
끝났음을 알립니다.
Line~24 는 이어서 수행중인 CPU 의 RCU 콜백들을 진행시키기 위해
\co{rcu_process_gp_end()} 를 실행하고, 마지막으로 line~26 은 새로운 grace
period 를 시작하기 위해 \co{rcu_start_gp()} 를 실행합니다.
\iffalse

If line~18 breaks from the loop, we know that the current grace period
has ended, as the only way that all bits can be cleared in the
root \co{rcu_node} structure is if all CPUs have passed through
quiescent states.
In this case, line~24 updates the \co{rcu_state} structure's
\co{->completed} field to match the number of the newly ended grace
period, indicating that the grace period has in fact ended.
Line~24 then invokes \co{rcu_process_gp_end()} to advance the
running CPU's RCU callbacks,
and, finally, line~26 invokes \co{rcu_start_gp()} in order to
start a new grace period should any remaining callbacks on the currently
running CPU require one.
\fi

\begin{figure}[tbp]
{ \scriptsize
\begin{verbatim}
  1 static void rcu_do_batch(struct rcu_data *rdp)
  2 {
  3   unsigned long flags;
  4   struct rcu_head *next, *list, **tail;
  5   int count;
  6
  7   if (!cpu_has_callbacks_ready_to_invoke(rdp))
  8     return;
  9   local_irq_save(flags);
 10   list = rdp->nxtlist;
 11   rdp->nxtlist = *rdp->nxttail[RCU_DONE_TAIL];
 12   *rdp->nxttail[RCU_DONE_TAIL] = NULL;
 13   tail = rdp->nxttail[RCU_DONE_TAIL];
 14   for (count = RCU_NEXT_SIZE - 1; count >= 0; count--)
 15     if (rdp->nxttail[count] ==
 16         rdp->nxttail[RCU_DONE_TAIL])
 17       rdp->nxttail[count] = &rdp->nxtlist;
 18   local_irq_restore(flags);
 19   count = 0;
 20   while (list) {
 21     next = list->next;
 22     prefetch(next);
 23     list->func(list);
 24     list = next;
 25     if (++count >= rdp->blimit)
 26       break;
 27   }
 28   local_irq_save(flags);
 29   rdp->qlen -= count;
 30   if (list != NULL) {
 31     *tail = rdp->nxtlist;
 32     rdp->nxtlist = list;
 33     for (count = 0; count < RCU_NEXT_SIZE; count++)
 34       if (&rdp->nxtlist == rdp->nxttail[count])
 35         rdp->nxttail[count] = tail;
 36       else
 37         break;
 38   }
 39   if (rdp->blimit == LONG_MAX && rdp->qlen <= qlowmark)
 40     rdp->blimit = blimit;
 41   local_irq_restore(flags);
 42   if (cpu_has_callbacks_ready_to_invoke(rdp))
 43     raise_softirq(RCU_SOFTIRQ);
 44 }
\end{verbatim}
}
\caption{Code for {\tt rcu\_do\_batch()}}
\label{fig:app:rcuimpl:rcutreewt:Code for rcu-do-batch}
\end{figure}

Figure~\ref{fig:app:rcuimpl:rcutreewt:Code for rcu-do-batch}
는 grace period 가 끝난 RCU 콜백들을 호출하는 \co{rcu_do_batch()} 함수를
보입니다.
수행중인 CPU 위에서의 콜백들만이 호출됩니다---다른 CPU 들은 각자의 콜백들을
호출해야만 합니다.
\iffalse

Figure~\ref{fig:app:rcuimpl:rcutreewt:Code for rcu-do-batch}
shows \co{rcu_do_batch()}, which invokes RCU callbacks
whose grace periods have ended.
Only callbacks on the running CPU will be invoked---other CPUs must
invoke their own callbacks.
\fi

\QuickQuiz{}
	Dynticks-idle 이나 offline CPU 들에 있는 RCU 콜백들은 어떻게
	호출되나요?
	\iffalse

	How do RCU callbacks on dynticks-idle or offline CPUs
	get invoked?
	\fi
\QuickQuizAnswer{
	그것들은 호출되지 않습니다.
	RCU 콜백들을 가진 CPU 들은 dynticks-idle 모드에 들어갈 수 없게
	되어있으므로, dynticks-idle CPU 들은 RCU 콜백을 가질 수 없습니다.
	CPU 가 offline 이 될 때에는, 이것들의 RCU 콜백들은 online CPU 로
	넘어가고, 따라서 offline CPU 는 역시 RCU 콜백을 가질 수 없습니다.
	따라서, dynticks-idle 이나 offline CPU 에서의 콜백들은 호출될 필요가
	없습니다.
	\iffalse

	They don't.
	CPUs with RCU callbacks are not permitted to enter dynticks-idle
	mode, so dynticks-idle CPUs never have RCU callbacks.
	When CPUs go offline, their RCU callbacks are migrated to
	an online CPU, so offline CPUs never have RCU callbacks, either.
	Thus, there is no need to invoke callbacks on dynticks-idle
	or offline CPUs.
	\fi
} \QuickQuizEnd

Line~7 은 이 CPU 가 grace period 가 종료된 RCU 콜백들을 가지고 있는지 보기 위해
\co{cpu_has_callbacks_ready_to_invoke()} 을 호출하고, 만약 그렇지 않다면 line~8
에서 리턴합니다.
Line~9 와 18 은 인터럽트를 각각 비활성화시키고 재활성화 시킵니다.
Line~11-13 은 호출될 준비가 된 콜백들을 \co{->nxtlist} 에서 제거하고,
line~14-17 은 tail 포인터에 모든 필요한 조정을 가합니다.
\iffalse

Line~7 invokes \co{cpu_has_callbacks_ready_to_invoke()} to see if
this CPU has any RCU callbacks whose grace period has completed,
and, if not, line~8 returns.
Lines~9 and 18 disable and re-enable interrupts, respectively.
Lines~11-13 remove the ready-to-invoke callbacks from \co{->nxtlist},
and lines~14-17 make any needed adjustments to the tail pointers.
\fi

\QuickQuiz{}
	Figure~\ref{fig:app:rcuimpl:rcutreewt:Code for rcu-do-batch}
	의 line~14-17 은 왜 tail 포인터를 조정하는 거죠?
	\iffalse

	Why would lines~14-17 in
	Figure~\ref{fig:app:rcuimpl:rcutreewt:Code for rcu-do-batch}
	need to adjust the tail pointers?
	\fi
\QuickQuizAnswer{
	어떤 tail 포인터라도 호출될 준비가 된 sublist 의 마지막 콜백을 가리키고
	있었다면, 그것들은 \co{->nxtlist} 포인터를 가리키도록 바뀌어야만
	합니다.
	이 상황은 호출될 준비가 된 sublist 를 따르는 sublist 가 텅 비었을 경우
	발생합니다.
	\iffalse

	If any of the tail pointers reference the last callback
	in the sublist that was ready to invoke, they must be
	changed to instead reference the \co{->nxtlist} pointer.
	This situation occurs when the sublists
	immediately following the ready-to-invoke sublist are empty.
	\fi
} \QuickQuizEnd

Line~19 는 실제로 호출될 콜백의 수를 셀 준비를 위해 \co{count} 로컬 변수를
0으로 초기화 합니다.
Line~20-27 의 루프의 각 단계는 콜백을 수행하고 그 수를 세는데, line~25-26 은
너무 많은 콜백들이 한번에 호출될 예정이면 이 루프를 빠져나갑니다 (이렇게
함으로써 반응성을 지킵니다).
이 함수의 나머지 부분은 이 한계로 인해 실행될 수 없는 모든 콜백들을 대기열에
다시 집어넣습니다.
\iffalse

Line~19 initializes local variable \co{count} to zero in preparation
for counting the number of callbacks that will actually be invoked.
Each pass through the loop spanning lines~20-27 invokes and counts
a callback, with lines~25-26 exiting the loop if too many callbacks
are to be invoked at a time (thus preserving responsiveness).
The remainder of the function then requeues any callbacks that could
not be invoked due to this limit.
\fi

Line~28 과 41 은 각각 인터럽트를 비활성화시키고 재활성화 시킵니다.
Line~29 는 이 CPU 의 RCU 콜백들의 전체 갯수를 갖는 \co{->qlen} 필드를 업데이트
합니다.
Line~30 은 원래 수행될 수 있었으나 그때에 수행될 수 있는 콜백 갯수의 제한
문제로 수행되지 못한, 호출될 준비가 된 콜백들의 갯수를 체크합니다.
만약 그런 콜백들이 있다면, line~30-38 은 그것들을 다시 대기열에 넣고, 다시 tail
포인터들을 필요한대로 조정합니다.
Line~39-40 은 만약 한번에 수행할 콜백 개숫의 한계가 지나친 콜백 backlog 로
증가했다면 다시 복구하고, line~42-43 은 남아있는 호출될 준비가 된 콜백들이
존재한다면 스케쥴 될 추가적인 RCU 처리를 행합니다.
\iffalse

Lines~28 and 41 disable and re-enable interrupts, respectively.
Line~29 updates the \co{->qlen} field, which maintains a count
of the total number of RCU callbacks for this CPU.
Line~30 checks to see if there were any ready-to-invoke callbacks
that could not be invoked at the moment due to the limit on the
number that may be invoked at a given time.
If such callbacks remain, lines~30-38 requeue them, again adjusting
the tail pointers as needed.
Lines~39-40 restore the batch limit if it was increased due to
excessive callback backlog, and lines~42-43 cause additional RCU
processing to be scheduled if there are any ready-to-invoke
callbacks remaining.
\fi

\subsection{Dyntick-Idle Functions}
\label{app:rcuimpl:rcutreewt:Dyntick-Idle Functions}

이 섹션의 함수들은 리눅스 커널의 \co{CONFIG_NO_HZ} 빌드에서만 정의되는데, 일부
경우에는 extended-no-op 버전들이 존재하긴 합니다.
이 함수들은 RCU 가 특정 CPU 에게 관심을 기울일지 안기울일지를 조정합니다.
Dynticks-idle 모드의 CPU 들은 무시됩니다만, 그것들이 현재 인터럽트나 NMI 핸들러
안에 있지 않을 때 뿐입니다.
이 섹션의 함수들은 이 CPU 의 RCU 로의 상태를 소통합니다.

이 함수들의 집합은 Preemptible RCU 에서 사용되었던 것들에 비해 상당히 단순화
되어 있는데, 앞의 더 복잡한 모델에 대한 설명을 위해선
Section~\ref{sec:formal:Promela Parable: dynticks and Preemptible RCU} 를
참고하시기 바랍니다.
Manfred Spraul 은 이 단순화된 인터페이스를 위한 아이디어를 그의 상태 기반 RCU
패치들 중 하나로
제공했습니다~\cite{ManfredSpraul2008StateMachineRCU,ManfredSpraul2008dyntickIRQNMI}.
\iffalse

The functions in this section are defined only in \co{CONFIG_NO_HZ}
builds of the Linux kernel,
though in some cases, extended-no-op versions are present otherwise.
These functions control whether or not RCU pays attention to a given CPU.
CPUs in dynticks-idle mode are ignored, but only if they are not
currently in an interrupt or NMI handler.
The functions in this section communicate this CPU state to RCU.

This set of functions is greatly simplified from that used in
preemptible RCU, see
Section~\ref{sec:formal:Promela Parable: dynticks and Preemptible RCU}
for a description of the earlier more-complex model.
Manfred Spraul put forth the idea for this simplified interface in
one of his state-based RCU
patches~\cite{ManfredSpraul2008StateMachineRCU,ManfredSpraul2008dyntickIRQNMI}.
\fi

Section~\ref{app:rcuimpl:rcutreewt:Entering and Exiting Dyntick-Idle Mode}
은 프로세스 컨텍스트로부터 dynticks-idle 모드에 들어가고 빠져나오는 함수들을
설명하고,
Section~\ref{app:rcuimpl:rcutreewt:NMIs from Dyntick-Idle Mode}
은 dynticks-idle 모드로부터의 NMI 처리를 설명하며,
Section~\ref{app:rcuimpl:rcutreewt:Interrupts from Dyntick-Idle Mode}
은 dynticks-idle 모드로부터의 인터럽트 처리를 다루고,
Section~\ref{app:rcuimpl:rcutreewt:Checking for Dyntick-Idle Mode}
은 어떤 다른 CPU 가 현재 dynticks-idle 모드에 있는지 체크하는 함수들을
보입니다.
\iffalse

Section~\ref{app:rcuimpl:rcutreewt:Entering and Exiting Dyntick-Idle Mode}
describes the functions that enter and exit dynticks-idle mode from
process context,
Section~\ref{app:rcuimpl:rcutreewt:NMIs from Dyntick-Idle Mode}
describes the handling of NMIs from dynticks-idle mode,
Section~\ref{app:rcuimpl:rcutreewt:Interrupts from Dyntick-Idle Mode}
covers handling of interrupts from dynticks-idle mode, and
Section~\ref{app:rcuimpl:rcutreewt:Checking for Dyntick-Idle Mode}
presents functions that check whether some other CPU is currently in
dynticks-idle mode.
\fi

\subsubsection{Entering and Exiting Dyntick-Idle Mode}
\label{app:rcuimpl:rcutreewt:Entering and Exiting Dyntick-Idle Mode}

\begin{figure}[tbp]
{ \scriptsize
\begin{verbatim}
  1 void rcu_enter_nohz(void)
  2 {
  3   unsigned long flags;
  4   struct rcu_dynticks *rdtp;
  5
  6   smp_mb();
  7   local_irq_save(flags);
  8   rdtp = &__get_cpu_var(rcu_dynticks);
  9   rdtp->dynticks++;
 10   rdtp->dynticks_nesting--;
 11   local_irq_restore(flags);
 12 }
 13
 14 void rcu_exit_nohz(void)
 15 {
 16   unsigned long flags;
 17   struct rcu_dynticks *rdtp;
 18
 19   local_irq_save(flags);
 20   rdtp = &__get_cpu_var(rcu_dynticks);
 21   rdtp->dynticks++;
 22   rdtp->dynticks_nesting++;
 23   local_irq_restore(flags);
 24   smp_mb();
 25 }
\end{verbatim}
}
\caption{Entering and Exiting Dyntick-Idle Mode}
\label{fig:app:rcuimpl:rcutreewt:Entering and Exiting Dyntick-Idle Mode}
\end{figure}

Figure~\ref{fig:app:rcuimpl:rcutreewt:Entering and Exiting Dyntick-Idle Mode}
는 스케쥴러가 dynticks-idle 모드로 들어가고 그로부터 나오는 걸 가능하게 하는
\co{rcu_enter_nohz()} 와 \co{rcu_exit_nohz()} 함수들을 보입니다.
따라서, \co{rcu_enter_nohz()} 함수가 호출된 뒤에는, RCU 는 적어도 다음
\co{rcu_exit_nohz()}, 다음 인터럽트, 또는 다음 NMI 전까지는 이 CPU 를 무시할
겁니다.
\iffalse

Figure~\ref{fig:app:rcuimpl:rcutreewt:Entering and Exiting Dyntick-Idle Mode}
shows the \co{rcu_enter_nohz()} and \co{rcu_exit_nohz()} functions
that allow the scheduler to transition to and from dynticks-idle
mode.
Therefore, after \co{rcu_enter_nohz()} has been call, RCU will ignore
it, at least until the next \co{rcu_exit_nohz()}, the next interrupt,
or the next NMI.
\fi

\co{rcu_enter_nohz()} 의 line~6 는 모든 앞의 RCU read-side 크리티컬 섹션들은
RCU 이게 이 CPU 를 무시하라고 말하는, 뒤따르는 코드 전에 일어난 것으로 보이도록
보장하기 위해 메모리 배리어를 실행합니다.
Line~7 과 11 은 상태 변경과의 간섭을 막기 위해 인터럽트를 비활성화하고 재활성화
합니다.
Line~8 은 수행중인 CPU 의 \co{rcu_dynticks} 구조체로의 포인터를 가져오고,
line~9 는 (이제는 이 CPU 가 무시될 수 있다는 걸 알리는 짝수여야 하는)
\co{->dynticks} 필드의 값을 증가시키고, 마지막으로 line~10 에서는 (이 CPU 에
관심을 줄 필요가 없다는 걸 알리는 값 0을 가져야 하는) \co{->dynticks_nesting}
필드의 값을 감소시킵니다.
\iffalse

Line~6 of \co{rcu_enter_nohz()} executes a memory barrier to ensure
that any preceding RCU read-side critical sections are seen to have
occurred before the following code that tells RCU to ignore this CPU.
Lines~7 and 11 disable and restore interrupts in order to avoid
interference with the state change.
Line~8 picks up a pointer to the running CPU's \co{rcu_dynticks}
structure, line~9 increments the \co{->dynticks} field (which now
must be even to indicate that this CPU may be ignored), and finally line~10
decrements the \co{->dynticks_nesting} field (which now must be
zero to indicate that there is no reason to pay attention to this CPU).
\fi

\co{rcu_exit_nohz()} 의 line~19 와 23 은 역시 간섭을 막기 위해 인터럽트를
비활성화 하고 재활성화 합니다.
Line~20 은 이 CPU 의 \co{rcu_dynticks} 구조체로의 포인터를 가져오고, line~21 은
(이제 RCU 에게 이 CPU 에 다시 관심을 기울여야 함을 알리기 위해 홀수 값을 가져야
하는) \co{->dynticks} 필드의 값을 증가시키고, line~22 는 (이제 이 CPU 에 관심을
기울여야 할 하나의 이유가 있음을 알리도록 값 1을 가져야 하는)
\co{->dynticks_nesting} 필드의 값을 증가시킵니다.
\iffalse

Lines~19 and 23 of \co{rcu_exit_nohz()} disable and re-enable interrupts,
again to avoid interference.
Line~20 obtains a pointer to this CPU's \co{rcu_dynticks} structure,
line~21 increments the \co{->dynticks} field (which now must be odd
in order to indicate that RCU must once again pay attention to this
CPU), and line~22 increments the \co{->dynticks_nesting} field
(which now must have the value 1 to indicate that there is one
reason to pay attention to this CPU).
\fi

\subsubsection{NMIs from Dyntick-Idle Mode}
\label{app:rcuimpl:rcutreewt:NMIs from Dyntick-Idle Mode}

\begin{figure}[tbp]
{ \scriptsize
\begin{verbatim}
  1 void rcu_nmi_enter(void)
  2 {
  3   struct rcu_dynticks *rdtp;
  4
  5   rdtp = &__get_cpu_var(rcu_dynticks);
  6   if (rdtp->dynticks & 0x1)
  7     return;
  8   rdtp->dynticks_nmi++;
  9   smp_mb();
 10 }
 11
 12 void rcu_nmi_exit(void)
 13 {
 14   struct rcu_dynticks *rdtp;
 15
 16   rdtp = &__get_cpu_var(rcu_dynticks);
 17   if (rdtp->dynticks & 0x1)
 18     return;
 19   smp_mb();
 20   rdtp->dynticks_nmi++;
\end{verbatim}
}
\caption{NMIs from Dyntick-Idle Mode}
\label{fig:app:rcuimpl:rcutreewt:NMIs from Dyntick-Idle Mode}
\end{figure}

Figure~\ref{fig:app:rcuimpl:rcutreewt:NMIs from Dyntick-Idle Mode}
는 각각 NMI 에 들어가는 것과 그로부터 나오는 것을 처리하는 \co{rcu_nmi_enter()}
와 \co{rcu_nmi_exit()} 을 보입니다.
NMI 핸들러에 들어가는 것은 dyntick-idle 모드를 나오게 된다는 점과 그 반대도
마찬가지라는 점을 알아둘 필요가 있는데, 달리 말하면 RCU 는 NMI 핸들러를 수행
중인 동안 dyntick-idle 모드로 들어가려 시도하는 CPU 들에 관심을 기울일 필요가
있다는 이야기로, NMI 핸들러는 RCU read-side 크리티컬 섹션을 가질 수 있기
때문입니다.
이 반대적인 역할은 상당히 헷갈릴 수 있습니다: 경고 드렸습니다.
\iffalse

Figure~\ref{fig:app:rcuimpl:rcutreewt:NMIs from Dyntick-Idle Mode}
shows \co{rcu_nmi_enter()} and \co{rcu_nmi_exit()}, which handle
NMI entry and exit, respectively.
It is important to keep in mind that entering an NMI handler
exits dyntick-idle mode and vice versa, in other words, RCU must
pay attention to CPUs that claim to be in dyntick-idle mode while
they are executing NMI handlers, due to the fact that NMI handlers
can contain RCU read-side critical sections.
This reversal of roles can be quite confusing: you have been warned.
\fi

\co{rcu_nmi_enter()} 의 line~5 는 이 CPU 의 \co{rcu_dynticks} 구조체로의
포인터를 얻어오고, line~6 는 이 CPU 가 이미 RCU 에 의해 srcutiny 아래 있는지
확인해보고, 만약 그렇다면 line~7 에서 조용히 리턴합니다.
그렇지 않다면, line~8 에서 \co{->dynticks_nmi} 필드의 값을 증가시키는데, 이
필드는 이제 홀수 값을 가져야만 합니다.
마지막으로, line~9 는 앞의 \co{->dynticks_nmi} 값의 증가가 다른 모든 CPU 들에게
뒤따르는 모든 RCU read-side 크리티컬 섹션보다 먼저 일어난 것으로 보이게끔
메모리 배리어를 실행합니다.
\iffalse

Line~5 of \co{rcu_nmi_enter()} obtains a pointer to this CPU's
\co{rcu_dynticks} structure, and line~6 checks to see if this
CPU is already under scrutiny by RCU, with line~7 silently returning
if so.
Otherwise, line~8 increments the \co{->dynticks_nmi} field, which
must now have an odd-numbered value.
Finally, line~9 executes a memory barrier to ensure that the prior
increment of \co{->dynticks_nmi} is see by all CPUs to happen
before any subsequent RCU read-side critical section.
\fi

\co{rcu_nmi_exit()} 의 line~16 은 다시 이 CPU 의 \co{rcu_dynticks} 구조체로의
포인터를 가져오고, line~17 은 이 CPU 가 NMI 아래 있징낳았다 하더라도 RCU 가
관심을 가질 것인지 체크하고 만약 그렇다면 line~18 에서 조용히 리턴합니다.
그렇지 않다면, line~19 에서 이 핸들러 안의 모든 RCU read-side 크리티컬 섹션들이
모든 CPU 들에게 line~20 에서의 \co{->dynticks_nmi} 필드의 값 증가 전에 일어난
것으로 보이게끔 메모리 배리어를 실행합니다.
이 필드의 새로운 값은 이제 짝수가 되어야만 합니다.
\iffalse

Line~16 of \co{rcu_nmi_exit()} again fetches a pointer to this CPU's
\co{rcu_dynticks} structure, and line~17 checks to see if RCU would
be paying attention to this CPU even if it were not in an NMI,
with line~18 silently returning if so.
Otherwise, line~19 executes a memory barrier to ensure that any
RCU read-side critical sections within the handler are seen by all
CPUs to happen before the increment of the \co{->dynticks_nmi} field
on line~20.
The new value of this field must now be even.
\fi

\QuickQuiz{}
	Figure~\ref{fig:app:rcuimpl:rcutreewt:NMIs from Dyntick-Idle Mode}
	의 코드는 어떻게 중첩된 NMI 들을 처리하나요?
	\iffalse

	But how does the code in
	Figure~\ref{fig:app:rcuimpl:rcutreewt:NMIs from Dyntick-Idle Mode}
	handle nested NMIs?
	\fi
\QuickQuizAnswer{
	NMI 들은 중첩되지 않으므로 중첩된 NMI 들을 처리할 필요가 없습니다.
	\iffalse

	It does not have to handle nested NMIs, because NMIs do not nest.
	\fi
} \QuickQuizEnd

\subsubsection{Interrupts from Dyntick-Idle Mode}
\label{app:rcuimpl:rcutreewt:Interrupts from Dyntick-Idle Mode}

\begin{figure}[tbp]
{ \scriptsize
\begin{verbatim}
  1 void rcu_irq_enter(void)
  2 {
  3   struct rcu_dynticks *rdtp;
  4
  5   rdtp = &__get_cpu_var(rcu_dynticks);
  6   if (rdtp->dynticks_nesting++)
  7     return;
  8   rdtp->dynticks++;
  9   smp_mb();
 10 }
 11
 12 void rcu_irq_exit(void)
 13 {
 14   struct rcu_dynticks *rdtp;
 15
 16   rdtp = &__get_cpu_var(rcu_dynticks);
 17   if (--rdtp->dynticks_nesting)
 18     return;
 19   smp_mb();
 20   rdtp->dynticks++;
 21   if (__get_cpu_var(rcu_data).nxtlist ||
 22       __get_cpu_var(rcu_bh_data).nxtlist)
 23     set_need_resched();
 24 }
\end{verbatim}
}
\caption{Interrupts from Dyntick-Idle Mode}
\label{fig:app:rcuimpl:rcutreewt:Interrupts from Dyntick-Idle Mode}
\end{figure}

Figure~\ref{fig:app:rcuimpl:rcutreewt:Interrupts from Dyntick-Idle Mode}
는 각각 인터럽트에 들어가는 것과 그로부터 나오는 걸 처리하는
\co{rcu_irq_enter()} 와 \co{rcu_irq_exit()} 를 보입니다.
NMI 에서와 같이, 인터럽트 핸들러에 들어가는 것은 dyntick-idle 모드를 빠져나오는
것이 되고 그 반대도 마찬가지인데, 인터럽트 핸들러 안에 RCU read-side 크리티컬
섹션이 있을 수 있다는 사실 때문입니다.
\iffalse

Figure~\ref{fig:app:rcuimpl:rcutreewt:Interrupts from Dyntick-Idle Mode}
shows \co{rcu_irq_enter()} and \co{rcu_irq_exit()}, which handle
interrupt entry and exit, respectively.
As with NMIs, it is important to note that entering an interrupt
handler exits dyntick-idle mode and vice versa, due to the fact
that RCU read-side critical sections can appear in interrupt handlers.
\fi

\co{rcu_irq_enter()} 의 line~5 는 현재 CPU 의 \co{rcu_dynticks} 구조체로의
레퍼런스를 얻어옵니다.
Line~6 는 \co{->dynticks_nesting} 필드의 값을 증가시키고, 만약 그 원래 값이
이미 0이 아니었다면 (달리 말해서, RCU 가 이미 이 CPU 에 관심을 기울이고
있었다면), line~7 에서 조용히 리턴합니다.
그렇지 않다면, line~8 에서 \co{->dynticks} 필드의 ㄱ밧을 증가시키는데, 이 값은
이제 홀수 값을 가져야만 합니다.
마지막으로, line~9 는 이 값 증가가 모든 CPU 들에게 인터럽트 핸들러 안에 있을 수
있는 모든 RCU read-side 크리티컬 섹션들보다 먼저 일어난 것으로 보일 수 있도록
메모리 배리어를 실행합니다.
\iffalse

Line~5 of \co{rcu_irq_enter()} once again acquires a reference to
the current CPU's \co{rcu_dynticks} structure.
Line~6 increments the \co{->dynticks_nesting} field, and if the
original value was already non-zero (in other words, RCU was
already paying attention to this CPU), line~7 silently returns.
Otherwise, line~8 increments the  \co{->dynticks} field, which
then must have an odd-numbered value.
Finally, line~9 executes a memory barrier so that this increment
is seen by all CPUs as happening before any RCU read-side critical
sections that might be in the interrupt handler.
\fi

\co{rcu_irq_exit()} 의 line~16 은 이제는 전통적인 현재 수행중인 CPU 의
\co{rcu_dynticks} 구조체로의 레퍼런스 얻어오기를 수행합니다.
Line~17 은 \co{->dynticks_nesting} 필드의 값을 감소시키고, 만약 그 결과가 0이
아니라면 (달리 말해서, RCU 가 이 인터럽트 핸들러를 빠져나감에도 불구하고 이 CPU
에 관심을 기울여야만 한다면), line~18 에서 조용히 리턴합니다.
그렇지 않다면, line~19 에서 이 인터럽트 핸들러 안에 있을 수 있는 모든 RCU
read-side 크리티컬 섹션들이 모든 CPU 들에 lin~20 에서의 \co{->dynticks} 필드의
값 증가 (이제는 짝수의 값을 가져야만 합니다) 보다 먼저 일어난 것으로 보일 수
있게끔 메모리 배리어를 실행합니다.
Line~21 과 22 는 이 인터럽트 핸들러가 어떤 ``rcu'' 나 ``rcu\_bh'' 콜백들을
신청했는지 확인해보고, 만약 그렇다면 line~23 에서 이 CPU 가 dynticks-idle
모드로부터 강제로 나가게 되는 side-effect 를 가진,  schedule 재실행을
강제하는데, 이는 RCU 가 이 콜백들에 의해 필요시 되는 grace period 를 처리할 수
있도록 하는데 필요한 대로입니다.
\iffalse

Line~16 of \co{rcu_irq_exit()} does the by-now traditional acquisition
of a reference to the currently running CPU's \co{rcu_dynticks} structure.
Line~17 decrements the \co{->dynticks_nesting} field, and, if the
result is non-zero (in other words, RCU must still pay attention
to this CPU despite exiting this interrupt handler), then line~18
silently returns.
Otherwise, line~19 executes a memory barrier so that any RCU read-side
critical sections that might have been in the interrupt handler are
seen by all CPUs as having happened before the increment on line~20
of the \co{->dynticks} field (which must now have an even-numbered
value).
Lines~21 and 22 check to see if the interrupt handler posted any
``rcu'' or ``rcu\_bh'' callbacks, and, if so, line~23 forces this
CPU to reschedule, which has the side-effect of forcing it out of
dynticks-idle mode, as is required to allow RCU to handle the
grace period required by these callbacks.
\fi

\subsubsection{Checking for Dyntick-Idle Mode}
\label{app:rcuimpl:rcutreewt:Checking for Dyntick-Idle Mode}

CPU 가 dynticks-idle 모드에 있는지를 체크하는데
\co{dyntick_save_progress_counter()} 와 \co{rcu_implicit_dynticks_qs()}
함수들이 사용됩니다.
\co{dyntick_save_progress_counter()} 함수가 먼저 호출되고, 해당 CPU 가 현재
dynticks-idle 모드에 있다면 0이 아닌 값을 리턴합니다.
만약 해당 CPU 가 예를 들어 현재 인터럽트나 NMI 를 처리하고 있거나 해서
dynticks-idle 모드에 있지 않다면, 몇 jiffy 후에 \co{rcu_implicit_dynticks_qs()}
함수가 호출됩니다.
이 함수는 앞의 \co{dyntick_save_progress_counter()} 호출로 저장되어 있는 상태와
함께 현재 상태를 살펴보고, 해당 CPU 가 dynticks-idle 모드에 있거나 그 사이에
dynticks-idle 모드에 있었던 적이 있다면 0이 아닌 값을 리턴합니다.
\co{rcu_implicit_dynticks_qs()} 함수는 필요하다면 true 를 리턴할 때까지
반복적으로 호출될 수 있습니다.
\iffalse

The \co{dyntick_save_progress_counter()} and
\co{rcu_implicit_dynticks_qs()} functions are used to check
whether a CPU is in dynticks-idle mode.
The \co{dyntick_save_progress_counter()} function is invoked first,
and returns non-zero if the CPU is currently in dynticks-idle mode.
If the CPU was not in dynticks-idle mode, for example, because it is
currently handling an interrupt or NMI, then the
\co{rcu_implicit_dynticks_qs()} function is called some jiffies later.
This function looks at the current state in conjunction with state
stored away by the earlier call to \co{dyntick_save_progress_counter()},
again returning non-zero if the CPU either is in dynticks-idle mode or
was in dynticks-idle mode during the intervening time.
The \co{rcu_implicit_dynticks_qs()} function may be invoked repeatedly,
if need be, until it returns true.
\fi

\begin{figure}[tbp]
{ \scriptsize
\begin{verbatim}
  1 static int
  2 dyntick_save_progress_counter(struct rcu_data *rdp)
  3 {
  4   int ret;
  5   int snap;
  6   int snap_nmi;
  7
  8   snap = rdp->dynticks->dynticks;
  9   snap_nmi = rdp->dynticks->dynticks_nmi;
 10   smp_mb();
 11   rdp->dynticks_snap = snap;
 12   rdp->dynticks_nmi_snap = snap_nmi;
 13   ret = ((snap & 0x1) == 0) && ((snap_nmi & 0x1) == 0);
 14   if (ret)
 15     rdp->dynticks_fqs++;
 16   return ret;
 17 }
\end{verbatim}
}
\caption{Code for {\tt dyntick\_\-save\_\-progress\_\-counter()}}
\label{fig:app:rcuimpl:rcutreewt:Code for dyntick-save-progress-counter}
\end{figure}

Figure~\ref{fig:app:rcuimpl:rcutreewt:Code for dyntick-save-progress-counter}
는 특정 CPU-\co{rcu_state} 짝의 \co{rcu_data} 구조체를 전달받는
\co{dyntick_save_progress_counter()} 의 코드를 보입니다.
Line~8 과 9 는 해당 CPU 의 \co{rcu_dynticks} 구조체의 \co{->dynticks} 와
\co{->dynticks_nmi} 필드의 값을 얻어오고, 이어서 line~10 에서는 이 얻어온 값이
모든 CPU 들에게 이 값에 의존적으로 행해지는 모든 처리들보다 먼저 보여진 것으로
보이게끔 메모리 배리어를 실행합니다.
이 메모리 배리어는 \co{rcu_enter_nohz()}, \co{rcu_exit_nohz()},
\co{rcu_nmi_enter()}, \co{rcu_nmi_exit()}, \co{rcu_irq_enter()}, 그리고
\co{rcu_irq_exit()} 의 것들과 짝을 이룹니다.
Line~11 과 12 는 이 두개의 값들을 저장해 두어서 이것들이 나중의
\co{rcu_implicit_dynticks_qs()} 호출에서 접근할 수 있도록 만듭니다.
Line~13 은 두개의 값들이 모두 짝수 값을 가져서 이 CPU 가 non-idle process
상태에도, 인터럽트 핸들러에도, NMI 핸들러에도 있지 않았음을 알리는지
체크합니다.
만약 그렇다면, line~14 와 15 에서 tracing 에만 사용되는 통계적 카운터인
\co{->dynticks_fqs} 의 값을 증가시킵니다.
어느 쪽이든, line~16 은 해당 CPU 가 dynticks-idle 모드에 있었는지 여부를 알리는
리턴을 합니다.
\iffalse

Figure~\ref{fig:app:rcuimpl:rcutreewt:Code for dyntick-save-progress-counter}
shows the code for \co{dyntick_save_progress_counter()}, which
is passed a given CPU-\co{rcu_state} pair's \co{rcu_data} structure.
Lines~8 and 9 take snapshots of the CPU's \co{rcu_dynticks} structure's
\co{->dynticks} and \co{->dynticks_nmi} fields,
and then line~10 executes a memory barrier to ensure that the snapshot
is seen by all CPUs to have happened before any later processing
depending on these values.
This memory barrier pairs up with those in \co{rcu_enter_nohz()},
\co{rcu_exit_nohz()}, \co{rcu_nmi_enter()}, \co{rcu_nmi_exit()},
\co{rcu_irq_enter()}, and \co{rcu_irq_exit()}.
Lines~11 and 12 store these two snapshots away so that they can be
accessed by a later call to \co{rcu_implicit_dynticks_qs()}.
Line~13 checks to see if both snapshots have even-numbered values,
indicating that the CPU in question was in neither non-idle process
state, an interrupt handler, nor an NMI handler.
If so, lines~14 and 15 increment the statistical counter
\co{->dynticks_fqs}, which is used only for tracing.
Either way, line~16 returns the indication of whether the CPU was
in dynticks-idle mode.
\fi

\QuickQuiz{}
	Figure~\ref{fig:app:rcuimpl:rcutreewt:Code for dyntick-save-progress-counter}
	의 line~8 과 9 사이에는 왜 메모리 배리어가 없죠?
	이는 \co{->dynticks} 와 \co{->dynticks_nmi} 필드의 두 값이 동시에
	0이었던 적이 없는데도 짝수 값들을 가져오도록 할 수 있지 않나요?
	\iffalse

	Why isn't there a memory barrier between lines~8 and 9 of
	Figure~\ref{fig:app:rcuimpl:rcutreewt:Code for dyntick-save-progress-counter}?
	Couldn't this cause the code to fetch even-numbered values
	from both the \co{->dynticks} and \co{->dynticks_nmi} fields,
	even though these two fields never were zero at the same time?
	\fi
\QuickQuizAnswer{
	먼저,
	Figure~\ref{fig:app:rcuimpl:rcutreewt:Entering and Exiting Dyntick-Idle Mode},
	\ref{fig:app:rcuimpl:rcutreewt:NMIs from Dyntick-Idle Mode}, 그리고
	\ref{fig:app:rcuimpl:rcutreewt:Interrupts from Dyntick-Idle Mode} 를
	다시 한번 보시고 \co{dynticks} 와 \co{dynticks_nmi} 는 결코 동시에 홀수
	값을 가질 수 없음을 알아두시기 바랍니다 (특히
	Figure~\ref{fig:app:rcuimpl:rcutreewt:NMIs from Dyntick-Idle Mode}
	의 line~6 과 17 을 잘 보시고, NMI 에서 인터럽트는 일어날 수 없음을 다시
	상기하세요).

	물론, 이 함수들의 메모리 배리어의 위치를 놓고 보면, 다른 CPU 에겐 두
	카운터들이 동시에 홀수인 것으로 \emph{보일} 수도 있습니다만, 논리적으로
	이는 일어날 수 없고, 해당 CPU 가 실제로는 dynticks-idle 모드를 지났음을
	알리게 될 겁니다.
	\iffalse

	First, review the code in
	Figures~\ref{fig:app:rcuimpl:rcutreewt:Entering and Exiting Dyntick-Idle Mode},
	\ref{fig:app:rcuimpl:rcutreewt:NMIs from Dyntick-Idle Mode}, and
	\ref{fig:app:rcuimpl:rcutreewt:Interrupts from Dyntick-Idle Mode},
	and note that \co{dynticks} and \co{dynticks_nmi} will never
	have odd values simultaneously (see especially lines~6 and 17 of
	Figure~\ref{fig:app:rcuimpl:rcutreewt:NMIs from Dyntick-Idle Mode},
	and recall that interrupts cannot happen from NMIs).

	Of course, given the placement of the memory barriers in these
	functions, it might \emph{appear} to another CPU that both
	counters were odd at the same time, but logically this cannot
	happen, and would indicate that the CPU had in fact passed
	through dynticks-idle mode.
	\fi

	이제, line~8 이 \co{->dynticks} 의 값을 얻어오는 시점에서
	\co{->dynticks_nmi} 가 홀수 값을 가지고 있었고, line~9 가
	\co{->dynticks_nmi} 의 값을 얻어온느 시점에서 \co{->dynticks} 값은 홀수
	값이었다고 생각해 봅시다.
	두개의 카운터들은 동시에 홀수일 수 없다는 점을 놓고 보면, 이 두개의 값
	가져오기 동작 사이에 두개의 카운터가 모두 짝수 값을 갖는 시점, 즉 해당
	CPU 가 dynticks-idle 모드였던 시간이 있었다는 이야기가 되는데, 이는 곧
	필요시되는대로, quiescent state 입니다.

	그런데, 왜
	Figure~\ref{fig:app:rcuimpl:rcutreewt:Code for dyntick-save-progress-counter}
	의 line~13 의 \co{&&} 는 \co{==} 로 바뀔 수 없을까요?
	음, 그럴 수도 있지만, 이는 도움이 되기보다는 헷갈리기만 할 수 있습니다.
	\iffalse

	Now, let's suppose that at the time line~8 fetches \co{->dynticks},
	the value of \co{->dynticks_nmi} was at odd number, and that at the
	time line~9 fetches \co{->dynticks_nmi}, the value of
	\co{->dynticks} was an odd number.
	Given that both counters cannot be odd simultaneously, there must
	have been a time between these two fetches when both counters
	were even, and thus a time when the CPU was in dynticks-idle
	mode, which is a quiescent state, as required.

	So, why can't the \co{&&} on line~13 of
	Figure~\ref{fig:app:rcuimpl:rcutreewt:Code for dyntick-save-progress-counter}
	be replaced with an \co{==}?
	Well, it could be, but this would likely be more confusing
	than helpful.
	\fi
} \QuickQuizEnd

\begin{figure}[tbp]
{ \scriptsize
\begin{verbatim}
  1 static int
  2 rcu_implicit_dynticks_qs(struct rcu_data *rdp)
  3 {
  4   long curr;
  5   long curr_nmi;
  6   long snap;
  7   long snap_nmi;
  8
  9   curr = rdp->dynticks->dynticks;
 10   snap = rdp->dynticks_snap;
 11   curr_nmi = rdp->dynticks->dynticks_nmi;
 12   snap_nmi = rdp->dynticks_nmi_snap;
 13   smp_mb();
 14   if ((curr != snap || (curr & 0x1) == 0) &&
 15       (curr_nmi != snap_nmi || (curr_nmi & 0x1) == 0)) {
 16     rdp->dynticks_fqs++;
 17     return 1;
 18   }
 19   return rcu_implicit_offline_qs(rdp);
 20 }
\end{verbatim}
}
\caption{Code for {\tt rcu\_\-implicit\_\-dynticks\_\-qs()}}
\label{fig:app:rcuimpl:rcutreewt:Code for rcu-implicit-dynticks-qs}
\end{figure}

Figure~\ref{fig:app:rcuimpl:rcutreewt:Code for rcu-implicit-dynticks-qs}
는 \co{rcu_implicit_dynticks_qs()} 의 코드를 보입니다.
Line~9-12 는 해당 CPU 의 \co{rcu_dynticks} 구조체의 \co{->dynticks} 와
\co{->dynticks_nmi} 필드의 새 값들을 가져오고, 또한 최근의
\co{dyntick_save_progress_counter()} 호출로부터 얻어온 스냅샷도 얻어옵니다.
Line~13 은 이어서 이 값들이 다른 CPU 들에 뒤따르는 RCU 처리보다 먼저 얻어온
것으로 보이게끔 메모리 배리어를 실행합니다.
\co{dyntick_save_progress_counter()} 에서와 같이, 이 메모리 배리어는
\co{rcu_enter_nohz()}, \co{rcu_exit_nohz()}, \co{rcu_nmi_enter()},
\co{rcu_nmi_exit()}, \co{rcu_irq_enter()}, 그리고 \co{rcu_irq_exit()} 의 것들과
짝을 이룹니다.
이어서 line~14-15 는 이 CPU 가 dynticks-idle 모드 (\co{(curr & 0x1) == 0} 이고
\co{curr_nmi & 0x1) == 0}) 이거나 지난번의 \co{dyntick_save_progress_counter()}
호출 후에 dynticks-idle 모드를 지났거나 (\co{curr != snap} 이고 \co{curr_nmi !=
snap_nmi}) 인지를 체크합니다.
만약 그렇다면, line~16 에서 \co{->dynticks_fqs} 통계적 카운터 (이번에도,
tracing 을 위해서만 사용됩니다) 를 증가시키고 line~17 에서 0이 아닌 값을 리턴해
해당 CPU 가 quiescent state 를 지났음을 알립니다.
그렇지 않다면, line~19 에서
(Section~\ref{app:rcuimpl:rcutreewt:Forcing Quiescent States} 에서 설명된)
\co{rcu_implicit_offline_qs()} 를 호출해서 해당 CPU 가 현재 offline 인지
체크합니다.
\iffalse

Figure~\ref{fig:app:rcuimpl:rcutreewt:Code for rcu-implicit-dynticks-qs}
shows the code for \co{rcu_implicit_dynticks_qs()}.
Lines~9-12 pick up both new values for the CPU's \co{rcu_dynticks}
structure's \co{->dynticks} and \co{->dynticks_nmi} fields, as well
as the snapshots taken by the last call to
\co{dyntick_save_progress_counter()}.
Line~13 then executes a memory barrier to ensure that the values are
seen by other CPUs to be gathered prior to subsequent RCU processing.
As with \co{dyntick_save_progress_counter()}, this memory barrier
pairs with those in \co{rcu_enter_nohz()},
\co{rcu_exit_nohz()}, \co{rcu_nmi_enter()}, \co{rcu_nmi_exit()},
\co{rcu_irq_enter()}, and \co{rcu_irq_exit()}.
Lines~14-15 then check to make sure that this CPU is either currently
in dynticks-idle mode (\co{(curr & 0x1) == 0} and
\co{(curr_nmi & 0x1) == 0}) or has passed through dynticks-idle mode
since the last call to \co{dyntick_save_progress_counter()}
(\co{curr != snap} and \co{curr_nmi != snap_nmi}).
If so, line~16 increments the \co{->dynticks_fqs} statistical
counter (again, used only for tracing) and line~17 returns non-zero
to indicate that the specified CPU has passed through a quiescent state.
Otherwise, line~19 invokes \co{rcu_implicit_offline_qs()}
(described in Section~\ref{app:rcuimpl:rcutreewt:Forcing Quiescent States})
to check whether the specified CPU is currently offline.
\fi

\subsection{Forcing Quiescent States}
\label{app:rcuimpl:rcutreewt:Forcing Quiescent States}

Normally, CPUs pass through quiescent states which are duly recorded,
so that grace periods end in a timely manner.
However, any of the following three conditions can prevent CPUs from
passing through quiescent states:

\begin{enumerate}
\item	The CPU is in dyntick-idle state, and is sleeping in a low-power
	mode.
	Although such a CPU is officially in an extended quiescent state,
	because it is not executing instructions, it cannot do anything
	on its own.
\item	The CPU is in the process of coming online, and RCU has been
	informed that it is online, but this CPU is not yet actually
	executing code, nor is it marked as online in \co{cpu_online_map}.
	The current grace period will therefore wait on it, but it cannot
	yet pass through quiescent states on its own.
\item	The CPU is running user-level code, but has avoided
	entering the scheduler for an extended time period.
\end{enumerate}

In each of these cases, RCU needs to take action on behalf of the
non-responding CPU.
The following sections describe the functions that take such action.
Section~\ref{app:rcuimpl:rcutreewt:Recording and Recalling Dynticks-Idle Grace Period}
describes the functions that record and recall the dynticks-idle
grace-period number (in order to avoid incorrectly applying a dynticks-idle
quiescent state to the wrong grace period),
Section~\ref{app:rcuimpl:rcutreewt:Handling Offline and Holdout CPUs}
describes functions that detect offline and holdout CPUs,
Section~\ref{app:rcuimpl:rcutreewt:Scanning for Holdout CPUs}
covers \co{rcu_process_dyntick()}, which scans for holdout CPUs, and
Section~\ref{app:rcuimpl:rcutreewt:Code for force-quiescent-state}
describes \co{force_quiescent_state()}, which drives the process of
detecting extended quiescent states and forcing quiescent states on
holdout CPUs.

\subsubsection{Recording and Recalling Dynticks-Idle Grace Period}
\label{app:rcuimpl:rcutreewt:Recording and Recalling Dynticks-Idle Grace Period}

\begin{figure}[tbp]
{ \scriptsize
\begin{verbatim}
  1 static void
  2 dyntick_record_completed(struct rcu_state *rsp,
  3                          long comp)
  4 {
  5   rsp->dynticks_completed = comp;
  6 }
  7
  8 static long
  9 dyntick_recall_completed(struct rcu_state *rsp)
 10 {
 11   return rsp->dynticks_completed;
 12 }
\end{verbatim}
}
\caption{Recording and Recalling Dynticks-Idle Grace Period}
\label{fig:app:rcuimpl:rcutreewt:Recording and Recalling Dynticks-Idle Grace Period}
\end{figure}

Figure~\ref{fig:app:rcuimpl:rcutreewt:Recording and Recalling Dynticks-Idle Grace Period}
shows the code for \co{dyntick_record_completed()} and
\co{dyntick_recall_completed()}.
These functions are defined as shown only if dynticks
is enabled (in other words, the \co{CONFIG_NO_HZ} kernel parameter
is selected), otherwise they are essentially no-ops.
The purpose of these functions is to ensure that a given observation
of a CPU in dynticks-idle mode is associated with the correct
grace period in face of races between reporting this CPU in
dynticks-idle mode and this CPU coming out of dynticks-idle mode
and reporting a quiescent state on its own.

Lines~1-6 show \co{dyntick_record_completed()}, which stores the
value specified by its \co{comp} argument into the specified
\co{rcu_state} structure's \co{->dynticks_completed} field.
Lines~8-12 show \co{dyntick_recall_completed()}, which returns
the value stored by the most recent call to
\co{dyntick_record_completed()} for this combination of CPU and
\co{rcu_state} structure.

\subsubsection{Handling Offline and Holdout CPUs}
\label{app:rcuimpl:rcutreewt:Handling Offline and Holdout CPUs}

\begin{figure}[tbp]
{ \scriptsize
\begin{verbatim}
  1 static int rcu_implicit_offline_qs(struct rcu_data *rdp)
  2 {
  3   if (cpu_is_offline(rdp->cpu)) {
  4     rdp->offline_fqs++;
  5     return 1;
  6   }
  7   if (rdp->cpu != smp_processor_id())
  8     smp_send_reschedule(rdp->cpu);
  9   else
 10     set_need_resched();
 11   rdp->resched_ipi++;
 12   return 0;
 13 }
\end{verbatim}
}
\caption{Handling Offline and Holdout CPUs}
\label{fig:app:rcuimpl:rcutreewt:Handling Offline and Holdout CPUs}
\end{figure}

Figure~\ref{fig:app:rcuimpl:rcutreewt:Handling Offline and Holdout CPUs}
shows the code for \co{rcu_implicit_offline_qs()}, which checks for
offline CPUs and forcing online holdout CPUs to enter a quiescent state.

Line~3 checks to see if the specified CPU is offline, and, if so,
line~4 increments statistical counter \co{->offline_fqs} (which is
used only for tracing), and line~5 returns non-zero to indicate
that the CPU is in an extended quiescent state.

Otherwise, the CPU is online, not in dynticks-idle mode (or this
function would not have been called in the first place), and has
not yet passed through a quiescent state for this grace period.
Line~7 checks to see if the holdout CPU is the current running
CPU, and, if not, line~8 sends the holdout CPU a reschedule IPI.
Otherwise, line~10 sets the \co{TIF_NEED_RESCHED} flag for the
current task, forcing the current CPU into the scheduler.
In either case, the CPU should then quickly enter a quiescent
state.
Line~11 increments statistical counter \co{resched_ipi}, which is
again used only for tracing.
Finally, line~12 returns zero to indicate that the holdout CPU is
still refusing to pass through a quiescent state.

\subsubsection{Scanning for Holdout CPUs}
\label{app:rcuimpl:rcutreewt:Scanning for Holdout CPUs}

\begin{figure}[tbp]
{ \scriptsize
\begin{verbatim}
  1 static int
  2 rcu_process_dyntick(struct rcu_state *rsp,
  3                     long lastcomp,
  4                     int (*f)(struct rcu_data *))
  5 {
  6   unsigned long bit;
  7   int cpu;
  8   unsigned long flags;
  9   unsigned long mask;
 10   struct rcu_node *rnp_cur;
 11   struct rcu_node *rnp_end;
 12
 13   rnp_cur = rsp->level[NUM_RCU_LVLS - 1];
 14   rnp_end = &rsp->node[NUM_RCU_NODES];
 15   for (; rnp_cur < rnp_end; rnp_cur++) {
 16     mask = 0;
 17     spin_lock_irqsave(&rnp_cur->lock, flags);
 18     if (rsp->completed != lastcomp) {
 19       spin_unlock_irqrestore(&rnp_cur->lock, flags);
 20       return 1;
 21     }
 22     if (rnp_cur->qsmask == 0) {
 23       spin_unlock_irqrestore(&rnp_cur->lock, flags);
 24       continue;
 25     }
 26     cpu = rnp_cur->grplo;
 27     bit = 1;
 28     for (; cpu <= rnp_cur->grphi; cpu++, bit <<= 1) {
 29       if ((rnp_cur->qsmask & bit) != 0 &&
 30           f(rsp->rda[cpu]))
 31         mask |= bit;
 32     }
 33     if (mask != 0 && rsp->completed == lastcomp) {
 34       cpu_quiet_msk(mask, rsp, rnp_cur, flags);
 35       continue;
 36     }
 37     spin_unlock_irqrestore(&rnp_cur->lock, flags);
 38   }
 39   return 0;
 40 }
\end{verbatim}
}
\caption{Scanning for Holdout CPUs}
\label{fig:app:rcuimpl:rcutreewt:Scanning for Holdout CPUs}
\end{figure}

\begin{figure*}[tb]
\centering
\resizebox{6in}{!}{\includegraphics{appendix/rcuimpl/RCUTreeLeafScan}}
\caption{Scanning Leaf {\tt rcu\_node} Structures}
\label{fig:app:rcuimpl:rcutree:Scanning Leaf rcu-node Structures}
\end{figure*}

Figure~\ref{fig:app:rcuimpl:rcutreewt:Scanning for Holdout CPUs}
shows the code for \co{rcu_process_dyntick()}, which scans the
leaf \co{rcu_node} structures in search of holdout CPUs,
as illustrated by the blue arrow in
Figure~\ref{fig:app:rcuimpl:rcutree:Scanning Leaf rcu-node Structures}.
It invokes the function passed in through argument \co{f} on each
such CPU's \co{rcu_data} structure, and returns non-zero if
the grace period specified by the \co{lastcomp} argument has ended.

Lines~13 and 14 acquire references to the first and the last leaf
\co{rcu_node} structures, respectively.
Each pass through the loop spanning lines~15-38 processes one of
the leaf \co{rcu_node} structures.

Line~16 sets the local variable \co{mask} to zero.
This variable will be used to accumulate the CPUs within the current
leaf \co{rcu_node} structure that are in extended quiescent states, and
can thus be reported as such.
Line~17 acquires the current leaf \co{rcu_node} structure's lock,
and line~18 checks to see if the current grace period has completed,
and, if so, line~19 releases the lock and line~20 returns non-zero.
Otherwise, line~22 checks for holdout CPUs associated with this
\co{rcu_node} structure, and, if there are none, line~23 releases
the lock and line~24 restarts the loop from the beginning on the
next leaf \co{rcu_node} structure.

Execution reaches line~26 if there is at least one holdout CPU associated
with this \co{rcu_node} structure.
Lines~26 and 27 set local variables \co{cpu} and \co{bit} to reference
the lowest-numbered CPU associated with this \co{rcu_node} structure.
Each pass through the loop spanning lines~28-32 checks one of the
CPUs associated with the current \co{rcu_node} structure.
Line~29 checks to see if the this CPU is still holding out or if
it has already passed through a quiescent state.
If it is still a holdout, line~30 invokes the specified function
(either \co{dyntick_save_progress_counter()} or
\co{rcu_implicit_dynticks_qs()}, as specified by the caller), and
if that function returns non-zero (indicating that the current CPU
is in an extended quiescent state), then line~31 sets the current
CPU's bit in \co{mask}.

Line 33 then checks to see if any CPUs were identified as being
in extended quiescent states and if the current grace period is
still in force, and, if so, line~34 invokes \co{cpu_quiet_msk()}
to report that the grace period need no longer wait for those
CPUs and then line~35 restarts the loop with the next \co{rcu_node}
structure.
(Note that \co{cpu_quiet_msk()} releases the current \co{rcu_node}
structure's lock, and might well end the current grace period.)
Otherwise, if all holdout CPUs really are still holding out, line~37
releases the current \co{rcu_node} structure's lock.

Once all of the leaf \co{rcu_node} structures have been processed,
the loop exits, and line~39 returns zero to indicate that the current
grace period is still in full force.
(Recall that line~20 returns non-zero should the current grace period
come to an end.)

\subsubsection{Code for {\tt force\_quiescent\_state()}}
\label{app:rcuimpl:rcutreewt:Code for force-quiescent-state}

\begin{figure}[tbp]
{ \scriptsize
\begin{verbatim}
  1 static void
  2 force_quiescent_state(struct rcu_state *rsp, int relaxed)
  3 {
  4   unsigned long flags;
  5   long lastcomp;
  6   struct rcu_data *rdp = rsp->rda[smp_processor_id()];
  7   struct rcu_node *rnp = rcu_get_root(rsp);
  8   u8 signaled;
  9
 10   if (ACCESS_ONCE(rsp->completed) ==
 11       ACCESS_ONCE(rsp->gpnum))
 12     return;
 13   if (!spin_trylock_irqsave(&rsp->fqslock, flags)) {
 14     rsp->n_force_qs_lh++;
 15     return;
 16   }
 17   if (relaxed &&
 18       (long)(rsp->jiffies_force_qs - jiffies) >= 0 &&
 19       (rdp->n_rcu_pending_force_qs -
 20        rdp->n_rcu_pending) >= 0)
 21     goto unlock_ret;
 22   rsp->n_force_qs++;
 23   spin_lock(&rnp->lock);
 24   lastcomp = rsp->completed;
 25   signaled = rsp->signaled;
 26   rsp->jiffies_force_qs =
 27     jiffies + RCU_JIFFIES_TILL_FORCE_QS;
 28   rdp->n_rcu_pending_force_qs =
 29     rdp->n_rcu_pending +
 30     RCU_JIFFIES_TILL_FORCE_QS;
 31   if (lastcomp == rsp->gpnum) {
 32     rsp->n_force_qs_ngp++;
 33     spin_unlock(&rnp->lock);
 34     goto unlock_ret;
 35   }
 36   spin_unlock(&rnp->lock);
 37   switch (signaled) {
 38   case RCU_GP_INIT:
 39     break;
 40   case RCU_SAVE_DYNTICK:
 41     if (RCU_SIGNAL_INIT != RCU_SAVE_DYNTICK)
 42       break;
 43     if (rcu_process_dyntick(rsp, lastcomp,
 44           dyntick_save_progress_counter))
 45       goto unlock_ret;
 46     spin_lock(&rnp->lock);
 47     if (lastcomp == rsp->completed) {
 48       rsp->signaled = RCU_FORCE_QS;
 49       dyntick_record_completed(rsp, lastcomp);
 50     }
 51     spin_unlock(&rnp->lock);
 52     break;
 53   case RCU_FORCE_QS:
 54     if (rcu_process_dyntick(rsp,
 55           dyntick_recall_completed(rsp),
 56           rcu_implicit_dynticks_qs))
 57       goto unlock_ret;
 58     break;
 59   }
 60 unlock_ret:
 61   spin_unlock_irqrestore(&rsp->fqslock, flags);
 62 }
\end{verbatim}
}
\caption{{\tt force\_quiescent\_state()} Code}
\label{fig:app:rcuimpl:rcutreewt:Code for rcutree force-quiescent-state}
\end{figure}

Figure~\ref{fig:app:rcuimpl:rcutreewt:Code for rcutree force-quiescent-state}
shows the code for \co{force_quiescent_state()} for
\co{CONFIG_SMP},\footnote{
	For non-\co{CONFIG_SMP}, \co{force_quiescent_state} is a
	simple wrapper around \co{set_need_resched()}.}
which is invoked when RCU feels the need to expedite the current
grace period by forcing CPUs through quiescent states.
RCU feels this need when either:
\begin{enumerate}
\item	the current grace period has gone on for more than three jiffies
	(or as specified by the compile-time value of
	\co{RCU_JIFFIES_TILL_FORCE_QS}), or
\item	a CPU enqueuing an RCU callback via either \co{call_rcu()}
	or \co{call_rcu_bh()} sees more than 10,000 callbacks enqueued
	(or as specified by the boot-time parameter \co{qhimark}).
\end{enumerate}

Lines~10-12 check to see if there is a grace period in progress,
silently exiting if not.
Lines~13-16 attempt to acquire \co{->fqslock}, which prevents concurrent
attempts to expedite a grace period.
The \co{->n_force_qs_lh} counter is incremented when this lock is
already held, and is visible via the \co{fqlh=} field
in the \co{rcuhier} debugfs file when the \co{CONFIG_RCU_TRACE} kernel
parameter is enabled.
Lines~17-21 check to see if it is really necessary to expedite the
current grace period, in other words, if (1) the current CPU has 10,000
RCU callbacks waiting, or (2) at least three jiffies have passed
since either the beginning of the current grace period or since the
last attempt to expedite the current grace period, measured either
by the \co{jiffies} counter or by the number of calls to
\co{rcu_pending}.
Line~22 then counts the number of attempts to expedite grace periods.

Lines~23-36 are executed with the root \co{rcu_node} structure's lock
held in order to prevent confusion should the current grace period
happen to end just as we try to expedite it.
Lines~24 and 25 snapshot the \co{->completed} and \co{->signaled} fields,
lines~26-30 set the soonest time that a subsequent non-relaxed
\co{force_quiescent_state()} will be allowed to actually do
any expediting, and lines~31-35 check to see if the grace period
ended while we were acquiring the \co{rcu_node} structure's lock,
releasing this lock and returning if so.

Lines~37-59 drive the \co{force_quiescent_state()} state machine.
If the grace period is still in the midst of initialization,
lines~41 and 42 simply return, allowing \co{force_quiescent_state()}
to be called again at a later time, presumably after initialization
has completed.
If dynticks are enabled (via the \co{CONFIG_NO_HZ} kernel
parameter), the first post-initialization call
to \co{force_quiescent_state()} in a given grace period will
execute lines~40-52, and the second and subsequent calls will
execute lines~53-59.
On the other hand, if dynticks is not enabled, then all post-initialization
calls to \co{force_quiescent_state()} will execute lines~53-59.

The purpose of lines~40-52 is to record the current dynticks-idle state
of all CPUs that have not yet passed through a quiescent state, and
to record a quiescent state for any that are currently in dynticks-idle
state (but not currently in an irq or NMI handler).
Lines~41-42 serve to inform gcc that this branch of the switch statement
is dead code for non-\co{CONFIG_NO_HZ} kernels.
Lines~43-45 invoke \co{rcu_process_dyntick()} in order to invoke
\co{dyntick_save_progress_counter()} for each CPU that has not yet
passed through a quiescent state for the current grace period,
exiting \co{force_quiescent_state()} if the grace period ends in
the meantime (possibly due to having found that all the CPUs that
had not yet passed through a quiescent state were sleeping in
dyntick-idle mode).
Lines~46 and 51 acquire and release the root \co{rcu_node} structure's
lock, again to avoid possible confusion with a concurrent end of the
current grace period.
Line~47 checks to see if the current grace period is still in force, and,
if so, line~48 advances the state machine to the \co{RCU_FORCE_QS} state
and line~49 saves the current grace-period number for the benefit of
the next invocation of \co{force_quiescent_state()}.
The reason for saving the current grace-period number is to correctly
handle race conditions involving the current grace period ending
concurrently with the next invocation of \co{force_quiescent_state()}.

As noted earlier, lines~53-58 handle the second and subsequent invocations
of \co{force_quiescent_state()} in \co{CONFIG_NO_HZ} kernels, and \emph{all}
invocations in non-\co{CONFIG_NO_HZ} kernels.
Lines~54 and 58 invoke \co{rcu_process_dyntick()}, which cycles through
the CPUs that have still not passed through a quiescent state, invoking
\co{rcu_implicit_dynticks_qs()} on them, which in turn checks to see
if any of these CPUs have passed through dyntick-idle state (if
\co{CONFIG_NO_HZ} is enabled), checks to see if we are waiting on
any offline CPUs, and finally sends a reschedule IPI to any remaining
CPUs not in the first two groups.

\subsection{CPU-Stall Detection}
\label{app:rcuimpl:rcutreewt:CPU-Stall Detection}

RCU checks for stalled CPUs when the \co{CONFIG_RCU_CPU_STALL_DETECTOR}
kernel parameter is selected.
``Stalled CPUs'' are those spinning in the kernel with preemption disabled,
which degrades response time.
These checks are implemented via the \co{record_gp_stall_check_time()},
\co{check_cpu_stall()}, \co{print_cpu_stall()}, and
\co{print_other_cpu_stall()} functions, each of which is described
below.
All of these functions are no-ops when the \co{CONFIG_RCU_CPU_STALL_DETECTOR}
kernel parameter is not selected.

\begin{figure}[tbp]
{ \scriptsize
\begin{verbatim}
  1 static void
  2 record_gp_stall_check_time(struct rcu_state *rsp)
  3 {
  4   rsp->gp_start = jiffies;
  5   rsp->jiffies_stall =
  6       jiffies + RCU_SECONDS_TILL_STALL_CHECK;
  7 }
\end{verbatim}
}
\caption{{\tt record\_gp\_stall\_check\_time()} Code}
\label{fig:app:rcuimpl:rcutreewt:Code for record-gp-stall-check-time}
\end{figure}

Figure~\ref{fig:app:rcuimpl:rcutreewt:Code for record-gp-stall-check-time}
shows the code for \co{record_gp_stall_check_time()}.
Line~4 records the current time (of the start of the grace period)
in jiffies, and lines~5-6 record the time at which CPU stalls should
be checked for, should the grace period run on that long.

\begin{figure}[tbp]
{ \scriptsize
\begin{verbatim}
  1 static void
  2 check_cpu_stall(struct rcu_state *rsp,
  3                 struct rcu_data *rdp)
  4 {
  5   long delta;
  6   struct rcu_node *rnp;
  7
  8   delta = jiffies - rsp->jiffies_stall;
  9   rnp = rdp->mynode;
 10   if ((rnp->qsmask & rdp->grpmask) && delta >= 0) {
 11     print_cpu_stall(rsp);
 12   } else if (rsp->gpnum != rsp->completed &&
 13        delta >= RCU_STALL_RAT_DELAY) {
 14     print_other_cpu_stall(rsp);
 15   }
 16 }
\end{verbatim}
}
\caption{{\tt check\_cpu\_stall()} Code}
\label{fig:app:rcuimpl:rcutreewt:Code for check-cpu-stall}
\end{figure}

Figure~\ref{fig:app:rcuimpl:rcutreewt:Code for check-cpu-stall}
shows the code for \co{check_cpu_stall}, which checks to see
if the grace period has stretched on too long, invoking either
\co{print_cpu_stall()} or \co{print_other_cpu_stall()} in order
to print a CPU-stall warning message if so.

Line~8 computes the number of jiffies since the time at which stall
warnings should be printed, which will be negative if it is not
yet time to print warnings.
Line~9 obtains a pointer to the leaf \co{rcu_node}
structure corresponding to the current CPU,
and line~10 checks to see if the current CPU has not yet passed through
a quiescent state and if the grace period has extended too long
(in other words, if the current CPU is stalled),
with line~11 invoking \co{print_cpu_stall()} if so.

Otherwise, lines~12-13 check to see if the grace period is still in effect
and if it has extended a couple of jiffies past the CPU-stall warning
duration, with line~14 invoking \co{print_other_cpu_stall()} if so.

\QuickQuiz{}
	Why wait the extra couple jiffies on lines~12-13 in
	Figure~\ref{fig:app:rcuimpl:rcutreewt:Code for check-cpu-stall}?
\QuickQuizAnswer{
	This added delay gives the offending CPU a better chance of
	reporting on itself, thus getting a decent stack trace of
	the stalled code.
	Of course, if the offending CPU is spinning with interrupts
	disabled, it will never report on itself, so other CPUs
	do so after a short delay.
} \QuickQuizEnd

\begin{figure}[tbp]
{ \scriptsize
\begin{verbatim}
  1 static void print_cpu_stall(struct rcu_state *rsp)
  2 {
  3   unsigned long flags;
  4   struct rcu_node *rnp = rcu_get_root(rsp);
  5
  6   printk(KERN_ERR
  7          "INFO: RCU detected CPU %d stall "
  8          "(t=%lu jiffies)\n",
  9     smp_processor_id(),
 10     jiffies - rsp->gp_start);
 11   dump_stack();
 12   spin_lock_irqsave(&rnp->lock, flags);
 13   if ((long)(jiffies - rsp->jiffies_stall) >= 0)
 14     rsp->jiffies_stall =
 15       jiffies + RCU_SECONDS_TILL_STALL_RECHECK;
 16   spin_unlock_irqrestore(&rnp->lock, flags);
 17   set_need_resched();
 18 }
\end{verbatim}
}
\caption{{\tt print\_cpu\_stall()} Code}
\label{fig:app:rcuimpl:rcutreewt:Code for print-cpu-stall}
\end{figure}

Figure~\ref{fig:app:rcuimpl:rcutreewt:Code for print-cpu-stall}
shows the code for \co{print_cpu_stall()}.

Lines~6-11 print a console message and dump the current CPU's stack,
while lines~12-17 compute the time to the next CPU stall warning, should
the grace period stretch on that much additional time.

\QuickQuiz{}
	What prevents the grace period from ending before the
	stall warning is printed in
	Figure~\ref{fig:app:rcuimpl:rcutreewt:Code for print-cpu-stall}?
\QuickQuizAnswer{
	The caller checked that this CPU still had not reported a
	quiescent state, and because preemption is disabled, there is
	no way that a quiescent state could have been reported in
	the meantime.
} \QuickQuizEnd

\begin{figure}[tbp]
{ \scriptsize
\begin{verbatim}
  1 static void print_other_cpu_stall(struct rcu_state *rsp)
  2 {
  3   int cpu;
  4   long delta;
  5   unsigned long flags;
  6   struct rcu_node *rnp = rcu_get_root(rsp);
  7   struct rcu_node *rnp_cur;
  8   struct rcu_node *rnp_end;
  9
 10   rnp_cur = rsp->level[NUM_RCU_LVLS - 1];
 11   rnp_end = &rsp->node[NUM_RCU_NODES];
 12   spin_lock_irqsave(&rnp->lock, flags);
 13   delta = jiffies - rsp->jiffies_stall;
 14   if (delta < RCU_STALL_RAT_DELAY ||
 15       rsp->gpnum == rsp->completed) {
 16     spin_unlock_irqrestore(&rnp->lock, flags);
 17     return;
 18   }
 19   rsp->jiffies_stall = jiffies +
 20       RCU_SECONDS_TILL_STALL_RECHECK;
 21   spin_unlock_irqrestore(&rnp->lock, flags);
 22   printk(KERN_ERR "INFO: RCU detected CPU stalls:");
 23   for (; rnp_cur < rnp_end; rnp_cur++) {
 24     if (rnp_cur->qsmask == 0)
 25       continue;
 26     cpu = 0;
 27     for (; cpu <= rnp_cur->grphi - rnp_cur->grplo; cpu++)
 28       if (rnp_cur->qsmask & (1UL << cpu))
 29         printk(" %d", rnp_cur->grplo + cpu);
 30   }
 31   printk(" (detected by %d, t=%ld jiffies)\n",
 32          smp_processor_id(),
 33          (long)(jiffies - rsp->gp_start));
 34   force_quiescent_state(rsp, 0);
 35 }
\end{verbatim}
}
\caption{{\tt print\_other\_cpu\_stall()} Code}
\label{fig:app:rcuimpl:rcutreewt:Code for print-other-cpu-stall}
\end{figure}

Figure~\ref{fig:app:rcuimpl:rcutreewt:Code for print-other-cpu-stall}
shows the code for \co{print_other_cpu_stall()}, which prints out
stall warnings for CPUs other than the currently running CPU.

Lines~10 and 11 pick up references to the first leaf \co{rcu_node}
structure and one past the last leaf \co{rcu_node} structure,
respectively.
Line~12 acquires the root \co{rcu_node} structure's lock, and also
disables interrupts.
Line~13 calculates the how long ago the CPU-stall warning time occurred
(which will be negative if it has not yet occurred), and lines~14 and 15
check to see if the CPU-stall warning time has passed and if the
grace period has not yet ended,
with line~16 releasing the lock (and re-enabling interrupts) and
line~17 returning if so.

\QuickQuiz{}
	Why does \co{print_other_cpu_stall()} in
	Figure~\ref{fig:app:rcuimpl:rcutreewt:Code for print-other-cpu-stall}
	need to check for the grace period ending when
	\co{print_cpu_stall()} did not?
\QuickQuizAnswer{
	The other CPUs might pass through a quiescent state at any time,
	so the grace period might well have ended in the meantime.
} \QuickQuizEnd

Otherwise, lines~19 and 20 compute the next time that CPU stall warnings
should be printed (if the grace period extends that long) and
line~21 releases the lock and re-enables interrupts.
Lines~23-33 print a list of the stalled CPUs, and, finally,
line~34 invokes \co{force_quiescent_state()} in order to nudge the
offending CPUs into passing through a quiescent state.

\subsection{Possible Flaws and Changes}
\label{app:rcuimpl:rcutreewt:Possible Flaws and Changes}

The biggest possible issue with Hierarchical RCU put forward as of this
writing is the fact that \co{force_quiescent_state()} involves a
potential walk through all CPUs' \co{rcu_data} structures.
On a machine with thousands of CPUs, this could potentially represent
an excessive impact on scheduling latency, given that this scan is
conducted with interrupts disabled.

Should this become a problem in real life, one fix is to maintain
separate \co{force_quiescent_state()} sequencing on a
per-leaf-\co{rcu_node} basis as well as the current per-\co{rcu_state}
\co{->signaled} state variable.
This would allow incremental forcing of quiescent states on a
per-leaf-\co{rcu_node} basis, greatly reducing the worst-case degradation
of scheduling latency.

In the meantime, those caring deeply about scheduling latency can
limit the number of CPUs in the system or use the preemptible RCU
implementation.
