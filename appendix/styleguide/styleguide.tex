% appendix/styleguide/styleguide.tex
% SPDX-License-Identifier: CC-BY-SA-3.0

\chapter{Style Guide}
\label{chp:app:styleguide:Style Guide}
%
\Epigraph{De gustibus non est disputandum.}{\emph{Latin maxim}}

This appendix is a collection of style guides which is intended
as a reference to improve consistency in perfbook. It also contains
several suggestions and their experimental examples.

Section~\ref{sec:app:styleguide:Paul's Conventions} describes basic
punctuation and spelling rules.
Section~\ref{sec:app:styleguide:NIST Style Guide} explains rules
related to unit symbols.
Section~\ref{sec:app:styleguide:LaTeX Conventions} summarizes
\LaTeX-specific conventions.

\section{Paul's Conventions}
\label{sec:app:styleguide:Paul's Conventions}

Following is the list of Paul's conventions assembled from his
answers to Akira's questions regarding perfbook's punctuation policy.

\begin{itemize}
\item (On punctuations and quotations)
  Despite being American myself, for this sort of book, the UK approach
  is better because it removes ambiguities like the following:
  \begin{quote}
    Type ``\nbco{ls -a},'' look for the file ``\co{.},''
    and file a bug if you don't see it.
  \end{quote}

  The following is much more clear:
  \begin{quote}
    Type ``\nbco{ls -a}'', look for the file ``\co{.}'',
    and file a bug if you don't see it.
  \end{quote}
\item American English spelling: ``color'' rather than ``colour''.
\item Oxford comma: ``a, b, and~c'' rather than ``a, b and~c''.
  This is arbitrary.  Cases where the Oxford comma results in ambiguity
  should be reworded, for example, by introducing numbering:  ``a,
  b, and c and~d'' should be ``(1)~a, (2)~b, and (3)~c and~d''.
\item Italic for emphasis.  Use sparingly.
\item \verb|\co{}| for identifiers, \verb|\url{}| for URLs,
  \verb|\path{}| for filenames.
\item Dates should use an unambiguous format.  Never ``mm/dd/yy''
  or ``dd/mm/yy'', but rather ``July 26, 2016'' or ``26 July 2016''
  or ``26-Jul-2016'' or ``2016/07/26''.  I tend to use
  \path{yyyy.mm.ddA} for filenames, for example.
\item North American rules on periods and abbreviations.
  For example neither of the following can reasonably be interpreted
  as two sentences:
  \begin{itemize}
  \item Say hello, to Mr. Jones.
  \item If it looks like she sprained her ankle, call Dr. Smith and
    then tell her to keep the ankle iced and elevated.
  \end{itemize}

  An ambiguous example:
  \begin{quote}
    If I take the cow, the pig, the horse, etc. George will be upset.
  \end{quote}
  can be written with more words:
  \begin{quote}
    If I take the cow, the pig, the horse, or much of anything else,
    George will be upset.
  \end{quote}
  or:
  \begin{quote}
    If I take the cow, the pig, the horse, etc., George will be upset.
  \end{quote}
\item I don't like ampersand (``\&'') in headings, but will sometimes
  use it if doing so prevents a line break in that heading.
\item When mentioning words, I use quotations.  When introducing
  a new word, I use \verb|\emph{}|.
\end{itemize}

\section{NIST Style Guide}
\label{sec:app:styleguide:NIST Style Guide}

\subsection{Unit Symbol}
\label{sec:app:styleguide:Unit Symbol}

\subsubsection{SI Unit Symbol}
\label{sec:app:styleguide:SI Unit Symbol}

NIST style guide\footnote{
  \url{https://www.nist.gov/pml/nist-guide-si-chapter-7-rules-and-style-conventions-expressing-values-quantities}}
states the following rules (rephrased for perfbook).
% cite: https://www.nist.gov/pml/nist-guide-si-chapter-7-rules-and-style-conventions-expressing-values-quantities

\begin{itemize}
\item When SI unit symbols such as ``ns'', ``MHz'', and ``K'' (kelvin)
are used behind numerical values, narrow spaces should be placed between
the values and the symbols.

A narrow space can be coded in \LaTeX{} by the sequence of
\qco{\\,}.
For example,
\begin{quote}
  ``2.4\,GHz'', rather then ``2.4GHz''.
\end{quote}

\item Even when the value is used in adjectival sense, a narrow space
should be placed. For example,
\begin{quote}
  ``a~10\,ms interval'', rather than ``a~10\=/ms interval'' nor
  ``a~10ms interval''.
\end{quote}
\end{itemize}

The symbol of micro (\micro :$10^{-6}$) can be typeset easily by
the help of ``gensymb'' \LaTeX\ package.
A macro \qco{\\micro} can be used in both text and
math modes. To typeset the symbol of ``microsecond'', you can do
so by \qco{\\micro s}. For example,
\begin{quote}
  10\,\micro s
\end{quote}

Note that math mode \qco{\\mu} is italic by default and should not
be used as a prefix. An improper example:
\begin{quote}
  10\,$\mu $s (math mode \qco{\\mu})
\end{quote}

\subsubsection{Non-SI Unit Symbol}
\label{sec:app:styleguide:Non-SI Unit Symbol}

Although NIST style guide does not cover non-SI unit symbols
such as ``KB'', ``MB'', and ``GB'', the same rule should be followed.

Example:

\begin{quote}
  ``A~240\,GB hard drive'', rather than ``a~240\=/GB hard drive''
  nor ``a~240GB hard drive''.
\end{quote}

Strictly speaking, NIST guide requires us to use the binary prefixes
``Ki'', ``Mi'', or ``Gi'' to represent powers of~$2^{10}$.
However, we accept the JEDEC conventions to use ``K'', ``M'',
and ``G'' as binary prefixes in describing memory capacity.\footnote{
  \url{https://www.jedec.org/standards-documents/dictionary/terms/mega-m-prefix-units-semiconductor-storage-capacity}}

An acceptable example:
\begin{quote}
  ``8\,GB of main memory'', meaning ``8\,GiB of main memory''.
\end{quote}

Also, it is acceptable to use just ``K'', ``M'', or ``G'' as abbreviations
appended to a numerical value, e.g., ``4K~entries''. In such cases, no space
before an abbreviation is required. For example,

\begin{quote}
  ``8K entries'', rather than ``8\,K entries''.
\end{quote}

If you put a space in between, the symbol looks like a unit symbol and
is confusing.
Note that ``K'' and ``k'' represent $2^{10}$ and $10^3$, respectively.
``M'' can represent either $2^{20}$ or $10^6$, and ``G'' can represent
either $2^{30}$ or $10^9$. These ambiguities should not be confusing
in discussing approximate order.

\subsubsection{Degree Symbol}
\label{sec:app:styleguide:Degree Symbol}

The angular-degree symbol (\degree) does not require any space in front
of it. NIST style guide clearly states so.

The symbol of degree can also be typeset easily by the help of gensymb
package.
A macro \qco{\\degree} can be used in both text and math modes.

Example:

\begin{quote}
  $45\degree$, rather than $45\,\degree$.
\end{quote}

\subsubsection{Percent Symbol}
\label{sec:app:styleguide:Percent Symbol}

NIST style guide treats the percent symbol (\%) as the same as SI unit
symbols.

\begin{quote}
  50\,\% possibility, rather than 50\% possibility.
\end{quote}

\subsubsection{Font Style}
\label{sec:app:styleguide:Font Style}

Quote from NIST check list:\footnote{
  \#6 in \url{https://physics.nist.gov/cuu/Units/checklist.html}
}

\begin{quote}
  Variables and quantity symbols are in italic type. Unit symbols
  are in roman type. Numbers should generally be written in roman
  type. These rules apply irrespective of the typeface used in
  the surrounding text.
\end{quote}

For example,
\begin{quote}
  {\textit e} (elementary charge)
\end{quote}

On the other hand, mathematical constants such as the base
of natural logarithms should be roman.\footnote{
  \url{https://physics.nist.gov/cuu/pdf/typefaces.pdf}
}
For example,

\begin{quote}
  $\mathrm{e}^x$
\end{quote}

%\footnote{
%  See \url{https://tex.stackexchange.com/questions/119248/}
%  for the historical reason.}

\subsection{NIST Guide Yet To Be Followed}
\label{sec:app:styleguide:NIST Guides Yet To Be Followed}

There are a few cases where NIST style guide is not followed.
Other English conventions are followed in such cases.

\subsubsection{Digit Grouping}
\label{sec:app:styleguide:Digit Grouping}

Quote from NIST check list:\footnote{
  \#16 in \url{http://physics.nist.gov/cuu/Units/checklist.html}.
}

\begin{quote}
  The digits of numerical values having more than four digits on either
  side of the decimal marker are separated into groups of three using
  a thin, fixed space counting from both the left and right of the decimal
  marker. Commas are not used to separate digits into groups of three.
\end{quote}

\begin{quote}
\begin{tabular}{ll}
  NIST Example:& 15\,739.012\,53\,ms\\
  Our convention:& 15,739.01253\,ms\\
\end{tabular}
\end{quote}

In \LaTeX\ coding, it is cumbersome to place thin spaces as are recommended
in NIST guide. The \verb|\num{}| command provided by the ``siunitx''
package would be of help for us to follow this rule.
It would also help us overcome different conventions.
We can select a specific digit-grouping style as
a default in preamble, or specify an option to each \verb|\num{}|
command as is shown in
Table~\ref{tab:app:styleguide:Digit-Grouping Style}.

\newcommand{\NumDigitGrpA}{12 345}
\newcommand{\NumDigitGrpB}{12.345}
\newcommand{\NumDigitGrp}{1 234 567.89}
\begin{table}[htbp]
\small\centering
\begin{tabular}{lrrr}\toprule
  Style & \multicolumn{3}{c}{Outputs of \co{\\num\{\}}} \\
  \midrule
  NIST/SI (English) & \num{\NumDigitGrpA} & \num{\NumDigitGrpB} & \num{\NumDigitGrp} \\
  SI (French) & \num[locale=FR]{\NumDigitGrpA} &
    \num[locale=FR]{\NumDigitGrpB} & \num[locale=FR]{\NumDigitGrp} \\
  English & \num[group-separator={,},group-digits=integer]{\NumDigitGrpA} &
    \num[group-separator={,},group-digits=integer]{\NumDigitGrpB} &
      \num[group-separator={,},group-digits=integer]{\NumDigitGrp} \\
  French & \num[locale=FR]{\NumDigitGrpA} &
    \num[locale=FR]{\NumDigitGrpB} & \num[locale=FR]{\NumDigitGrp} \\
  Other Europe & \num[group-separator={.},output-decimal-marker={,},group-digits=integer]{\NumDigitGrpA} &
    \num[group-separator={.},output-decimal-marker={,},group-digits=integer]{\NumDigitGrpB} &
      \num[group-separator={.},output-decimal-marker={,},group-digits=integer]{\NumDigitGrp} \\
\bottomrule
\end{tabular}
\caption{Digit-Grouping Style}
\label{tab:app:styleguide:Digit-Grouping Style}
\end{table}

As are evident in
Table~\ref{tab:app:styleguide:Digit-Grouping Style},
periods and commas used as other than decimal markers are confusing
and should be avoided, especially in documents expecting global
audiences.

By marking up constant decimal values by \verb|\num{}| commands,
the \LaTeX\ source would be exempted from any particular conventions.

Because of its open-source policy, this approach should give
more ``portability'' to perfbook.

\section{\LaTeX\ Conventions}
\label{sec:app:styleguide:LaTeX Conventions}

Good looking \LaTeX\ documents require further considerations
on proper use of font styles, line break exceptions, etc.
This section summarizes guidelines specific to \LaTeX.

\subsection{Monospace Font}
\label{sec:app:styleguide:Monospace Font}

Monospace font (or typewriter font) is heavily used in this textbook.
First policy regarding monospace font in perfbook is to avoid
directly using \qco{\\texttt} or \qco{\\tt} macro.
It is highly recommended to use a macro or an environment
indicating the reason why you want the font.

This section explains the use cases of such macros and environments.

\subsubsection{Code Snippet}
\label{sec:app:styleguide:Code Snippet}

Although the ``verbatim'' environment is commonly used to include
listings, we use the ``verbbox'' environment provided by the
``verbatimbox'' package for most code snippets to enable the
centering layout.

% Another option would be the ``lstlisting'' environment provided
%  by the ``listings'' package. We are already using its ``lstinline''
%  command in the definition of \co{\\co\{\}} macro.

\begin{listing}[tbh]
{ \scriptsize
\verbfilebox[{\makebox[5ex][r]{\arabic{VerbboxLineNo}:\hspace{2ex}}}]
	{appendix/styleguide/samplecodesnippetlst.tex}
}
\centering
\theverbbox
\caption{\LaTeX\ Source of Sample Code Snippet (Current)}
\label{lst:app:styleguide:LaTeX Source of Sample Code Snippet (Current)}
\end{listing}

\begin{listing}
{ \scriptsize
\begin{verbbox}[\LstLineNo]
/*
 * Sample Code Snippet
 */
#include <stdio.h>
int main(void)
{
  printf("Hello world!\n");
  return 0;
}
\end{verbbox}
}
\centering
\theverbbox
\caption{Sample Code Snippet}
\label{lst:app:styleguide:Sample Code Snippet}
\end{listing}


The \LaTeX\ source of a sample code snippet is shown in
Listing~\ref{lst:app:styleguide:LaTeX Source of Sample Code Snippet (Current)}
and is typeset as show in
Listing~\ref{lst:app:styleguide:Sample Code Snippet}.

Note that the verbbox environment is placed inside the listing environment.
This is to avoid a side effect of verbbox environment that interferes
with the ``afterheading\-/ness'' of a section's first sentence
when a verbbox is placed just below a heading.

Until recently, code snippets were coded using a different scheme.
A sample \LaTeX\ source is shown in
Listing~\ref{lst:app:styleguide:LaTeX Source of Sample Code Snippet (Obsolescent)}
and is typeset as shown in
Figure~\ref{fig:app:styleguide:Sample Code Snippet}.

\begin{listing}[tbh]
{ \scriptsize
\verbfilebox[{\makebox[5ex][r]{\arabic{VerbboxLineNo}:\hspace{2ex}}}]
	{appendix/styleguide/samplecodesnippetfig.tex}
}
\centering
\theverbbox
\caption{\LaTeX\ Source of Sample Code Snippet (Obsolescent)}
\label{lst:app:styleguide:LaTeX Source of Sample Code Snippet (Obsolescent)}
\end{listing}

\begin{figure}[tbh]
{ \scriptsize
\begin{verbbox}
  1 /*
  2  * Sample Code Snippet
  3  */
  4  #include <stdio.h>
  5  int main(void)
  6  {
  7    printf("Hello world!\n");
  8    return 0;
  9  }
\end{verbbox}
}
\centering
\theverbbox
\caption{Sample Code Snippet}
\label{fig:app:styleguide:Sample Code Snippet}
\end{figure}


In Listing~\ref{lst:app:styleguide:LaTeX Source of Sample Code Snippet (Obsolescent)},
the code snippet is coded as a ``figure'' object.
Line numbers are manually placed for ease of referencing them within
\LaTeX\ sources.

This is how most code snippets are coded as of this writing.
However, strictly speaking, code snippets are \emph{not} figures
and they deserve their own floating environment.
The ``float'' package provides the feature to define additional
floating environments.\footnote{
  The ``floatrow'' package provides us even more flexible
  control of floating objects. However, because of an issue
  in two-column layout, we can not use it at the moment.}

Transition to the auto-numbering scheme of verbbox and
the ``listing'' environment defined for code snippets
has recently started in
Chapter~\ref{chp:memorder:Memory Ordering}.
The transition also permits us to choose distinct looks
for code snippets, including moving captions to top of the
listings
(discussed in Section~\ref{sec:app:styleguide:Position of Caption}).

The auto\-/numbering feature of verbbox is enabled by
the \verb|\LstLineNo| macro specified in the option to verbbox
(line~3 in
Listing~\ref{lst:app:styleguide:LaTeX Source of Sample Code Snippet (Current)}).
The macro is defined in the preamble of \path{perfbook.tex}
as the following:

\noindent\begin{minipage}{\columnwidth}
{ \scriptsize
\begin{verbbox}
\newcommand{\LstLineNo}
  {\makebox[5ex][r]{\arabic{VerbboxLineNo}\hspace{2ex}}}
\end{verbbox}
}
\vspace{10pt}
\centering
\theverbbox
\vspace{10pt}
\end{minipage}

The verbatim environment is used for listings with too many lines
to fit in a column. It is also used to avoid overwhelming
\LaTeX\ with a lot of floating objects.

\subsubsection{Identifier}
\label{sec:app:styleguide:Identifier}

We use ``\verb|\co{}|'' macro for inline identifiers.
(``co'' stands for ``code''.)

By putting them into \verb|\co{}|, underscore characters in
their names are free of escaping in \LaTeX\ source. It is convenient
to search them in source files. Also, \verb|\co{}|
macro has a capability to permit line breaks at particular
sequences of letters. Current definition permits a line break at
an underscore (\tco{_}), two consecutive underscores (\tco{__}),
a white space, or an operator \tco{->}.

\subsubsection{Identifier inside Table and Heading}
\label{sec:app:styleguide:Identifier inside Table and Heading}

Although \verb|\co{}| command is convenient for inlining within text,
it is fragile because of its capability of line break.
When it is used inside a ``tabular'' environment or its derivative
such as ``tabularx'', it confuses column width
estimation of those environments.
Furthermore, \verb|\co{}| can not be safely used in section headings nor
description headings.

As a workaround, we use ``\verb|\tco{}|'' command
inside tables and headings. It has no capability of line break
at particular sequences, but still frees us from escaping
underscores.

When used in text, \verb|\tco{}| permits line breaks at
white spaces.

\subsubsection{Other Use Cases of Monospace Font}
\label{sec:app:styleguide:Other Use Cases of Monospace Font}

For URLs, we use ``\verb|\url{}|'' command provided by the
``hyperref'' package. It will generate hyper references to the
URLs.

For path names, we use ``\verb|\path{}|'' command. It won't
generate hyper references.

Both \verb|\url{}| and \verb|\path{}| permit line breaks
at \qco{/}, \qco{-}, and \qco{.}.\footnote{
  Overfill can be a problem if the URL or the path name contains
  long runs of unbreakable characters.
}

For short monospace statements not to be line broken, we use
the ``\verb|\nbco{}|'' (non-breakable co) macro.

\subsubsection{Limitations}
\label{sec:app:styleguide:Limitations}

There are a few cases where macros introduced in this section
do not work as expected.
Table~\ref{tab:app:styleguide:Limitation of Monospace Macro}
lists such limitations.

\begin{table}[tbh]
\renewcommand*{\arraystretch}{1.2}\centering\footnotesize
\begin{tabular}{@{}lll@{}}\toprule
  Macro &  Need Escape & Should Avoid \\
  \midrule
  \co{\\co}, \co{\\nbco} & \co{\\}, \%, \{, \} & \\
  \co{\\tco}  & \# & \%, \{, \}, \co{\\} \\
  \bottomrule
\end{tabular}
\caption{Limitation of Monospace Macro}
\label{tab:app:styleguide:Limitation of Monospace Macro}
\end{table}

While \verb|\co{}| requires some characters to be escaped,
it can contain any character.

On the other hand, \verb|\tco{}| can not handle
\qco{\%}, \qco{\{}, \qco{\}}, nor \qco{\\} properly.
If they are escaped by a~\qco{\\},
they appear in the end result with the escape character.
The \qco{\\verb} macro can be used in running text if you
need to use monospace font for a string which contains
many characters to escape.\footnote{
  \co{\\verb} is not almighty though. For example, you can't use it
  within a footnote. If you do so, you will see a fatal latex error.
  There are several workarounds of this problem, but as for perfbook,
  \co{\\co\{\}} should suffice.}

\subsection{Non Breakable Spaces}
\label{sec:app:styleguide:Non Breakable Spaces}

In \LaTeX\ conventions, proper use of non-breakable white spaces
is highly recommended. They can prevent widowing and orphaning
of single digit numbers or short variable names, which would
cause the text to be confusing at first glance.

The thin space mentioned earlier to be placed in front of a unit
symbol is non breakable.

Other cases to use a non-breakable space (``\verb|~|'' in \LaTeX\
source, often referred to as ``nbsp'')
are the following (inexhaustive).

\begin{itemize}
\item Reference to a Chapter or a Section:
  \begin{quote}
    Please refer to Section~\ref{sec:app:styleguide:NIST Style Guide}.
  \end{quote}
\item Calling out CPU number or Thread name:
  \begin{quote}
    After they load the pointer, CPUs~1 and~2 will see the stored
    value.
  \end{quote}
\item Short variable name:
  \begin{quote}
    The results will be stored in variables~\co{a} and~\co{b}.
  \end{quote}
\end{itemize}

\subsection{Hyphenation and Dashes}
\label{sec:app:styleguide:Hyphenation and Dashes}

\subsubsection{Hyphenation in Compound Word}
\label{sec:app:styleguide:Hyphenation in Compound Word}

In plain \LaTeX, compound words such as ``high-frequency''
can be hyphenated only at the hyphen. This sometimes results
in poor typesetting. For example:

\begin{center}\begin{minipage}{2.55in}\vspace{0.6\baselineskip}
  High-frequency radio wave, high-frequency radio wave,
  high-frequency radio wave, high-frequency radio wave,
  high-frequency radio wave, high-frequency radio wave.
\vspace{0.6\baselineskip}\end{minipage}\end{center}

By using a shortcut \qco{\\-/} provided by the
``extdash'' package, hyphenation in elements of compound
words is enabled in perfbook.\footnote{
  In exchange for enabling the shortcut, we can't use plain
  \LaTeX's shortcut \qco{\\-} to specify hyphenation points.
  Use \path{pfhyphex.tex} to add such exceptions.
}

Example with \qco{\\-/}:

\begin{center}\begin{minipage}{2.55in}\vspace{0.6\baselineskip}
  High\-/frequency radio wave, high\-/frequency radio wave,
  high\-/frequency radio wave, high\-/frequency radio wave,
  high\-/frequency radio wave, high\-/frequency radio wave.
\vspace{0.6\baselineskip}\end{minipage}\end{center}

\subsubsection{Non Breakable Hyphen}
\label{sec:app:styleguide:Non Breakable Hyphen}

We want hyphenated compound terms such as ``x\=/coordinate'',
``y\=/coordinate'', etc. not to be broken at the hyphen
following a single letter.

To make a hyphen unbreakable, we can use a short cut
\qco{\\=/} also provided by the ``extdash'' package.

Example without a shortcut:

\begin{center}\begin{minipage}{2.55in}\vspace{0.6\baselineskip}
x-, y-, and z-coordinates; x-, y-, and z-coordinates;
x-, y-, and z-coordinates; x-, y-, and z-coordinates;
x-, y-, and z-coordinates; x-, y-, and z-coordinates;
\vspace{0.6\baselineskip}\end{minipage}\end{center}

Example with \qco{\\-/}:

\begin{center}\begin{minipage}{2.55in}\vspace{0.6\baselineskip}
x-, y-, and z\-/coordinates; x-, y-, and z\-/coordinates;
x-, y-, and z\-/coordinates; x-, y-, and z\-/coordinates;
x-, y-, and z\-/coordinates; x-, y-, and z\-/coordinates;
\vspace{0.6\baselineskip}\end{minipage}\end{center}

Example with \qco{\\=/}:

\begin{center}\begin{minipage}{2.55in}\vspace{0.6\baselineskip}
x-, y-, and z\=/coordinates; x-, y-, and z\=/coordinates;
x-, y-, and z\=/coordinates; x-, y-, and z\=/coordinates;
x-, y-, and z\=/coordinates; x-, y-, and z\=/coordinates;
\vspace{0.6\baselineskip}\end{minipage}\end{center}

Note that \qco{\\=/} enables hyphenation in elements
of compound words as the same as \qco{\\-/} does.

\subsubsection{Em Dash}
\label{sec:app:styleguide:Em Dash}

Em dashes are used to indicate parenthetic expression. In perfbook,
em dashes are placed without spaces around it. In \LaTeX\ source,
an em dash is represented by \qco{---}.

Example (quote from Section~\ref{sec:app:whymb:Cache Structure}):
\begin{quote}
  This disparity in speed---more than two orders of magnitude---has
  resulted in the multi-megabyte caches found on modern CPUs.
\end{quote}

\subsubsection{En Dash}
\label{sec:app;styleguide:En Dash}

In \LaTeX\ convention, en~dashes (\==) are used for a range of (mostly)
numbers.
However, this is not followed in perfbook at all.
Because of the heavy use of dashes (\=/) for such cases
in plain-text communication, to make the \LaTeX\ sources compatible
with them, plain dashes are kept unmodified in the sources.

As a compromise, for those who are accustomed to en~dashes representing
ranges, there is a script to substitute en~dashes for plain dashes.

If you have the git repository of perfbook, by using a script
\path{utilities/dohyphen2endash.sh}, you can do the substitutions.
The script works only when you are in a clean git repository.
Otherwise it will just abort to prevent you from losing local
changes.

Example with a simple dash:

\begin{quote}
  Lines~4\=/12 in
  Listing~\ref{lst:app:styleguide:LaTeX Source of Sample Code Snippet (Current)}
  are the contents of the verbbox environment. The box is output
  by the \co{\\theverbbox} macro on line~16.
\end{quote}

Example with an en dash:

\begin{quote}
  Lines~4\==12 in
  Listing~\ref{lst:app:styleguide:LaTeX Source of Sample Code Snippet (Current)}
  are the contents of the verbbox environment. The box is output
  by the \co{\\theverbbox} macro on line~16.
\end{quote}

\subsubsection{Numerical Minus Sign}
\label{sec:app:styleguide:Numerical Minus Sign}

Numerical minus signs should be coded as math mode minus signs,
namely \qco{$-$}.\footnote{This rule assumes that math mode uses the
  same upright glyph as text mode. Our default font choice meets
  the assumption.
\IfSansSerif{
  One of the experimental targets ``1csf'' \emph{does} use a differnt font
  for math mode figures as of October 2017.}{}
}
For example,

\begin{quote}
  $-30$, rather than -30.
\end{quote}

\subsection{Punctuation}
\label{sec:app:styleguide:Punctuation}

\subsubsection{Ellipsis}
\label{sec:app:styleguide:Ellipsis}

In monospace fonts, ellipses can be expressed by
series of periods. For example:

\begin{quote}
  \verb|Great ... So how do I fix it?|
\end{quote}

However, in proportional fonts, the series of periods is printed
with tight spaces as follows:

\begin{quote}
  Great ... So how do I fix it?
\end{quote}

Standard \LaTeX\ defines the \verb|\dots| macro for this purpose.
However, it has a kludge in the evenness of spaces.
The ``ellipsis'' package redefines the \verb|\dots| macro to fix
the issue.\footnote{To be exact, it is the \co{\\textellipsis} macro
  that is redefined. The behavior of \co{\\dots} macro in math
  mode is not affected. The ``amsmath'' package has another definition
  of \co{\\dots}. It is not used in perfbook at the moment.}
By using \verb|\dots|, the above example is typeset as the following:

\begin{quote}
  Great \dots So how do I fix it?
\end{quote}

Note that the ``xspace'' option specified to the ``ellipsis'' package
adjusts the spaces after ellipses depending on what follows them.

For example:

\begin{itemize}[itemsep=.2ex]
\item He said, ``I~\dots really don't remember~\dots''
\item Sequence A: (one, two, three, \dots)
\item Sequence B: (4, 5, \dots, $n$)
\end{itemize}

As you can see, extra space is placed before the comma.

\verb|\dots| macro can also be used in math mode:

\begin{itemize}[itemsep=.2ex]
\item Sequence C: $(1, 2, 3, 5, 8, \dots)$
\item Sequence D: $(10, 12, \dots, 20)$
\end{itemize}

The \verb|\ldots| macro behaves the same as the \verb|\dots| macro.

\subsection{Improvement Candidates}
\label{sec:app:styleguide:Improvement Candidates}

\begin{figure*}[tbh]\centering
\begin{minipage}[t][][t]{2.1in}
\resizebox{2.1in}{!}{\includegraphics{cartoons/1kHz}}
\caption{Timer Wheel at 1\,kHz}
\label{fig:app:styleguide:Timer Wheel at 1kHz}
\end{minipage}
\qquad
\begin{minipage}[t][][t]{2.3in}
\resizebox{2.3in}{!}{\includegraphics{cartoons/100kHz}}
\caption{Timer Wheel at 100\,kHz}
\label{fig:app:styleguide:Timer Wheel at 100kHz}
\end{minipage}
\end{figure*}

\floatstyle{ruled}
\restylefloat{listing}

\begin{listing*}[tbh]%
\caption{Message-Passing Litmus Test (by subfig)}%
\label{lst:app:styleguide:Message-Passing Litmus Test (subfig)}%
{\scriptsize%
\begin{verbbox}[\LstLineNo]
C C-MP+o-wmb-o+o-o.litmus

{
}

P0(int* x0, int* x1) {

  WRITE_ONCE(*x0, 2);
  smp_wmb();
  WRITE_ONCE(*x1, 2);

}

P1(int* x0, int* x1) {

  int r2;
  int r3;

  r2 = READ_ONCE(*x1);
  r3 = READ_ONCE(*x0);

}


exists (1:r2=2 /\ 1:r3=0)
\end{verbbox}
}
\centering
\hspace*{\fill}
\subfloat[Not Enforcing Order]{
  \theverbbox
  \label{sublst:app:styleguide:Not Enforcing Order}
}
\hspace{\fill}
{\scriptsize%
\begin{verbbox}[\LstLineNo]
C C-MP+o-wmb-o+o-rmb-o.litmus

{
}

P0(int* x0, int* x1) {

  WRITE_ONCE(*x0, 2);
  smp_wmb();
  WRITE_ONCE(*x1, 2);

}

P1(int* x0, int* x1) {

  int r2;
  int r3;

  r2 = READ_ONCE(*x1);
  smp_rmb();
  r3 = READ_ONCE(*x0);

}

exists (1:r2=2 /\ 1:r3=0)
\end{verbbox}
}%
\subfloat[Enforcing Order]{%
  \theverbbox
  \label{sublst:app:styleguide:Enforcing Order}
}\hspace*{\fill}%
\end{listing*}

There are a few areas yet to be attempted in perfbook
which would further improve its appearance.
This section lists up such candidates.

\subsubsection{Position of Caption}
\label{sec:app:styleguide:Position of Caption}

In \LaTeX\ conventions, captions of tables are usually placed
above them. The reason is the flow of your eye movement
when you look at them. Most tables have a row of heading at the
top. You naturally look at the top of a table at first. Captions at
the bottom of tables disturb this flow.
The same can be said of code snippets, which are read from
top to bottom.

For code snippets, the ``ruled'' style chosen for listing
environment places the caption at the top.
See Listing~\ref{lst:app:styleguide:Sample Code Snippet}
for an example.

As for tables, the position of caption can be tweaked by
\verb|\floatstyle{}| and \verb|\restylefloat{}| macros
in preamble.

Currently, as most code snippets are figures with their captions
at the bottom, captions of tables at the top might look inconsistent.
Once the transition of code snippets to listing environment
completes, there would be fewer figures and the caption of tables
at the top would hopefully be acceptable.

Vertical space between captions at the top and the table bodies
can be reduced by the help of ``ctable'' package.

In the sample tables shown in
Sections~\ref{sec:app:styleguide:Ruled Line in Table}
and~\ref{sec:app:styleguide:Table Layout Experiment},
the vertical skip is manually reduced by setting a negative value to the
\verb|\abovetopsep| variable which controls the behavior of
\verb|\toprule| of the ``booktabs'' package.
It should be regarded as a band-aid tweak.

\subsubsection{Grouping Related Figures/Listings}
\label{sec:app:styleguide:Grouping Related Figures/Listings}

To prevent a pair of closely related figures or listings
from being placed in different pages, it is desirable to group
them into a single floating object.
The ``subfig'' package provides the features to do so.\footnote{
  One problem of grouping figures might be the complexity in
  \LaTeX\ source.}

Two floating objects can be placed side by side by using
\co{\\parbox} or \co{minipage}.
For example,
Figures~\ref{fig:rt:Timer Wheel at 1kHz}
and~\ref{fig:rt:Timer Wheel at 100kHz}
can be grouped together by using a pair of \co{minipage}s
as shown in
Figures~\ref{fig:app:styleguide:Timer Wheel at 1kHz}
and~\ref{fig:app:styleguide:Timer Wheel at 100kHz}.

By using subfig package,
Listings~\ref{lst:memorder:Message-Passing Litmus Test (No Ordering)}
and~\ref{lst:memorder:Enforcing Order of Message-Passing Litmus Test}
can be grouped together as shown in
Listing~\ref{lst:app:styleguide:Message-Passing Litmus Test (subfig)}
with sub\-/captions (with a minor change of blank line).

Note that they can not be grouped in the same way as
Figures~\ref{fig:app:styleguide:Timer Wheel at 1kHz}
and~\ref{fig:app:styleguide:Timer Wheel at 100kHz}
because the ``ruled'' style prevents their captions
from being properly typeset.

The sub\-/caption can be cited by combining a \verb|\ref{}| macro
and a \verb|\subref{}| macro, for example,
``Listing~\ref{lst:app:styleguide:Message-Passing Litmus Test (subfig)}\,%
\subref{sublst:app:styleguide:Not Enforcing Order}''.

It can also be cited by a \verb|\ref{}| macro, for example,
``Listing~\ref{sublst:app:styleguide:Enforcing Order}''.
Note the difference in the resulting format. For the citing by
a \verb|\ref{}| to work, you need to place the \verb|\label{}|
macro of the combined floating object ahead of the definition of
subfloats.
Otherwise, the resulting caption number would be off by one
from the actual number.

\subsubsection{Ruled Line in Table}
\label{sec:app:styleguide:Ruled Line in Table}

They say that tables drawn by using ruled lines of plain \LaTeX\
look ugly.\footnote{
  \url{https://www.inf.ethz.ch/personal/markusp/teaching/guides/guide-tables.pdf}
}
Vertical lines should be avoided and horizontal lines should be
used sparingly, especially in tables of simple structure.

\floatstyle{plaintop}
\restylefloat{table}
\captionsetup[table]{position=top,hangindent=30pt}
\renewcommand*{\abovetopsep}{-7pt}

For example,
Table~\ref{tab:future:Refrigeration Power Consumption}
can be tweaked as shown in
Table~\ref{tab:app:styleguide:Refrigeration Power Consumption}
with the help of ``booktabs'' and ``xcolor'' packages.

\begin{table}[tbhp]
\rowcolors{1}{}{lightgray}
\renewcommand*{\arraystretch}{1.2}\centering\small
\begin{tabular}{lrrr}\toprule
Situation
	& $T$ (K)
		& $\CPf$ & \parbox[b]{.75in}{\raggedleft Power per watt\par waste heat (W)} \\
\midrule
Dry Ice
	& $195$
		& $1.990$
			& 0.5 \\
Liquid N$_2$
	& $77$
		& $0.356$
			& 2.8 \\
Liquid H$_2$
	& $20$
		& $0.073$
			& 13.7 \\
Liquid He
	& $4$
		& $0.0138$
			& 72.3 \\
IBM~Q	& $0.015$
		& $0.000051$
			& 19,500.0 \\
\bottomrule
\end{tabular}
\caption{Refrigeration Power Consumption}
\label{tab:app:styleguide:Refrigeration Power Consumption}
\end{table}

Note that ruled lines of booktabs can not be mixed with
vertical lines in a table.\footnote{
  There is another package named ``arydshln'' which provides dashed lines
  to be used in tables. A couple of experimental examples are presented in
  Section~\ref{sec:app:styleguide:Table Layout Experiment}.
}

\subsubsection{Table Layout Experiment}
\label{sec:app:styleguide:Table Layout Experiment}

To see how far we can go without vertical rules in tables,
several experiments using booktabs, xcolors, and arydshln packages
are presented in this section.

\begin{table}[tb]
\rowcolors{1}{}{lightgray}
\renewcommand*{\arraystretch}{1.1}
\sisetup{group-minimum-digits=4,group-separator={,}}
\centering\small
\begin{tabular}
  {
    l
    S[table-format = 9.1]
    S[table-format = 9.1]
  }
	\toprule
	Operation		& \multicolumn{1}{r}{Cost (ns)}
			& {\parbox[b]{.7in}{\raggedleft Ratio\\(cost/clock)}} \\
	\midrule
	Clock period		&           0.6	&           1.0 \\
	Best-case CAS		&          37.9	&          63.2 \\
	Best-case lock		&          65.6	&         109.3 \\
	Single cache miss	&         139.5	&         232.5 \\
	CAS cache miss		&         306.0	&         510.0 \\
	Comms Fabric		&       5 000	&       8 330	\\
	Global Comms		& 195 000 000	& 325 000 000   \\
	\bottomrule
\end{tabular}
\caption{Performance of Synchronization Mechanisms of 4-CPU 1.8\,GHz AMD Opteron 844 System}
\label{tab:app:styleguide:Performance of Synchronization Mechanisms of 4-CPU 1.8GHz AMD Opteron 844 System}
\end{table}

Table~\ref{tab:cpu:Performance of Synchronization Mechanisms on 4-CPU 1.8GHz AMD Opteron 844 System}
can be tweaked as is shown in
Table~\ref{tab:app:styleguide:Performance of Synchronization Mechanisms of 4-CPU 1.8GHz AMD Opteron 844 System} using booktabs and xcolors.
In this table, original tabular source contains tweaks with
\verb|\textcolor{}| commands. They are removed by using ``S'' column
specifiers provided by the ``siunitx'' package.

Table~\ref{tab:app:styleguide:Reference Counting and Synchronization Mechanisms}
is a tweaked version of
Table~\ref{tab:together:Reference Counting and Synchronization Mechanisms},
which has more complex header than the tables experimented so far.

In
Table~\ref{tab:app:styleguide:Reference Counting and Synchronization Mechanisms},
the gap in the mid-rule corresponds to the distinction
which is represented by double vertical rules in
Table~\ref{tab:together:Reference Counting and Synchronization Mechanisms}.
The legends in the frame box explain the abbreviations used in the matrix.
Two types of memory barrier are denoted by subscripts here.

\begin{table}[tb]
\small
\centering
\renewcommand*{\arraystretch}{1.25}
\rowcolors{3}{}{lightgray}
\begin{tabular}{lccc}
	\toprule
	& \multicolumn{3}{c}{Release Synchronization} \\
	\cmidrule(l){2-4}
	\parbox[c]{.8in}{Acquisition\\Synchronization}
			& Locking
				& \parbox[c]{.5in}{Reference\\Counting}
				        & RCU \\
	\cmidrule{1-1} \cmidrule(l){2-4}
	Locking		& $-$	& CAM\textsubscript{R}	& CA  \\
	\parbox[c][6ex]{.8in}{Reference\\Counting}
			& A	& AM\textsubscript{R}	& A   \\
	RCU		& CA	& M\textsubscript{A}CA	& CA  \\
	\bottomrule
\end{tabular}

\vspace{5pt}\hfill
\framebox[\width]{\footnotesize\setlength{\tabcolsep}{3pt}
\rowcolors{1}{}{}
  \begin{tabular}{lrp{2in}}
	Key:	& A: & Atomic counting \\
		& C: & Check combined with the atomic acquisition operation \\
		& M\textsubscript{R}: & Memory barriers required only on release \\
		& M\textsubscript{A}: & Memory barriers required on acquire \\
  \end{tabular}
}
\caption{Reference Counting and Synchronization Mechanisms}
\label{tab:app:styleguide:Reference Counting and Synchronization Mechanisms}
\end{table}

Table~\ref{tab:app:whymb:Cache Coherence Example}
can be tweaked as in
Table~\ref{tab:app:styleguide:Cache Coherence Example}
in a similar manner.
It is not a ``table'' in the narrow sense, rather a sequence diagram.
A ``figure'' environment might be a proper choice here.

\begin{table*}[tbh]
\small
\centering
\renewcommand*{\arraystretch}{1.2}
\rowcolors{6}{}{lightgray}
% "6" is chosen due to disturbance of row count by cmidrule.
% The command definition is:
%     \rowcolors{<row>}{<odd-row color>}{<even-row color>}
% Here, <row> specifies the row count where the coloring start.
% In this table, the "Seq = 0" row is the 3rd row, so a "3" would
% be a right choice.
% However, because of the \cmidrule{} commands used in the heading,
% internal row count of the "Seq = 0" row becomes "6".
% This is why the 3rd row has the background color of <even-row color>.
%
% \cline of plain LaTeX also interfares the row count.
\begin{tabular}{rclcccccc}
	\toprule
	& & & \multicolumn{4}{c}{CPU Cache} & \multicolumn{2}{c}{Memory} \\
	\cmidrule(lr){4-7} \cmidrule(l){8-9}
	Sequence \# & CPU \# & Operation & 0 & 1 & 2 & 3 & 0 & 8 \\
	\cmidrule(r){1-3} \cmidrule(lr){4-7} \cmidrule(l){8-9}
%	Seq CPU Operation	------------- CPU -------------   - Memory -
%				   0	   1	   2	   3	    0   8
	0 &   & Initial State	& $-$/I & $-$/I & $-$/I & $-$/I   & V & V \\
	1 & 0 & Load		& 0/S &   $-$/I & $-$/I & $-$/I   & V & V \\
	2 & 3 & Load		& 0/S &   $-$/I & $-$/I & 0/S     & V & V \\
	3 & 0 & Invalidation	& 8/S &   $-$/I & $-$/I & 0/S     & V & V \\
	4 & 2 & RMW		& 8/S &   $-$/I & 0/E &   $-$/I   & V & V \\
	5 & 2 & Store		& 8/S &   $-$/I & 0/M &   $-$/I   & I & V \\
	6 & 1 & Atomic Inc	& 8/S &   0/M &   $-$/I & $-$/I   & I & V \\
	7 & 1 & Writeback	& 8/S &   8/S &   $-$/I & $-$/I   & V & V \\
	\bottomrule
\end{tabular}
\caption{Cache Coherence Example}
\label{tab:app:styleguide:Cache Coherence Example}
\end{table*}

Table~\ref{tab:app:styleguide:RCU Publish-Subscribe and Version Maintenance APIs}
is a tweaked version of
Table~\ref{tab:defer:RCU Publish-Subscribe and Version Maintenance APIs}.
Here, the ``Category'' column in the original is removed
and the categories are indicated in rows of bold-face font
just below the mid-rules. This change makes it easier for
\verb|\rowcolors{}| command of ``xcolor'' package to work
properly.

Table~\ref{tab:app:styleguide:RCU Publish-Subscribe and Version Maintenance APIs (colortbl)}
is another example which keeps original columns and colors rows only where
a category has multiple rows. This is done by combining \verb|\rowcolors{}|
of ``xcolor'' and \verb|\cellcolor{}| commands of the ``colortbl''
package (\verb|\cellcolor{}| overrides \verb|\rowcolors{}|).

For consistent looks, the latter layout might be our choice.

\begin{table*}[tbh]
\rowcolors{2}{}{blue!15}
\renewcommand*{\arraystretch}{1.1}
\footnotesize
\centering
\begin{tabular}{lll}
\toprule
	Primitives &
		Availability &
			Overhead \\
\midrule
	\multicolumn{3}{l}{\bfseries List traversal} \\
	\tco{list_for_each_entry_rcu()} &
		2.5.59 &
			Simple instructions (memory barrier on Alpha) \\
\midrule
	\multicolumn{3}{l}{\bfseries List update} \\
	\tco{list_add_rcu()} &
		2.5.44 &
			Memory barrier \\
	\rowcolor{lightgray}\tco{list_add_tail_rcu()} &
		2.5.44 &
			Memory barrier \\
	\tco{list_del_rcu()} &
		2.5.44 &
			Simple instructions \\
	\rowcolor{lightgray}\tco{list_replace_rcu()} &
		2.6.9 &
			Memory barrier \\
	\tco{list_splice_init_rcu()} &
		2.6.21 &
			Grace-period latency \\
\midrule
	\multicolumn{3}{l}{\bfseries Hlist traversal} \\
	\tco{hlist_for_each_entry_rcu()} &
		2.6.8 &
			Simple instructions (memory barrier on Alpha) \\
\midrule
	\multicolumn{3}{l}{\bfseries Hlist update} \\
	\tco{hlist_add_after_rcu()} &
		2.6.14 &
			Memory barrier \\
	\rowcolor{lightgray}\tco{hlist_add_before_rcu()} &
		2.6.14 &
			Memory barrier \\
	\tco{hlist_add_head_rcu()} &
		2.5.64 &
			Memory barrier \\
	\rowcolor{lightgray}\tco{hlist_del_rcu()} &
		2.5.64 &
			Simple instructions \\
	\tco{hlist_replace_rcu()} &
		2.6.15 &
			Memory barrier \\
\midrule
	\multicolumn{3}{l}{\bfseries Pointer traversal} \\
	\tco{rcu_dereference()} &
		2.6.9 &
			Simple instructions (memory barrier on Alpha) \\
\midrule
	\multicolumn{3}{l}{\bfseries Pointer update} \\
	\tco{rcu_assign_pointer()} &
		2.6.10 &
			Memory barrier \\
\bottomrule
\end{tabular}
\caption{RCU Publish-Subscribe and Version Maintenance APIs}
\label{tab:app:styleguide:RCU Publish-Subscribe and Version Maintenance APIs}
\end{table*}

\begin{table*}[tbhp]
\renewcommand*{\arraystretch}{1.2}
\rowcolors{3}{lightgray}{}
\footnotesize
\centering
\begin{tabular}{lllp{1.2in}}\toprule
Category &
	Primitives &
		Availability &
			Overhead \\
\midrule
List traversal &
	\tco{list_for_each_entry_rcu()} &
		2.5.59 &
			Simple instructions (memory barrier on Alpha) \\
\midrule
\cellcolor{white}List update &
	\tco{list_add_rcu()} &
		2.5.44 &
			Memory barrier \\
&
	\tco{list_add_tail_rcu()} &
		2.5.44 &
			Memory barrier \\
\cellcolor{white} &
	\tco{list_del_rcu()} &
		2.5.44 &
			Simple instructions \\
&
	\tco{list_replace_rcu()} &
		2.6.9 &
			Memory barrier \\
\cellcolor{white} &
	\tco{list_splice_init_rcu()} &
		2.6.21 &
			Grace-period latency \\
\midrule
Hlist traversal &
	\tco{hlist_for_each_entry_rcu()} &
		2.6.8 &
			Simple instructions (memory barrier on Alpha) \\
\midrule
\cellcolor{white}Hlist update &
	\tco{hlist_add_after_rcu()} &
		2.6.14 &
			Memory barrier \\
&
	\tco{hlist_add_before_rcu()} &
		2.6.14 &
			Memory barrier \\
\cellcolor{white} &
	\tco{hlist_add_head_rcu()} &
		2.5.64 &
			Memory barrier \\
&
	\tco{hlist_del_rcu()} &
		2.5.64 &
			Simple instructions \\
\cellcolor{white} &
	\tco{hlist_replace_rcu()} &
		2.6.15 &
			Memory barrier \\
\midrule\hiderowcolors
Pointer traversal &
	\tco{rcu_dereference()} &
		2.6.9 &
			Simple instructions (memory barrier on Alpha) \\
\midrule
Pointer update &
	\tco{rcu_assign_pointer()} &
		2.6.10 &
			Memory barrier \\
\bottomrule
\end{tabular}
\caption{RCU Publish-Subscribe and Version Maintenance APIs}
\label{tab:app:styleguide:RCU Publish-Subscribe and Version Maintenance APIs (colortbl)}
\end{table*}

Table~\ref{tab:memorder:Memory Misordering: Store-Buffering Sequence of Events}
can be tweaked as shown in
Table~\ref{tab:app:styleguide:Memory Misordering: Store-Buffering Sequence of Events}.
It is also a sequence diagram drawn as a tabular object.

\begin{table*}[tbh]
\rowcolors{6}{}{lightgray}
\renewcommand*{\arraystretch}{1.1}
\small
\centering\OneColumnHSpace{-0.1in}
\begin{tabular}{rllllll}
	\toprule
	& \multicolumn{3}{c}{CPU 0} & \multicolumn{3}{c}{CPU 1} \\
	\cmidrule(l){2-4} \cmidrule(l){5-7}
	& Instruction & Store Buffer & Cache &
		Instruction & Store Buffer & Cache \\
	\cmidrule{1-1} \cmidrule(l){2-4} \cmidrule(l){5-7}
	1 & (Initial state) & & \tco{x1==0} &
		(Initial state) & & \tco{x0==0} \\
	2 & \tco{x0 = 2;} & \tco{x0==2} & \tco{x1==0} &
		\tco{x1 = 2;} & \tco{x1==2} & \tco{x0==0} \\
	3 & \tco{r2 = x1;} (0) & \tco{x0==2} & \tco{x1==0} &
		\tco{r2 = x0;} (0) & \tco{x1==2} & \tco{x0==0} \\
	4 & (Read-invalidate) & \tco{x0==2} & \tco{x0==0} &
		(Read-invalidate) & \tco{x1==2} & \tco{x1==0} \\
	5 & (Finish store) & & \tco{x0==2} &
		(Finish store) & & \tco{x1==2} \\
	\bottomrule
\end{tabular}
\caption{Memory Misordering: Store-Buffering Sequence of Events}
\label{tab:app:styleguide:Memory Misordering: Store-Buffering Sequence of Events}
\end{table*}

Table~\ref{tab:app:styleguide:Refrigeration Power Consumption (arydshln)}
shows another version of
Table~\ref{tab:future:Refrigeration Power Consumption}
with dashed horizontal and vertical rules of the arydshln package.

\setlength\dashlinedash{.5pt}
\setlength\dashlinegap{1pt}

\begin{table}[H]
\renewcommand*{\arraystretch}{1.2}\centering\small
\begin{tabular}{l:r:r:r}\toprule
Situation
	& $T$ (K)
		& $\CPf$ & \parbox[b]{.75in}{\raggedleft Power per watt\par waste heat (W)} \\
\hline
Dry Ice
	& $195$
		& $1.990$
			& 0.5 \\ \hdashline
Liquid N$_2$
	& $77$
		& $0.356$
			& 2.8 \\ \hdashline
Liquid H$_2$
	& $20$
		& $0.073$
			& 13.7 \\ \hdashline
Liquid He
	& $4$
		& $0.0138$
			& 72.3 \\ \hdashline
IBM~Q	& $0.015$
		& $0.000051$
			& 19,500.0 \\
\bottomrule
\end{tabular}
\caption{Refrigeration Power Consumption}
\label{tab:app:styleguide:Refrigeration Power Consumption (arydshln)}
\end{table}

In this case, the vertical dashed rules seems unnecessary.
The one without the vertical rules is shown in
Table~\ref{tab:app:styleguide:Refrigeration Power Consumption (arydshln-2)}.

\begin{table}[H]
\renewcommand*{\arraystretch}{1.2}\centering\small
\begin{tabular}{lrrr}\toprule
Situation
	& $T$ (K)
		& $\CPf$ & \parbox[b]{.75in}{\raggedleft Power per watt\par waste heat (W)} \\
\midrule
Dry Ice
	& $195$
		& $1.990$
			& 0.5 \\ \hdashline
Liquid N$_2$
	& $77$
		& $0.356$
			& 2.8 \\ \hdashline
Liquid H$_2$
	& $20$
		& $0.073$
			& 13.7 \\ \hdashline
Liquid He
	& $4$
		& $0.0138$
			& 72.3 \\ \hdashline
IBM~Q	& $0.015$
		& $0.000051$
			& 19,500.0 \\
\bottomrule
\end{tabular}
\caption{Refrigeration Power Consumption}
\label{tab:app:styleguide:Refrigeration Power Consumption (arydshln-2)}
\end{table}

Tables in Chapter~\ref{chp:memorder:Memory Ordering}
have been recently converted to the scheme presented in this section.
Refer to \path{memorder/memorder.tex}
for examples of tables with complex headings.

\floatstyle{plain}
\restylefloat{table}
\captionsetup[table]{position=bottom,hangindent=0pt}
\renewcommand*{\abovetopsep}{0pt}

\subsubsection{Miscellaneous Candidates}
\label{sec:app:styleguide:Miscellaneous Candidates}

Other improvement candidates are listed in the source of this
section as comments.

% Trademarks:
%    As the Legal page covers trademarks, there is no need to
%    use trademark symbol in the text. They seems to have been
%    imported from original publications.
%
% Ugly line break by \co{}
%                                 __
%        atomic_store()
%
%                           seqlock_
%        t
%
%   Is there any way to prevent these breaks?
%   Maybe we need an on-the-fly script to convert such \co{}s
%   to couples of \co{}s.
%   Example:
%     \co{__atomic_store()} -> \co{__}\co{atomic_store()}
%     \co{seqlock_t} ->        \co{seqlock_}\co{t}
