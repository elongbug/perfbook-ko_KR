% future/htm.tex
% mainfile: ../perfbook.tex
% SPDX-License-Identifier: CC-BY-SA-3.0

\section{Hardware Transactional Memory}
\label{sec:future:Hardware Transactional Memory}
%
\epigraph{Make sure your report system is reasonably clean and efficient
	  before you automate.
	  Otherwise, your new computer will just speed up the mess.}
	 {\emph{Robert Townsend}}
% If at first you do succeed---try to hide your astonishment.
% Harry F.~Banks

2021년 기준, 하드웨어 트랜잭셔널 메모리 (HTM) 은 상업적으로 접근 가능한 상품
컴퓨터 시스템의 여러 형태로 수년째
가용해졌습니다~\cite{Yoo:2013:PEI:2503210.2503232,RickMerrit2011PowerTM,ChristianJacobi2012MainframeTM,TimothyHayes2020ARM-HTM}.
이 섹션은 병렬 프로그래머의 도구상자에서 HTM 의 위치를 정의해 보려 합니다.

\iffalse

As of 2021, hardware transactional memory (HTM) has been available for many
years on several types of commercially available commodity computer
systems~\cite{Yoo:2013:PEI:2503210.2503232,RickMerrit2011PowerTM,ChristianJacobi2012MainframeTM,TimothyHayes2020ARM-HTM}.
This section makes an attempt to identify HTM's place in the parallel
programmer's toolbox.

\fi

컨셉적 관점에서, HTM 은 선정된 선언문들이 (``트랜잭션'') 다른 프로세서에서
수행중인 다른 트랜잭션의 관점에서 볼 때 어토믹하게 효과를 발휘하도록 하게 하기
위해 프로세서 캐쉬와 예측적 수행을 사용합니다.
이 트랜잭션은 begin-transaction 기계 명령에 의해 시작되고 commit-transaction
기계 명령에 의해 완료됩니다.
일반적으로 abort-transaction 기계 명령도 존재하는데, 예측을 찌그러뜨리고
(begin-transaction 명령과 뒤따르는 명령들이 수행되지 않은 것처럼) 실패
처리기에서의 수행을 시작합니다.
실패 처리기의 위치는 일반적으로 begin-transaction 명령에 의해 명시되는데,
명시적 실패 처리기 주소가 주어지거나 명령 자체에 의해 설정되는 조건 코드를 통해
주어집니다.
각 트랜잭션은 모든 다른 트랜잭션에 대해 원자적으로 수행됩니다.

HTM 은 여러 중요한 이득을 갖는데, 데이터 구조의 자동화된 동적 파티셔닝, 동기화
기능의 캐쉬 미스 감소, 상당한 수의 실용적 어플리케이션의 지원이 포함됩니다.

\iffalse

From a conceptual viewpoint, HTM uses processor caches and speculative
execution to make a designated group of statements (a ``transaction'')
take effect atomically
from the viewpoint of any other transactions running on other processors.
This transaction is initiated by a
begin-transaction machine instruction and completed by a commit-transaction
machine instruction.
There is typically also an abort-transaction machine instruction, which
squashes the speculation (as if the begin-transaction instruction and
all following instructions had not executed) and commences execution
at a failure handler.
The location of the failure handler is typically specified by the
begin-transaction instruction, either as an explicit failure-handler
address or via a condition code set by the instruction itself.
Each transaction executes atomically with respect to all other transactions.

HTM has a number of important benefits, including automatic
dynamic partitioning of data structures, reducing synchronization-primitive
cache misses, and supporting a fair number of practical applications.

\fi

그러나, 제대로 된 문서를 읽을 것이 항상 필요하며, HTM 도 예외가 아닙니다.
이 섹션의 주요 지점은 어떤 조건에서 HTM 의 이득이 그것의 제대로 된 문서에 숨어
있는 복잡도를 이겨내는가를 알아내는 것입니다.
따라서, \cref{sec:future:HTM Benefits WRT Locking} 는 HTM 의 이득을 이야기하고,
\cref{sec:future:HTM Weaknesses WRT Locking} 는 그 약점을 이야기 합니다.
이는 이전의 섹션들과
논문들에서와~\cite{McKenney2007PLOSTM,PaulEMcKenney2010OSRGrassGreener} 같은
방법입니다.\footnote{
	다른 저자들, Maged Michael, Josh Triplett, Jonathan Walpole, Andi Kleen
	등과의 많은 자극적 토론에 소중한 감사를 표합니다.}

이어서
\cref{sec:future:HTM Weaknesses WRT Locking When Augmented} 는 리눅스 커널
(그리고 많은 유저 스페이스 어플리케이션) 에서 사용되는 동기화 기능들과의 조합
관점에서의 HTM 의 약점을 설명합니다.
\Cref{sec:future:Where Does HTM Best Fit In?} 는 병렬 프로그래머의 도구상자
내에서 어디에 HTM 이 가장 잘 맞을지 알아보고,
\cref{sec:future:Potential Game Changers} 는 HTM 의 영역과 인상을 크게 개선시킬
수도 있을 이벤트를 일부 나열해 봅니다.
마지막으로, \cref{sec:future:Conclusions} 는 결론을 짓습니다.

\iffalse

However, it always pays to read the fine print, and HTM is no exception.
A major point of this section is determining under what conditions HTM's
benefits outweigh the complications hidden in its fine print.
To this end, \cref{sec:future:HTM Benefits WRT Locking}
describes HTM's benefits and
\cref{sec:future:HTM Weaknesses WRT Locking} describes its weaknesses.
This is the same approach used in earlier
papers~\cite{McKenney2007PLOSTM,PaulEMcKenney2010OSRGrassGreener}
and also in the previous section.\footnote{
	I gratefully acknowledge many stimulating
	discussions with the other authors, Maged Michael, Josh Triplett,
	and Jonathan Walpole, as well as with Andi Kleen.}

\Cref{sec:future:HTM Weaknesses WRT Locking When Augmented} then describes
HTM's weaknesses with respect to the combination of synchronization
primitives used in the Linux kernel (and in many user-space applications).
\Cref{sec:future:Where Does HTM Best Fit In?} looks at where HTM
might best fit into the parallel programmer's toolbox, and
\cref{sec:future:Potential Game Changers} lists some events that might
greatly increase HTM's scope and appeal.
Finally, \cref{sec:future:Conclusions}
presents concluding remarks.

\fi

\subsection{HTM Benefits WRT Locking}
\label{sec:future:HTM Benefits WRT Locking}

HTM 의 주요 장점은
(1)~다른 동기화 기능에 의해 종종 일어나는 캐쉬 미스의 회피,
(2)~동적으로 데이터 구조를 파티셔닝하는 능력, 그리고
(3)~상당한 수의 실용적 어플리케이션을 갖는다는 사실 입니다.
저는 TM 에 대한 전통과 달리 두가지 이유로 쉬운 사용성을 여기에 포함시키지
않습니다.
첫째로, 쉬운 사용성은 HTM 의 주요 장점으로부터 생겨나는 것인데, 그게 이 섹션이
주목하는 부분입니다.
둘째로, 기본적 프로그래밍 재능에 대한
검사와~\cite{RichardBornat2006SheepGoats,SaeedDehnadi2009SheepGoats,ElizabethPatitsas2020GradesNotBimodal}
심지어 취업 면접에서의 작은 프로그래밍 테스트의 사용에
대한~\cite{RegBraithwaite2007FizzBuzz} 상당한 노란이 있었습니다.
이는 무엇이 프로그래밍을 쉽게 하는지 어렵게 하는지에 대한 확고한 이해를 정말로
갖지 못하고 있음을 보입니다.
따라서, 이 섹션의 나머지 부분은 앞에 나열된 세가지 장점에 집중합니다.

\iffalse

The primary benefits of HTM are
(1)~its avoidance of the cache misses that are often incurred by
other synchronization primitives,
(2)~its ability to dynamically partition
data structures,
and (3)~the fact that it has
a fair number of practical applications.
I break from TM tradition by not listing ease of use separately
for two reasons.
First, ease of use should stem from HTM's primary benefits,
which this section focuses on.
Second, there has been considerable controversy surrounding attempts to
test for raw programming
talent~\cite{RichardBornat2006SheepGoats,SaeedDehnadi2009SheepGoats,ElizabethPatitsas2020GradesNotBimodal}
and even around the use of small programming exercises in job
interviews~\cite{RegBraithwaite2007FizzBuzz}.
This indicates that we really do not have a firm grasp on what makes
programming easy or hard.
Therefore, the remainder of this section focuses on the three benefits
listed above.

\fi

\subsubsection{Avoiding Synchronization Cache Misses}
\label{sec:future:Avoiding Synchronization Cache Misses}

대부분의 동기화 메커니즘은 어토믹 명령에 의해 운영되는 데이터 구조에
기반합니다.
이 어토믹 명령들은 일반적으로 일단 연관된 캐쉬라인이 그들이 수행중인 CPU 에
의해 소유되게 하여, 다른 CPU 에서의 같은 동기화 기능의 사용은 캐쉬 미스를
일으키게 합니다.
이 통신을 위한 캐쉬 미스는 전통적 동기화 메커니즘의 성능과 확장성을 크게
악화시킵니다~\cite[Section 4.2.3]{Anderson97}.

\iffalse

Most synchronization mechanisms are based on data structures that are
operated on by atomic instructions.
Because these atomic instructions normally operate by first causing
the relevant cache line to be owned by the CPU that they are running on,
a subsequent execution
of the same instance of that synchronization primitive on some other
CPU will result in a cache miss.
These communications cache misses severely degrade both the performance and
scalability of conventional synchronization
mechanisms~\cite[Section 4.2.3]{Anderson97}.

\fi

대조적으로, HTM 은 해당 CPU 의 캐쉬를 이용해 동기화를 하므로 별도의 동기화
데이터 구조와 그로 인한 캐쉬 미스를 회피합니다.
HTM 의 장점은 락 데이터 구조가 별도의 캐쉬 라인에 위치하는 경우 가장 커지는데,
이 경우 크리티컬 섹션을 HTM 트랜잭션으로 변환시키는 것은 크리티컬 섹션의 캐쉬
미스로 인한 오버헤드를 완전히 줄일 수 있습니다.
이로 인한 절약은 짧은 크리티컬 섹션이라는 흔한 경우 상당히 클 수 있는데, 최소한
제거된 락이 그 락에 의해 보호되는 변수의 캐쉬 라인을 공유하는 많은 경우가 아닐
경우에는 그렇습니다.

\iffalse

In contrast, HTM synchronizes by using the CPU's cache, avoiding the need
for a separate synchronization data structure and resultant cache misses.
HTM's advantage is greatest in cases where a lock data structure is
placed in a separate cache line, in which case, converting a given
critical section to an HTM transaction can reduce that critical section's
overhead by a full cache miss.
These savings can be quite significant for the common case of short
critical sections, at least for those situations where the elided lock
does not share a cache line with an oft-written variable protected by
that lock.

\fi

\QuickQuiz{
	왜 락 변수와 보호되는 변수가 캐쉬 라인을 공유하는 일반적 경우가
	중요하죠?

	\iffalse

	Why would it matter that oft-written variables shared the cache
	line with the lock variable?

	\fi

}\QuickQuizAnswer{
	그 락이 자신이 보호하는 변수들과 같은 캐쉬라인에 있다면, 한 CPU 에 의한
	이 변수들로의 쓰기는 모든 다른 CPU 의 캐쉬 라인을 무효화 시킬 겁니다.
	이 무효화는 큰 수의 충돌과 재시도를 발생시키는데, 이는 아마 락킹에
	비해서도 성능과 확장성을 더 떨어뜨릴 수도 있습니다.

	\iffalse

	If the lock is in the same cacheline as some of the variables
	that it is protecting, then writes to those variables by one CPU
	will invalidate that cache line for all the other CPUs.
	These invalidations will
	generate large numbers of conflicts and retries, perhaps even
	degrading performance and scalability compared to locking.

	\fi

}\QuickQuizEnd

\subsubsection{Dynamic Partitioning of Data Structures}
\label{sec:future:Dynamic Partitioning of Data Structures}

일부 전통적인 동기화 메커니즘의 사용에 있어서의 주요 장애물은 정적으로 데이터
구조를 분할해야 한다는 필요성입니다.
가장 흔한 예로는 각 해쉬 체인이 하나의 조각이 되는 해쉬 테이블과 같은 쉽게
분할될 수 있는 데이터 구조도 여럿 있습니다.
각 해쉬 체인을 위한 락을 할당하는 것은 간단히 이 해쉬 테이블을 해당 체인으로
국한된 오퍼레이션에 대해 분할할 수 있습니다.\footnote{
	그리고 이 방법을 여러 해쉬 체인을 접근하는 오퍼레이션에 대해서는 연관된
	모든 락을 해쉬 순서로 획득하게 하는 식으로 쉽게 확장할 수 있습니다.}
분할하기는 비슷하게 배열, 래딕스 트리, 스킵리스트, 그리고 여러 다른 데이터
구조에 있어서도 간단합니다.

그러나, 여러 종류의 트리와 그래프에 있어 분할하기는 상당히 어려우며, 그
결과물은 종종 복잡합니다~\cite{Ellis80}.
일반적 데이터 구조를 분할하기 위해선
ㅅwo-phased 락킹과 락의 해슁된 배열을 사용하는게 가능하지만,
\cref{sec:future:HTM Weaknesses WRT Locking When Augmented} 에서 곧 논의할
것이지만 다른 기법이 선호된다는게
밝혀졌습니다~\cite{DavidSMiller2006HashedLocking}.
동기화 캐쉬 미스의 회피 덕에, HTM 은 거대한 분할 불가능한 데이터 구조에 대해
최소한 상대적으로 적은 업데이트를 상정하면 사용 가능성이 있습니다.

\iffalse

A major obstacle to the use of some conventional synchronization mechanisms
is the need to statically partition data structures.
There are a number of data structures that are trivially
partitionable, with the most prominent example being hash tables,
where each hash chain constitutes a partition.
Allocating a lock for each hash chain then trivially parallelizes
the hash table for operations confined to a given chain.\footnote{
	And it is also easy to extend this scheme to operations accessing
	multiple hash chains by having such operations acquire the
	locks for all relevant chains in hash order.}
Partitioning is similarly trivial for arrays, radix trees, skiplists, and
several other data structures.

However, partitioning for many types of trees and graphs is quite
difficult, and the results are often quite complex~\cite{Ellis80}.
Although it is possible to use two-phased locking and hashed arrays
of locks to partition general data structures, other techniques
have proven preferable~\cite{DavidSMiller2006HashedLocking},
as will be discussed in
\cref{sec:future:HTM Weaknesses WRT Locking When Augmented}.
Given its avoidance of synchronization cache misses,
HTM is therefore a very real possibility for large non-partitionable
data structures, at least assuming relatively small updates.

\fi

\QuickQuiz{
	HTM 의 성능과 확장성에 있어 상대적으로 적은 업데이트가 왜 중요한가요?

	\iffalse

	Why are relatively small updates important to HTM performance
	and scalability?

	\fi

}\QuickQuizAnswer{
	업데이트가 클수록, 충돌의 가능성이 커지며, 따라서 재시도의 가능성이
	높아지는데, 이는 성능을 떨어뜨립니다.

	\iffalse

	The larger the updates, the greater the probability of conflict,
	and thus the greater probability of retries, which degrade
	performance.

	\fi

}\QuickQuizEnd

\subsubsection{Practical Value}
\label{sec:future:Practical Value}

HTM 의 실용적 가치에 대한 일부 증거가 여러 하드웨어 플랫폼에서 선보였는데,
Sun Rock~\cite{DaveDice2009ASPLOSRockHTM},
Azul Vega~\cite{CliffClick2009AzulHTM},
IBM Blue Gene/Q~\cite{RickMerrit2011PowerTM},
Intel Haswell TSX~\cite{RaviRajwar2012TSX}, 그리고
IBM System z~\cite{ChristianJacobi2012MainframeTM} 이 포함됩니다.

예상되는 실용적 이득은 다음을 포함합니다:

\iffalse

Some evidence of HTM's practical value has been demonstrated in a number
of hardware platforms, including
Sun Rock~\cite{DaveDice2009ASPLOSRockHTM},
Azul Vega~\cite{CliffClick2009AzulHTM},
IBM Blue Gene/Q~\cite{RickMerrit2011PowerTM},
Intel Haswell TSX~\cite{RaviRajwar2012TSX}, and
IBM System z~\cite{ChristianJacobi2012MainframeTM}.

Expected practical benefits include:

\fi

\begin{enumerate}
\item	메모리 내 데이터 접근과 업데이트 시의 락
	제거~\cite{Martinez01a,Rajwar02a}.
\item	거대한 분할 불가한 데이터 구조로의 동시의 액세스와 작은 무작위
	업데이트.

\iffalse

\item	Lock elision for in-memory data access and
	update~\cite{Martinez01a,Rajwar02a}.
\item	Concurrent access and small random updates to large non-partitionable
	data structures.

\fi

\end{enumerate}

그러나, HTM 은 또한 매우 실제적인 단점을 갖는데, 다음 섹션에서 다룹니다.

\iffalse

However, HTM also has some very real shortcomings, which will be discussed
in the next section.

\fi

\subsection{HTM Weaknesses WRT Locking}
\label{sec:future:HTM Weaknesses WRT Locking}

HTM 의 컨셉은 상당히 간단합니다: 한 그룹의 액세스와 업데이트가 원자적으로
이루어집니다.
그러나, 많은 간단한 아이디어의 경우처럼, 실제 세계의 실제 시스템에 이를 적용할
때 복잡도가 나타납니다.
이 복잡도는 다음과 같습니다:

\iffalse

The concept of HTM is quite simple: A group of accesses and updates to
memory occurs atomically.
However, as is the case with many simple ideas, complications arise
when you apply it to real systems in the real world.
These complications are as follows:

\fi

\begin{enumerate}
\item	트랜잭션 크기 제한.
\item	충돌 처리.
\item	취소하기와 되돌리기.
\item	진행 보장의 부재.
\item	취소 불가한 오퍼레이션.
\item	의미상의 차이.

\iffalse

\item	Transaction-size limitations.
\item	Conflict handling.
\item	Aborts and rollbacks.
\item	Lack of forward-progress guarantees.
\item	Irrevocable operations.
\item	Semantic differences.

\fi

\end{enumerate}

이 복잡성 각각이 다음 섹션에서 다루어지며, 그 뒤에 요약이 나옵니다.

\iffalse

Each of these complications is covered in the following sections,
followed by a summary.

\fi

\subsubsection{Transaction-Size Limitations}
\label{sec:future:Transaction-Size Limitations}

현재 HTM 구현의 트랜잭션 크기 제한은 트랜잭션에 의해 영향받는 데이터를 프로세서
캐쉬에 둔다는 점에서 기인합니다.
이게 특정 CPU 가 트랜잭션을 자신의 캐쉬에 국한시켜서 수행함으로써 이를
원자적으로 행하는 것으로 다른 CPU 에게 보이게 하지만, 이는 또한 여기 들어맞지
않는 트랜잭션은 커밋될 수 없음을 의미합니다.
더 나아가, 인터럽트, 시스템콜, 예외, 트랩, 그리고 컨텍스트 스위치 같은 수행
컨텍스트를 바꾸는 이벤트는 해당 CPU 의 진행중인 트랜잭션을 취소시키거나 이 다른
수행 문맥의 캐쉬 사용량으로 인해 트랜잭션 크기를 더 제한해야 합니다.

\iffalse

The transaction-size limitations of current HTM implementations
stem from the use of the processor caches to hold the data
affected by the transaction.
Although this allows a given CPU to make the transaction appear atomic to
other CPUs by executing the transaction within the confines of its cache,
it also means that any transaction that does not fit cannot commit.
Furthermore, events that change execution context, such as interrupts,
system calls, exceptions, traps, and context switches either must
abort any ongoing transaction on the CPU in question or must further
restrict transaction size due to the cache footprint of the other
execution context.

\fi

물론, 현대의 CPU 는 큰 캐쉬를 갖는 경향을 보이며, 많은 트랜잭션에 필요한
데이터는 쉽게 1 메가바이트 캐쉬에 담길 겁니다.
불행히도, 캐쉬에 있어 작은 크기가 전부는 아닙니다.
문제는 대부분의 캐쉬는 하드웨어로 구현된 해쉬 테이블로 생각될 수 있습니다.
그러나, 하드웨어 캐쉬는 버킷을 (일반적으로 \emph{set} 이라 불립니다) 연결시키지
않고, 고정된 수의 set 별 캐쉬라인을 제공합니다.
각 set 에 의해 제공된 원소의 갯수를 해당 캐쉬의 \emph{associativity} 라
불립니다.

\iffalse

Of course, modern CPUs tend to have large caches, and the data required
for many transactions would fit easily in a one-megabyte cache.
Unfortunately, with caches, sheer size is not all that matters.
The problem is that most caches
can be thought of hash tables implemented in hardware.
However, hardware caches do not chain their buckets (which are normally
called \emph{sets}), but rather
provide a fixed number of cachelines per set.
The number of elements provided for each set in a given cache
is termed that cache's \emph{associativity}.

\fi

캐쉬 associativity 는 다양하지만, 제가 이를 타이핑 하는 랩탑에 있는 8-way
associativity 레벨 0 캐쉬는 드물지 않습니다.
이게 의미하는 바는 특정 트랜잭션이 아홉개의 캐쉬 라인을 건드려야 하며 그
아홉개의 캐쉬 라인이 모두 같은 set 으로 매핑된다면 해당 캐쉬에 얼마나 많은
추가적 공간이 존재하는가와 관계없이 이 트랜잭션은 결코 완료될 수 없다는 겁니다.
그렇습니다, 특정 데이터 구조에서 데이터 원소가 무작위적으로 선택된다면 그
트랜잭션이 커밋될 가능성은 상당히 높습니다, 그러나 보장은
없습니다~\cite{PaulEMcKenney2012HTMCacheGeometry}.

\iffalse

Although cache associativity varies, the eight-way associativity of
the level-0 cache on the laptop I am typing this on is not unusual.
What this means is that if a given transaction needed to touch
nine cache lines, and if all nine cache lines mapped to the same
set, then that transaction cannot possibly complete, never mind how
many megabytes of additional space might be available in that cache.
Yes, given randomly selected data elements in a given data structure,
the probability of that transaction being able to commit is quite
high, but there can be no guarantee~\cite{PaulEMcKenney2012HTMCacheGeometry}.

\fi

이 제한을 완화시키려는 일부 연구가 있었습니다.
완전한 associativity 의 \emph{victim cache} 는 이 ssociativity 제한을 완화시킬
테지만, victim 캐쉬의 크기에 대한 까다로운 성능과 에너지 효율 제한이 있습니다.
그렇다고 하나, 수정되지 않은 캐쉬 라인을 위한 HTM victim 캐쉬는 상당히 작을 수
있는데, 주소만 가지면 되기 때문입니다:
데이터 자체는 메모리에 저장되거나 다른 캐쉬에 의해 따라갈 수 있으며, 주소
자체만으로도 쓰기와의 충돌을 탐지하기 충분합니다~\cite{RaviRajwar2012TSX}.

\iffalse

There has been some research work to alleviate this limitation.
Fully associative \emph{victim caches} would alleviate the associativity
constraints, but there are currently stringent performance and
energy-efficiency constraints on the sizes of victim caches.
That said, HTM victim caches for unmodified cache lines can be quite
small, as they need to retain only the address:
The data itself can be written to memory or shadowed by other caches,
while the address itself is sufficient to detect a conflicting
write~\cite{RaviRajwar2012TSX}.

\fi

\emph{Unbounded transactional memory} (UTM)
방법은~\cite{CScottAnanian2006,KevinEMoore2006}
DRAM 을 극단적으로 큰 victim 캐쉬로 사용합니다만, 그런 방법을 제품 품질의 캐쉬
일관성 메커니즘과 결합하는 것은 여전히 해결되지 않은 문제입니다.
또한, DRAM 을 victim 캐쉬로 사용하는 것은 불행한 성능과 에너지 효율성 결론을
가질 수 있으며, 특히 victim 캐쉬가 완전한 ssociativity 를 갖는다면 그렇습니다.
마지막으로, ``unbounded'' 라는 UTM 의 속성은 모든 DRAM 이 victim 캐쉬로 사용될
수 있음을 가정하는데, 실제로는 크지만 여전히 고정된 양의 DRAM 이 특정 CPU 에
할당되므로 해당 CPU 의 트랜잭션의 크기는 제한될 겁니다.
다른 방법들은 하드웨어와 소프트웨어 트랜잭션 메모리의 결합을
사용하며~\cite{SanjeevKumar2006} 여기서 STM 은 HTM 의 fallback 메커니즘으로
생각될 수 있습니다.

그러나, 제가 알기로는 TM 읽기 집합의 표현의 간략화라는 예외가 있지만 현재 사용
가능한 시스템들은 이런 연구 아이디어들을 구현하지 않았는데, 아마도 좋은 이유가
있었을 겁니다.

\iffalse

\emph{Unbounded transactional memory} (UTM)
schemes~\cite{CScottAnanian2006,KevinEMoore2006}
use DRAM as an extremely large victim cache, but integrating such schemes
into a production-quality cache-coherence mechanism is still an unsolved
problem.
In addition, use of DRAM as a victim cache may have unfortunate
performance and energy-efficiency consequences, particularly
if the victim cache is to be fully associative.
Finally, the ``unbounded'' aspect of UTM assumes that all of DRAM
could be used as a victim cache, while in reality
the large but still fixed amount of DRAM assigned to a given CPU
would limit the size of that CPU's transactions.
Other schemes use a combination of hardware and software transactional
memory~\cite{SanjeevKumar2006} and one could imagine using STM as a
fallback mechanism for HTM\@.

However, to the best of my knowledge, with the exception of abbreviating
representation of TM read sets, currently available systems do not
implement any of these research ideas, and perhaps for good reason.

\fi

\subsubsection{Conflict Handling}
\label{sec:future:Conflict Handling}

첫번째 복잡성은 \emph{충돌} 가능성입니다.
예를 들어, 트랜잭션~A 와~B 가 다음과 같이 정의되었다고 해봅시다.

\iffalse

The first complication is the possibility of \emph{conflicts}.
For example, suppose that transactions~A and~B are defined as follows:

\fi

\begin{VerbatimU}
Transaction A       Transaction B

x = 1;              y = 2;
y = 3;              x = 4;
\end{VerbatimU}

각 트랜잭션이 각자의 프로세서에서 동시에 수행된다고 해봅시다.
트랜잭션~A \co{x} 로의 스토어를 트랜잭션~B 의 \co{y} 스토어와 동시에 행하면,
어떤 트랜잭션도 진행될 수 없습니다.
이를 자세히 보기 위해, 트랜잭션~A 가 \co{y} 로의 스토어를 하는걸 봅시다.
그럼 트랜잭션~A 는 트랜잭션~B 와 엮이게 되어, 각자에 대해 트랜잭션에 원자적으로
수행되어야 한다는 요구사항을 어기게 됩니다.
트랜잭션~B 가 \co{x} 로의 스토어를 하게 하는 것 역시 비슷하게 원자적 수행
요구사항을 어기게 합니다.
이 상황은 \emph{충돌 (conflict)} 라 명명되었는데, 두 동시의 트랜잭션이 같은
변수에 액세스 하며 그 액세스 중 최소 하나는 스토어일 때 발생합니다.
따라서 이 시스템은 수행이 진행될 수 있게 하기 위해 트랜잭션 중 하나 또는 둘
다를 중지시켜야 합니다.
어떤 트랜잭션을 중지시킬지의 선택은 박사 학위논문을 만들기 충분할 만큼의 능력을
얻게 할만큼 흥미로운 주제일텐데 그런 예도
있습니다~\cite{EgeAkpinar2011HTM2TLE}.\footnote{
	Liu 와 Spear 의 ``Toxic
	Transactions''~\cite{YujieLiu2011ToxicTransactions} 는 특히
	교훈적입니다.}
이 섹션의 목표에 집중하기 위해, 우린 시스템이 무작위적 선택을 한다고 가정할 수
있습니다.

\iffalse

Suppose that each transaction executes concurrently on its own processor.
If transaction~A stores to \co{x} at the same time that transaction~B
stores to \co{y}, neither transaction can progress.
To see this, suppose that transaction~A executes its store to \co{y}.
Then transaction~A will be interleaved within transaction~B, in violation
of the requirement that transactions execute atomically with respect to
each other.
Allowing transaction~B to execute its store to \co{x} similarly violates
the atomic-execution requirement.
This situation is termed a \emph{conflict}, which happens whenever two
concurrent transactions access the same variable where at least one of
the accesses is a store.
The system is therefore obligated to abort one or both of the transactions
in order to allow execution to progress.
The choice of exactly which transaction to abort is an interesting topic
that will very likely retain the ability to generate Ph.D. dissertations for
some time to come, see for
example~\cite{EgeAkpinar2011HTM2TLE}.\footnote{
	Liu's and Spear's paper entitled ``Toxic
	Transactions''~\cite{YujieLiu2011ToxicTransactions} is
	particularly instructive.}
For the purposes of this section, we can assume that the system makes
a random choice.

\fi

또다른 복잡성은 충돌 탐지로, 최소한 가장 간단한 경우에는 비교적 간단합니다.
프로세서는 트랜잭션을 수행할 때 이 트랜잭션에 의해 만져진 모든 캐쉬라인을
표시합니다.
만약 이 프로세서의 캐쉬가 현재 트랜잭션에 의해 접촉된 것으로 표시된 캐쉬 라인에
연관된 요청을 받는다면, 잠재적 충돌이 일어난 겁니다.
더 정교한 시스템은 현재 프로세서의 트랜잭션이 그 요청을 보낸 프로세서의 것보다
앞서 수행되도록 노력할 것이고, 이 과정을 최적화 하는 것은 역시나 박사
학위논문을 만들 능력을 얻을 수 있게 할 겁니다.
그러나 이 섹션은 매우 간단한 충돌 탐지 전략을 가정합니다.

\iffalse

Another complication is conflict detection, which is comparatively
straightforward, at least in the simplest case.
When a processor is executing a transaction, it marks every cache line
touched by that transaction.
If the processor's cache receives a request involving a cache line that
has been marked as touched by the current transaction, a potential
conflict has occurred.
More sophisticated systems might try to order the current processors'
transaction to precede that of the processor sending the request, and
optimizing this process will likely also retain the ability to generate
Ph.D. dissertations for quite some time.
However this section assumes a very simple conflict-detection strategy.

\fi

그러나, HTM 이 효과적으로 동작하기 위해선 충돌의 가능성이 상당히 낮아야 하는데,
이는 결국 데이터 구조가 충분히 낮은 충돌 확률을 유지하게끔 짜여져야 할 것을
필요로 합니다.
예를 들어, 간단한 삽입, 삭제, 그리고 탐색을 하는 red-black 트리는 이 경우에
들어맞습니다만, 트리의 원소들의 정확한 수를 유지하는 red-black 트리는 그렇지
않습니다.\footnote{
	이 카운트를 업데이트 해야 한다는 필요성이 트리에의 삽입과 삭제가 서로
	충돌하게 만들어, 강력한 non-commutativity 를
	초래합니다~\cite{HagitAttiya2011LawsOfOrder,Attiya:2011:LOE:1925844.1926442,PaulEMcKenney2011SNC}.}
또다른 예로, 하나의 트랜잭션 내에서 트리의 모든 원소를 접근하는 red-black
트리는 높은 충돌 가능성을 가져서 성능과 확장성을 떨어뜨립니다.
그 결과, 많은 순차적 프로그램들은 HTM 이 효과적으로 동작하기 전에 일부 재구축을
필요로 할 겁니다.
어떤 경우에는, 실무자들은 추가적 단계들을 취하고 (red-black 트리의 경우, radix
트리나 해쉬 테이블 같은 분할 가능한 데이터 구조로의 변경 같은) 간단히 락킹을
하는걸 선호할 텐데, 특히 HTM 이 모든 관련 아키텍쳐에서 사용될 준비가 될
때까지는 그럴 겁니다~\cite{CliffClick2009AzulHTM}.

\iffalse

However, for HTM to work effectively, the probability of conflict must
be quite low, which in turn requires that the data structures
be organized so as to maintain a sufficiently low probability of conflict.
For example, a red-black tree with simple insertion, deletion, and search
operations fits this description, but a red-black
tree that maintains an accurate count of the number of elements in
the tree does not.\footnote{
	The need to update the count would result in additions to and
	deletions from the tree conflicting with each other, resulting
	in strong non-commutativity~\cite{HagitAttiya2011LawsOfOrder,Attiya:2011:LOE:1925844.1926442,PaulEMcKenney2011SNC}.}
For another example, a red-black tree that enumerates all elements in
the tree in a single transaction will have high conflict probabilities,
degrading performance and scalability.
As a result, many serial programs will require some restructuring before
HTM can work effectively.
In some cases, practitioners will prefer to take the extra steps
(in the red-black-tree case, perhaps switching to a partitionable
data structure such as a radix tree or a hash table), and just
use locking, particularly until such time as HTM is readily available
on all relevant
architectures~\cite{CliffClick2009AzulHTM}.

\fi

\QuickQuiz{
	동기화 메커니즘의 선택에 관계 없이 red-black 트리가 어떻게 모든 원소를
	효율적으로 접근할 수 있죠???

	\iffalse

	How could a red-black tree possibly efficiently enumerate all
	elements of the tree regardless of choice of synchronization
	mechanism???

	\fi

}\QuickQuizAnswer{
	많은 경우, 그 접근은 정확할 필요가 없습니다.
	이 경우, 읽기 쓰레드를 보호하기 위해 어떤 삽입이나 삭제와도 낮은 충돌
	확률을 제공하는 해저드 포인터나 RCU 를 사용할 수 있을 겁니다.

	\iffalse

	In many cases, the enumeration need not be exact.
	In these cases, hazard pointers or RCU may be used to protect
	readers, which provides low probability of conflict with any
	given insertion or deletion.

	\fi

}\QuickQuizEnd

더 나아가, 동시의 트랜잭션 사이에서의 충돌하는 액세스의 가능성이 실패를 초래할
수 있습니다.
그런 실패를 처리하는 것에 대해 다음 섹션에서 다룹니다.

\iffalse

Furthermore, the potential for conflicting accesses among concurrent
transactions can result in failure.
Handling such failure is discussed in the next section.

\fi

\subsubsection{Aborts and Rollbacks}
\label{sec:future:Aborts and Rollbacks}

Because any transaction might be aborted at any time, it is important
that transactions contain no statements that cannot be rolled back.
This means that transactions cannot do I/O, system calls, or debugging
breakpoints (no single stepping in the debugger for HTM transactions!!!).
Instead, transactions must confine themselves to accessing normal
cached memory.
Furthermore, on some systems, interrupts, exceptions, traps,
TLB misses, and other events will also abort transactions.
Given the number of bugs that have resulted from improper handling
of error conditions, it is fair to ask what impact aborts and rollbacks
have on ease of use.

\QuickQuiz{
	But why can't a debugger emulate single stepping by setting
	breakpoints at successive lines of the transaction, relying
	on the retry to retrace the steps of the earlier instances
	of the transaction?
}\QuickQuizAnswer{
	This scheme might work with reasonably high probability, but it
	can fail in ways that would be quite surprising to most users.
	To see this, consider the following transaction:

\begin{fcvlabel}[ln:future:htm:debug rollbacks]
\begin{VerbatimN}[commandchars=\\\[\]]
begin_trans();
if (a) {
	do_one_thing();
	do_another_thing();	\lnlbl[another]
} else {
	do_a_third_thing();
	do_a_fourth_thing();
}
end_trans();
\end{VerbatimN}
\end{fcvlabel}

	\begin{fcvref}[ln:future:htm:debug rollbacks]
	Suppose that the user sets a breakpoint at \clnref{another},
	which triggers,
	aborting the transaction and entering the debugger.
	\end{fcvref}
	Suppose that between the time that the breakpoint triggers
	and the debugger gets around to stopping all the threads, some
	other thread sets the value of \co{a} to zero.
	When the poor user attempts to single-step the program, surprise!
	The program is now in the else-clause instead of the then-clause.

	This is \emph{not} what I call an easy-to-use debugger.
}\QuickQuizEnd

Of course, aborts and rollbacks raise the question of whether HTM can
be useful for hard real-time systems.
Do the performance benefits of HTM outweigh the costs of the aborts
and rollbacks, and if so under what conditions?
Can transactions use priority boosting?
Or should transactions for high-priority threads instead preferentially
abort those of low-priority threads?
If so, how is the hardware efficiently informed of priorities?
The literature on real-time use of HTM is quite sparse, perhaps
because there are more than enough problems in making HTM work well in
non-real-time environments.

Because current HTM implementations might deterministically abort a
given transaction, software must provide fallback code.
This fallback code must use some other form of synchronization, for
example, locking.
If a lock-based fallback is ever used, then all the limitations of locking,
including the possibility of deadlock, reappear.
One can of course hope that the fallback isn't used often, which might
allow simpler and less deadlock-prone locking designs to be used.
But this raises the question of how the system transitions from using
the lock-based fallbacks back to transactions.\footnote{
	The possibility of an application getting stuck in fallback
	mode has been termed the ``lemming effect'', a term that
	Dave Dice has been credited with coining.}
One approach is to use a test-and-test-and-set discipline~\cite{Martinez02a},
so that everyone holds off until the lock is released, allowing the
system to start from a clean slate in transactional mode at that point.
However, this could result in quite a bit of spinning, which might not
be wise if the lock holder has blocked or been preempted.
Another approach is to allow transactions to proceed in parallel with
a thread holding a lock~\cite{Martinez02a}, but this raises difficulties
in maintaining atomicity, especially if the reason that the thread is
holding the lock is because the corresponding transaction would not fit
into cache.

Finally, dealing with the possibility of aborts and rollbacks seems to
put an additional burden on the developer, who must correctly handle
all combinations of possible error conditions.

It is clear that users of HTM must put considerable validation effort
into testing both the fallback code paths and transition from fallback
code back to transactional code.
Nor is there any reason to believe that the validation requirements of
HTM hardware are any less daunting.

\subsubsection{Lack of Forward-Progress Guarantees}
\label{sec:future:Lack of Forward-Progress Guarantees}

Even though transaction size, conflicts, and aborts/rollbacks can all
cause transactions to abort, one might hope that sufficiently small and
short-duration transactions could be guaranteed to eventually succeed.
This would permit a transaction to be unconditionally retried, in the
same way that compare-and-swap (CAS) and load-linked/store-conditional
(LL/SC) operations are unconditionally retried in code that uses these
instructions to implement atomic operations.

Unfortunately, other than low-clock-rate academic research
prototypes~\cite{MartinSchoeberl2010realtimeTM},
currently available HTM implementations refuse to make any
sort of forward-progress guarantee.
As noted earlier, HTM therefore cannot be used to avoid deadlock on
those systems.
Hopefully future implementations of HTM will provide some sort of
forward-progress guarantees.
Until that time, HTM must be used with extreme caution in real-time
applications.

The one exception to this gloomy picture as of 2021 is
the IBM mainframe, which provides
\emph{constrained transactions}~\cite{ChristianJacobi2012MainframeTM}.
The constraints are quite severe, and are presented in
\cref{sec:future:Forward-Progress Guarantees}.
It will be interesting to see if HTM forward-progress guarantees migrate
from the mainframe to commodity CPU families.

\subsubsection{Irrevocable Operations}
\label{sec:future:Irrevocable Operations}

Another consequence of aborts and rollbacks is that HTM transactions
cannot accommodate irrevocable operations.
Current HTM implementations typically enforce this limitation by
requiring that all of the accesses in the transaction be to cacheable
memory (thus prohibiting MMIO accesses) and aborting transactions on
interrupts, traps, and exceptions (thus prohibiting system calls).

Note that buffered I/O can be accommodated by HTM transactions as
long as the buffer fill/flush operations occur extra-transactionally.
The reason that this works is that adding data to and removing data
from the buffer is revocable: Only the actual buffer fill/flush
operations are irrevocable.
Of course, this buffered-I/O approach has the effect of including the I/O
in the transaction's footprint, increasing the size of the transaction
and thus increasing the probability of failure.

\subsubsection{Semantic Differences}
\label{sec:future:Semantic Differences}

Although HTM can in many cases be used as a drop-in replacement for locking
(hence the name transactional lock
elision~\cite{DaveDice2008TransactLockElision}),
there are subtle differences in semantics.
A particularly nasty example involving coordinated lock-based critical
sections that results in deadlock or livelock when executed transactionally
was given by Blundell~\cite{Blundell2006TMdeadlock}, but a much simpler
example is the empty critical section.

In a lock-based program, an empty critical section will guarantee
that all processes that had previously been holding that lock have
now released it.
This idiom was used by the 2.4 Linux kernel's networking stack to
coordinate changes in configuration.
But if this empty critical section is translated to a transaction,
the result is a no-op.
The guarantee that all prior critical sections have terminated is
lost.
In other words, transactional lock elision preserves the data-protection
semantics of locking, but loses locking's time-based messaging semantics.

\QuickQuizSeries{%
\QuickQuizB{
	But why would \emph{anyone} need an empty lock-based critical
	section???
}\QuickQuizAnswerB{
	See the answer to \QuickQuizARef{\QlockingQemptycriticalsection} in
	\cref{sec:locking:Exclusive Locks}.

	However, it is claimed that given a strongly atomic HTM
	implementation without forward-progress guarantees, any
	memory-based locking design based on empty critical sections
	will operate correctly in the presence of transactional
	lock elision.
	Although I have not seen a proof of this statement, there
	is a straightforward rationale for this claim.
	The main idea is that in a strongly atomic HTM implementation,
	the results of a given transaction are not visible until
	after the transaction completes successfully.
	Therefore, if you can see that a transaction has started,
	it is guaranteed to have already completed, which means
	that a subsequent empty lock-based critical section will
	successfully ``wait'' on it---after all, there is no waiting
	required.

	This line of reasoning does not apply to weakly atomic
	systems (including many STM implementation), and it also
	does not apply to lock-based programs that use means other
	than memory to communicate.
	One such means is the passage of time (for example, in
	hard real-time systems) or flow of priority (for example,
	in soft real-time systems).

	Locking designs that rely on priority boosting are of particular
	interest.
}\QuickQuizEndB
%
\QuickQuizM{
	Can't transactional lock elision trivially handle locking's
	time-based messaging semantics
	by simply choosing not to elide empty lock-based critical sections?
}\QuickQuizAnswerM{
	It could do so, but this would be both unnecessary and
	insufficient.

	It would be unnecessary in cases where the empty critical section
	was due to conditional compilation.
	Here, it might well be that the only purpose of the lock was to
	protect data, so eliding it completely would be the right thing
	to do.
	In fact, leaving the empty lock-based critical section would
	degrade performance and scalability.

	On the other hand, it is possible for a non-empty lock-based
	critical section to be relying on both the data-protection
	and time-based and messaging semantics of locking.
	Using transactional lock elision in such a case would be
	incorrect, and would result in bugs.
}\QuickQuizEndM
%
\QuickQuizE{
	Given modern hardware~\cite{PeterOkech2009InherentRandomness},
	how can anyone possibly expect parallel software relying
	on timing to work?
}\QuickQuizAnswerE{
	The short answer is that on commonplace commodity hardware,
	synchronization designs based on any sort of fine-grained
	timing are foolhardy and cannot be expected to operate correctly
	under all conditions.

	That said, there are systems designed for hard real-time use
	that are much more deterministic.
	In the (very unlikely) event that you are using such a system,
	here is a toy example showing how time-based synchronization can
	work.
	Again, do \emph{not} try this on commodity microprocessors,
	as they have highly nondeterministic performance characteristics.

	This example uses multiple worker threads along with a control
	thread.
	Each worker thread corresponds to an outbound data feed, and
	records the current time (for example, from the
	\co{clock_gettime()} system call) in a per-thread
	\co{my_timestamp} variable after executing each unit
	of work.
	The real-time nature of this example results in the following
	set of constraints:

	\begin{enumerate}
	\item	It is a fatal error for a given worker thread to fail
		to update its timestamp for a time period of more than
		\co{MAX_LOOP_TIME}.
	\item	Locks are used sparingly to access and update global
		state.
	\item	Locks are granted in strict FIFO order within
		a given thread priority.
	\end{enumerate}

	When worker threads complete their feed, they must disentangle
	themselves from the rest of the application and place a status
	value in a per-thread \co{my_status} variable that is initialized
	to $-1$.
	Threads do not exit; they instead are placed on a thread pool
	to accommodate later processing requirements.
	The control thread assigns (and re-assigns) worker threads as
	needed, and also maintains a histogram of thread statuses.
	The control thread runs at a real-time priority no higher than
	that of the worker threads.

	Worker threads' code is as follows:

\begin{VerbatimN}
	int my_status = -1;  /* Thread local. */

	while (continue_working()) {
		enqueue_any_new_work();
		wp = dequeue_work();
		do_work(wp);
		my_timestamp = clock_gettime(...);
	}

	acquire_lock(&departing_thread_lock);

	/*
	 * Disentangle from application, might
	 * acquire other locks, can take much longer
	 * than MAX_LOOP_TIME, especially if many
	 * threads exit concurrently.
	 */
	my_status = get_return_status();
	release_lock(&departing_thread_lock);

	/* thread awaits repurposing. */
\end{VerbatimN}

	The control thread's code is as follows:

\begin{fcvlabel}[ln:future:htm:control thread]
\begin{VerbatimN}[commandchars=\\\@\$]
	for (;;) {
		for_each_thread(t) {
			ct = clock_gettime(...);
			d = ct - per_thread(my_timestamp, t);
			if (d >= MAX_LOOP_TIME) {	\lnlbl@if$
				/* thread departing. */	\lnlbl@dep:b$
				acquire_lock(&departing_thread_lock); \lnlbl@acq$
				release_lock(&departing_thread_lock); \lnlbl@rel$
				i = per_thread(my_status, t);
				status_hist[i]++; /* Bug if TLE! */ \lnlbl@dep:e$
			}
		}
		/* Repurpose threads as needed. */
	}
\end{VerbatimN}
\end{fcvlabel}

	\begin{fcvref}[ln:future:htm:control thread]
	\Clnref{if} uses the passage of time to deduce that the thread
	has exited, executing \clnref{dep:b,dep:e} if so.
	The empty lock-based critical section on \clnref{acq,rel}
	guarantees that any thread in the process of exiting
	completes (remember that locks are granted in FIFO order!).
	\end{fcvref}

	Once again, do not try this sort of thing on commodity
	microprocessors.
	After all, it is difficult enough to get this right on systems
	specifically designed for hard real-time use!
}\QuickQuizEndE
}

One important semantic difference between locking and transactions
is the priority boosting that is used to avoid priority inversion
in lock-based real-time programs.
One way in which priority inversion can occur is when a
low-priority thread holding a lock
is preempted by a medium-priority CPU-bound thread.
If there is at least one such medium-priority thread per CPU, the
low-priority thread will never get a chance to run.
If a high-priority thread now attempts to acquire the lock,
it will block.
It cannot acquire the lock until the low-priority thread releases it,
the low-priority thread cannot release the lock until it gets a chance
to run, and it cannot get a chance to run until one of the medium-priority
threads gives up its CPU\@.
Therefore, the medium-priority threads are in effect blocking the
high-priority process, which is the rationale for the name ``priority
inversion.''

\begin{listing}[tbp]
\begin{fcvlabel}[ln:future:Exploiting Priority Boosting]
\begin{VerbatimL}[commandchars=\\\@\$]
void boostee(void)		\lnlbl@low:b$
{
	int i = 0;

	acquire_lock(&boost_lock[i]);	\lnlbl@1stacq$
	for (;;) {
		acquire_lock(&boost_lock[!i]);
		release_lock(&boost_lock[i]);
		i = i ^ 1;
		do_something();
	}
}				\lnlbl@low:e$

void booster(void)		\lnlbl@high:b$
{
	int i = 0;

	for (;;) {
		usleep(500); /* sleep 0.5 ms. */
		acquire_lock(&boost_lock[i]);	\lnlbl@acq$
		release_lock(&boost_lock[i]);	\lnlbl@rel$
		i = i ^ 1;
	}
}                               \lnlbl@high:e$
\end{VerbatimL}
\end{fcvlabel}
\caption{Exploiting Priority Boosting}
\label{lst:future:Exploiting Priority Boosting}
\end{listing}

One way to avoid priority inversion is \emph{priority inheritance},
in which a high-priority thread blocked on a lock temporarily donates
its priority to the lock's holder, which is also called \emph{priority
boosting}.
However, priority boosting can be used for things other than avoiding
priority inversion, as shown in
\cref{lst:future:Exploiting Priority Boosting}.
\begin{fcvref}[ln:future:Exploiting Priority Boosting]
\Clnrefrange{low:b}{low:e} of this listing show a low-priority process that must
nevertheless run every millisecond or so, while \clnrefrange{high:b}{high:e} of
this same listing show a high-priority process that uses priority
boosting to ensure that \co{boostee()} runs periodically as needed.

The \co{boostee()} function arranges this by always holding one of
the two \co{boost_lock[]} locks, so that \clnrefrange{acq}{rel} of
\co{booster()} can boost priority as needed.
\end{fcvref}

\QuickQuiz{
	But the \co{boostee()} function in
	\cref{lst:future:Exploiting Priority Boosting}
	alternatively acquires its locks in reverse order!
	Won't this result in deadlock?
}\QuickQuizAnswer{
	No deadlock will result.
	To arrive at deadlock, two different threads must each
	acquire the two locks in opposite orders, which does not
	happen in this example.
	However, deadlock detectors such as
	lockdep~\cite{JonathanCorbet2006lockdep}
	will flag this as a false positive.
}\QuickQuizEnd

\begin{fcvref}[ln:future:Exploiting Priority Boosting]
This arrangement requires that \co{boostee()} acquire its first
lock on \clnref{1stacq} before the system becomes busy, but this is easily
arranged, even on modern hardware.

Unfortunately, this arrangement can break down in presence of transactional
lock elision.
The \co{boostee()} function's overlapping critical sections become
one infinite transaction, which will sooner or later abort,
for example, on the first time that the thread running
the \co{boostee()} function is preempted.
At this point, \co{boostee()} will fall back to locking, but given
its low priority and that the quiet initialization period is now
complete (which after all is why \co{boostee()} was preempted),
this thread might never again get a chance to run.

And if the \co{boostee()} thread is not holding the lock, then
the \co{booster()} thread's empty critical section on \clnref{acq,rel} of
\cref{lst:future:Exploiting Priority Boosting}
will become an empty transaction that has no effect, so that
\co{boostee()} never runs.
This example illustrates some of the subtle consequences of
transactional memory's rollback-and-retry semantics.
\end{fcvref}

Given that experience will likely uncover additional subtle semantic
differences, application of HTM-based lock elision to large programs
should be undertaken with caution.
That said, where it does apply, HTM-based lock elision can eliminate
the cache misses associated with the lock variable, which has resulted
in tens of percent performance increases in large real-world software
systems as of early 2015.
We can therefore expect to see substantial use of this technique on
hardware providing reliable support for it.

\QuickQuiz{
	So a bunch of people set out to supplant locking, and they
	mostly end up just optimizing locking???
}\QuickQuizAnswer{
	At least they accomplished something useful!
	And perhaps there will continue to be additional HTM progress
	over time~\cite{Siakavaras2017CombiningHA,DimitriosSiakavaras2020RCU-HTM-B+Trees,ChristinaGiannoula2018HTM-RCU-graphcoloring,SeongJaePark2020HTMRCUlock}.
}\QuickQuizEnd

\subsubsection{Summary}
\label{sec:future:HTM Weaknesses WRT Locking: Summary}

% future/HTMtable.tex
% SPDX-License-Identifier: CC-BY-SA-3.0

\begin{table*}[p]
\centering
% \scriptsize
\small
\begin{tabular}{p{1.0in}||c|p{2.0in}||c|p{2.0in}}
& \multicolumn{2}{c||}{Locking} & \multicolumn{2}{c}{Hardware Transactional Memory} \\
\hline
\hline
Basic Idea
	& \multicolumn{2}{p{2.2in}||}{
	  Allow only one thread at a time to access a given set of objects.}
		& \multicolumn{2}{p{2.2in}}{
		  Cause a given operation over a set of objects to execute
		  atomically.} \\
\hline
\hline
Scope
	& $+$
	& Handles all operations.
		& $+$
		& Handles revocable operations. \\
\cline{4-5}
	& &
		& $-$
		& Irrevocable operations force fallback (typically
		  to locking). \\
\hline
Composability
	& $\Downarrow$
	& Limited by deadlock.
		& $\Downarrow$
		& Limited by irrevocable operations, transaction size,
		  and deadlock (assuming lock-based fallback code). \\
\hline
Scalability \& Performance
	& $-$
	& Data must be partitionable to avoid lock contention.
		& $-$
		& Data must be partionable to avoid conflicts. \\
\cline{2-5}
	& $\Downarrow$
	& Partioning must typically be fixed at design time.
		& $+$
		& Dynamic adjustment of partitioning carried out
		  automatically down to cacheline boundaries. \\
\cline{4-5}
	&
	&
		& $-$
		& Partitioning required for fallbacks (less important
		  for rare fallbacks). \\
\cline{2-5}
	& $\Downarrow$
	& Locking primitives typically result in expensive cache misses
	  and memory-barrier instructions.
		& $-$
		& Transactions begin/end instructions typically do not
		  result in cache misses, but do have memory-ordering
		  consequences. \\
\cline{2-5}
	& $+$
	& Contention effects are focused on acquisition and release, so
	  that the critical section runs at full speed.
		& $-$
		& Contention aborts conflicting transactions, even
		  if they have been running for a long time. \\
\cline{2-5}
	& $+$
	& Privatization operations are simple, intuitive, performant,
	  and scalable.
		& $-$
		& Privatized data contributes to transaction size. \\
\hline
Hardware Support
	& $+$
	& Commodity hardware suffices.
		& $-$
		& New hardware required (and is starting to become
		  available). \\
\cline{2-5}
	& $+$
	& Performance is insensitive to cache-geometry details.
		& $-$
		& Performance depends critically on cache geometry. \\
\hline
Software Support
	& $+$
	& APIs exist, large body of code and experience, debuggers operate
	  naturally.
		& $-$
		& APIs emerging, little experience outside of DBMS,
		  breakpoints mid-transaction can be problematic. \\
\hline
Interaction With Other Mechanisms
	& $+$
	& Long experience of successful interaction.
		& $\Downarrow$
		& Just beginning investigation of interaction. \\
\hline
Practical Apps
	& $+$
	& Yes.
		& $+$
		& Yes. \\
\hline
Wide Applicability
	& $+$
	& Yes.
		& $-$
		& Jury still out, but likely to win significant use. \\
\end{tabular}
\caption{Comparison of Locking and HTM (``$+$'' is Advantage, ``$-$'' is Disadvantage, ``$\Downarrow$'' is Strong Disadvantage)}
\label{tab:future:Comparison of Locking and HTM}
\end{table*}


Although it seems likely that HTM will have compelling use cases,
current implementations have serious transaction-size limitations,
conflict-handling complications, abort-and-rollback issues, and
semantic differences that will require careful handling.
HTM's current situation relative to locking is summarized in
\cref{tab:future:Comparison of Locking and HTM}.
As can be seen, although the current state of HTM alleviates some
serious shortcomings of locking,\footnote{
	In fairness, it is important to emphasize that locking's shortcomings
	do have well-known and heavily used engineering solutions, including
	deadlock detectors~\cite{JonathanCorbet2006lockdep}, a wealth
	of data structures that have been adapted to locking, and
	a long history of augmentation, as discussed in
	\cref{sec:future:HTM Weaknesses WRT Locking When Augmented}.
	In addition, if locking really were as horrible as a quick skim
	of many academic papers might reasonably lead one to believe,
	where did all the large lock-based parallel programs (both
	FOSS and proprietary) come from, anyway?}
it does so by introducing a significant
number of shortcomings of its own.
These shortcomings are acknowledged by leaders in the TM
community~\cite{AlexanderMatveev2012PessimisticTM}.\footnote{
	In addition, in early 2011, I was invited to deliver a critique of
	some of the assumptions underlying transactional
	memory~\cite{PaulEMcKenney2011Verico}.
	The audience was surprisingly non-hostile, though perhaps they
	were taking it easy on me due to the fact that I was heavily
	jet-lagged while giving the presentation.}

In addition, this is not the whole story.
Locking is not normally used by itself, but is instead typically
augmented by other synchronization mechanisms,
including reference counting, atomic operations, non-blocking data structures,
hazard pointers~\cite{MagedMichael04a,HerlihyLM02},
and RCU~\cite{McKenney98,McKenney01a,ThomasEHart2007a,PaulEMcKenney2012ELCbattery}.
The next section looks at how such augmentation changes the equation.

\subsection{HTM Weaknesses WRT Locking When Augmented}
\label{sec:future:HTM Weaknesses WRT Locking When Augmented}

% future/HTMtableRCU.tex
% SPDX-License-Identifier: CC-BY-SA-3.0

\begin{table*}[p]
\centering
\small\OneColumnHSpace{-.8in}
%\raggedright
\begin{tabular}{p{1.0in}||c|p{2.0in}||c|p{2.0in}}
& \multicolumn{2}{c||}{Locking with RCU or Hazard Pointers} & \multicolumn{2}{c}{Hardware Transactional Memory} \\
\hline
\hline
Basic Idea
	& \multicolumn{2}{p{2.2in}||}{
	  Allow only one thread at a time to access a given set of objects.}
		& \multicolumn{2}{p{2.2in}}{
		  Cause a given operation over a set of objects to execute
		  atomically.} \\
\hline
\hline
Scope
	& $+$
	& Handles all operations.
		& $+$
		& Handles revocable operations. \\
\cline{4-5}
	& &
		& $-$
		& Irrevocable operations force fallback (typically
		  to locking). \\
\hline
Composability
	& $+$
	& Readers limited only by grace-period-wait operations.
		& $\Downarrow$
		& Limited by irrevocable operations, transaction size,
		  and deadlock. \\
\cline{2-3}
	& $-$
	& Updaters limited by deadlock.  Readers reduce deadlock.
		&
		& (Assuming lock-based fallback code.) \\
\hline
Scalability \& Performance
	& $-$
	& Data must be partitionable to avoid lock contention among
	  updaters.
		& $-$
		& Data must be partionable to avoid conflicts. \\
\cline{2-3}
	& $+$
	& Partitioning not needed for readers.
		&
		& \\
\cline{2-5}
	& $\Downarrow$
	& Partioning for updaters must typically be fixed at design time.
		& $+$
		& Dynamic adjustment of partitioning carried out
		  automatically down to cacheline boundaries. \\
\cline{2-5}
	& $+$
	& Partitioning not needed for readers.
		& $-$
		& Partitioning required for fallbacks (less important
		  for rare fallbacks). \\
\cline{2-5}
	& $\Downarrow$
	& Updater locking primitives typically result in expensive cache
	  misses and memory-barrier instructions.
		& $-$
		& Transactions begin/end instructions typically do not
		  result in cache misses, but do have memory-ordering
		  consequences. \\
\cline{2-5}
	& $+$
	& Update-side contention effects are focused on acquisition and
	  release, so that the critical section runs at full speed.
		& $-$
		& Contention aborts conflicting transactions, even
		  if they have been running for a long time. \\
\cline{2-3}
	& $+$
	& Readers do not contend with updaters or with each other.
		&
		& \\
\cline{2-5}
	& $+$
	& Read-side primitives are typically wait-free with low
	  overhead.  (Lock-free for hazard pointers.)
		& $-$
		& Read-only transactions subject to conflicts and rollbacks.
		  No forward-progress guarantees other than those supplied
		  by fallback code. \\
\cline{2-5}
	& $+$
	& Privatization operations are simple, intuitive, performant,
	  and scalable when data is visible only to updaters.
		& $-$
		& Privatized data contributes to transaction size. \\
\cline{2-3}
	& $-$
	& Privitization operations are expensive (though still intuitive
	  and scalable) for reader-visible data.
		&
		& \\
\hline
Hardware Support
	& $+$
	& Commodity hardware suffices.
		& $-$
		& New hardware required (and is starting to become
		  available). \\
\cline{2-5}
	& $+$
	& Performance is insensitive to cache-geometry details.
		& $-$
		& Performance depends critically on cache geometry. \\
\hline
Software Support
	& $+$
	& APIs exist, large body of code and experience, debuggers operate
	  naturally.
		& $-$
		& APIs emerging, little experience outside of DBMS,
		  breakpoints mid-transaction can be problematic. \\
\hline
Interaction With Other Mechanisms
	& $+$
	& Long experience of successful interaction.
		& $\Downarrow$
		& Just beginning investigation of interaction. \\
\hline
Practical Apps
	& $+$
	& Yes.
		& $+$
		& Yes. \\
\hline
Wide Applicability
	& $+$
	& Yes.
		& $-$
		& Jury still out, but likely to win significant use. \\
\end{tabular}
\caption{Comparison of Locking (Augmented by RCU or Hazard Pointers) and HTM (``$+$'' is Advantage, ``$-$'' is Disadvantage, ``$\Downarrow$'' is Strong Disadvantage)}
\label{tab:future:Comparison of Locking (Augmented by RCU or Hazard Pointers) and HTM}
\end{table*}


Practitioners have long used reference counting, atomic operations,
non-blocking data structures, hazard pointers, and RCU to avoid some
of the shortcomings of locking.
For example, deadlock can be avoided in many cases by using reference
counts, hazard pointers, or RCU to protect data structures,
particularly for read-only critical
sections~\cite{MagedMichael04a,HerlihyLM02,MathieuDesnoyers2012URCU,DinakarGuniguntala2008IBMSysJ,ThomasEHart2007a}.
These approaches also reduce the need to partition data
structures, as was seen in \cref{chp:Data Structures}.
RCU further provides contention-free bounded wait-free read-side
primitives~\cite{McKenney98,MathieuDesnoyers2012URCU}, while hazard pointers
provides lock-free read-side
primitives~\cite{Michael02a,HerlihyLM02,MagedMichael04a}.
Adding these considerations to
\cref{tab:future:Comparison of Locking and HTM}
results in the updated comparison between augmented locking and HTM
shown in
\cref{tab:future:Comparison of Locking (Augmented by RCU or Hazard Pointers) and HTM}.
A summary of the differences between the two tables is as follows:

\begin{enumerate}
\item	Use of non-blocking read-side mechanisms alleviates deadlock issues.
\item	Read-side mechanisms such as hazard pointers and RCU can operate
	efficiently on non-partitionable data.
\item	Hazard pointers and RCU do not contend with each other or with
	updaters, allowing excellent performance and scalability for
	read-mostly workloads.
\item	Hazard pointers and RCU provide forward-progress guarantees
	(lock freedom and bounded wait-freedom, respectively).
\item	Privatization operations for hazard pointers and RCU are
	straightforward.
\end{enumerate}

% future/HTMtableFull.tex
% SPDX-License-Identifier: CC-BY-SA-3.0

\begin{sidewaystable*}[htbp]
% future/HTMtableColor.tex
% SPDX-License-Identifier: CC-BY-SA-3.0

\definecolor{plus}{cmyk}{0.1,0,0,0}
\definecolor{minus}{cmyk}{0,0.05,0.2,0.05}
\definecolor{down}{cmyk}{0,0.15,0.15,0.1}
\newcommand{\Pl}{\cellcolor{plus}}
\newcommand{\Mn}{\cellcolor{minus}}
\newcommand{\Dw}{\cellcolor{down}}

\centering
\caption{Comparison of Locking (Plain and Augmented) and HTM
  (\colorbox{plus}{Advantage}, \colorbox{minus}{Disadvantage},
  \colorbox{down}{Strong Disadvantage})}
\label{tab:future:Comparison of Locking (Plain and Augmented) and HTM}
\footnotesize
\setstretch{0.95}
%\renewcommand*{\arraystretch}{1.4}
\setlength{\tabcolsep}{3pt}
\resizebox{8in}{!}{
\begin{tabularx}{8.5in}{p{.85in}cXcXcX}
\toprule
  &
    & \multicolumn{1}{c}{Locking} &
      & \multicolumn{1}{c}{Locking with Userspace RCU or Hazard Pointers} &
        & \multicolumn{1}{c}{Hardware Transactional Memory} \\
\midrule
  Basic Idea &
    & Allow only one thread at a time to access a given set of objects. &
      & Allow only one thread at a time to access a given set of objects. &
        & Cause a given operation over a set of objects to execute atomically. \\
\midrule
  Scope &
    & \Pl Handles all operations. &
      & \Pl Handles all operations. &
        & \Pl Handles revocable operations. \\
\addlinespace[4pt]
  &
    & &
      & &
        & \Mn Irrevocable operations force fallback (typically to locking). \\
\midrule
  Composability &
    & \Dw Limited by deadlock. &
      & \Pl Readers limited only by grace-period-wait operations. &
        & \Dw Limited by irrevocable operations, transaction size, and deadlock.
          (Assuming lock-based fallback code.) \\
\addlinespace[4pt]
  &
    & &
      & \Mn Updaters limited by deadlock. Readers reduce deadlock. &
        & \\
\midrule
  Scalability \& Performance &
    & \Mn Data must be partitionable to avoid lock contention. &
      & \Mn Data must be partitionable to avoid lock contention among updaters. &
        & \Mn Data must be partitionable to avoid conflicts. \\
\addlinespace[4pt]
  &
    & &
      & \Pl Partitioning not needed for readers. &
        & \\
\cmidrule{3-7}
  &
    & \Dw Partitioning must typically be fixed at design time. &
      & \Dw Partitioning for updaters must typically be fixed at design time. &
        & \Pl Dynamic adjustment of partitioning carried out automatically
          down to cacheline boundaries. \\
\addlinespace[4pt]
  &
    & &
      & \Pl Partitioning not needed for readers. &
        & \Mn Partitioning required for fallbacks (less important for rare
          fallbacks). \\
\cmidrule{3-7}
  &
    & \Dw Locking primitives typically result in expensive cache misses and
      memory-barrier instructions.&
      & \Dw Updater locking primitives typically result in expensive cache
        misses and memory-barrier instructions. &
        & \Mn Transactions begin/end instructions typically do not result in
          cache misses, but do have memory-ordering consequences. \\
\cmidrule{3-7}
  &
    & \Pl Contention effects are focused on acquisition and release, so that
      the critical section runs at full speed. &
      & \Pl Update-side contention effects are focused on acquisition and
        release, so that the critical section runs at full speed. &
        & \Mn Contention aborts conflicting transactions, even if they have been
          running for a long time. \\
\addlinespace[4pt]
  &
    & &
      & \Pl Readers do not contend with updaters or with each other. &
        & \\
\cmidrule{3-7}
  &
    & &
      & \Pl Read-side primitives are typically wait-free with low overhead.
        (Lock-free for hazard pointers.) &
        & \Mn Read-only transactions subject to conflicts and rollbacks. No
          forward-progress guarantees other than those supplied by fallback
          code. \\
\cmidrule{3-7}
  &
    & \Pl Privatization operations are simple, intuitive, performant,
      and scalable. &
      & \Pl Privatization operations are simple, intuitive, performant,
        and scalable when data is visible only to updaters. &
        & \Mn Privatized data contributes to transaction size. \\
\addlinespace[4pt]
  &
    & &
      & \Mn Privatization operations are expensive (though still intuitive
        and scalable) for reader-visible data. &
        & \\
\midrule
  Hardware Support &
    & \Pl Commodity hardware suffices. &
      & \Pl Commodity hardware suffices. &
        & \Mn New hardware required (and is starting to become available). \\
\cmidrule{3-7}
  &
    & \Pl Performance is insensitive to cache-geometry details. &
      & \Pl Performance is insensitive to cache-geometry details. &
        & \Mn Performance depends critically on cache geometry. \\
\midrule
  Software Support &
    & \Pl APIs exist, large body of code and experience, debuggers operate
      naturally. &
      & \Pl APIs exist, large body of code and experience, debuggers operate
        naturally. &
        & \Mn APIs emerging, little experience outside of DBMS, breakpoints
              mid-transaction can be problematic. \\
\midrule
  Interaction With Other Mechanisms &
    & \Pl Long experience of successful interaction. &
      & \Pl Long experience of successful interaction. &
        & \Dw Just beginning investigation of interaction. \\
\midrule
  Practical Apps &
    & \Pl Yes. &
      & \Pl Yes. &
        & \Pl Yes. \\
\midrule
  Wide Applicability &
    & \Pl Yes. &
      & \Pl Yes. &
        & \Mn Jury still out, but likely to win significant use. \\
\bottomrule
\end{tabularx}
}
\end{sidewaystable*}


For those with good eyesight,
\cref{tab:future:Comparison of Locking (Plain and Augmented) and HTM}
combines
\cref{tab:future:Comparison of Locking and HTM,%
tab:future:Comparison of Locking (Augmented by RCU or Hazard Pointers) and HTM}.

Of course, it is also possible to augment HTM,
as discussed in the next section.

\subsection{Where Does HTM Best Fit In?}
\label{sec:future:Where Does HTM Best Fit In?}

Although it will likely be some time before HTM's area of applicability
can be as crisply delineated as that shown for RCU in
\cref{fig:defer:RCU Areas of Applicability} on
page~\pageref{fig:defer:RCU Areas of Applicability}, that is no reason not to
start moving in that direction.

HTM seems best suited to update-heavy workloads involving relatively
small changes to disparate portions of relatively large in-memory
data structures running on large multiprocessors,
as this meets the size restrictions of current HTM implementations while
minimizing the probability of conflicts and attendant aborts and
rollbacks.
This scenario is also one that is relatively difficult to handle given
current synchronization primitives.

Use of locking in conjunction with HTM seems likely to overcome HTM's
difficulties with irrevocable operations, while use of RCU or
hazard pointers might alleviate HTM's transaction-size limitations
for read-only operations that traverse large fractions of the data
structure~\cite{SeongJaePark2020HTMRCUlock}.
Current HTM implementations unconditionally abort an update transaction
that conflicts with an RCU or hazard-pointer reader, but perhaps future
HTM implementations will interoperate more smoothly with these
synchronization mechanisms.
In the meantime, the probability of an update conflicting with a
large RCU or hazard-pointer read-side critical section should be
much smaller than the probability of conflicting with the equivalent
read-only transaction.\footnote{
	It is quite ironic that strictly transactional mechanisms are
	appearing in shared-memory systems at just about the time
	that NoSQL databases are relaxing the traditional
	database-application reliance on strict transactions.
	Nevertheless, HTM has in fact realized the ease-of-use promise
	of TM, albeit for black-hat attacks on the Linux kernel's
	address-space randomization defense
	mechanism~\cite{YeongjinJang2016TSXbreakKASLR,Jang:2016:BKA:2976749.2978321}.}
Nevertheless, it is quite possible that a steady stream of RCU or
hazard-pointer readers might starve updaters due to a corresponding
steady stream of conflicts.
This vulnerability could be eliminated (at significant
hardware cost and complexity) by giving extra-transactional
reads the pre-transaction copy of the memory location being loaded.

The fact that HTM transactions must have fallbacks might in some cases
force static partitionability of data structures back onto HTM\@.
This limitation might be alleviated if future HTM implementations
provide forward-progress guarantees, which might eliminate the need
for fallback code in some cases, which in turn might allow HTM to
be used efficiently in situations with higher conflict probabilities.

In short, although HTM is likely to have important uses and applications,
it is another tool in the parallel programmer's toolbox, not a replacement
for the toolbox in its entirety.

\subsection{Potential Game Changers}
\label{sec:future:Potential Game Changers}

Game changers that could greatly increase the need for HTM include
the following:

\begin{enumerate}
\item	Forward-progress guarantees.
\item	Transaction-size increases.
\item	Improved debugging support.
\item	Weak atomicity.
\end{enumerate}

These are expanded upon in the following sections.

\subsubsection{Forward-Progress Guarantees}
\label{sec:future:Forward-Progress Guarantees}

As was discussed in
\cref{sec:future:Lack of Forward-Progress Guarantees},
current HTM implementations lack forward-progress guarantees, which requires
that fallback software is available to handle HTM failures.
Of course, it is easy to demand guarantees, but not always easy
to provide them.
In the case of HTM, obstacles to guarantees can include cache size and
associativity, TLB size and associativity, transaction duration and
interrupt frequency, and scheduler implementation.

Cache size and associativity was discussed in
\cref{sec:future:Transaction-Size Limitations},
along with some research intended to work around current limitations.
However, HTM forward-progress guarantees would
come with size limits, large though these limits might one day be.
So why don't current HTM implementations provide forward-progress
guarantees for small transactions, for example, limited to the
associativity of the cache?
One potential reason might be the need to deal with hardware failure.
For example, a failing cache SRAM cell might be handled by deactivating
the failing cell, thus reducing the associativity of the cache and
therefore also the maximum size of transactions that can be guaranteed
forward progress.
Given that this would simply decrease the guaranteed transaction size,
it seems likely that other reasons are at work.
Perhaps providing forward progress guarantees on production-quality
hardware is more difficult than one might think, an entirely plausible
explanation given the difficulty of making forward-progress guarantees
in software.
Moving a problem from software to hardware does not necessarily make
it easier to solve~\cite{ChristianJacobi2012MainframeTM}.

Given a physically tagged and indexed cache, it is not enough for the
transaction to fit in the cache.
Its address translations must also fit in the TLB\@.
Any forward-progress guarantees must therefore also take TLB size
and associativity into account.

Given that interrupts, traps, and exceptions abort transactions in current
HTM implementations, it is necessary that the execution duration of
a given transaction be shorter than the expected interval between
interrupts.
No matter how little data a given transaction touches, if it runs too
long, it will be aborted.
Therefore, any forward-progress guarantees must be conditioned not only
on transaction size, but also on transaction duration.

Forward-progress guarantees depend critically on the ability to determine
which of several conflicting transactions should be aborted.
It is all too easy to imagine an endless series of transactions, each
aborting an earlier transaction only to itself be aborted by a later
transactions, so that none of the transactions actually commit.
The complexity of conflict handling is
evidenced by the large number of HTM conflict-resolution policies
that have been proposed~\cite{EgeAkpinar2011HTM2TLE,YujieLiu2011ToxicTransactions}.
Additional complications are introduced by extra-transactional accesses,
as noted by Blundell~\cite{Blundell2006TMdeadlock}.
It is easy to blame the extra-transactional accesses for all of these
problems, but the folly of this line of thinking is easily demonstrated
by placing each of the extra-transactional accesses into its own
single-access transaction.
It is the pattern of accesses that is the issue, not whether or not they
happen to be enclosed in a transaction.

Finally, any forward-progress guarantees for transactions also
depend on the scheduler, which must let the thread executing the
transaction run long enough to successfully commit.

So there are significant obstacles to HTM vendors offering forward-progress
guarantees.
However, the impact of any of them doing so would be enormous.
It would mean that HTM transactions would no longer need software
fallbacks, which would mean that HTM could finally deliver on the
TM promise of deadlock elimination.

However, in late 2012, the IBM Mainframe announced an HTM implementation
that includes \emph{constrained transactions} in addition to the usual
best-effort HTM
implementation~\cite{ChristianJacobi2012MainframeTM}.
A constrained transaction starts with the \co{tbeginc} instruction
instead of the \co{tbegin} instruction that is used for best-effort
transactions.
Constrained transactions are guaranteed to always complete (eventually),
so if a transaction aborts, rather than branching to a fallback path
(as is done for best-effort transactions), the hardware instead restarts
the transaction at the \co{tbeginc} instruction.

The Mainframe architects needed to take extreme measures to deliver on
this forward-progress guarantee.
If a given constrained transaction repeatedly fails, the CPU
might disable branch prediction, force in-order execution, and even
disable pipelining.
If the repeated failures are due to high contention, the CPU might
disable speculative fetches, introduce random delays, and even
serialize execution of the conflicting CPUs.
``Interesting'' forward-progress scenarios involve as few as two CPUs
or as many as one hundred CPUs.
Perhaps these extreme measures provide some insight as to why other CPUs
have thus far refrained from offering constrained transactions.

As the name implies, constrained transactions are in fact severely constrained:

\begin{enumerate}
\item	The maximum data footprint is four blocks of memory,
	where each block can be no larger than 32 bytes.
\item	The maximum code footprint is 256 bytes.
\item	If a given 4K page contains a constrained transaction's code,
	then that page may not contain that transaction's data.
\item	The maximum number of assembly instructions that may be executed
	is 32.
\item	Backwards branches are forbidden.
\end{enumerate}

Nevertheless, these constraints support a number of important data structures,
including linked lists, stacks, queues, and arrays.
Constrained HTM therefore seems likely to become an important tool in
the parallel programmer's toolbox.

Note that these forward-progress guarantees need not be absolute.
For example, suppose that a use of HTM uses a global lock as fallback.
Assuming that the fallback mechanism has been carefully designed to
avoid the ``lemming effect'' discussed in
\cref{sec:future:Aborts and Rollbacks},
then if HTM rollbacks are sufficiently infrequent, the global lock
will not be a bottleneck.
That said, the larger the system, the longer the critical sections,
and the longer the time required to recover from the ``lemming effect'',
the more rare ``sufficiently infrequent'' needs to be.

\subsubsection{Transaction-Size Increases}
\label{sec:future:Transaction-Size Increases}

Forward-progress guarantees are important, but as we saw, they will
be conditional guarantees based on transaction size and duration.
There has been some progress, for example, some commercially available
HTM implementations use approximation techniques to support extremely
large HTM read sets~\cite{RaviRajwar2012TSX}.
For another example, \Power{8} HTM supports suspended transations, which
avoid adding irrelevant accesses to the suspended transation's read and
write sets~\cite{Le:2015:TMS:3266491.3266500}.
This capability has been used to produce a high performance
reader-writer lock~\cite{PascalFelber2016rwlockElision}.

It is important to note that even small-sized guarantees will be
quite useful.
For example,
a guarantee of two cache lines is sufficient for a stack, queue, or dequeue.
However, larger data structures require larger guarantees, for example,
traversing a tree in order requires a guarantee equal to the number
of nodes in the tree.
Therefore, even modest increases in the size of the guarantee also
increases the usefulness of HTM, thereby increasing the need for CPUs
to either provide it or provide good-and-sufficient workarounds.

\subsubsection{Improved Debugging Support}
\label{sec:future:Improved Debugging Support}

Another inhibitor to transaction size is the need to debug the transactions.
The problem with current mechanisms is that a single-step exception
aborts the enclosing transaction.
There are a number of workarounds for this issue, including emulating
the processor (slow!), substituting STM for HTM (slow and slightly
different semantics!),
playback techniques using repeated retries to emulate forward
progress (strange failure modes!), and
full support of debugging HTM transactions (complex!).

Should one of the HTM vendors produce an HTM system that allows
straightforward use of classical debugging techniques within
transactions, including breakpoints, single stepping, and
print statements, this will make HTM much more compelling.
Some transactional-memory researchers started to recognize this
problem in 2013, with at least one proposal involving hardware-assisted
debugging facilities~\cite{JustinGottschlich2013TMdebug}.
Of course, this proposal depends on readily available hardware gaining such
facilities~\cite{TimothyHayes2020ARM-HTM,Intel2020TSXdevguide}.
Worse yet, some cutting-edge debugging facilities are incompatible
with HTM~\cite{RobertOCallahan2020DebuggingHTM}.

\subsubsection{Weak Atomicity}
\label{sec:future:Weak Atomicity}

Given that HTM is likely to face some sort of size limitations for the
foreseeable future, it will be necessary for HTM to interoperate
smoothly with other mechanisms.
HTM's interoperability with read-mostly mechanisms such as hazard pointers
and RCU would be improved if extra-transactional reads did not
unconditionally abort transactions with conflicting writes---instead,
the read could simply be provided with the pre-transaction value.
In this way, hazard pointers and RCU could be used to allow HTM to handle
larger data structures and to reduce conflict probabilities.

This is not necessarily simple, however.
The most straightforward way of implementing this requires an additional
state in each cache line and on the bus, which is a non-trivial added
expense.
The benefit that goes along with this expense is permitting
large-footprint readers without the risk of starving updaters due
to continual conflicts.
An alternative approach, applied to great effect to binary search trees
by Siakavaras et al.~\cite{Siakavaras2017CombiningHA},
is to use RCU for read-only traversals and HTM
only for the actual updates themselves.
This combination outperformed other transactional-memory techniques by
up to 220\,\%, a speedup similar to that observed by
Howard and Walpole~\cite{PhilHoward2011RCUTMRBTree}
when they combined RCU with STM\@.
In both cases, the weak atomicity is implemented in software rather than
in hardware.
It would nevertheless be interesting to see what additional speedups
could be obtained by implementing weak atomicity in both hardware and
software.

\subsection{Conclusions}
\label{sec:future:Conclusions}

Although current HTM implementations have delivered real performance
benefits in some situations, they also have significant shortcomings.
The most significant shortcomings appear to be
limited transaction sizes,
the need for conflict handling, the need for aborts and rollbacks,
the lack of forward-progress guarantees,
the inability to handle irrevocable operations,
and subtle semantic differences
from locking.

Some of these shortcomings might be alleviated in future implementations,
but it appears that there will continue to be a strong need to make
HTM work well with the many other types of synchronization mechanisms,
as noted earlier~\cite{McKenney2007PLOSTM,PaulEMcKenney2010OSRGrassGreener}.
Although there has been some work using HTM with
RCU~\cite{Siakavaras2017CombiningHA,DimitriosSiakavaras2020RCU-HTM-B+Trees,ChristinaGiannoula2018HTM-RCU-graphcoloring,SeongJaePark2020HTMRCUlock},
there has been little evidence of progress towards HTM work better with
RCU and with other deferred-reclamation mechanisms.

In short, current HTM implementations appear to be welcome and useful
additions to the parallel programmer's toolbox, and much interesting
and challenging work is required to make use of them.
However, they cannot be
considered to be a magic wand with which to wave away all parallel-programming
problems.
