% locking/locking-existence.tex
% mainfile: ../perfbook.tex
% SPDX-License-Identifier: CC-BY-SA-3.0

\section{Lock-Based Existence Guarantees}
\label{sec:locking:Lock-Based Existence Guarantees}
%
\epigraph{Existence precedes and rules essence.}{\emph{Jean-Paul Sartre}}

\begin{listing}[tbp]
\begin{fcvlabel}[ln:locking:Per-Element Locking Without Existence Guarantees]
\begin{VerbatimL}[commandchars=\\\@\$]
int delete(int key)
{
	int b;
	struct element *p;

	b = hashfunction(key);
	p = hashtable[b];
	if (p == NULL || p->key != key)		\lnlbl@chk_first$
		return 0;
	spin_lock(&p->lock);			\lnlbl@acq$
	hashtable[b] = NULL;			\lnlbl@NULL$
	spin_unlock(&p->lock);
	kfree(p);
	return 1;				\lnlbl@return1$
}
\end{VerbatimL}
\end{fcvlabel}
\caption{Per-Element Locking Without Existence Guarantees}
\label{lst:locking:Per-Element Locking Without Existence Guarantees}
\end{listing}

병렬 프로그래밍에서의 핵심 도전사항은 \emph{존재 보장}~\cite{Gamsa99} 을
제공하여, 특정 객체로의 액세스 시도가 해당 액세스 시도 동안은 객체의 존재에
의존할 수 있게 하는 것입니다.
어떤 경우, 존재 보장은 묵시적입니다:

\iffalse

A key challenge in parallel programming is to provide
\emph{existence guarantees}~\cite{Gamsa99},
so that attempts to access a given object can rely on that object
being in existence throughout a given access attempt.
In some cases, existence guarantees are implicit:

\fi

\begin{enumerate}
\item	기본 모듈의 전역 변수와 정적 로컬 변수들은 해당 어플리케이션이 수행되는
	동안은 존재합니다.
\item	로드된 모듈의 전역 변수와 정적 로컬 변수는 이 모듈이 로드된 상태를
	유지하는 동안은 존재합니다.
\item	모듈은 그것의 함수들이 활동 상태의 인스턴스를 하나라도 가지고 있는
	동안은 로드된 상태를 유지합니다.
\item	특정 함수 인스턴스의 스택 상 변수들은 이 인스턴스가 리턴할 때까지는
	존재합니다.
\item	여러분이 어떤 함수 내에서 수행 중이거나 (직접적으로 또는 간접적으로)
	해당 함수에서 호출되었다면, 이 함수는 활동 상태 인스턴스를 갖습니다.

\iffalse

\item	Global variables and static local variables in the base module
	will exist as long as the application is running.
\item	Global variables and static local variables in a loaded module
	will exist as long as that module remains loaded.
\item	A module will remain loaded as long as at least one of its functions
	has an active instance.
\item	A given function instance's on-stack variables will exist until
	that instance returns.
\item	If you are executing within a given function or have been
	called (directly or indirectly) from that function,
	then the given function has an active instance.

\fi

\end{enumerate}

이 묵시적 존재 보장은 간단합니다만, 묵시적 존재 보장에 관련된 버그는 실제로
일어날 수 있습니다.

\iffalse

These implicit existence guarantees are straightforward, though
bugs involving implicit existence guarantees really can happen.

\fi

\QuickQuiz{
	묵시적 존재 보장에 기대는게 어떻게 버그에 이를 수 있죠?

	\iffalse

	How can relying on implicit existence guarantees result in
	a bug?

	\fi

}\QuickQuizAnswer{
	여기 올바르지 않은 묵시적 존재 보장의 사용으로부터 나올 수 있는
	버그들이 있습니다:
	\begin{enumerate}
	\item	한 프로그램이 한 전역 변수로의 주소를 어떤 파일에 쓰고, 같은
		프로그램의 나중 인스턴스가 이 주소를 읽고 역참조 하려 합니다.
		이는 주소 공간 무작위화 (address-space randomization) 때문에
		실패할 수 있으며, 이 프로그램의 재 컴파일링에 대해선 말할 것도
		없습니다.
	\item	어떤 모듈은 자신의 변수로의 주소를 어떤 다른 모듈에 위치한
		포인터에 저장하고, 이 모듈이 언로드된 다음에 이 포인터를 역참조
		하려 할 수 있습니다.
	\item	어떤 함수는 자신의 스택 상의 변수의 주소를 어떤 전역 변수에
		저장하고, 이 함수가 리턴된 다음에 어떤 다른 함수가 그 주소를
		역참조 할 수도 있습니다.
	\end{enumerate}
	여러분은 추가적인 가능성을 가져올 수 있으리라 확신합니다.

	\iffalse

	Here are some bugs resulting from improper use of implicit
	existence guarantees:
	\begin{enumerate}
	\item	A program writes the address of a global variable to
		a file, then a later instance of that same program
		reads that address and attempts to dereference it.
		This can fail due to address-space randomization,
		to say nothing of recompilation of the program.
	\item	A module can record the address of one of its variables
		in a pointer located in some other module, then attempt
		to dereference that pointer after the module has
		been unloaded.
	\item	A function can record the address of one of its on-stack
		variables into a global pointer, which some other
		function might attempt to dereference after that function
		has returned.
	\end{enumerate}
	I am sure that you can come up with additional possibilities.

	\fi

}\QuickQuizEnd

하지만 더 흥미로운---그리고 문제가 있을 수 있는---보장은 힙 메모리와
연관됩니다: 동적으로 할당된 데이터 구조는 그것이 해제될 때까지 존재할 겁니다.
해결되야 하는 문제는 이 구조체의 해제를 같은 구조체로의 동시의 액세스와 동기화
시켜야 한다는 것입니다.
이를 위한 한가지 방법은 락킹과 같은 \emph{명시적 보장} 입니다.
주어진 구조체가 이 락을 잡은 상태에서만 해제된다면, 이 락을 잡는 것은 이
구조체의 존재를 보장합니다.

하지만 이 보장은 이 락 자체의 존재에 의존합니다.
이 락의 존재를 보장하는 한가지 간단한 방법은 이 락을 전역 변수에 두는
것입니다만, 전역 락킹은 확장성 제한의 단점을 갖습니다.
데이터 구조체의 크기의 증가에 따라 개선되는 확장성을 제공하는 한가지 방법은 이
구조체의 각 원소마다 락을 두는 것입니다.
불행히도, 데이터 원소를 보호해야 하는 락을 데이터 원소 자체에 두는 것은
Listing~\ref{lst:locking:Per-Element Locking Without Existence Guarantees}
에 보인 것과 같이 일부 레이스 컨디션을 일으킵니다.

\iffalse

But the more interesting---and troublesome---guarantee involves
heap memory: A dynamically allocated data structure will exist until it
is freed.
The problem to be solved is to synchronize the freeing of the structure
with concurrent accesses to that same structure.
One way to do this is with \emph{explicit guarantees}, such as locking.
If a given structure may only be freed while holding a given lock, then holding
that lock guarantees that structure's existence.

But this guarantee depends on the existence of the lock itself.
One straightforward way to guarantee the lock's existence is to
place the lock in a global variable, but global locking has the disadvantage
of limiting scalability.
One way of providing scalability that improves as the size of the
data structure increases is to place a lock in each element of the
structure.
Unfortunately, putting the lock that is to protect a data element
in the data element itself is subject to subtle race conditions,
as shown in
Listing~\ref{lst:locking:Per-Element Locking Without Existence Guarantees}.

\fi

\QuickQuiz{
	\begin{fcvref}[ln:locking:Per-Element Locking Without Existence Guarantees]
	우리가 제거해야 하는 원소가
	Listing~\ref{lst:locking:Per-Element Locking Without Existence Guarantees}
	의 라인~\lnref{chk_first} 에 있는 리스트의 첫번째 원소가 아니라면
	어떻게 되는 걸까요?
        \end{fcvref}

	\iffalse

	\begin{fcvref}[ln:locking:Per-Element Locking Without Existence Guarantees]
	What if the element we need to delete is not the first element
	of the list on line~\lnref{chk_first} of
	Listing~\ref{lst:locking:Per-Element Locking Without Existence Guarantees}?
        \end{fcvref}

	\fi

}\QuickQuizAnswer{
	이건 체이닝이 없는 매우 간단한 해쉬 테이블이며, 따라서 특정 버킷에
	들어있는 유일한 원소는 첫번째 원소입니다.
	독자 여러분은 이 예를 완전한 체이닝을 갖는 해쉬 테이블에 적용해 보셔도
	좋겠습니다.

	\iffalse

	This is a very simple hash table with no chaining, so the only
	element in a given bucket is the first element.
	The reader is invited to adapt this example to a hash table with
	full chaining.

	\fi

}\QuickQuizEnd

\begin{fcvref}[ln:locking:Per-Element Locking Without Existence Guarantees]
이런 레이스 컨디션들 중 하나를 자세히 보기 위해, 다음 이벤트들을 생각해 봅시다:
\begin{enumerate}
	\item	쓰레드~0 이 \co{delete(0)} 를 호출하고, 이 listing 의
		라인~\lnref{acq} 에 도달, 락을 잡습니다.
	\item	쓰레드~1 이 \co{delete(0)} 을 호출하고, 라인~\lnref{acq} 에
		도달하지만 쓰레드~0 이 이미 락을 잡고 있으므로 여기서
		맴돕니다.
	\item	쓰레드~0 이 \clnrefrange{NULL}{return1} 을 수행하고, 해쉬
		테이블에서 원소를 제거하고, 락을 해제하고, 원소를 메모리 할당
		해제합니다.
	\item	쓰레드~0 이 수행을 계속해서 메모리를 할당받아 방금 해제한 바로
		그 메모리 블록을 받습니다.
	\item	쓰레드~0 은 이제 이 메모리 블록을 어떤 다른 타입의 구조체로
		초기화 시킵니다.
	\item	쓰레드~1 의 \co{spin_lock()} 오퍼레이션은 자신이 \co{p->lock}
		이라 믿었던 게 더이상 스핀락이 아니게 되었으므로 실패합니다.
\end{enumerate}
어떤 존재 보장이 없으므로, 이 데이터 원소의 정체성은 이 쓰레드가 이 원소의 락을
라인~\lnref{acq} 에서 획득하려 하는 사이에 바뀔 수 있습니다!
\end{fcvref}

\iffalse

\begin{fcvref}[ln:locking:Per-Element Locking Without Existence Guarantees]
To see one of these race conditions, consider the following sequence
of events:
\begin{enumerate}
\item	Thread~0 invokes \co{delete(0)}, and reaches line~\lnref{acq} of
	the listing, acquiring the lock.
\item	Thread~1 concurrently invokes \co{delete(0)}, reaching
	line~\lnref{acq}, but spins on the lock because Thread~0 holds it.
\item	Thread~0 executes \clnrefrange{NULL}{return1}, removing the element from
	the hashtable, releasing the lock, and then freeing the
	element.
\item	Thread~0 continues execution, and allocates memory, getting
	the exact block of memory that it just freed.
\item	Thread~0 then initializes this block of memory as some
	other type of structure.
\item	Thread~1's \co{spin_lock()} operation fails due to the
	fact that what it believes to be \co{p->lock} is no longer
	a spinlock.
\end{enumerate}
Because there is no existence guarantee, the identity of the
data element can change while a thread is attempting to acquire
that element's lock on line~\lnref{acq}!
\end{fcvref}

\fi

\begin{listing}[tbp]
\begin{fcvlabel}[ln:locking:Per-Element Locking With Lock-Based Existence Guarantees]
\begin{VerbatimL}[commandchars=\\\@\$]
int delete(int key)
{
	int b;
	struct element *p;
	spinlock_t *sp;

	b = hashfunction(key);
	sp = &locktable[b];
	spin_lock(sp);				\lnlbl@acq$
	p = hashtable[b];			\lnlbl@getp$
	if (p == NULL || p->key != key) {
		spin_unlock(sp);
		return 0;
	}
	hashtable[b] = NULL;
	spin_unlock(sp);
	kfree(p);
	return 1;
}
\end{VerbatimL}
\end{fcvlabel}
\caption{Per-Element Locking With Lock-Based Existence Guarantees}
\label{lst:locking:Per-Element Locking With Lock-Based Existence Guarantees}
\end{listing}

\begin{fcvref}[ln:locking:Per-Element Locking With Lock-Based Existence Guarantees]
이 예를 고치는 한가지 방법은 해시된 전역 락 집합을 사용해서 각 해시 버킷이
각자의 락을 가지게 하는 것으로,
Listing~\ref{lst:locking:Per-Element Locking With Lock-Based Existence Guarantees}
에 보여져 있습니다.
이 방법은 데이터 원소로의 포인터를 얻기 전에 (라인~\lnref{getp}) 올바른 락을
획득할 수 있게 (라인~\lnref{acq}) 해줍니다.
이 방법이 이 listing 에 보인 해시 테이블과 같이 단일 분할 가능 데이터 구조체에
포함되어 있는 원소들에는 상당히 잘 동작하지만, 주어진 데이터 원소가 여러 해시
테이블 또는 트리나 graph 같은 더 복잡한 데이터 구조체에서는 문제가 있을 수
있습니다.
해결책은 이 문제만 해결할 뿐만 아니라, 락 기반 소프트웨어 트랜잭셔널 메모리
구현의 기반을 형성합니다~\cite{Shavit95,DaveDice2006DISC}.
하지만,
Chapter~\ref{chp:Deferred Processing} 는 존재 보장을 제공하는 더
간단한---그리고 더 빠른---방법을 설명합니다.
\end{fcvref}

\iffalse

\begin{fcvref}[ln:locking:Per-Element Locking With Lock-Based Existence Guarantees]
One way to fix this example is to use a hashed set of global locks, so
that each hash bucket has its own lock, as shown in
Listing~\ref{lst:locking:Per-Element Locking With Lock-Based Existence Guarantees}.
This approach allows acquiring the proper lock (on line~\lnref{acq}) before
gaining a pointer to the data element (on line~\lnref{getp}).
Although this approach works quite well for elements contained in a
single partitionable data structure such as the hash table shown in the
listing, it can be problematic if a given data element can be a member
of multiple hash tables or given more-complex data structures such
as trees or graphs.
Not only can these problems be solved, but the solutions also form
the basis of lock-based software transactional memory
implementations~\cite{Shavit95,DaveDice2006DISC}.
However,
Chapter~\ref{chp:Deferred Processing}
describes simpler---and faster---ways of providing existence guarantees.
\end{fcvref}

\fi
