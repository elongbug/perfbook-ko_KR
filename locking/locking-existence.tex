% locking-existence.tex

\section{Lock-Based Existence Guarantees}
\label{sec:locking:Lock-Based Existence Guarantees}

\begin{figure}[tbp]
{ \scriptsize
\begin{verbbox}
  1 int delete(int key)
  2 {
  3   int b;
  4   struct element *p;
  5
  6   b = hashfunction(key);
  7   p = hashtable[b];
  8   if (p == NULL || p->key != key)
  9     return 0;
 10   spin_lock(&p->lock);
 11   hashtable[b] = NULL;
 12   spin_unlock(&p->lock);
 13   kfree(p);
 14   return 1;
 15 }
\end{verbbox}
}
\centering
\theverbbox
\caption{Per-Element Locking Without Existence Guarantees}
\label{fig:locking:Per-Element Locking Without Existence Guarantees}
\end{figure}

병렬 프로그래밍에서의 핵심적인 난제는 \emph{존재 보장}~\cite{Gamsa99} 을
제공하는 것으로, 어떤 객체에 접근하려 시도하는 것은 그 객체가 이 접근 시도 동안
존재하는지 여부에 의존적입니다.
어떤 경우에는 존재 보장이 명시적이지 않기도 합니다:
\iffalse

A key challenge in parallel programming is to provide
\emph{existence guarantees}~\cite{Gamsa99},
so that attempts to access a given object can rely on that object
being in existence throughout a given access attempt.
In some cases, existence guarantees are implicit:
\fi

\begin{enumerate}
\item	기본 모듈의 글로벌 변수와 static 지역 변수들은 어플리케이션이 돌아가는
	동안은 존재할 것입니다.
\item	로드된 모듈의 글로벌 변수들과 static 지역 변수들은 그 모듈이 로드된
	상태를 유지하는 동안은 존재할 것입니다.
\item	하나의 모듈은 자신의 함수들 중 최소한 하나는 실행중인 실행 흐름을
	가지고 있되 동안은 로드된 상태를 유지할 것입니다.
\item	어떤 함수의 실행중인 흐름의 스택 위에 할당되는 변수들은 그 함수의 실행
	흐름이 리턴을 하기 전까지는 존재할 것입니다.
\item	어떤 함수 위에서 코드를 수행 중이거나 (직접적이든 간접적이든) 그
	함수로부터 호출되었다면, 그 함수는 실행중인 흐름을 가지고 있는
	것입니다.
\iffalse
\item	Global variables and static local variables in the base module
	will exist as long as the application is running.
\item	Global variables and static local variables in a loaded module
	will exist as long as that module remains loaded.
\item	A module will remain loaded as long as at least one of its functions
	has an active instance.
\item	A given function instance's on-stack variables will exist until
	that instance returns.
\item	If you are executing within a given function or have been
	called (directly or indirectly) from that function,
	then the given function has an active instance.
\fi
\end{enumerate}

비록 명시적이지 않은 존재 보장과 연관된 버그들이 실제로 만들어질 수 있긴 하지만
이런 명시적이지 않은 존재 보장은 직관적인 이야기입니다.
\iffalse

These implicit existence guarantees are straightforward, though
bugs involving implicit existence guarantees really can happen.
\fi

\QuickQuiz{}
	어떻게 비명시적인 존재 보장에 의존하는게 버그를 만들어낼 수 있나요?
	\iffalse

	How can relying on implicit existence guarantees result in
	a bug?
	\fi
\QuickQuizAnswer{
	비명시적 존재 보장의 잘못된 사용에 의해 초래될 수 있는 일부 버그들이
	여기 있습니다:
	\iffalse

	Here are some bugs resulting from improper use of implicit
	existence guarantees:
	\fi
	\begin{enumerate}
	\item	어떤 프로그램이 한 글로벌 변수의 주소를 한 파일에 쓰고, 그 후에
		똑같은 프로그램의 새로운 실행 흐름이 그 주소를 읽고는 그 주소의
		값을 읽어보려 시도합니다.
		이는 그 프로그램의 기존 수행에서의 내용을 기반으로 다음
		수행에서의 구성을 예측할 수 없게 하는 주소 공간 임의 추출
		(address space randomization) 기법으로 인해 실패할 수 있습니다.
	\item	어떤 모듈은 자신의 변수들 가운데 하나의 주소를 다른 모듈에
		위치해 있는 한 포인터에 기록해 두고는 그 포인터가 가리키는 값을
		모듈이 내려간 (unloaded) 다음에 읽어보려 시도할 수 있습니다.
	\item	어떤 함수는 자신의 스택 위에 할당된 변수들 가운데 하나의 주소를
		어떤 글로벌 포인터에 기록해 두고는 그 함수가 리턴한 후에 어떤
		다른 함수가 그 포인터가 가리키는 값을 읽어보려 시도할 수
		있습니다.
	\iffalse

	\item	A program writes the address of a global variable to
		a file, then a later instance of that same program
		reads that address and attempts to dereference it.
		This can fail due to address-space randomization,
		to say nothing of recompilation of the program.
	\item	A module can record the address of one of its variables
		in a pointer located in some other module, then attempt
		to dereference that pointer after the module has
		been unloaded.
	\item	A function can record the address of one of its on-stack
		variables into a global pointer, which some other
		function might attempt to dereference after that function
		has returned.
	\fi
	\end{enumerate}
	그 외에도 여러가지 추가적인 가능성이 있을 수 있다고 확신합니다.
	\iffalse

	I am sure that you can come up with additional possibilities.
	\fi
} \QuickQuizEnd

하지만 그보다 더 흥미로운---그리고 문제를 일으키는---보장은 heap 메모리에
연관된 것입니다: 동적으로 할당된 데이터 구조체는 그 메모리가 해제되기 전까지
존재할 것입니다.
여기서 해결해야 할 문제는 구조체에 대한 동시적인 접근들과 그 구조체가 존재하는
메모리의 해제를 동기화 시키는 것입니다.
이를 해결하는 방법 가운데 하나는 락킹과 같은 \emph{명시적 보장} 입니다.
만약 어떤 구조체가 어떤 주어진 락을 잡은 채로만 메모리 해제될 수 있다고 한다면,
그 락을 잡는 행위는 그 구조체의 존재를 보장합니다.
\iffalse

But the more interesting---and troublesome---guarantee involves
heap memory: A dynamically allocated data structure will exist until it
is freed.
The problem to be solved is to synchronize the freeing of the structure
with concurrent accesses to that same structure.
One way to do this is with \emph{explicit guarantees}, such as locking.
If a given structure may only be freed while holding a given lock, then holding
that lock guarantees that structure's existence.
\fi

하지만 이 보장은 또다시 그 락 자체의 존재 여부에 의존적입니다.
이 락의 존재를 보장하기 위한 한가지 직선적인 방법은 그 락을 글로벌 변수로
위치시키는 것입니다만, 글로벌한 락킹은 제약된 확장성이라는 단점을 갖습니다.
데이터 구조체의 크기가 커질수록 향상되는 확장성을 제공하는 한가지 방법은 락을
구조체의 각 원소마다에 위치시키는 것입니다.
불행히도 데이터 원소 자체 내에 그 데이터 원소를 지키기 위한 락을 위치시키는
것은 Figure~\ref{fig:locking:Per-Element Locking Without Existence Guarantees}
미묘한 경쟁 상황을 일으키기 쉽습니다.
\iffalse

But this guarantee depends on the existence of the lock itself.
One straightforward way to guarantee the lock's existence is to
place the lock in a global variable, but global locking has the disadvantage
of limiting scalability.
One way of providing scalability that improves as the size of the
data structure increases is to place a lock in each element of the
structure.
Unfortunately, putting the lock that is to protect a data element
in the data element itself is subject to subtle race conditions,
as shown in
Figure~\ref{fig:locking:Per-Element Locking Without Existence Guarantees}.
\fi

\QuickQuiz{}
	우리가 삭제해야 하는 원소가
	Figure~\ref{fig:locking:Per-Element Locking Without Existence Guarantees}
	의 line~8 의 리스트의 첫번째 원소가 아니면 어떻게 하죠?
	\iffalse

	What if the element we need to delete is not the first element
	of the list on line~8 of
	Figure~\ref{fig:locking:Per-Element Locking Without Existence Guarantees}?
	\fi
\QuickQuizAnswer{
	이건 체이닝을 사용하지 않는 매우 간단한 해시 테이블이므로, 주어진
	bucket 의 원소는 첫번째 원소 뿐입니다.
	독자 여러분은 이 간단한 예제 해시 테이블에 체이닝을 적용해 보셔도 좋을
	겁니다.
	\iffalse

	This is a very simple hash table with no chaining, so the only
	element in a given bucket is the first element.
	The reader is invited to adapt this example to a hash table with
	full chaining.
	\fi
} \QuickQuizEnd

\QuickQuiz{}
	Figure~\ref{fig:locking:Per-Element Locking Without Existence Guarantees}
	에서 어떤 경쟁 조건이 발생할 수 있나요?
	\iffalse

	What race condition can occur in
	Figure~\ref{fig:locking:Per-Element Locking Without Existence Guarantees}?
	\fi
\QuickQuizAnswer{
	다음과 같은 일련의 이벤트들을 생각해 봅시다:
	\iffalse

	Consider the following sequence of events:
	\fi
	\begin{enumerate}
	\item	Thread~0 이 \co{delete(0)} 을 호출하고, 그림의 line~10 에
		도달해서 락을 얻습니다.
	\item	Thread~1 이 동시적으로 \co{delete(0)} 를 호출하고, line~10 에
		도착합니다만 Thread~0 이 이미 락을 잡고 있으니 그 락을 기다리며
		루프를 돕니다.
	\item	thread~0 이 line~11-14 를 수행하고 해시테이블에서 그 원소를
		삭제하고 락을 해제하고, 그 원소를 메모리 해제시킵니다.
	\item	Thread~0 는 수행을 계속해서 메모리를 할당받는데 이 때 정확히
		방금 해제한 메모리 블락을 받습니다.
	\item	Thread~0 는 이제 이 메모리 블락을 어떤 다른 타입의 구조체로
		초기화 시킵니다.
	\item	Thread~1 의 \co{spin_lock()} 은 스핀락이라 생각했던
		\co{p->lock} 이 더이상 스핀락이 아니기 때문에 실패합니다.
	\iffalse

	\item	Thread~0 invokes \co{delete(0)}, and reaches line~10 of
		the figure, acquiring the lock.
	\item	Thread~1 concurrently invokes \co{delete(0)}, reaching
		line~10, but spins on the lock because Thread~0 holds it.
	\item	Thread~0 executes lines~11-14, removing the element from
		the hashtable, releasing the lock, and then freeing the
		element.
	\item	Thread~0 continues execution, and allocates memory, getting
		the exact block of memory that it just freed.
	\item	Thread~0 then initializes this block of memory as some
		other type of structure.
	\item	Thread~1's \co{spin_lock()} operation fails due to the
		fact that what it believes to be \co{p->lock} is no longer
		a spinlock.
	\fi
	\end{enumerate}
	존재에 대한 보장이 없기 때문에 해당 데이터 원소의 정체성은 어떤
	쓰레드가 line~10 에서 그 원소의 락을 얻기 위해 시도중인 상황 중에도
	바뀔 수가 있는 겁니다!
	\iffalse

	Because there is no existence guarantee, the identity of the
	data element can change while a thread is attempting to acquire
	that element's lock on line~10!
	\fi
} \QuickQuizEnd

\begin{figure}[tbp]
{ \scriptsize
\begin{verbbox}
  1 int delete(int key)
  2 {
  3   int b;
  4   struct element *p;
  5   spinlock_t *sp;
  6
  7   b = hashfunction(key);
  8   sp = &locktable[b];
  9   spin_lock(sp);
 10   p = hashtable[b];
 11   if (p == NULL || p->key != key) {
 12     spin_unlock(sp);
 13     return 0;
 14   }
 15   hashtable[b] = NULL;
 16   spin_unlock(sp);
 17   kfree(p);
 18   return 1;
 19 }
\end{verbbox}
}
\centering
\theverbbox
\caption{Per-Element Locking With Lock-Based Existence Guarantees}
\label{fig:locking:Per-Element Locking With Lock-Based Existence Guarantees}
\end{figure}

이 예제를 고치는 한가지 방법은 해싱을 사용한 글로벌 락들의 집합을 사용하는
것으로,
Figure~\ref{fig:locking:Per-Element Locking With Lock-Based Existence Guarantees}
에 보여진 것처럼 각각의 해시 bucket 이 자신의 락을 가지고 있는 형태입니다.
이 방법은 그 데이터 원소로의 포인터를 얻어오기 (line~10 에서) 전에 올바른 락을
얻어올 수 있도록 합니다 (line~9 에서).
이 접근법은 이 그림에 보여진 해시 테이블처럼 하나의 분할 가능한 데이터 구조체에
대해서는 잘 동작하지만 주어진 데이터 원소가 여러 해시 테이블의 멤버이거나
트리나 그래프와 같은 보다 복잡한 데이터 구조체의 멤버라면 문제가 생길 수도
있습니다.
사실, 이런 문제들은 해결될 수 있는데, 그런 해결책들이 락에 기반한 소프트웨어
트랜잭셔널 메모리 구현의 기반을 이룹니다~\cite{Shavit95,DaveDice2006DISC}.
하지만, Chapter~\ref{chp:Deferred Processing} 에서 더 간단한---그리고 더
빠른---존재 보장을 제공하는 방법을 알아보겠습니다.
\iffalse

One way to fix this example is to use a hashed set of global locks, so
that each hash bucket has its own lock, as shown in
Figure~\ref{fig:locking:Per-Element Locking With Lock-Based Existence Guarantees}.
This approach allows acquiring the proper lock (on line~9) before
gaining a pointer to the data element (on line~10).
Although this approach works quite well for elements contained in a
single partitionable data structure such as the hash table shown in the
figure, it can be problematic if a given data element can be a member
of multiple hash tables or given more-complex data structures such
as trees or graphs.
These problems can be solved, in fact, such solutions form the basis
of lock-based software transactional memory
implementations~\cite{Shavit95,DaveDice2006DISC}.
However,
Chapter~\ref{chp:Deferred Processing}
describes simpler---and faster---ways of providing existence guarantees.
\fi
