% formal/axiomatic.tex

\section{Axiomatic Approaches}
\label{sec:formal:Axiomatic Approaches}
\OriginallyPublished{Section}{sec:formal:Axiomatic Approaches}{Axiomatic Approaches}{Linux Weekly News}{PaulEMcKenney2014weakaxiom}

\begin{listing*}[tb]
{ \scriptsize
\begin{verbbox}
 1 PPC IRIW.litmus
 2 ""
 3 (* Traditional IRIW. *)
 4 {
 5 0:r1=1; 0:r2=x;
 6 1:r1=1;         1:r4=y;
 7         2:r2=x; 2:r4=y;
 8         3:r2=x; 3:r4=y;
 9 }
10 P0           | P1           | P2           | P3           ;
11 stw r1,0(r2) | stw r1,0(r4) | lwz r3,0(r2) | lwz r3,0(r4) ;
12              |              | sync         | sync         ;
13              |              | lwz r5,0(r4) | lwz r5,0(r2) ;
14 
15 exists
16 (2:r3=1 /\ 2:r5=0 /\ 3:r3=1 /\ 3:r5=0)
\end{verbbox}
}
\centering
\theverbbox
\caption{IRIW Litmus Test}
\label{lst:formal:IRIW Litmus Test}
\end{listing*}

PPCMEM 도구가
Listing~\ref{lst:formal:IRIW Litmus Test} 에 보여진,
유명한 ``independent reads of independent writes'' (IRIW) 리트머스 테스트를
해곃할 수 있긴 합니다만, 그러기 위해선 14 시간 이상의 CPU 시간과 10 기가바이트
이상의 상태공간을 필요로 합니다.
그렇다곤 하나, 이 상황은 이 문제를 풀기 위해선 커다란 레퍼런스 매뉴얼을
뒤져보고, 증명을 시도하고, 전문가와 토론을 하고, 마지막으로 내린 답에 대해서도
확신할 수 없었던, PPCMEM 이 나오기 전에 비하면 커다란 개선이 이뤄진 것입니다.
비록 14 시간은 긴 시간처럼 보일 수 있겠지만, 이는 수 주나 수 개월에 비하면
너무나도 짧은 시간입니다.

하지만, 두개의 쓰레드가 두개의 별도의 변수에 값을 쓰고 두개의 다른 쓰레드가 이
두개의 변수로부터 반대 순서로 값을 읽어들일 뿐인 해당 리트머스 테스트의
단순성을 놓고 보면 요구되는 해당 시간은 조금 놀랍습니다.
단정문은 두개의 값을 읽어들이는 쓰레드들이 두개의 값 저장의 순서에 대해 서로
다른 의견을 갖는다면 터집니다.
이 리트머스 테스트는 긴단한데, 표준적 메모리 순서 리트머스 테스트들을 놓고 봐도
그렇습니다.
\iffalse

Although the PPCMEM tool can solve the famous ``independent reads of
independent writes'' (IRIW) litmus test shown in
Listing~\ref{lst:formal:IRIW Litmus Test}, doing so requires no less than
fourteen CPU hours and generates no less than ten gigabytes of state space.
That said, this situation is a great improvement over that before the advent
of PPCMEM, where solving this problem required perusing volumes of
reference manuals, attempting proofs, discussing with experts, and
being unsure of the final answer.
Although fourteen hours can seem like a long time, it is much shorter
than weeks or even months.

However, the time required is a bit surprising given the simplicity
of the litmus test, which has two threads storing to two separate variables
and two other threads loading from these two variables in opposite
orders.
The assertion triggers if the two loading threads disagree on the order
of the two stores.
This litmus test is simple, even by the standards of memory-order litmus
tests.
\fi

소모되는 시간과 공간의 양에 대한 한가지 이유는 PPCMEM 이 추적 기반의 전체 상태
공간 탐색을 한다는 것으로, 이는 아키텍쳐 단계에서 이벤트들의 모든 가능한 순서와
조합을 만들어내고 수행해 봐야 함을 의미합니다.
이 단계에서, 화려한 이벤트와 액션들의 시퀀스에 연관된 로드와 스토어는 모두 모두
탐색되어야만 하는 매우 커다란 상태 공간을 초래하게 되고, 이는 커다란 메모리와
CPU 소모로 이어지게 됩니다.

물론, 그런 추적들 가운데 많은 것들은 다른 것들과 상당히 유사해서, 비슷한
추적들을 하나로 취급하는 것이 성능을 개선시킬 수도 있을 것임을 시사합니다.
그런 한가지 방법이 Alglave 등의 공리적 집합론
방법~\cite{Alglave:2014:HCM:2594291.2594347} 으로, 여기선 메모리 모델을
나타내기 위한 공리 집합을 만들어내고 리트머스 테스트들을 이 공리들의 집합으로
증명되거나 반증될 수 있는 정리로 변환시킵니다.
이렇게 만들어진, ``herd'' 라 불리는 도구는 편리하게 PPCMEM 에서와 같은 리트머스
테스트들을 입력으로 받는데,
Listing~\ref{lst:formal:IRIW Litmus Test} 에 보인 IRIW 리트머스 테스트도
포함됩니다.
\iffalse

One reason for the amount of time and space consumed is that PPCMEM does
a trace-based full-state-space search, which means that it must generate
and evaluate all possible orders and combinations of events at the
architectural level.
At this level, both loads and stores correspond to ornate sequences
of events and actions, resulting in a very large state space that must
be completely searched, in turn resulting in large memory and CPU
consumption.

Of course, many of the traces are quite similar to one another, which
suggests that an approach that treated similar traces as one might
improve performace.
One such approach is the axiomatic approach of
Alglave et al.~\cite{Alglave:2014:HCM:2594291.2594347},
which creates a set of axioms to represent the memory model and then
converts litmus tests to theorems that might be proven or disproven
over this set of axioms.
The resulting tool, called ``herd'',  conveniently takes as input the
same litmus tests as PPCMEM, including the IRIW litmus test shown in
Listing~\ref{lst:formal:IRIW Litmus Test}.
\fi

\begin{listing*}[tb]
{ \scriptsize
\begin{verbbox}
 1 PPC IRIW5.litmus
 2 ""
 3 (* Traditional IRIW, but with five stores instead of just one. *)
 4 {
 5 0:r1=1; 0:r2=x;
 6 1:r1=1;         1:r4=y;
 7         2:r2=x; 2:r4=y;
 8         3:r2=x; 3:r4=y;
 9 }
10 P0           | P1           | P2           | P3           ;
11 stw r1,0(r2) | stw r1,0(r4) | lwz r3,0(r2) | lwz r3,0(r4) ;
12 addi r1,r1,1 | addi r1,r1,1 | sync         | sync         ;
13 stw r1,0(r2) | stw r1,0(r4) | lwz r5,0(r4) | lwz r5,0(r2) ;
14 addi r1,r1,1 | addi r1,r1,1 |              |              ;
15 stw r1,0(r2) | stw r1,0(r4) |              |              ;
16 addi r1,r1,1 | addi r1,r1,1 |              |              ;
17 stw r1,0(r2) | stw r1,0(r4) |              |              ;
18 addi r1,r1,1 | addi r1,r1,1 |              |              ;
19 stw r1,0(r2) | stw r1,0(r4) |              |              ;
20 
21 exists
22 (2:r3=1 /\ 2:r5=0 /\ 3:r3=1 /\ 3:r5=0)
\end{verbbox}
}
\centering
\theverbbox
\caption{Expanded IRIW Litmus Test}
\label{lst:formal:Expanded IRIW Litmus Test}
\end{listing*}

IRIW 를 푸는데에 PPCMEM 이 14 CPU 시간을 필요로 하는 반면, herd 는 17
밀리세컨드만에 IRIW 를 푸는데, 이는 백만배 이상의 속도 향상을 의미합니다.
그렇다곤 하나, 문제는 근본적으로 기하급수적이므로, 더 커다란 문제들에 있어서는
herd 역시 기하급수적으로 느려질 것을 예상해야 합니다.
그리고 이는 실제로 일어나는 일인데, 예를 들어 우리가
Listing~\ref{lst:formal:Expanded IRIW Litmus Test} 에 보여진 것처럼 쓰기를 하는
CPU 마다 네개의 쓰기를 추가하면, herd 는 50,000 배 이상 느려져서 15 \emph{분}
이상의 CPU 시간을 필요로 하게 됩니다.
쓰레드를 추가하는 것 역시 기하급수적인 속도저하를
초래합니다~\cite{PaulEMcKenney2014weakaxiom}.
\iffalse

However, where PPCMEM requires 14 CPU hours to solve IRIW, herd does so
in 17 milliseconds, which represents a speedup of more than six orders
of magnitude.
That said, the problem is exponential in nature, so we should expect
herd to exhibit exponential slowdowns for larger problems.
And this is exactly what happens, for example, if we add four more writes
per writing CPU as shown in
Listing~\ref{lst:formal:Expanded IRIW Litmus Test},
herd slows down by a factor of more than 50,000, requiring more than
15 \emph{minutes} of CPU time.
Adding threads also results in exponential
slowdowns~\cite{PaulEMcKenney2014weakaxiom}.
\fi

이런 근본적 기하급수적 성질에도 불구하고, PPCMEM 과 herd 는 x86 시스템에서의
queued-lock handoff 를 포함해서 핵심적 병렬 알고리즘을 체크하는데에 상당히
유용한 것으로 증명되었습니다.
Herd 의 약점은
Section~\ref{sec:formal:PPCMEM Discussion} 에서 이야기한 PPCMEM 의 그것과
유사합니다.
PPCMEM 과 herd 가 동의하지 않게 되는 불분명한 (하지만 매우 현실적인) 경우들이
존재하는데 2014년 말 현재까지는 이 이견문제를 해결하는 노력이 진행 중입니다.

장기적으로, 희망은 공리적 방법이 더 높은 단계의 소프트웨어의 것들을 설명하는
공리들을 포함하는 것입니다.
이는 잠재적으로 훨씬 더 커다란 소프트웨어 시스템의 공리적 증명을 가능하게
할것입니다.
또다른 대안은 다음 섹션에서 설명하는 것처럼 이진 논리의 공리들을 제공하는
것으로, 다음 섹션에서 다룹니다.
\iffalse

Despite their exponential nature, both PPCMEM and herd have proven quite
useful for checking key parallel algorithms, including the queued-lock
handoff on x86 systems.
The weaknesses of the herd tool are similar to those of PPCMEM, which
were described in
Section~\ref{sec:formal:PPCMEM Discussion}.
There are some obscure (but very real) cases for which the PPCMEM and
herd tools disagree, and as of late 2014 resolving these disagreements
was ongoing.

Longer term, the hope is that the axiomatic approaches incorporate
axioms describing higher-level software artifacts.
This could potentially allow axiomatic verification of much larger
software systems.
Another alternative is to press the axioms of boolean logic into service,
as described in the next section.
\fi
