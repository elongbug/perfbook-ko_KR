% formal/axiomatic.tex
% mainfile: ../perfbook.tex
% SPDX-License-Identifier: CC-BY-SA-3.0

\section{Axiomatic Approaches}
\label{sec:formal:Axiomatic Approaches}
\OriginallyPublished{Section}{sec:formal:Axiomatic Approaches}{Axiomatic Approaches}{Linux Weekly News}{PaulEMcKenney2014weakaxiom}
%
\epigraph{Theory helps us to bear our ignorance of facts.}
	{\emph{George Santayana}}

\begin{listing}[tb]
\begin{fcvlabel}[ln:formal:IRIW Litmus Test]
\begin{VerbatimL}[commandchars=\%\@\$]
PPC IRIW.litmus
""
(* Traditional IRIW. *)
{
0:r1=1; 0:r2=x;
1:r1=1;         1:r4=y;
        2:r2=x; 2:r4=y;
        3:r2=x; 3:r4=y;
}
P0           | P1           | P2           | P3           ;
stw r1,0(r2) | stw r1,0(r4) | lwz r3,0(r2) | lwz r3,0(r4) ;
             |              | sync         | sync         ;
             |              | lwz r5,0(r4) | lwz r5,0(r2) ;

exists
(2:r3=1 /\ 2:r5=0 /\ 3:r3=1 /\ 3:r5=0)
\end{VerbatimL}
\end{fcvlabel}
\caption{IRIW Litmus Test}
\label{lst:formal:IRIW Litmus Test}
\end{listing}

PPCMEM 도구가 Listing~\ref{lst:formal:IRIW Litmus Test} 에 보인 유명한
``independent reads of independent writes'' (IRIW) 리트머스 테스트를 해결할 수
있지만, 그러기 위해선 최소 CPU 시간 14 시간을 필요로 하며 최소 10 gigabyte 상태
공간을 생성합니다.
그러나, 이상황은 이 문제를 풀기 위해선 방대한 레퍼런스 매뉴얼을 숙독하고 증명을
시도하고 전문가들과 논의하고 그 마지막 결과에 대해 의심해야 했던,  PPCMEM 이
발전하기 전 상황에 비하면 커다란 진보입니다.
14시간이 긴 시간처럼 보일 수 있지만, 몇주 또는 심지어 몇달에 비하면 훨씬 짧은
시간입니다.

그러나, 이 필요한 시간은 두개의 별개 변수에 값을 저장하는 두개의 쓰레드와 이
두개의 변수들에서 반대 순서로 읽기를 하는 두개의 또다른 쓰레드로 구성된 이
리트머스 테스트의 간단성을 놓고 보면 약간 놀랍기도 합니다.
단정문은 이 두개의 값 읽기 쓰레드가 두개의 값 저장의 순서에 대해 다른 의견을
표출하면 터집니다.
표준 메모리 순서 규칙 리트머스 테스트들 중에서도 이건 꽤 간단한 편입니다.

\iffalse

Although the PPCMEM tool can solve the famous ``independent reads of
independent writes'' (IRIW) litmus test shown in
Listing~\ref{lst:formal:IRIW Litmus Test}, doing so requires no less than
fourteen CPU hours and generates no less than ten gigabytes of state space.
That said, this situation is a great improvement over that before the advent
of PPCMEM, where solving this problem required perusing volumes of
reference manuals, attempting proofs, discussing with experts, and
being unsure of the final answer.
Although fourteen hours can seem like a long time, it is much shorter
than weeks or even months.

However, the time required is a bit surprising given the simplicity
of the litmus test, which has two threads storing to two separate variables
and two other threads loading from these two variables in opposite
orders.
The assertion triggers if the two loading threads disagree on the order
of the two stores.
Even by the standards of memory-order litmus tests, this is quite simple.

\fi

이 소모되는 시간과 공간의 양에 대한 한가지 이유는 PPCMEM 이 추적 기반의 전체
상태 공간 탐색을 한다는 것으로, 이는 이 도구가 아키텍쳐 수준에서 발생 가능한
모든 이벤트의 순서 조합을 생성하고 평가해야 함을 의미합니다.
이 수준에서는 읽기와 스기가 모두 화려하게 장식된 이벤트와 동작들의 조합으로
연관되어서, 완전히 탐색되어야만 하는 매우 거대한 상태 공간을 초래하며, 이는
결국 큰 메모리와 CPU 소모로 이어집니다.

물론, 이 추적들 중 많은 것들은 다른 것들과 비슷한데, 비슷한 추적들을 하나의
이벤트로 취급하는게 성능을 개선할 수도 있음을 시사합니다.
그런 한가지 접근법은 
\pplsur{Jade}{Alglave} 등의~\cite{Alglave:2014:HCM:2594291.2594347} 공리적
접근으로, 메모리 모델을 표현하기 위한 공리 집합을 만들고 리트머스 테스트들을 이
공리들에 의해 증명될 수 있는 이론들로 변환합니다.
그에 의해 만들어지는, \qco{herd} 라 불리는 도구는
Listing~\ref{lst:formal:IRIW Litmus Test} 에 보인 IRIW 리트머스 테스트를 포함해
PPCMEM 같은 도구에 입력되는 것과 똑같은 리트머스 테스트를 받습니다.

\iffalse

One reason for the amount of time and space consumed is that PPCMEM does
a trace-based full-state-space search, which means that it must generate
and evaluate all possible orders and combinations of events at the
architectural level.
At this level, both loads and stores correspond to ornate sequences
of events and actions, resulting in a very large state space that must
be completely searched, in turn resulting in large memory and CPU
consumption.

Of course, many of the traces are quite similar to one another, which
suggests that an approach that treated similar traces as one might
improve performace.
One such approach is the axiomatic approach of
\pplsur{Jade}{Alglave} et al.~\cite{Alglave:2014:HCM:2594291.2594347},
which creates a set of axioms to represent the memory model and then
converts litmus tests to theorems that might be proven or disproven
over this set of axioms.
The resulting tool, called \qco{herd},  conveniently takes as input the
same litmus tests as PPCMEM, including the IRIW litmus test shown in
Listing~\ref{lst:formal:IRIW Litmus Test}.

\fi

\begin{listing}[tb]
\begin{fcvlabel}[ln:formal:Expanded IRIW Litmus Test]
\begin{VerbatimL}[commandchars=\%\@\$]
PPC IRIW5.litmus
""
(* Traditional IRIW, but with five stores instead of *)
(* just one.                                         *)
{
0:r1=1; 0:r2=x;
1:r1=1;         1:r4=y;
        2:r2=x; 2:r4=y;
        3:r2=x; 3:r4=y;
}
P0           | P1           | P2           | P3           ;
stw r1,0(r2) | stw r1,0(r4) | lwz r3,0(r2) | lwz r3,0(r4) ;
addi r1,r1,1 | addi r1,r1,1 | sync         | sync         ;
stw r1,0(r2) | stw r1,0(r4) | lwz r5,0(r4) | lwz r5,0(r2) ;
addi r1,r1,1 | addi r1,r1,1 |              |              ;
stw r1,0(r2) | stw r1,0(r4) |              |              ;
addi r1,r1,1 | addi r1,r1,1 |              |              ;
stw r1,0(r2) | stw r1,0(r4) |              |              ;
addi r1,r1,1 | addi r1,r1,1 |              |              ;
stw r1,0(r2) | stw r1,0(r4) |              |              ;

exists
(2:r3=1 /\ 2:r5=0 /\ 3:r3=1 /\ 3:r5=0)
\end{VerbatimL}
\end{fcvlabel}
\caption{Expanded IRIW Litmus Test}
\label{lst:formal:Expanded IRIW Litmus Test}
\end{listing}

그러나, PPCMEM 은 IRIW 를 푸는데 14 CPU 시간을 소요하는데 반해 \co{herd} 는 17
밀리세컨드만에 이를 완료하여, 백만배 이상의 속도향상을 보입니다.
그렇다고는 하나, 이 문제는 근본적으로 지수적이어서 우린 \co{herd} 가 큰 문제에
있어서는 지수적으로 속도하락을 보일 것이라 예상할 수 있습니다.
그리고 이게 실제로 일어나는 일인데, 예를 들어,
Listing~\ref{lst:formal:Expanded IRIW Litmus Test} 에 보인 것과 같이 쓰기 CPU
마다 네개의 쓰기를 더하면, \co{herd} 는 50,000 배 이상 느려져서 15 \emph{분} 이 넘는
CPU 시간을 필요로 합니다.
쓰레드를 추가하는 것 역시 폭발적인 속도 하락을
초래합니다~\cite{PaulEMcKenney2014weakaxiom}.

\iffalse

Howeved, where PPCMEM requires 14 CPU hours to solve IRIW, \co{herd} does so
in 17 milliseconds, which represents a speedup of more than six orders
of magnitude.
That said, the problem is exponential in nature, so we should expect
\co{herd} to exhibit exponential slowdowns for larger problems.
And this is exactly what happens, for example, if we add four more writes
per writing CPU as shown in
Listing~\ref{lst:formal:Expanded IRIW Litmus Test},
\co{herd} slows down by a factor of more than 50,000, requiring more than
15 \emph{minutes} of CPU time.
Adding threads also results in exponential
slowdowns~\cite{PaulEMcKenney2014weakaxiom}.

\fi

그 폭발적 본성에도 불구하고, PPCMEM 과 \co{herd} 는 모두 x86 시스템에서의
queued-lock handoff 를 포함한 핵심 병렬 알고리즘을 검사하는데 상당히 유용함이
증명되었습니다.
\co{herd} 의 약점은 Section~\ref{sec:formal:PPCMEM Discussion} 에서 보인 PPCMEM
의 그것과 비슷합니다.
PPCMEM 과 \co{herd} 가 다른 결과를 내는 눈에 잘 안띄는 (그러나 분명 존재하는)
경우들이 있으며, 2021년 기준으로 이런 서로 다른 결과의 문제들 중 전부는
아니지만 많은 것들이 해결되었습니다.

이 리트머스 테스트들이
(Listing~\ref{lst:formal:PPCMEM Litmus Test} 처럼).
어셈블리어가 아닌
(Listing~\ref{lst:formal:Meaning of PPCMEM Litmus Test} 처럼)
C 로 쓰여질 수 있다면 도움이 될 겁니다.
이는 이제 가능한데, 다음 섹션에서 이를 다룹니다.

\iffalse

Despite their exponential nature, both PPCMEM and \co{herd} have proven quite
useful for checking key parallel algorithms, including the queued-lock
handoff on x86 systems.
The weaknesses of the \co{herd} tool are similar to those of PPCMEM, which
were described in
Section~\ref{sec:formal:PPCMEM Discussion}.
There are some obscure (but very real) cases for which the PPCMEM and
\co{herd} tools disagree, and as of 2021 many but not all of these disagreements
was resolved.

It would be helpful if the litmus tests could be written in C
(as in Listing~\ref{lst:formal:Meaning of PPCMEM Litmus Test})
rather than assembly
(as in Listing~\ref{lst:formal:PPCMEM Litmus Test}).
This is now possible, as will be described in the following sections.

\fi

\subsection{Axiomatic Approaches and Locking}
\label{sec:formal:Axiomatic Approaches and Locking}

Axiomatic approaches may also be applied to higher-level
languages and also to higher-level synchronization primitives, as
exemplified by the lock-based litmus test shown in
Listing~\ref{lst:formal:Locking Example} (\path{C-Lock1.litmus}).
This litmus test can be modeled by the Linux kernel memory model
(LKMM)~\cite{Alglave:2018:FSC:3173162.3177156,LucMaranget2018lock.cat}.
As expected, the \co{herd} tool's output features the string \co{Never},
correctly indicating that \co{P1()} cannot see \co{x} having a value
of one.\footnote{
	The output of the \co{herd} tool is compatible with that
	of PPCMEM, so feel free to look at
	Listings~\ref{lst:formal:PPCMEM Detects an Error}
	and~\ref{lst:formal:PPCMEM on Repaired Litmus Test}
	for examples showing the output format.}

\begin{listing}[tb]
\input{CodeSamples/formal/herd/C-Lock1@whole.fcv}
\caption{Locking Example}
\label{lst:formal:Locking Example}
\end{listing}

\QuickQuiz{
	What do you have to do to run \co{herd} on litmus tests like
	that shown in Listing~\ref{lst:formal:Locking Example}?
}\QuickQuizAnswer{
	Get version v4.17 (or later) of the Linux-kernel source code,
	then follow the instructions in \path{tools/memory-model/README}
	to install the needed tools.
	Then follow the further instructions to run these tools on the
	litmus test of your choice.
}\QuickQuizEnd

\begin{listing}[tb]
\input{CodeSamples/formal/herd/C-Lock2@whole.fcv}
\caption{Broken Locking Example}
\label{lst:formal:Broken Locking Example}
\end{listing}

Of course, if \co{P0()} and \co{P1()} use different locks, as shown in
Listing~\ref{lst:formal:Broken Locking Example} (\path{C-Lock2.litmus}),
then all bets are off.
And in this case, the \co{herd} tool's output features the string
\co{Sometimes}, correctly indicating that use of different locks allows
\co{P1()} to see \co{x} having a value of one.

\QuickQuiz{
	Why bother modeling locking directly?
	Why not simply emulate locking with atomic operations?
}\QuickQuizAnswer{
	In a word, performance, as can be seen in
	Table~\ref{tab:formal:Locking: Modeling vs. Emulation Time (s)}.
	The first column shows the number of \co{herd} processes modeled.
	The second column shows the \co{herd} runtime when modeling
	\co{spin_lock()} and \co{spin_unlock()} directly in \co{herd}'s
	cat language.
	The third column shows the \co{herd} runtime when emulating
	\co{spin_lock()} with \co{cmpxchg_acquire()} and \co{spin_unlock()}
	with \co{smp_store_release()}, using the \co{herd} \co{filter}
	clause to reject executions that fail to acquire the lock.
	The fourth column is like the third, but using \co{xchg_acquire()}
	instead of \co{cmpxchg_acquire()}.
	The fifth and sixth columns are like the third and fourth,
	but instead using the \co{herd} \co{exists} clause to reject
	executions that fail to acquire the lock.

\begin{table}[tb]
\rowcolors{10}{}{lightgray}
\small
\centering
\newcommand{\lockfml}[1]{\multicolumn{1}{c}{\begin{picture}(6,8)(0,0)\rotatebox{90}{#1}\end{picture}}}
\begin{tabular}{rrrrrr}
	\toprule
	& Model & \multicolumn{4}{c}{Emulate} \\
	\cmidrule(l){2-2} \cmidrule(l){3-6}
	& & \multicolumn{2}{c}{\tco{filter}} & \multicolumn{2}{c}{\tco{exists}} \\
	\cmidrule(l){3-4} \cmidrule(l){5-6}
	\lockfml{\# Proc.}
	&
	& \tco{cmpxchg}
	& \tco{xchg}
	& \tco{cmpxchg}
	& \tco{xchg}
	\\
	\cmidrule{1-1} \cmidrule(l){2-2} \cmidrule(l){3-4} \cmidrule(l){5-6}
	2 & 0.004 &  0.022 &  0.027 &   0.039 &   0.058 \\
	3 & 0.041 &  0.743 &  0.968 &   1.653 &   3.203 \\
	4 & 0.374 & 59.565 & 74.818 & 151.962 & 500.960 \\
	5 & 4.905 \\
	\bottomrule
\end{tabular}
\caption{Locking: Modeling vs. Emulation Time (s)}
\label{tab:formal:Locking: Modeling vs. Emulation Time (s)}
\end{table}

	Note also that use of the \co{filter} clause is about twice
	as fast as is use of the \co{exists} clause.
	This is no surprise because the \co{filter} clause allows
	early abandoning of excluded executions, where the executions
	that are excluded are the ones in which the lock is concurrently
	held by more than one process.

	More important, modeling \co{spin_lock()} and \co{spin_unlock()}
	directly ranges from five times faster to more than two orders
	of magnitude faster than modeling emulated locking.
	This should also be no surprise, as direct modeling raises
	the level of abstraction, thus reducing the number of events
	that \co{herd} must model.
	Because almost everything that \co{herd} does is of exponential
	computational complexity, modest reductions in the number of
	events produces exponentially large reductions in runtime.

	Thus, in formal verification even more than in parallel
	programming itself, divide and conquer!!!
}\QuickQuizEnd

But locking is not the only synchronization primitive that can be
modeled directly:
The next section looks at RCU\@.

\subsection{Axiomatic Approaches and RCU}
\label{sec:formal:Axiomatic Approaches and RCU}

\begin{fcvref}[ln:formal:C-RCU-remove:whole]
Axiomatic approaches can also analyze litmus tests involving
RCU~\cite{Alglave:2018:FSC:3173162.3177156}.
To that end,
Listing~\ref{lst:formal:Canonical RCU Removal Litmus Test}
(\path{C-RCU-remove.litmus})
shows a litmus test corresponding to the canonical RCU-mediated
removal from a linked list.
As with the locking litmus test, this RCU litmus test can be
modeled by LKMM, with similar performance advantages compared
to modeling emulations of RCU\@.
Line~\lnref{head} shows \co{x} as the list head, initially
referencing \co{y}, which in turn is initialized to the value
\co{2} on line~\lnref{tail:1}.

\begin{listing}[tb]
\input{CodeSamples/formal/herd/C-RCU-remove@whole.fcv}
\caption{Canonical RCU Removal Litmus Test}
\label{lst:formal:Canonical RCU Removal Litmus Test}
\end{listing}

\co{P0()} on \clnrefrange{P0start}{P0end}
removes element \co{y} from the list by replacing it with element \co{z}
(line~\lnref{assignnewtail}),
waits for a grace period (line~\lnref{sync}),
and finally zeroes \co{y} to emulate \co{free()} (line~\lnref{free}).
\co{P1()} on \clnrefrange{P1start}{P1end}
executes within an RCU read-side critical section
(\clnrefrange{rl}{rul}),
picking up the list head (line~\lnref{rderef}) and then
loading the next element (line~\lnref{read}).
The next element should be non-zero, that is, not yet freed
(line~\lnref{exists_}).
Several other variables are output for debugging purposes
(line~\lnref{locations_}).

The output of the \co{herd} tool when running this litmus test features
\co{Never}, indicating that \co{P0()} never accesses a freed element,
as expected.
Also as expected, removing line~\lnref{sync} results in \co{P0()}
accessing a freed element, as indicated by the \co{Sometimes} in
the \co{herd} output.
\end{fcvref}

\begin{listing}[tb]
\input{CodeSamples/formal/herd/C-RomanPenyaev-list-rcu-rr@whole.fcv}
\caption{Complex RCU Litmus Test}
\label{lst:formal:Complex RCU Litmus Test}
\end{listing}

\begin{fcvref}[ln:formal:C-RomanPenyaev-list-rcu-rr:whole]
A litmus test for a more complex example proposed by
\ppl{Roman}{Penyaev}~\cite{RomanPenyaev2018rrRCU} is shown in
Listing~\ref{lst:formal:Complex RCU Litmus Test}
(\path{C-RomanPenyaev-list-rcu-rr.litmus}).
In this example, readers (modeled by \co{P0()} on
\clnrefrange{P0start}{P0end}) access a linked list
in a round-robin fashion by ``leaking'' a pointer to the last
list element accessed into variable \co{c}.
Updaters (modeled by \co{P1()} on \clnrefrange{P1start}{P1end})
remove an element, taking care to avoid disrupting current or future
readers.

\QuickQuiz{
	Wait!!!
	Isn't leaking pointers out of an RCU read-side critical
	section a critical bug???
}\QuickQuizAnswer{
	Yes, it usually is a critical bug.
	However, in this case, the updater has been cleverly constructed
	to properly handle such pointer leaks.
	But please don't make a habit of doing this sort of thing, and
	especially don't do this without having put a lot of thought
	into making some more conventional approach work.
}\QuickQuizEnd

\Clnrefrange{listtail}{listhead} define the initial linked
list, tail first.
In the Linux kernel, this would be a doubly linked circular list,
but \co{herd} is currently incapable of modeling such a beast.
The strategy is instead to use a singly linked linear list that
is long enough that the end is never reached.
Line~\lnref{rrcache} defines variable \co{c}, which is used to
cache the list pointer between successive RCU read-side critical
sections.

Again, \co{P0()} on \clnrefrange{P0start}{P0end} models readers.
This process models a pair of successive readers traversing round-robin
through the list, with the first reader on \clnrefrange{rl1}{rul1}
and the second reader on \clnrefrange{rl2}{rul2}.
Line~\lnref{rdcache} fetches the pointer cached in \co{c}, and if
line~\lnref{rdckcache} sees that the pointer was \co{NULL},
line~\lnref{rdinitcache} restarts at the beginning of the list.
In either case, line~\lnref{rdnext} advances to the next list element,
and line~\lnref{rdupdcache} stores a pointer to this element back into
variable \co{c}.
\Clnrefrange{rl2}{rul2} repeat this process, but using
registers \co{r3} and \co{r4} instead of \co{r1} and \co{r2}.
As with
Listing~\ref{lst:formal:Canonical RCU Removal Litmus Test},
this litmus test stores zero to emulate \co{free()}, so
line~\lnref{exists_} checks for any of these four registers being
\co{NULL}, also known as zero.

Because \co{P0()} leaks an RCU-protected pointer from its first
RCU read-side critical section to its second, \co{P1()} must carry
out its update (removing \co{x}) very carefully.
Line~\lnref{updremove} removes \co{x} by linking \co{w} to \co{y}.
Line~\lnref{updsync1} waits for readers, after which no subsequent reader
has a path to \co{x} via the linked list.
Line~\lnref{updrdcache} fetches \co{c}, and if line~\lnref{updckcache}
determines that \co{c} references the newly removed \co{x},
line~\lnref{updinitcache} sets \co{c} to \co{NULL}
and line~\lnref{updsync2} again waits for readers, after which no
subsequent reader can fetch \co{x} from \co{c}.
In either case, line~\lnref{updfree} emulates \co{free()} by storing
zero to \co{x}.

\QuickQuiz{
	\begin{fcvref}[ln:formal:C-RomanPenyaev-list-rcu-rr:whole]
	In Listing~\ref{lst:formal:Complex RCU Litmus Test},
	why couldn't a reader fetch \co{c} just before \co{P1()}
	zeroed it on line~\lnref{updinitcache}, and then later
	store this same value back into \co{c} just after it was
	zeroed, thus defeating the zeroing operation?
	\end{fcvref}
}\QuickQuizAnswer{
	\begin{fcvref}[ln:formal:C-RomanPenyaev-list-rcu-rr:whole]
	Because the reader advances to the next element on
	line~\lnref{rdnext}, thus avoiding storing a pointer to the
	same element as was fetched.
	\end{fcvref}
}\QuickQuizEnd

The output of the \co{herd} tool when running this litmus test features
\co{Never}, indicating that \co{P0()} never accesses a freed element,
as expected.
Also as expected, removing either \co{synchronize_rcu()} results
in \co{P1()} accessing a freed element, as indicated by \co{Sometimes}
in the \co{herd} output.
\end{fcvref}

\QuickQuizSeries{%
\QuickQuizB{
	\begin{fcvref}[ln:formal:C-RomanPenyaev-list-rcu-rr:whole]
	In Listing~\ref{lst:formal:Complex RCU Litmus Test},
	why not have just one call to \co{synchronize_rcu()}
	immediately before line~\lnref{updfree}?
	\end{fcvref}
}\QuickQuizAnswerB{
	\begin{fcvref}[ln:formal:C-RomanPenyaev-list-rcu-rr:whole]
	Because this results in \co{P0()} accessing a freed element.
	But don't take my word for this, try it out in \co{herd}!
	\end{fcvref}
}\QuickQuizEndB
%
\QuickQuizE{
	\begin{fcvref}[ln:formal:C-RomanPenyaev-list-rcu-rr:whole]
	Also in Listing~\ref{lst:formal:Complex RCU Litmus Test},
	can't line~\lnref{updfree} be \co{WRITE_ONCE()} instead
	of \co{smp_store_release()}?
	\end{fcvref}
}\QuickQuizAnswerE{
	\begin{fcvref}[ln:formal:C-RomanPenyaev-list-rcu-rr:whole]
	That is an excellent question.
	As of late 2021, the answer is ``no one knows''.
	Much depends on the semantics of \ARMv8's conditional-move
	instruction.
	While awaiting clarity on these semantics, \co{smp_store_release()}
	is the safe choice.
	\end{fcvref}
}\QuickQuizEndE
}

These sections have shown how axiomatic approaches can successfully
model synchronization primitives such as locking and RCU in C-language
litmus tests.
Longer term, the hope is that the axiomatic approaches will model
even higher-level software artifacts, producing exponential
verification speedups.
This could potentially allow axiomatic verification of much larger
software systems.
Another alternative is to press the axioms of boolean logic into service,
as described in the next section.
