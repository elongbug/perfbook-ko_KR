% ppcmem.tex

\section{Special-Purpose State-Space Search}
\label{sec:formal:Special-Purpose State-Space Search}
%
\epigraph{Jack of all trades, master of none.}{\emph{Unknown}}

Promela 와 spin 이 어떤 (작은) 알고리즘이든 굉장히 많은 부분을 검증할 수 있게
해주긴 하지만, 그것들의 커다란 범용성은 어떨때에는 문제가 될수도 있습니다.
예를 들어, Promela 는 메모리 모델들이나 재배치 의미들을 이해하지 못합니다.
따라서 이 섹션은 제품화된 시스템들에서 사용되느 메모리 모델을 이해해서 완화된
순서 규칙의 코드의 검증을 단순화 시켜주는 상태 공간 탐색 도구들을 설명합니다.

예를 들어,
Section~\ref{sec:formal:Promela Example: QRCU}
는 완화된 메모리 순서 규칙을 처리하기 위해 Promela 를 어떻게 납득시켜야
하는지를 보았습니다.
이런 방법도 잘 동작할수 있지만, 이는 개발자가 해당 시스템의 메모리 모델을
이해하고 있을 것을 필요로 합니다.
안타깝게도, (존재한다 쳐도) 일부 개발자들만이 현대 CPU 들의 복잡한 메모리
모델을 완전하게 이해하고 있습니다.
\iffalse

Although Promela and spin allow you to verify pretty much any (smallish)
algorithm, their very generality can sometimes be a curse.
For example, Promela does not understand memory models or any sort
of reordering semantics.
This section therefore describes some state-space search tools that
understand memory models used by production systems, greatly simplifying the
verification of weakly ordered code.

For example,
Section~\ref{sec:formal:Promela Example: QRCU}
showed how to convince Promela to account for weak memory ordering.
Although this approach can work well, it requires that the developer
fully understand the system's memory model.
Unfortunately, few (if any) developers fully understand the complex
memory models of modern CPUs.
\fi

따라서, 또다른 방법은
Cambridge 대학의 Peter Sewell 과 Susmit Sarkar, INRIA 의 Luc Maranget,
Francesco Zappa Nardelli, 그리고 Pankaj Pawan, 그리고 Oxford 대학의 Jade
Alglave 가 IBM 의 Derek Williams 와 협력해서 만든
PPCMEM~\cite{JadeAlglave2011ppcmem} 과 같은, 이미 이 메모리 순서 규칙을
이해하고 있는 도구들을 활용하는 것입니다.
이 그룹은 Power, ARM, x86 뿐만 아니라 C/C++11 표준~\cite{PeteBecker2011N3242}
의 메모리 모델을 형식화 시키고 Power 와 ARM 형식에 기초하여 PPCMEM 도구를
만들었습니다.
\iffalse

Therefore, another approach is to use a tool that already understands
this memory ordering, such as the PPCMEM tool produced by
Peter Sewell and Susmit Sarkar at the University of Cambridge, Luc
Maranget, Francesco Zappa Nardelli, and Pankaj Pawan at INRIA, and Jade
Alglave at Oxford University, in cooperation with Derek Williams of
IBM~\cite{JadeAlglave2011ppcmem}.
This group formalized the memory models of Power, ARM, x86, as well
as that of the C/C++11 standard~\cite{PeteBecker2011N3242}, and
produced the PPCMEM tool based on the Power and ARM formalizations.
\fi

\QuickQuiz{}
	하지만 x86 은 강한 메모리 순서 규칙을 가지고 있어요!
	왜 이 메모리 모델을 형식화 시켜야 하는거죠?
	\iffalse

	But x86 has strong memory ordering!  Why would you need to
	formalize its memory model?
	\fi
\QuickQuizAnswer{
	사실, 학계에서는 x86 메모리 모델을 완화된 것으로 여기는데, 앞의 store
	가 뒤의 load 와 재배치될 수 있기 때문입니다.
	학계의 시점에서 볼때, 강한 메모리 모델은 어떠한 재배치도 절대적으로
	허용하지 않아서 모든 쓰레드들이 그들에게 보일 수 있는 모든
	오퍼레이션들의 순서에 대해 동의할 수 있는 것입니다.
	\iffalse

	Actually, academics consider the x86 memory model to be weak
	because it can allow prior stores to be reordered with
	subsequent loads.
	From an academic viewpoint, a strong memory model is one
	that allows absolutely no reordering, so that all threads
	agree on the order of all operations visible to them.
	\fi
} \QuickQuizEnd

PPCMEM 도구는 입력으로 \emph{리트머스 테스트} 를 받습니다.
예제 리트머스 테스트가
Section~\ref{sec:formal:Anatomy of a Litmus Test} 에 있습니다.
Section~\ref{sec:formal:What Does This Litmus Test Mean?} 은 이 리트머스
테스트를 동일한 C-언어 프로그램으로 관계지어 보고,
Section~\ref{sec:formal:Running a Litmus Test} 는 PPCMEM 을 이 리트머스
테스트에 어떻게 적용하는지를 설명하며,
Section~\ref{sec:formal:PPCMEM Discussion} 이로부터의 의미를 이야기해 봅니다.
\iffalse

The PPCMEM tool takes \emph{litmus tests} as input.
A sample litmus test is presented in
Section~\ref{sec:formal:Anatomy of a Litmus Test}.
Section~\ref{sec:formal:What Does This Litmus Test Mean?}
relates this litmus test to the equivalent C-language program,
Section~\ref{sec:formal:Running a Litmus Test} describes how to
apply PPCMEM to this litmus test, and
Section~\ref{sec:formal:PPCMEM Discussion}
discusses the implications.
\fi

\subsection{Anatomy of a Litmus Test}
\label{sec:formal:Anatomy of a Litmus Test}

\begin{listing}[tbp]
{ \scriptsize
\begin{verbbox}
 1 PPC SB+lwsync-RMW-lwsync+isync-simple
 2 ""
 3 {
 4 0:r2=x; 0:r3=2; 0:r4=y; 0:r10=0; 0:r11=0; 0:r12=z;
 5 1:r2=y; 1:r4=x;
 6 }
 7  P0                 | P1           ;
 8  li r1,1            | li r1,1      ;
 9  stw r1,0(r2)       | stw r1,0(r2) ;
10  lwsync             | sync         ;
11                     | lwz r3,0(r4) ;
12  lwarx  r11,r10,r12 | ;
13  stwcx. r11,r10,r12 | ;
14  bne Fail1          | ;
15  isync              | ;
16  lwz r3,0(r4)       | ;
17  Fail1:             | ;
18 
19 exists
20 (0:r3=0 /\ 1:r3=0)
\end{verbbox}
}
\centering
\theverbbox
\caption{PPCMEM Litmus Test}
\label{lst:formal:PPCMEM Litmus Test}
\end{listing}

PPCMEM 을 위한 예제 PowerPC 리트머스 테스트가
Listing~\ref{lst:sec:formal:PPCMEM Litmus Test}
에 보여져 있습니다.
ARM 인터페이스도 정확히 같은 방식으로 동작합니다만, 그러기 위해선 첫줄의
``PPC'' 가 ``ARM'' 으로 바뀌어야 하고 Power 인스트럭션(instruction) 이 ARM
인스트럭션으로 대체되어야 합니다.
뒤에서 이야기될 웹페이지의 ``Change to ARM model'' 을 클릭하는 것으로 ARM
인터페이스를 선택할 수 있습니다.

예제에서, line~1 은 시스템의 종류 (``ARM'' 또는 ``PPC'') 를 밝히고 모델의
이름을 포함합니다.
Line~2 는 테스트의 대안적 이름을 위한 공간을 제공하는데, 일반적으로는 앞의
예제에서 보인 것처럼 빈줄로 남겨두게 될 겁니다.
주석은 line~2 와 3 사이에 Ocaml (또는 Pascal) 문법의 \nbco{(* *)} 를 이용해
들어갈 수 있습니다.
\iffalse

An example PowerPC litmus test for PPCMEM is shown in
Listing~\ref{lst:formal:PPCMEM Litmus Test}.
The ARM interface works exactly the same way, but with ARM instructions
substituted for the Power instructions and with the initial ``PPC''
replaced by ``ARM''. You can select the ARM interface by clicking on
``Change to ARM Model'' at the web page called out above.

In the example, line~1 identifies the type of system (``ARM'' or ``PPC'')
and contains the title for the model. Line~2 provides a place for an
alternative name for the test, which you will usually want to leave
blank as shown in the above example. Comments can be inserted between
lines~2 and~3 using the OCaml (or Pascal) syntax of \nbco{(* *)}.
\fi

Line~3-6 는 모든 레지스터들의 초기 값들을 제공합니다; 각각은 \co{P:R=V} 의
형태로, \co{P} 는 프로세스 식별자이고, \co{R} 은 레지스터 식별자이며, \co{V} 는
값입니다.
예를 들어, 프로세스 0의 레지스터 r3 는 초기에 2라는 값을 담고 있습니다.
만약 값이 변수 (예제의 \co{x}, \co{y}, 또는 \co{z}) 라면 레지스터는 해당 변수의
주소로 초기화 됩니다.
변수의 값을 초기화 시키는 것도 가능한데, 예를 들어 \co{x=1} 은 \co{x} 의 값을
1로 초기화 시킵니다.
초기화 되지않은 변수들은 기본값으로 0을 갖게 되어서, 이 예제에서는 \co{x},
\co{y}, 그리고 \co{z} 는 모두 초기에 0의 값을 갖습니다.

Line~7 은 두 프로세스를 위한 식별자를 제공하므로, line~4 의 \co{0:r3=2} 는
\co{P0:r3=2} 로 대신 쓰여질 수 있습니다.
Line~7 은 있어야만 하며, 식별자들은 \co{Pn} 의 형태여야만 하며, \co{n} 은 열
수로, 가장 왼쪽의 열을 0으로 세는 것부터 시작합니다.
이는 불필요하게 엄정한 것으로 보여질 수도 있겠습니다만, 이는 실제 사용에 있어
많은 혼란을 방지해 줍니다.
\iffalse

Lines~3-6 give initial values for all registers; each is of the form
\co{P:R=V}, where \co{P} is the process identifier, \co{R} is the register
identifier, and \co{V} is the value. For example, process 0's register
r3 initially contains the value 2. If the value is a variable (\co{x},
\co{y}, or \co{z} in the example) then the register is initialized to the
address of the variable. It is also possible to initialize the contents
of variables, for example, \co{x=1} initializes the value of \co{x} to
1. Uninitialized variables default to the value zero, so that in the
example, \co{x}, \co{y}, and \co{z} are all initially zero.

Line~7 provides identifiers for the two processes, so that the \co{0:r3=2}
on line~4 could instead have been written \co{P0:r3=2}. Line~7 is
required, and the identifiers must be of the form \co{Pn}, where \co{n}
is the column number, starting from zero for the left-most column. This
may seem unnecessarily strict, but it does prevent considerable confusion
in actual use.
\fi

\QuickQuiz{}
	Listing~\ref{lst:sec:formal:PPCMEM Litmus Test}
	의 line~8 은 왜 레지스터들을 초기화 시키죠?
	왜 그것들을 line~4 와 5 에서 대신 초기화 시키지 않나요?
	\iffalse

	Why does line~8
	of Listing~\ref{lst:formal:PPCMEM Litmus Test}
	initialize the registers?
	Why not instead initialize them on lines~4 and~5?
	\fi
\QuickQuizAnswer{
	두 방식 모두 잘 동작합니다.
	하지만, 일반적으로는, 명시적 인스트럭션 사용보다는 초기화를 사용하는게
	낫습니다.
	이 예제에서의 명시적 인스트럭션들은 그 사용법을 보이기 위해
	사용되었습니다.
	또한, 이 도구의 웹사이트 (\url{http://www.cl.cam.ac.uk/~pes20/ppcmem/})
	에서 사용 가능한 리트머스 테스트들 가운데 많은 것들은 명시적 초기화
	인스트럭션들을 사용하도록 자동으로 생성되었습니다.
	\iffalse

	Either way works.
	However, in general, it is better to use initialization than
	explicit instructions.
	The explicit instructions are used in this example to demonstrate
	their use.
	In addition, many of the litmus tests available on the tool's
	web site (\url{http://www.cl.cam.ac.uk/~pes20/ppcmem/}) were
	automatically generated, which generates explicit
	initialization instructions.
	\fi
} \QuickQuizEnd

Line~8-17 은 각각의 프로세스를 위한 코드들입니다.
프로세스는 P0 의 line~11 과 P1 의 line~12-17 처럼 빈 줄을 가질 수도 있습니다.
라벨과 브랜치들 역시 사용 가능한데, line~14 의 브랜치와 line~17 의 라벨로
사용이 보여져 있습니다.
그렇다곤 하나, 너무 자유로운 브랜치 사용은 상태 공간의 크기를 폭증시킬 수
있습니다.
루프의 사용은 특히나 상태 공간의 크기를 폭증시키기 위한 좋은 방법입니다.

Line~19-20 은 단정문을 보이는데, 이 경우에는 두 쓰레드가 모두 실행을 완료한
후에 P0 의, 그리고 P1 의 r3 레지스터들이 모두 0을 가지고 있을 수 있는지에
우리가 관심을 갖고 있음을 보입니다.
먄약 P0 와 P1 이 모두 각각의 r3 레지스터에서 0을 보게 된다면 안타깝게도 실패할
수 있는 수많은 경우들이 존재할 수 있기에 이 단정문은 중요합니다.
\iffalse

Lines~8-17 are the lines of code for each process. A given process
can have empty lines, as is the case for P0's line~11 and P1's
lines~12-17.
Labels and branches are permitted, as demonstrated by the branch
on line~14 to the label on line~17. That said, too-free use of branches
will expand the state space. Use of loops is a particularly good way to
explode your state space.

Lines~19-20 show the assertion, which in this case indicates that we
are interested in whether P0's and P1's r3 registers can both contain
zero after both threads complete execution. This assertion is important
because there are a number of use cases that would fail miserably if
both P0 and P1 saw zero in their respective r3 registers.
\fi

이것만으로도 간단한 리트머스 테스트들을 구성하는데 충분한 정보가 될 겁니다.
몇가지 추가적인 문서화가 가능합니다만, 이런 추가적인 문서화들은 대부분 테스트를
실제 하드웨어 위에서 수행하기 위한 다른 연구용 도구를 위한 것입니다.
어쩌면 더 중요한 것은, 많은 수의 이미 존재하는 리트머스 테스트들은 온라인
도구를 통해서도 (``Select ARM Test'' 와 ``Select POWER Test'' 벝느들을 통해서)
사용 가능하다는 것입니다.
이런 미리 존재하는 리트머스 테스트들 가운데 하나가 당신의 Power 나 ARM 메모리
순서 규칙에 대한 질문에 대한 답이 될수도 있을 겁니다.
\iffalse

This should give you enough information to construct simple litmus
tests. Some additional documentation is available, though much of this
additional documentation is intended for a different research tool that
runs tests on actual hardware. Perhaps more importantly, a large number of
pre-existing litmus tests are available with the online tool (available
via the ``Select ARM Test'' and ``Select POWER Test'' buttons). It is
quite likely that one of these pre-existing litmus tests will answer
your Power or ARM memory-ordering question.
\fi

\subsection{What Does This Litmus Test Mean?}
\label{sec:formal:What Does This Litmus Test Mean?}

P0 의 line~8 과 9 는 C 언어의 \co{x=1} 과 같은데, line~4 에서 P0 의 레지스터
\co{r2} 가 \co{x} 의 주소를 갖도록 정의했기 때문입니다.
P0 의 line~12 와 13 은 각각 load-linked (ARM 용어로는 ``load register
exclusive'' 이고 Power 용어로는 ``load reserve'') 와 store-conditional (ARM
용어로는 ``store register exclusive'') 입니다.
이것들이 함께 사용되면, x86 에서는 \co{lock;cmpxchg} 인스트럭션으로 표현되는
compare-and-swap 시퀀스와 유사한 어토믹 인스트럭션을 형성하게 됩니다.
더 높은 차원의 추상화 단계로 이야기 하면, line~10-15 의 시퀀스는 리눅스 커널의
\co{atomic_add_return(&z, o)} 와 같습니다.
마지막으로, line~16 은 C 언어의 \co{r3=y} 와 같습니다.

P1 의 line~8 과 9 는 C 언어의 \co{y=1} 과 같으며, line~10 은 리눅스 커널의
\co{smp_mb()} 와 동일한 메모리 배리어이며, line~11 은 C 언어의 \co{r3=x} 와
같습니다.
\iffalse

P0's lines~8 and~9 are equivalent to the C statement \co{x=1} because
line~4 defines P0's register \co{r2} to be the address of \co{x}. P0's
lines~12 and~13 are the mnemonics for load-linked (``load register
exclusive'' in ARM parlance and ``load reserve'' in Power parlance)
and store-conditional (``store register exclusive'' in ARM parlance),
respectively. When these are used together, they form an atomic
instruction sequence, roughly similar to the compare-and-swap sequences
exemplified by the x86 \co{lock;cmpxchg} instruction. Moving to a higher
level of abstraction, the sequence from lines~10-15 is equivalent to
the Linux kernel's \co{atomic_add_return(&z, 0)}. Finally, line~16 is
roughly equivalent to the C statement \co{r3=y}.

P1's lines~8 and~9 are equivalent to the C statement \co{y=1}, line~10
is a memory barrier, equivalent to the Linux kernel statement \co{smp_mb()},
and line~11 is equivalent to the C statement \co{r3=x}.
\fi

\QuickQuiz{}
	Listing~\ref{lst:sec:formal:PPCMEM Litmus Test} 의 line~17 의 \co{Fail:}
	라벨에서는 무슨 일이 벌어지게 되는거죠?
	\iffalse

	But whatever happened to line~17 of
	Listing~\ref{lst:formal:PPCMEM Litmus Test},
	the one that is the \co{Fail:} label?
	\fi
\QuickQuizAnswer{
	\co{atomic_add_return()} 의 Powerpc 버전 구현은 \co{stwcx} 인스트럭션이
	condition-code 레지스터에 있는 0이 아닌 상태값을 설정함으로써 통신을
	하며 \co{bne} 인스트럭션을 통해 테스트 하게 되는 성공 여부가 실패로
	드러나면 루프를 돌게 됩니다.
	실제 루프를 모델링하게 되는 것은 상태 공간의 폭증을 유발하게 되므로,
	그대신에 우리는 Fail: 라벨로의 브랜치를 하고, P0 의 \co{r3}
	레지스터의 값이 초기값 2인 형태로 모델링을 종료하여서, exists 단정문의
	실패를 유발하지 않게 합니다.

	이 트릭이 일반적으로 적용 가능한 것인지에 대해서는 일부 논쟁이
	있습니다만, 전 아직 이게 실패하는 예제를 본적이 없습니다.
	\iffalse

	The implementation of powerpc version of \co{atomic_add_return()}
	loops when the \co{stwcx} instruction fails, which it communicates
	by setting non-zero status in the condition-code register,
	which in turn is tested by the \co{bne} instruction. Because actually
	modeling the loop would result in state-space explosion, we
	instead branch to the Fail: label, terminating the model with
	the initial value of 2 in P0's \co{r3} register, which
	will not trigger the exists assertion.

	There is some debate about whether this trick is universally
	applicable, but I have not seen an example where it fails.
	\fi
} \QuickQuizEnd

\begin{listing}[tbp]
{ \scriptsize
\begin{verbbox}
 1 void P0(void)
 2 {
 3   int r3;
 4 
 5   x = 1; /* Lines 8 and 9 */
 6   atomic_add_return(&z, 0); /* Lines 10-15 */
 7   r3 = y; /* Line 16 */
 8 }
 9 
10 void P1(void)
11 {
12   int r3;
13 
14   y = 1; /* Lines 8-9 */
15   smp_mb(); /* Line 10 */
16   r3 = x; /* Line 11 */
17 }
\end{verbbox}
}
\centering
\theverbbox
\caption{Meaning of PPCMEM Litmus Test}
\label{lst:formal:Meaning of PPCMEM Litmus Test}
\end{listing}

이 모든 것들을 한자리에 모아서 만든, 전체 리트머스 테스트와 동이한 C-언어
코드가
Listing~\ref{lst:sec:formal:Meaning of PPCMEM Litmus Test}
에 보여져 있습니다.
핵심은, \co{atomic_add_return()} 이 (리눅스 커널의 요구사항이 그렇듯) 전체
메모리 배리어처럼 동작한다면, 실행이 완료된 후에 \co{P0()} 의, 그리고 \co{P1()}
의 \co{r3} 가 모두 0일 수는 없다는 것입니다.

다음 섹션은 이 리트머스 테스트를 어떻게 수행할 수 있는지 설명합니다.
\iffalse

Putting all this together, the C-language equivalent to the entire litmus
test is as shown in
Listing~\ref{lst:formal:Meaning of PPCMEM Litmus Test}.
The key point is that if \co{atomic_add_return()} acts as a full
memory barrier (as the Linux kernel requires it to), 
then it should be impossible for \co{P0()}'s and \co{P1()}'s \co{r3}
variables to both be zero after execution completes.

The next section describes how to run this litmus test.
\fi

\subsection{Running a Litmus Test}
\label{sec:formal:Running a Litmus Test}

리트머스 테스트들은
\url{http://www.cl.cam.ac.uk/~pes20/ppcmem/} 를 통해서 대화형으로 수행되어
메모리 모델에 대한 이해를 도울 수 있습니다.
하지만, 이 방법은 사용자가 직접 전체 상태 공간 탐색을 해야 할 것을 필요로
합니다.
모든 가능한 이벤트들의 시퀀스를 체크했다고 자신하긴 어려우므로, 이런 목적을
위한 별개의 도구가 제공됩니다~\cite{PaulEMcKenney2011ppcmem}.
\iffalse

Litmus tests may be run interactively via
\url{http://www.cl.cam.ac.uk/~pes20/ppcmem/}, which can help build an
understanding of the memory model.
However, this approach requires that the user manually carry out the
full state-space search.
Because it is very difficult to be sure that you have checked every
possible sequence of events, a separate tool is provided for this
purpose~\cite{PaulEMcKenney2011ppcmem}.
\fi

\begin{listing}[tbp]
{ \scriptsize
\begin{verbbox}
./ppcmem -model lwsync_read_block \
         -model coherence_points filename.litmus
...
States 6
0:r3=0; 1:r3=0;
0:r3=0; 1:r3=1;
0:r3=1; 1:r3=0;
0:r3=1; 1:r3=1;
0:r3=2; 1:r3=0;
0:r3=2; 1:r3=1;
Ok
Condition exists (0:r3=0 /\ 1:r3=0)
Hash=e2240ce2072a2610c034ccd4fc964e77
Observation SB+lwsync-RMW-lwsync+isync Sometimes 1
\end{verbbox}
}
\centering
\theverbbox
\caption{PPCMEM Detects an Error}
\label{lst:formal:PPCMEM Detects an Error}
\end{listing}

Listing~\ref{lst:sec:formal:PPCMEM Litmus Test}
에 보인 리트머스 테스트는 read-modify-write 인스트럭션들을 포함하고 있기
때문에, 커맨드 라인에 \co{-model} 인자를 추가해야만 합니다.
해당 리트머스 테스트가 \co{filename.litmus} 에 저장되어 있다면, 이는
Listing~\ref{lst:sec:formal:PPCMEM Detects an Error} 에 보여진 것과 같은 결과를
보일 것인데, 여기서 \co{...} 은 많은 양의 빌드 진행상의 출력을 대체하고
있습니다.
마지막 줄의 ``Sometimes'' 는 이를 증명합니다: 해당 단정문의 조건은 항상 그런건
아니지만 일부 수행에서는 일어날 수 있습니다.
\iffalse

Because the litmus test shown in
Listing~\ref{lst:formal:PPCMEM Litmus Test}
contains read-modify-write instructions, we must add \co{-model}
arguments to the command line.
If the litmus test is stored in \co{filename.litmus},
this will result in the output shown in
Listing~\ref{lst:formal:PPCMEM Detects an Error},
where the \co{...} stands for voluminous making-progress output.
The list of states includes \co{0:r3=0; 1:r3=0;}, indicating once again
that the old PowerPC implementation of \co{atomic_add_return()} does
not act as a full barrier.
The ``Sometimes'' on the last line confirms this: the assertion triggers
for some executions, but not all of the time.
\fi

\begin{listing}[tbp]
{ \scriptsize
\begin{verbbox}
./ppcmem -model lwsync_read_block \
         -model coherence_points filename.litmus
...
States 5
0:r3=0; 1:r3=1;
0:r3=1; 1:r3=0;
0:r3=1; 1:r3=1;
0:r3=2; 1:r3=0;
0:r3=2; 1:r3=1;
No (allowed not found)
Condition exists (0:r3=0 /\ 1:r3=0)
Hash=77dd723cda9981248ea4459fcdf6097d
Observation SB+lwsync-RMW-lwsync+sync Never 0 5
\end{verbbox}
}
\centering
\theverbbox
\caption{PPCMEM on Repaired Litmus Test}
\label{lst:formal:PPCMEM on Repaired Litmus Test}
\end{listing}

이 리눅스 커널 버그를 고치는 방법은 P0 의 \co{isync} 를 \co{sync} 로 교체하는
것으로, 이렇게 수정된 테스트는
Listing~\ref{lst:sec:formal:PPCMEM on Repaired Litmus Test} 에 보인 것과 같은
결과를 나오게 합니다.
볼 수 있듯, \co{0:r3=0; 1:r3=0;} 는 상태들의 리스트에 나타나 있지 않고, 마지막
줄은 ``Never'' 라고 이야기 합니다.
따라서, 해당 모델은 문제가 되는 실행 시퀀스는 일어날 수 없음을 예측합니다.
\iffalse

The fix to this Linux-kernel bug is to replace P0's \co{isync} with
\co{sync}, which results in the output shown in
Listing~\ref{lst:formal:PPCMEM on Repaired Litmus Test}.
As you can see, \co{0:r3=0; 1:r3=0;} does not appear in the list of states,
and the last line calls out ``Never''.
Therefore, the model predicts that the offending execution sequence
cannot happen.
\fi

\QuickQuiz{}
	ARM 리눅스 커널도 비슷한 버그를 가지고 있나요?
	\iffalse

	Does the ARM Linux kernel have a similar bug?
	\fi
\QuickQuizAnswer{
	ARM 에서 리눅스는 \co{atomic_add_return()} 함수의 어셈블리어 구현의
	앞과 뒤에 \co{smp_mb()} 를 넣어두기 때문에 이 특정한 버그는 ARM 에는
	존재하지 않습니다.
	PowerPC 는 더이상 이 버그를 가지고 있지 않습니다; 이 버그는 오래전에
	고쳐졌습니다.
	리눅스 커널이 가지고 있을수도 있는 또다른 버그들을 찾는 것은 독자
	여러분의 몫으로 남겨두겠습니다.
	\iffalse

	ARM does not have this particular bug because that it places
	\co{smp_mb()} before and after the \co{atomic_add_return()}
	function's assembly-language implementation.
	PowerPC no longer has this bug; it has long since been fixed.
	Finding any other bugs that the Linux kernel might have is left
	as an exercise for the reader.
	\fi
} \QuickQuizEnd

\subsection{PPCMEM Discussion}
\label{sec:formal:PPCMEM Discussion}

이 도구들은 ARM 과 Power 에서 돌아가는 낮은 단계의 병렬 기능들을 사용하는
사람들에겐 커다란 도움이 될 것을 약속합니다.
이 도구들은 일부 본질적인 한계점들을 가지고 있습니다:
\iffalse

These tools promise to be of great help to people working on low-level
parallel primitives that run on ARM and on Power. These tools do have
some intrinsic limitations:
\fi

\begin{enumerate}
\item	이 도구들은 연구용 프로토타입이고, 그것 자체에 있어서는 기능을 지원하지
	않습니다.
\item	이 도구들은 IBM 이나 ARM 으로부터 각각의 CPU 아키텍쳐들에 대한 공식적
	언급을 받은 바가 없습니다.
	예를 들어, 두 회사 모두 언제든 이 도구들의 어떤 버전들에 대해서도
	버그가 있다고 이야기할 수가 있습니다.
	따라서 이 도구들은 실제 하드웨어에서의 조심스러운 스트레스 테스트를
	대체제가 될수는 없습니다.
	더구나, 이것들의 기반을 형성하고 있는 도구들도 모델들도 모두 활발히
	개발되는 중이고 언제든 바뀔 수 있습니다.
	한편, 이 모델은 적절한 하드웨어 전문가의 자문에 의해 개발되었기에,
	이것은 해당 아키텍쳐들의 든든한 표현이라고 확신을 가져도 좋을 이유가
	있는 셈입니다.
\item	이 도구들은 현재 인스트럭션집합 중 부분집합만을 다루고 있습니다.
	이 부분집합은 제 목적에 있어서는 충분했습니다만, 여러분의 목적은 다양할
	수 있습니다.
	구체적으로, 이 도구는 워드 크기 (32 bits) 의 액세스들만을 다루며,
	이렇게 액세스 되는 워드들은 올바르게 정렬되어 있어야만 합니다.
	또한, 이 도구는 ARM 메모리 배리어 인스트럭션의 더 완화된 변종들도, 산술
	인스트럭션들도 다루지 않습니다.
\iffalse

\item	These tools are research prototypes, and as such are unsupported.
\item	These tools do not constitute official statements by IBM or ARM
	on their respective CPU architectures. For example, both
	corporations reserve the right to report a bug at any time against
	any version of any of these tools. These tools are therefore not a
	substitute for careful stress testing on real hardware. Moreover,
	both the tools and the model that they are based on are under
	active development and might change at any time. On the other
	hand, this model was developed in consultation with the relevant
	hardware experts, so there is good reason to be confident that
	it is a robust representation of the architectures.
\item	These tools currently handle a subset of the instruction set.
	This subset has been sufficient for my purposes, but your mileage
	may vary. In particular, the tool handles only word-sized accesses
	(32 bits), and the words accessed must be properly aligned. In
	addition, the tool does not handle some of the weaker variants
	of the ARM memory-barrier instructions, nor does it handle arithmetic.
\fi
\item	이 도구들은 적은 수의 쓰레드에서 돌아가는, 작고 루프를 사용하지 않는
	코드 조각들에서 사용될 것으로 제약되어 있습니다.
	더 커다란 예제들은 Promela 와 spin 같은 비슷한 도구들과 같이 상태
	공간의 폭증을 초래합니다.
\item	전체 상태 공간 탐색은 어떻게 각각의 문제되는 상태에 도달했는지를
	알려주지 않습니다.
	그렇다곤 하나, 특정 상태가 정말로 도달할 수 있다는 사실을 깨닫게 되면,
	대화형 도구를 사용해서 그 상태를 찾아나서는 것은 일반적으로 어렵지
	않습니다.
\item	이 도구들은 복잡한 데이터 구조에는 그다지 좋지 않습니다만, 극단적으로
	간단한 링크드 리스트를 ``\co{x=y; y=z; z=42;}'' 와 같은 형태의 초기화
	statement 들을 사용해서 생성하고 탐색하는 것은 가능하긴 합니다.
\item	이 도구들은 memory mapped I/O 나 디바이스 레지스터들에 대해서는
	처리하지 않습니다.
	물론, 그런 것들을 처리하는 것은 그것들이 형식화 될 것을 필요로 하는데,
	여기엔 존재하지 않습니다.
\item	이 도구들은 단정문을 통해 여러분이 코딩한 문제들에 대해서만 문제를
	찾아줄 겁니다.
	이 약점은 모든 형식적 방법들에 공통적인 것이고, 왜 테스트가 여전히
	중요한지에 대한 또다른 이유입니다.
	이 챕터의 시작에서 인용한 Donald Knuth 의 영원한 말처럼, ``다음 코드의
	버그들을 경계하세요; 전 그것이 올바르다는 걸 증명했을 뿐이지, 그걸
	사용해 보진 않았습니다.''
\iffalse

\item	The tools are restricted to small loop-free code fragments
	running on small numbers of threads. Larger examples result
	in state-space explosion, just as with similar tools such as
	Promela and spin.
\item	The full state-space search does not give any indication of how
	each offending state was reached. That said, once you realize
	that the state is in fact reachable, it is usually not too hard
	to find that state using the interactive tool.
\item	These tools are not much good for complex data structures, although
	it is possible to create and traverse extremely simple linked
	lists using initialization statements of the form
	``\co{x=y; y=z; z=42;}''.
\item	These tools do not handle memory mapped I/O or device registers.
	Of course, handling such things would require that they be
	formalized, which does not appear to be in the offing.
\item	The tools will detect only those problems for which you code an
	assertion. This weakness is common to all formal methods, and
	is yet another reason why testing remains important. In the
	immortal words of Donald Knuth quoted at the beginning of this
	chapter, ``Beware of bugs in the above
	code; I have only proved it correct, not tried it.''
\fi
\end{enumerate}

그렇다곤 하나, 이 도구들의 강점 가운데 하나는 해당 아키텍쳐에 의해서 허용되는,
합법적이지만 현재 하드웨어 구현이 부주의한 소프트웨어 개발자들에게 아직 가하지
않은 모든 동작들의 전체 범위를 모델링 하도록 설계되었다는 점입니다.
따라서, 이 도구들에서 진료된 알고리즘들은 실제 하드웨어에서 동작할 때에는
추가적인 안전성 마진을 가질 수 있을 확률이 큽니다.
더 나아가서, 실제 하드웨어에서 테스트를 해보는 것은 버그를 찾을 수 있게 해줄
뿐입니다; 그런 테스트로 해당 사용이 올바른지에 대한 증명을 하는 것은 본질적으로
불가능합니다.
이를 이해하기 위해, 연구자들이 그들의 모델의 검증을 위해 실제 하드웨어 위에서
천억개의 테스트를 일일이 수행하고 있는 것을 생각해 보세요.
어떤 경우에 있어서는, 아크텍쳐에 의해 허용되어 있는 행동이 1760억 회의 테스트
수행에도 불구하고 나타나지 않은 적도 있었습니다~\cite{JadeAlglave2011ppcmem}.
반면에, 이 전체 상태 공간 탐색은 해당 도구가 코드 조각의 정확성을 증명할 수
있게 해줍니다.
\iffalse

That said, one strength of these tools is that they are designed to
model the full range of behaviors allowed by the architectures, including
behaviors that are legal, but which current hardware implementations do
not yet inflict on unwary software developers. Therefore, an algorithm
that is vetted by these tools likely has some additional safety margin
when running on real hardware. Furthermore, testing on real hardware can
only find bugs; such testing is inherently incapable of proving a given
usage correct. To appreciate this, consider that the researchers
routinely ran in excess of 100 billion test runs on real hardware to
validate their model.
In one case, behavior that is allowed by the architecture did not occur,
despite 176 billion runs~\cite{JadeAlglave2011ppcmem}.
In contrast, the
full-state-space search allows the tool to prove code fragments correct.
\fi

형식적 방법들과 도구들은 테스트의 대체제가 아님을 다시 한번 이야기할 필요가
있습니다.
예를 들어 리눅스 커널과 같은 커다란 안정적인 동시적 소프트웨어 작품을
만들어내는 것은 상당히 어려운 일입니다.
따라서 개발자들은 이 목표를 위해 모든 가능한 도구들을 적용할 준비가 되어 있어야
합니다.
이 챕터에서 보인 도구들은 테스트를 통해 발생시키기 (그리고 추적하기) 상당히
어려운 버그들을 찾아내는 것을 가능하게 해줍니다.
한편, 테스트는 이 챕터에서 소개된 도구들로 처리할 수 있는 것들보다 훨씬 커다란
몸집의 소프트웨어들에 적용될 수 있습니다.
항상 그렇듯이, 작업에 걸맞는 도구를 사용하세요!

물론, 여러분의 병렬 코드를 쉽게 분할될 수 있도록 설계하고 (락, 시퀀스 카운터,
어토믹 오퍼레이션, 그리고 RCU 같은) 고차원의 도구들을 사용해서 일을 더 간단하게
되도록 함으로써 이런 단계에서의 일을 할 필요 자체를 없애는 게 항상 최선입니다.
그리고 여러분의 일을 처리하기 위해 반드시 낮은 단계의 메모리 배리어들과
read-modify-write 인스트럭션들을 사용해야만 하는 경우라 할지라도, 이 날카로운
도구들을 더 많이 보수적으로 사용할수록 여러분의 삶이 더 쉬워질 겁니다.
\iffalse

It is worth repeating that formal methods and tools are no substitute for
testing. The fact is that producing large reliable concurrent software
artifacts, the Linux kernel for example, is quite difficult. Developers
must therefore be prepared to apply every tool at their disposal towards
this goal. The tools presented in this chapter are able to locate bugs that
are quite difficult to produce (let alone track down) via testing. On the
other hand, testing can be applied to far larger bodies of software than
the tools presented in this chapter are ever likely to handle. As always,
use the right tools for the job!

Of course, it is always best to avoid the need to work at this level
by designing your parallel code to be easily partitioned and then
using higher-level primitives (such as locks, sequence counters, atomic
operations, and RCU) to get your job done more straightforwardly. And even
if you absolutely must use low-level memory barriers and read-modify-write
instructions to get your job done, the more conservative your use of
these sharp instruments, the easier your life is likely to be.
\fi
