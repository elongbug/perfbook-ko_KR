% formal/spinhint.html
% mainfile: ../perfbook.tex
% SPDX-License-Identifier: CC-BY-SA-3.0

\section{State-Space Search}
\label{sec:formal:State-Space Search}
%
\epigraph{Follow every byway / Every path you know.}
	 {\emph{``Climb Every Mountain'', Rodgers \& Hammerstein}}

이 섹션은 많은 종류의 멀티 쓰레드 코드의 전체 상태 공간 탐색을 해낼 수도 있는
범용 Promela 와 Spin 도구들을 소개합니다.
이것들은 데이터 통신 프로토콜을 검증하기 위해 사용됩니다.
Section~\ref{sec:formal:Promela and Spin}
은 Promela 와 Spin 을 소개하는데 어토믹과 어토믹하지 않은 값 증가를 검증하는
두개의 워밍업 연습을 포함합니다.
Section~\ref{sec:formal:How to Use Promela}
은 Promela 의 사용법을 소개하는데 Promela 의 문법의 C 의 그것과 비교와 예제
커맨드 라인들을 포함합니다.
Section~\ref{sec:formal:Promela Example: Locking}
은 락킹을 검증하기 위해 Promela 가 어떻게 사용되는지 알아보고,
\ref{sec:formal:Promela Example: QRCU}
는 ``QRCU'' 라 이름지어진 일반적이지 않은 RCU 구현을 검증하는데 Promela 를
사용해 보며, 마지막으로
Section~\ref{sec:formal:Promela Parable: dynticks and Preemptible RCU}
은 RCU 의 dyntick-idle 구현 초기 버전에 Promela 를 적용합니다.

\iffalse

This section features the general-purpose Promela and Spin tools,
which may be used to carry out a full
state-space search of many types of multi-threaded code.
They are used to verifying data communication protocols.
Section~\ref{sec:formal:Promela and Spin}
introduces Promela and Spin, including a couple of warm-up exercises
verifying both non-atomic and atomic increment.
Section~\ref{sec:formal:How to Use Promela}
describes use of Promela, including example command lines and a
comparison of Promela syntax to that of C\@.
Section~\ref{sec:formal:Promela Example: Locking}
shows how Promela may be used to verify locking,
\ref{sec:formal:Promela Example: QRCU}
uses Promela to verify an unusual implementation of RCU named ``QRCU'',
and finally
Section~\ref{sec:formal:Promela Parable: dynticks and Preemptible RCU}
applies Promela to early versions of RCU's dyntick-idle implementation.

\fi

\subsection{Promela and Spin}
\label{sec:formal:Promela and Spin}

Promela 는 프로토콜을 검증하기 위해 설계된 언어이지만 작은 병렬 알고리즘을
검증하는데에도 사용될 수 있습니다.
여러분은 여러분의 알고리즘과 정확성 제한들을 C 같은 Promela 언어로 재작성할 수
있으며 Spin 을 사용해 그것을 컴파일하고 수행할 수 있는 C 프로그램으로 변환할 수
있습니다.
그 결과 프로그램은 여러분의 알고리즘의 전체 상태 공간 탐색을 해내는데, 여러분이
Promela 프로그램에 짜넣은 단정들을 검증하거나 반례를 찾아냅니다.

이 전체 공감 탐색은 굉장히 강력할 수 있지만 또한 양날의 검이 될 수 있습니다.
여러분의 알고리즘이 너무 복잡하거나 여러분의 Promela 구현이 부주의 하다면,
메모리에 들어가는 것보다 많은 상태가 있을 수 있습니다.
더 나아가서, 충분한 메모리가 있더라도 이 상태 공간 탐색은 예상되는 우주의
수명보다 긴 시간동안 수행될 수도 있습니다.
따라서, 이 도구는 작지만 복잡한 병렬 알고리즘을 위해 사용하십시오.
이를 적당한 크기의 알고리즘에 (전체 리눅스 커널은 놔두세요) 생각없이 적용하는
것은 나쁜 결과를 초래할 겁니다.

\iffalse

Promela is a language designed to help verify protocols, but which
can also be used to verify small parallel algorithms.
You recode your algorithm and correctness constraints in the C-like
language Promela, and then use Spin to translate it into a C program
that you can compile and run.
The resulting program carries out a full state-space search of your
algorithm, either verifying or finding counter-examples for
assertions that you can associate with in your Promela program.

This full-state search can be extremely powerful, but can also be a two-edged
sword.
If your algorithm is too complex or your Promela implementation is
careless, there might be more states than fit in memory.
Furthermore, even given sufficient memory, the state-space search might
well run for longer than the expected lifetime of the universe.
Therefore, use this tool for compact but complex parallel algorithms.
Attempts to naively apply it to even moderate-scale algorithms (let alone
the full Linux kernel) will end badly.

\fi

Promela 와 Spin 은
\url{https://spinroot.com/spin/whatispin.html} 에서 다운로드 받을 수 있습니다.

위의 사이트는 또한 Gerard Holzmann 의 Promela 와 Spin 에 대한 훌륭한
교재~\cite{Holzmann03a} 로의 링크와 검색 가능한 온라인 레퍼런스들을
\url{https://www.spinroot.com/spin/Man/index.html} 에서 제공합니다.

이 섹션의 나머지는 병렬 알고리즘을 디버깅 하기 위해 어떻게 Promela 를
사용하는지, 간단한 예에서 시작해 복잡한 사용법으로 넘어가면서 알아봅니다.

\iffalse

Promela and Spin may be downloaded from
\url{https://spinroot.com/spin/whatispin.html}.

The above site also gives links to Gerard Holzmann's excellent
book~\cite{Holzmann03a} on Promela and Spin,
as well as searchable online references starting at:
\url{https://www.spinroot.com/spin/Man/index.html}.

The remainder of this section describes how to use Promela to debug
parallel algorithms, starting with simple examples and progressing to
more complex uses.

\fi

\subsubsection{Warm-Up: Non-Atomic Increment}
\label{sec:formal:Warm-Up: Non-Atomic Increment}

\begin{fcvref}[ln:formal:promela:increment:whole]
Listing~\ref{lst:formal:Promela Code for Non-Atomic Increment}
은 어토믹하지 않은 값 증가에서 초래되는 교재에 나오는 경주 조건을 선보입니다.
라인~\lnref{nprocs} 는 수행할 프로세스의 수를 정의하고 (상태 공간에의 영향을
보기 위해 우린 이 값을 변화시켜 볼 겁니다), 라인~\lnref{count} 는 카운터를
정의하며, 라인~\lnref{prog} 는 \clnrefrange{assert:b}{assert:e} 에 나오는
단정문을 구현하는데 사용됩니다.

\iffalse

\begin{fcvref}[ln:formal:promela:increment:whole]
Listing~\ref{lst:formal:Promela Code for Non-Atomic Increment}
demonstrates the textbook race condition
resulting from non-atomic increment.
Line~\lnref{nprocs} defines the number of processes to run (we will vary this
to see the effect on state space), line~\lnref{count} defines the counter,
and line~\lnref{prog} is used to implement the assertion that appears on
\clnrefrange{assert:b}{assert:e}.

\fi

\begin{listing}[tbp]
\input{CodeSamples/formal/promela/increment@whole.fcv}
\caption{Promela Code for Non-Atomic Increment}
\label{lst:formal:Promela Code for Non-Atomic Increment}
\end{listing}

\Clnrefrange{proc:b}{proc:e} 는 어토믹하지 않게 카운터를 증가시키는 프로세스
하나를 정의합니다.
인자 \co{me} 는 프로세스 수로, 코드의 뒤쪽에 있는 초기화 블록에서 설정됩니다.
간단한 Promela 선언문들은 각각 어토믹한 걸로 가정되므로, 우린 이 값 증가를
\clnrefrange{incr:b}{incr:e} 의 두개 선언문으로 쪼개야 합니다.
라인~\lnref{setprog} 에서의 값 할당은 이 프로세스의 완료를 표시합니다.
Spin 시스템은 상태 공간을 모든 가능한 상태 순서들을 포함해 완전히 탐색하므로,
전통적인 스트레스 테스트에 사용되는 반복문은 필요 없습니다.

\Clnrefrange{init:b}{init:e} 는 초기화 블록으로, 가장 먼저 수행됩니다.
\Clnrefrange{doinit:b}{doinit:e} 는 실제로 초기화를 하며,
\clnrefrange{assert:b}{assert:e} 는 단정을 수행합니다.
둘 다 불필요한 상태 공간 증가를 막기 위해 원자적 블록으로 정의됩니다: 이것들은
알고리즘의 부분이 아니므로, 이것들을 원자적으로 만듦으로써 검증 범위를 잃지는
않습니다.

\iffalse

\Clnrefrange{proc:b}{proc:e} define a process that increments
the counter non-atomically.
The argument \co{me} is the process number, set by the initialization
block later in the code.
Because simple Promela statements are each assumed atomic, we must
break the increment into the two statements on
\clnrefrange{incr:b}{incr:e}.
The assignment on line~\lnref{setprog} marks the process's completion.
Because the Spin system will fully search the state space, including
all possible sequences of states, there is no need for the loop
that would be used for conventional stress testing.

\Clnrefrange{init:b}{init:e} are the initialization block,
which is executed first.
\Clnrefrange{doinit:b}{doinit:e} actually do the initialization,
while \clnrefrange{assert:b}{assert:e}
perform the assertion.
Both are atomic blocks in order to avoid unnecessarily increasing
the state space: because they are not part of the algorithm proper,
we lose no verification coverage by making them atomic.

\fi

\Clnrefrange{dood1:b}{dood1:e} 의 \co{do-od} 구성은 C 에서의 case label 표현을
허용하는 \co{switch} 문을 담는 \co{for (;;)} 반복문으로 생각될 수 있는 Promela
반복문을 구현합니다.
조건 블록 (\co{::} 접두어로 표시됨) 은 비결정적으로 스캔됩니다만, 이 경우에는
한번에 단 하나의 조건만 잡힐 수 있습니다.
이 \co{do-od} 의
\clnrefrange{block1:b}{block1:e} 에 있는 첫번째 블록은 i 번째 값 증가 쓰레드의
진행 셀을 초기화 하고, i 번째 값 증가 쓰레드의 프로세스를 수행하며, 이어서 변수
\co{i} 의 값을 증가시킵니다.
라인~\lnref{block2} 에 있는 \co{do-od} 의 두번째 블록은 이 프로세스들이
시작되고 나면 이 반복문을 빠져나갑니다.

\iffalse

The \co{do-od} construct on \clnrefrange{dood1:b}{dood1:e}
implements a Promela loop,
which can be thought of as a C \co{for (;;)} loop containing a
\co{switch} statement that allows expressions in case labels.
The condition blocks (prefixed by \co{::})
are scanned non-deterministically,
though in this case only one of the conditions can possibly hold at a given
time.
The first block of the \co{do-od} from
\clnrefrange{block1:b}{block1:e}
initializes the i-th
incrementer's progress cell, runs the i-th incrementer's process, and
then increments the variable \co{i}.
The second block of the \co{do-od} on
line~\lnref{block2} exits the loop once
these processes have been started.

\fi

\Clnrefrange{assert:b}{assert:e} 의 어토믹 블록도 진행 카운터의 합을 구하는
비슷한 \co{do-od} 반복문을 갖습니다.
라인~\lnref{assert} 의 \co{assert()} 문은 모든 프로세스가 완료되었다면 모든
카운트가 올바르게 기록되었을 것을 검증합니다.
\end{fcvref}

\iffalse

The atomic block on \clnrefrange{assert:b}{assert:e} also contains
a similar \co{do-od}
loop that sums up the progress counters.
The \co{assert()} statement on line~\lnref{assert} verifies that
if all processes
have been completed, then all counts have been correctly recorded.
\end{fcvref}

\fi

여러분은 이 프로그램을 다음과 같이 빌드하고 수행할 수 있습니다:

\iffalse

You can build and run this program as follows:

\fi

\begin{VerbatimU}
spin -a increment.spin      # Translate the model to C
cc -DSAFETY -o pan pan.c    # Compile the model
./pan                       # Run the model
\end{VerbatimU}

\begin{listing}[tbp]
\VerbatimInput[numbers=none,fontsize=\scriptsize]{CodeSamples/formal/promela/increment.spin.lst}
\vspace*{-9pt}
\caption{Non-Atomic Increment Spin Output}
\label{lst:formal:Non-Atomic Increment Spin Output}
\end{listing}

이는
Listing~\ref{lst:formal:Non-Atomic Increment Spin Output} 에 보인 것과 같은
출력을 낼 겁니다.
첫번째 줄은 우리의 단정이 위배되었음을 (어토믹 하지 않은 값 증가를 했으니
예상된 바입니다!) 말합니다.
두번째 줄은 이 단정이 어떻게 위배되었는지에 대한 설명이 \co{trail} 파일에 쓰여
있음을 말합니다.
``Warning'' 줄은 우리의 모델이 완벽하지 않았음을 다시 말합니다.
두번째 문단은 진행된 상태 탐색의 종류를 설명하는데 이 경우 단정 위배와 올바르지
않은 종료 상태입니다.
세번째 문단은 상태 크기 통계를 보입니다: 이 작은 모델은 45개의 상태만을
가졌습니다.
마지막 라인은 메모리 사용량을 보입니다.

\co{trail} 파일은 다음과 같이 사람이 읽을 수 있는 형태로 변환될 수 있습니다:

\iffalse

This will produce output as shown in
Listing~\ref{lst:formal:Non-Atomic Increment Spin Output}.
The first line tells us that our assertion was violated (as expected
given the non-atomic increment!).
The second line that a \co{trail} file was written describing how the
assertion was violated.
The ``Warning'' line reiterates that all was not well with our model.
The second paragraph describes the type of state-search being carried out,
in this case for assertion violations and invalid end states.
The third paragraph gives state-size statistics: this small model had only
45 states.
The final line shows memory usage.

The \co{trail} file may be rendered human-readable as follows:

\fi

\begin{VerbatimU}
spin -t -p increment.spin
\end{VerbatimU}

\begin{listing*}[htbp]
\VerbatimInput[numbers=none,fontsize=\scriptsize]{CodeSamples/formal/promela/increment.spin.trail.lst}
\vspace*{-9pt}
\caption{Non-Atomic Increment Error Trail}
\label{lst:formal:Non-Atomic Increment Error Trail}
\end{listing*}

이는
Listing~\ref{lst:formal:Non-Atomic Increment Error Trail}
에 보인 것과 같은 출력을 냅니다.
보이듯, init 블록의 첫번째 부분은 두개의 값 증가 프로세스를 만들었는데, 둘 다
카운터를 먼저 읽어들이고, 값을 증가한 후 저장하여, 수를 잃습니다.
단정문이 이어서 실행되고 전역 상태가 표시됩니다.

\iffalse

This gives the output shown in
Listing~\ref{lst:formal:Non-Atomic Increment Error Trail}.
As can be seen, the first portion of the init block created both
incrementer processes, both of which first fetched the counter,
then both incremented and stored it, losing a count.
The assertion then triggered, after which the global state is displayed.

\fi

\subsubsection{Warm-Up: Atomic Increment}
\label{sec:formal:Warm-Up: Atomic Increment}

이 예를 고치는 건
Listing~\ref{lst:formal:Promela Code for Atomic Increment} 에 보인 것처럼 값
증가 프로세스의 몸체를 어토믹 블록에 두는 것처럼 쉽습니다.
어떤 사람은 간단히 두 쌍의 명령문을 \co{counter = counter + 1} 로 바꿀 수도
있을텐데, Promela 명령문은 어토믹 하기 때문입니다.
어떤 방법이든, 이 수정된 모델을 수행하는 것은
Listing~\ref{lst:formal:Atomic Increment Spin Output} 에 보인 것처럼 오류 없는
상태 공간 순회를 내놓습니다.

\iffalse

It is easy to fix this example by placing the body of the incrementer
processes in an atomic block as shown in
Listing~\ref{lst:formal:Promela Code for Atomic Increment}.
One could also have simply replaced the pair of statements with
\co{counter = counter + 1}, because Promela statements are
atomic.
Either way, running this modified model gives us an error-free traversal
of the state space, as shown in
Listing~\ref{lst:formal:Atomic Increment Spin Output}.

\fi

\begin{listing}[tbp]
\input{CodeSamples/formal/promela/atomicincrement@incrementer.fcv}
\caption{Promela Code for Atomic Increment}
\label{lst:formal:Promela Code for Atomic Increment}
\end{listing}

\begin{listing}[tbp]
\VerbatimInput[numbers=none,fontsize=\scriptsize]{CodeSamples/formal/promela/atomicincrement.spin.lst}
\vspace*{-9pt}
\caption{Atomic Increment Spin Output}
\label{lst:formal:Atomic Increment Spin Output}
\end{listing}

Table~\ref{tab:advsync:Memory Usage of Increment Model}
은 모델된 값 증가 쓰레드의 수에 따른 (\co{NUMPROCS} 재정의에 의한) 상태 갯수와
사용된 메모리를 보입니다.

\iffalse

Table~\ref{tab:advsync:Memory Usage of Increment Model}
shows the number of states and memory consumed
as a function of number of incrementers modeled
(by redefining \co{NUMPROCS}):

\fi

\begin{table}
\rowcolors{1}{}{lightgray}
\small
\renewcommand*{\arraystretch}{1.2}
\centering
\begin{tabular}{S[table-format = 1.0]S[table-format = 7.0]S[table-format = 3.1]}
	\toprule
	\multicolumn{1}{l}{\# incrementers} &
		\multicolumn{1}{r}{\# states} &
			\multicolumn{1}{r}{total memory usage (MB)} \\
	\midrule
	1 &		        11 &        128.7 \\
	2 &		        52 &        128.7 \\
	3 &		       372 &        128.7 \\
	4 &		     3 496 &        128.9 \\
	5 &		    40 221 &        131.7 \\
	6 &		   545 720 &        174.0 \\
	7 &		 8 521 446 &        881.9 \\
	\bottomrule
\end{tabular}
\caption{Memory Usage of Increment Model}
\label{tab:advsync:Memory Usage of Increment Model}
\end{table}

따라서 불필요하게 큰 모델을 수행하는 것은 장려되지 않는데, 882\,MB 는 현대
데스크탑과 랩톱 기계에서의 한계 내이긴 합니다.

이 예를 머리 속에 새겨두고, Promela 모델을 분석하는데 사용되는 커맨드를 더
자세히 알아보고 더 정교한 예를 들여다 봅시다.

\iffalse

Running unnecessarily large models is thus subtly discouraged, although
882\,MB is well within the limits of modern desktop and laptop machines.

With this example under our belt, let's take a closer look at the
commands used to analyze Promela models and then look at more
elaborate examples.

\fi

\subsection{How to Use Promela}
\label{sec:formal:How to Use Promela}

소스 파일 \path{qrcu.spin} 을 가지고, 다음과 같은 커맨드를 사용할 수 있습니다:

Given a source file \path{qrcu.spin}, one can use the following commands:

\begin{description}[style=nextline]
\item	[\tco{spin -a qrcu.spin}]
	상태 기계를 전체 탐색하는 \path{pan.c} 파일을 만듭니다.
\item	[\tco{cc -DSAFETY [-DCOLLAPSE] [-DMA=N] -o pan pan.c}]
	생성된 상태 기계 탐색을 컴파일 합니다.  \co{-DSAFETY} 는 여러분이
	단정문들만을 가지고 있다면 (그리고 어쩌면 \co{never} 명령문을) 적합한
	최적화를 생성합니다.  여러분이 liveness, fairness, 그리고
	forward-progress 체크를 갖는다면, 여러분은 \co{-DSAFETY} 없이 컴파일을
	해야할 수도 있습니다.  여러분이 사용할 수 있음에도 \co{-DSAFETY} 를
	사용하지 않는다면 프로그램은 이를 알려줄 겁니다.

	\co{-DSAFETY} 에 의해 생성되는 최적화는 상당히 속도를 향상시키므로
	사용할 수 있다면 여러분은 그걸 사용해야 합니다.
	여러분이 \co{-DSAFETY} 를 사용할 수 없는 상황 중 한 예는 \co{-DNP} 를
	통해 livelock (``non-pregress cycles'' 라고도 알려져 있습니다) 을
	검사할 때입니다.

	옵션 \co{-DCOLLAPSE} 는 state vector compression 모드를 위한 코드를
	생성합니다.

	또다른 옵션인 \co{-DMA=N} 은 느리지만 공격적인 상태 공간 메모리 압축
	모드를 위한 코드를 생성합니다.

\iffalse

\item	[\tco{spin -a qrcu.spin}]
	Create a file \path{pan.c} that fully searches the state machine.
\item	[\tco{cc -DSAFETY [-DCOLLAPSE] [-DMA=N] -o pan pan.c}]
	Compile the generated state-machine search.  The \co{-DSAFETY}
	generates optimizations that are appropriate if you have only
	assertions (and perhaps \co{never} statements).  If you have
	liveness, fairness, or forward-progress checks, you may need
	to compile without \co{-DSAFETY}.  If you leave off \co{-DSAFETY}
	when you could have used it, the program will let you know.

	The optimizations produced by \co{-DSAFETY} greatly speed things
	up, so you should use it when you can.
	An example situation where you cannot use \co{-DSAFETY} is
	when checking for livelocks (AKA ``non-progress cycles'')
	via \co{-DNP}.

	The optional \co{-DCOLLAPSE} generates code for a state vector
	compression mode.

	Another optional flag \co{-DMA=N} generates code for a slow
	but aggressive state-space memory compression mode.

\fi

\item	[\tco{./pan [-mN] [-wN]}]
	이는 실제로 상태 공간을 탐색합니다.  상태의 수는 매우 작은 상태 기계를
	가지고도 수천만에 이를 수 있으므로 여러분은 큰 메모리의 기계를 필요로
	할 겁니다.
	예를 들어, 3개의 업데이트 쓰레드와 2개의 읽기 쓰레드를 갖는
	\path{qrcu.spin} 은 \co{-DCOLLAPSE} 플래그를 가지고도 10.5\,GB 메모리를
	필요로 했습니다.

	\co{./pan} 이 다음과 같이 말한다면, 여러분은 완벽한 탐색을 위해
	\co{-mN} 옵션을 사용해 최대 깊이를 늘려야 합니다:
	\qco{error: max search depth too small}.
	기본값은 \co{-m10000} 입니다.

	\co{-wN} 옵션은 해쉬테이블 크기를 명시합니다.
	전체 상태 공간 탐색을 위한 기본 값은 \co{-w24} 입니다.\footnote{
		Spin 버전 6.4.6 과 6.4.8 에서.  2011년 7월 10일 자의 Spin
		온라인 매뉴얼에서는 전체 탐색 모드를 위한 기본 값은 \co{-w19}
		라고 이야기하는데, 실제 동작과 맞지 않습니다.}

\iffalse

\item	[\tco{./pan [-mN] [-wN]}]
	This actually searches the state space.  The number of states
	can reach into the tens of millions with very small state
	machines, so you will need a machine with large memory.
	For example, \path{qrcu.spin} with 3~updaters and 2~readers required
	10.5\,GB of memory even with the \co{-DCOLLAPSE} flag.

	If you see a message from \co{./pan} saying:
	\qco{error: max search depth too small}, you need to increase
	the maximum depth by a \co{-mN} option for a complete search.
	The default is \co{-m10000}.

	The \co{-wN} option specifies the hashtable size.
	The default for full state-space search is \co{-w24}.\footnote{
		As of Spin Version 6.4.6 and 6.4.8. In the online manual of
		Spin dated 10 July 2011, the default for exhaustive search
		mode is said to be \co{-w19}, which does not meet
		the actual behavior.}

\fi

	여러분의 기계가 충분한 메모리를 가지고 있는지 확신치 못한다면, 한
	윈도우에서 \co{top} 을 돌리고 다른 윈도우에서 \co{./pan} 을 돌리세요.
	\co{./pan} 윈도우에 포커스를 유지해서 필요하면 그 수행을 중지시킬 수
	있게 하세요.  CPU 시간이 100\,\% 아래로 떨어지는 순간 \co{./pan} 을
	강제 종료시키세요.  \co{./pan} 을 수행하는 윈도우로부터  포커스를
	제거했다면 윈도우 시스템이 다른 무언가를 하는데 충분한 메모리를 쥐는데
	많은 시간을 소요해 여러분을 기다려야 하게 만들 겁니다.

	메모리 소모를 막는 다른 방법은 \co{-DMEMLIM=N} 컴파일러 플래그입니다.
	\co{-DMEMLIM=2000} 은 그 최대값을 2\,GB 로 만듭니다.

	출력을 캡쳐해 두는걸, 특히 여러분이 원격의 기계에서 작업 중이라면 잊지
	마세요.

	여러분의 모델이 forward-progress 체크를 포함한다면 여러분은 \co{./pan}
	에의 \co{-f} 커맨드라인 인자를 통해 ``weak fairness'' 를 활성화 시켜야
	할 겁니다.
	여러분의 forward-progress 검사가 \co{accept} 라벨을 포함한다면, \co{-a}
	인자도 필요할 겁니다.

\iffalse

	If you aren't sure whether your machine has enough memory,
	run \co{top} in one window and \co{./pan} in another.  Keep the
	focus on the \co{./pan} window so that you can quickly kill
	execution if need be.  As soon as CPU time drops much below
	100\,\%, kill \co{./pan}.  If you have removed focus from the
	window running \co{./pan}, you may wait a long time for the
	windowing system to grab enough memory to do anything for
	you.

	Another option to avoid memory exhaustion is the
	\co{-DMEMLIM=N} compiler flag. \co{-DMEMLIM=2000}
	would set the maximum of 2\,GB.

	Don't forget to capture the output, especially
	if you are working on a remote machine.

	If your model includes forward-progress checks, you will likely
	need to enable ``weak fairness'' via the \co{-f} command-line
	argument to \co{./pan}.
	If your forward-progress checks involve \co{accept} labels,
	you will also need the \co{-a} argument.
	% forward reference to model: formal.2009.02.19a in
	% /home/linux/git/userspace-rcu/formal-model.

\fi

\item	[\tco{spin -t -p qrcu.spin}]
	오류를 발견한 수행에 의한 \co{trail} 파일 결과물을 가지고 그 오류를
	일으킨 단계들을 출력합니다.
	\co{-g} 플래그는 변경된 전역 변수들을 포함시키고 \co{-l} 플래그는
	변경된 지역 변수들도 포함시킵니다.

\iffalse

\item	[\tco{spin -t -p qrcu.spin}]
	Given \co{trail} file output by a run that encountered an
	error, output the sequence of steps leading to that error.
	The \co{-g} flag will also include the values of changed
	global variables, and the  \co{-l} flag will also include
	the values of changed local variables.

\fi

\end{description}

\subsubsection{Promela Peculiarities}
\label{sec:formal:Promela Peculiarities}

모든 프로그래밍 언어가 비슷한 기반을 갖지만, Promela 는 C, C++, 또는 Java 로
코드를 짜는데 익숙한 사람들에게 조금 놀라울 겁니다.

\iffalse

Although all computer languages have underlying similarities,
Promela will provide some surprises to people used to coding in C,
C++, or Java.

\fi

\begin{enumerate}
\item	C 에서, \qco{;} 는 문장을 끝냅니다.  Promela 에서는 그것들을
	구분합니다.
	다행히, 더 최신 버전의 Spin 은 ``여분의'' 세미콜론들을 더 포기했습니다.
\item	Promela 의 루프 구조물인 \co{do} 문은 조건을 취합니다.
	이 \co{do} 문은 if-then-else 반복문을 상당히 닮았습니다.
\item	C 의 \co{switch} 문에서는 매칭되는 경우가 없을 때 전체 구분이
	생략됩니다.  Promela 의 것에서는 알아볼 수 있는 연관된 에러 메세지 없이
	에러를 냅니다.
	따라서, 오류 결과물이 잘못 없는 코드를 가리킬 때에는 여러분이 \co{if}
	나 \co{do} 문에 조건을 남겨두지 않았는지 검사해 보세요.
\item	C 에서 스트레스 테스트를 할 때, 어떤 사람들은 의심되는 오퍼레이션을
	다른것들 각각에 대해 바복적으로 경주시켜볼 겁니다.  Promela 에서는 그
	대신 하나의 경주를 만드는데, Promela 는 그 하나의 경주로부터 나올 수
	있는 모든 가능한 경우를 탐색하기 때문입니다.  간혹 여러분은 Promela
	에서 반복문을 짜야하기도 한데, 예를 들어 여러 오퍼레이션들이 겹치지만
	그렇게 하는게 여러분의 상태 공간의 크기를 크게 증가시키는 경우입니다.

\iffalse

\item	In C, \qco{;} terminates statements.  In Promela it separates them.
	Fortunately, more recent versions of Spin have become
	much more forgiving of ``extra'' semicolons.
\item	Promela's looping construct, the \co{do} statement, takes
	conditions.
	This \co{do} statement closely resembles a looping if-then-else
	statement.
\item	In C's \co{switch} statement, if there is no matching case, the whole
	statement is skipped.  In Promela's equivalent, confusingly called
	\co{if}, if there is no matching guard expression, you get an error
	without a recognizable corresponding error message.
	So, if the error output indicates an innocent line of code,
	check to see if you left out a condition from an \co{if} or \co{do}
	statement.
\item	When creating stress tests in C, one usually races suspect operations
	against each other repeatedly.	In Promela, one instead sets up
	a single race, because Promela will search out all the possible
	outcomes from that single race.	Sometimes you do need to loop
	in Promela, for example, if multiple operations overlap, but
	doing so greatly increases the size of your state space.

\fi

\item	C 에서, 할 수 있는 가장 쉬운 일은 반복문의 진행과 종료를 추적하기 위해
	반복문 카운터를 갖는 것입니다.  Promela 에서, 반복문 카운터는 역병과
	같이 막아져야 하는데 그게 상태 공간을 폭발시키기 때문입니다.  다른
	한편, 변수들 중 어떤 것도 단조적으로 증가하거나 감소하지 않는 한 무한
	루프는 Promela 에서 단점을 갖지 않습니다---Promela 는 그 반복문을
	지나는 얼마나 많은 경우가 정말로 문제시 되는지 알아내고 자동으로 그
	지점 이후의 수행을 제거할 겁니다.
\item	C 고문 테스트 코드에서, 태스크별 제어 변수를 갖는게 종종 현명합니다.
	그것들은 읽기 쉽고, 테스트 코드의 디버깅을 상당히 쉽게 합니다.  Promela
	에서, 태스크별 제어 변수는 다른 대안이 없을 때에만 상요되어야 합니다.
	이를 자세히 보기 위해, 다섯개의 검증을 위한 태스크가 있고 각 태스크의
	완료를 알리는 비트를 하나씩 사용한다고 해봅시다.  이는 32개의 상태를
	만듭니다.  대조적으로, 간단한 카운터는 여섯개의 상태만을 가져서 다섯배
	감소를 시킵니다.  이 다섯배의 규모는 여러분이 10\,GB 메모리를 소모하는
	1500만개의 상태로 고생하고 있지 않을때까지는 문제로 보이지 않을수도
	있을 겁니다!

\iffalse

\item	In C, the easiest thing to do is to maintain a loop counter to track
	progress and terminate the loop.  In Promela, loop counters
	must be avoided like the plague because they cause the state
	space to explode.  On the other hand, there is no penalty for
	infinite loops in Promela as long as none of the variables
	monotonically increase or decrease---Promela will figure out
	how many passes through the loop really matter, and automatically
	prune execution beyond that point.
\item	In C torture-test code, it is often wise to keep per-task control
	variables.  They are cheap to read, and greatly aid in debugging the
	test code.  In Promela, per-task control variables should be used
	only when there is no other alternative.  To see this, consider
	a 5-task verification with one bit each to indicate completion.
	This gives 32 states.  In contrast, a simple counter would have
	only six states, more than a five-fold reduction.  That factor
	of five might not seem like a problem, at least not until you
	are struggling with a verification program possessing more than
	150 million states consuming more than 10\,GB of memory!

\fi

\item	C 고문 테스트 코드와 Promela 모두에서의 가장 어려운 일은 좋은 단정문을
	만드는 겁니다.  Promela 는 또한 각 코드 줄마다 복사되는 단정문 같이
	동작하는 \co{never} 문을 지원합니다.
\item	분할해 정복하기는 Promela 에서 상태 공간을 제어하에 두는데에 굉장히
	도움됩니다.  거대한 모델을 두개의 대략 동일한 절반으로 쪼개는 것은 각
	절반이 전체의 대략 제곱근 크기가 되게 합니다.
	예를 들어, 백만개의 상태가 결합된 모델은 천개 상태의 모델 한쌍으로
	크기가 줄어들 수도 있습니다.
	이 두개의 더 작은 모델들은 Promela 에 의해 더 적은 메모리로 더 빨리
	처리될 뿐 아니라 사람들이 이 두개의 더 작은 알고리즘들을 더 쉽게 이해할
	수 있게 합니다.

\iffalse

\item	One of the most challenging things both in C torture-test code and
	in Promela is formulating good assertions.  Promela also allows
	\co{never} claims that act like an assertion replicated
	between every line of code.
\item	Dividing and conquering is extremely helpful in Promela in keeping
	the state space under control.  Splitting a large model into two
	roughly equal halves will result in the state space of each
	half being roughly the square root of the whole.
	For example, a million-state combined model might reduce to a
	pair of thousand-state models.
	Not only will Promela handle the two smaller models much more
	quickly with much less memory, but the two smaller algorithms
	are easier for people to understand.

\fi

\end{enumerate}


\subsubsection{Promela Coding Tricks}
\label{sec:formal:Promela Coding Tricks}

Promela 는 프로토콜을 분석하기 위해 설계되었으므로 이를 병렬 프로그램에
사용하는건 약간 남용하는 것이긴 합니다.
다음 트릭들은 여러분이 Promela 를 안전히 남용하는 걸 도울겁니다:

\iffalse

Promela was designed to analyze protocols, so using it on parallel programs
is a bit abusive.
The following tricks can help you to abuse Promela safely:

\fi

\begin{enumerate}
\item	메모리 순서 재배치.  전역변수 \co{x} 와 \co{y} 를 지역변수 \co{r1} 과
	\co{r2} 로 복사하는 한쌍의 명령문이 있고, 순서가 중요하지만 (예: 락으로
	보호되지 않음), 메모리 배리어가 없는 경우를 생각해 봅시다.  이는
	Promela 에서 다음과 같이 모델링 될 수 있습니다:

\iffalse

\item	Memory reordering.  Suppose you have a pair of statements
	copying globals \co{x} and \co{y} to locals \co{r1} and \co{r2}, where
	ordering matters (e.g., unprotected by locks), but where you have no
	memory barriers.  This can be modeled in Promela as follows:

\fi

\begin{VerbatimN}[samepage=true]
if
:: 1 -> r1 = x;
        r2 = y
:: 1 -> r2 = y;
        r1 = x
fi
\end{VerbatimN}

	이 \co{if} 문에서의 두개의 갈래는 비결정적으로 선택될 것인데, 둘 다
	가능하기 때문입니다.
	전체 상태 공간이 탐색되므로, \emph{두개의} 선택들이 모든 경우에 결국은
	만들어질 겁니다.

	물론, 이 트릭은 너무 자주 사용되면 여러분의 상태 공간을 폭증하게 할
	겁니다.
	또한, 이는 여러분이 가능한 재배치를 예상할 것을 요구합니다.

\iffalse

	The two branches of the \co{if} statement will be selected
	nondeterministically, since they both are available.
	Because the full state space is searched, \emph{both} choices
	will eventually be made in all cases.

	Of course, this trick will cause your state space to explode
	if used too heavily.
	In addition, it requires you to anticipate possible reorderings.

\fi

\item	상태 축소.  복잡한 단정문이 있다면 그것들을 \co{atomic} 에서
	평가하세요.  어쨌건, 그것들은 알고리즘의 한 부분이 아닙니다.  복잡한
	단정문의 한 예는 (뒤에서 더 자세히 이야기 하겠습니다)
	Listing~\ref{lst:formal:Complex Promela Assertion} 에 보인 것과
	같습니다.

	이 단정문을 원자적이 아닌 방식으로 평가할 이유가 없는데, 그건
	알고리즘의 실제 부분이 아니기 때문입니다.
	각 명령문이 상태를 증가시키므로 우린 이를
	Listing~\ref{lst:formal:Atomic Block for Complex Promela Assertion}
	에 보인 것처럼 \co{atomic} 블록에 넣음으로써 쓸모없는 상태들의 수를
	줄일 수 있습니다.

\item	Promela 는 함수를 제공하지 않습니다.
	그대신 C 전처리기 매크로를 사용하셔야 합니다.
	그러나, 조합을 통한 폭발을 막기 위해 조심스럽게 사용하셔야 합니다.

\iffalse

\item	State reduction.  If you have complex assertions, evaluate
	them under \co{atomic}.  After all, they are not part of the
	algorithm.  One example of a complex assertion (to be discussed
	in more detail later) is as shown in
	Listing~\ref{lst:formal:Complex Promela Assertion}.

	There is no reason to evaluate this assertion
	non-atomically, since it is not actually part of the algorithm.
	Because each statement contributes to state, we can reduce
	the number of useless states by enclosing it in an \co{atomic}
	block as shown in
	Listing~\ref{lst:formal:Atomic Block for Complex Promela Assertion}.

\item	Promela does not provide functions.
	You must instead use C preprocessor macros.
	However, you must use them carefully in order to avoid
	combinatorial explosion.

\fi

\end{enumerate}

\begin{listing}[tbp]
\begin{VerbatimL}
i = 0;
sum = 0;
do
:: i < N_QRCU_READERS ->
	sum = sum + (readerstart[i] == 1 &&
	             readerprogress[i] == 1);
	i++
:: i >= N_QRCU_READERS ->
	assert(sum == 0);
	break
od
\end{VerbatimL}
\caption{Complex Promela Assertion}
\label{lst:formal:Complex Promela Assertion}
\end{listing}

\begin{listing}[tbp]
\begin{VerbatimL}
atomic {
	i = 0;
	sum = 0;
	do
	:: i < N_QRCU_READERS ->
		sum = sum + (readerstart[i] == 1 &&
		             readerprogress[i] == 1);
		i++
	:: i >= N_QRCU_READERS ->
		assert(sum == 0);
		break
	od
}
\end{VerbatimL}
\caption{Atomic Block for Complex Promela Assertion}
\label{lst:formal:Atomic Block for Complex Promela Assertion}
\end{listing}

이제 더 많은 예를 알아볼 준비가 되었습니다.

\iffalse

Now we are ready for further examples.

\fi

\subsection{Promela Example: Locking}
\label{sec:formal:Promela Example: Locking}

\begin{fcvref}[ln:formal:promela:lock:whole]
락은 일반적으로 유용하므로
Listing~\ref{lst:formal:Promela Code for Spinlock} 에 보인 것처럼 여러 Promela
모델에 포함될 수 있는 \path{lock.h} 에 의해 \co{spin_lock()} 과
\co{spin_unlock()} 이 제공됩니다.
\co{spin_lock()} 매크로는 \clnrefrange{dood:b}{dood:e} 의 \co{do-od}
무한반복문을 포함하는데, 라인~\lnref{one} 의 ``1'' 의 보호 덕입니다.
이 반복문의 몸체는 \co{if-fi} 문을 포함하는 하나의 어토믹 블록입니다.
이 \co{if-fi} 문은 반복이 아니라 한번의 수행만 된다는 점을 제외하곤 \co{do-od}
문과 비슷하게 구성됩니다.
락이 라인~\lnref{notheld} 에서 잡히면 라인~\lnref{acq} 는 이를 획득하고
라인~\lnref{break} 는 감싸고 있는 \co{do-od} 반복문을 깹니다 (그리고 이 어토믹
블록에서 나갑니다).
다른 한편 이 락이 이미 라인~\lnref{held} 에서 잡혀 있다면 우린 아무것도 하지
않고 (\co{skip}), \co{if-fi} 와 이 어토믹 블록을 실패로 끝내서 그 바깥 반복문의
다음 반복을 취해서 이 락이 획득 가능할 때까지 이를 반복합니다.
\end{fcvref}

\iffalse

\begin{fcvref}[ln:formal:promela:lock:whole]
Since locks are generally useful, \co{spin_lock()} and
\co{spin_unlock()}
macros are provided in \path{lock.h}, which may be included from
multiple Promela models, as shown in
Listing~\ref{lst:formal:Promela Code for Spinlock}.
The \co{spin_lock()} macro contains an infinite \co{do-od} loop
spanning \clnrefrange{dood:b}{dood:e},
courtesy of the single guard expression of ``1'' on line~\lnref{one}.
The body of this loop is a single atomic block that contains
an \co{if-fi} statement.
The \co{if-fi} construct is similar to the \co{do-od} construct, except
that it takes a single pass rather than looping.
If the lock is not held on line~\lnref{notheld}, then
line~\lnref{acq} acquires it and
line~\lnref{break} breaks out of the enclosing \co{do-od} loop (and also exits
the atomic block).
On the other hand, if the lock is already held on line~\lnref{held},
we do nothing (\co{skip}), and fall out of the \co{if-fi} and the
atomic block so as to take another pass through the outer
loop, repeating until the lock is available.
\end{fcvref}

\fi

\begin{listing}[tbp]
\input{CodeSamples/formal/promela/lock@whole.fcv}
\caption{Promela Code for Spinlock}
\label{lst:formal:Promela Code for Spinlock}
\end{listing}

\co{spin_unlock()} 매크로는 단순히 락을 잡혀있지 않다고 표시합니다.

Promela 는 완전한 순서규칙을 가정하므로 메모리 배리어는필요치 않음에
주의하십시오.
모든 Promela 상태에서, 모든 프로세스는 현재 상태와 그 위에서 우리에게 가해진
상태 변화의 순서에 동의합니다.
이는 일부 컴퓨터 시스템에서 (1990년대의 MPIS 와 PA-RISC 같은) 사용된
``순차적으로 일관적인 (sequentially consistent)'' 메모리 모델과 비슷합니다.
앞서 언급되었듯, 그리고 뒤의 예에서 살펴보게 되겠지만 완화된 메모리 순서 규칙은
명시적으로 코딩 되어야 합니다.

\iffalse

The \co{spin_unlock()} macro simply marks the lock as no
longer held.

Note that memory barriers are not needed because Promela assumes
full ordering.
In any given Promela state, all processes agree on both the current
state and the order of state changes that caused us to arrive at
the current state.
This is analogous to the ``sequentially consistent'' memory model
used by a few computer systems (such as 1990s MIPS and PA-RISC).
As noted earlier, and as will be seen in a later example,
weak memory ordering must be explicitly coded.

\fi

\begin{listing}[tb]
\input{CodeSamples/formal/promela/lock@spin.fcv}
\caption{Promela Code to Test Spinlocks}
\label{lst:formal:Promela Code to Test Spinlocks}
\end{listing}

\begin{fcvref}[ln:formal:promela:lock:spin]
이 매크로들은
Listing~\ref{lst:formal:Promela Code to Test Spinlocks}
에 보인 Promela 코드로 테스트 됩니다.
이 코드는 값 증가를 테스트하는데 사용된 것과 유사한데, 락을 사용하는 프로세스가
라인~\lnref{nlockers} 의 \co{N_LOCKERS} 매크로로 정의되어 있습니다.
뮤텍스 자체는 라인~\lnref{mutex} 에, 락 소유자를 추적하기 위한 배열은
라인~\lnref{array} 에 정의되어 있으며, 라인~\lnref{sum} 은 하나의 프로세스만이
락을 잡음을 검증하기 위한 단정문 코드에 의해 사용됩니다.
\end{fcvref}

\begin{fcvref}[ln:formal:promela:lock:spin:locker]
락을 사용하는 프로세스는 \clnrefrange{b}{e} 에 있는데, 단순히 라인~\lnref{lock}
에서 락을 잡고 라인~\lnref{claim} 에서 락을 잡았음을 표시하고
라인~\lnref{unclaim} 에서 이를 철회한 후 라인~\lnref{unlock} 에서 락을
내려놓기를 영원히 반복합니다.
\end{fcvref}

\iffalse

\begin{fcvref}[ln:formal:promela:lock:spin]
These macros are tested by the Promela code shown in
Listing~\ref{lst:formal:Promela Code to Test Spinlocks}.
This code is similar to that used to test the increments,
with the number of locking processes defined by the \co{N_LOCKERS}
macro definition on line~\lnref{nlockers}.
The mutex itself is defined on line~\lnref{mutex},
an array to track the lock owner
on line~\lnref{array}, and line~\lnref{sum} is used by assertion
code to verify that only one process holds the lock.
\end{fcvref}

\begin{fcvref}[ln:formal:promela:lock:spin:locker]
The locker process is on \clnrefrange{b}{e}, and simply loops forever
acquiring the lock on line~\lnref{lock}, claiming it on line~\lnref{claim},
unclaiming it on line~\lnref{unclaim}, and releasing it on line~\lnref{unlock}.
\end{fcvref}

\fi

\begin{fcvref}[ln:formal:promela:lock:spin:init]
\Clnrefrange{b}{e} 의 초기화 블록은 현재 락 사용 프로세스의 \co{havelock}
배열을 라인~\lnref{array} 에서 초기화하고 현재 락 사용 프로세스를
라인~\lnref{start} 에서 시작하며, 라인~\lnref{next} 에서 다음 락 사용
프로세스로 진행합니다.
일단 모든 락 사용 프로세스가 시작되면, \co{do-od} 반복문은
라인~\lnref{chkassert} 로 이동하는데, 단정문을 검사합니다.
라인~\lnref{sum} 과~\lnref{j} 는 통제 변수를 초기화하고,
\clnrefrange{atm:b}{atm:e} 는 원자적으로 \co{havelock} 배열의 원소들의 값을
합하며, 라인~\lnref{assert} 는 단정을 하고, 라인~\lnref{break} 에서 반복문을
나갑니다.
\end{fcvref}

Listing~\ref{lst:formal:Promela Code for Spinlock}
과~\ref{lst:formal:Promela Code to Test Spinlocks} 의 두 코드 조각을
\path{lock.h} 와 \path{lock.spin} 파일들에 각각 넣고 다음 명령들을 사용해
이 모델을 수행시킬 수 있습니다.

\iffalse

\begin{fcvref}[ln:formal:promela:lock:spin:init]
The init block on \clnrefrange{b}{e} initializes the current locker's
\co{havelock} array entry on line~\lnref{array}, starts the current locker on
line~\lnref{start}, and advances to the next locker on line~\lnref{next}.
Once all locker processes are spawned, the \co{do-od} loop
moves to line~\lnref{chkassert}, which checks the assertion.
Lines~\lnref{sum} and~\lnref{j} initialize the control variables,
\clnrefrange{atm:b}{atm:e} atomically sum the \co{havelock} array entries,
line~\lnref{assert} is the assertion, and line~\lnref{break} exits the loop.
\end{fcvref}

We can run this model by placing the two code fragments of
Listings~\ref{lst:formal:Promela Code for Spinlock}
and~\ref{lst:formal:Promela Code to Test Spinlocks} into
files named \path{lock.h} and \path{lock.spin}, respectively, and then running
the following commands:

\fi

\begin{VerbatimU}
spin -a lock.spin
cc -DSAFETY -o pan pan.c
./pan
\end{VerbatimU}

\begin{listing}[htbp]
\VerbatimInput[numbers=none,fontsize=\scriptsize]{CodeSamples/formal/promela/lock.spin.lst}
\vspace*{-9pt}
\caption{Output for Spinlock Test}
\label{lst:formal:Output for Spinlock Test}
\end{listing}

그 결과물은
Listing~\ref{lst:formal:Output for Spinlock Test}
에 보인 것과 비슷한 무엇일 겁니다.
예상대로, 이 수행은 실패를 갖지 않습니다 (\qco{errors: 0}).

\iffalse

The output will look something like that shown in
Listing~\ref{lst:formal:Output for Spinlock Test}.
As expected, this run has no assertion failures (\qco{errors: 0}).

\fi

\QuickQuizSeries{%
\QuickQuizB{
	락 사용 프로세스에 도달되지 못한 명령문이 있는 이유는 뭐죠?  어쨌건,
	이건 \emph{전체} 상태 공간 탐색 아니었나요?

	\iffalse

	Why is there an unreached statement in
	locker?  After all, isn't this a \emph{full} state-space
	search?

	\fi

}\QuickQuizAnswerB{
	이 락 사용 프로세스는 무한 반복문이므로, 제어는 이 프로세스의 마지막에
	결코 도달하지 못합니다.
	그러나, 단조증가하는 변수가 없으므로 Promela 는 이 무한 반복문을 작은
	수의 상태로 모델링할 수 있습니다.

	\iffalse

	The locker process is an infinite loop, so control
	never reaches the end of this process.
	However, since there are no monotonically increasing variables,
	Promela is able to model this infinite loop with a small
	number of states.

	\fi

}\QuickQuizEndB
%
\QuickQuizE{
	이 예에서 Promela 코드 스타일 문제는 무엇이 있을까요?

	\iffalse

	What are some Promela code-style issues with this example?

	\fi

}\QuickQuizAnswerE{
	여러개 있습니다:
	\begin{enumerate}
	\item	\co{sum} 의 선언은 초기화 블록으로 옮겨져야 하는데, 어디서도
		사용되지 않기 때문입니다.
	\item	단정문 코드는 초기화 반복문 바깥으로 옮겨져야 합니다.  이
		초기화 반복문은 그러면 어토믹 블록에 위치해서 상태 공간을 크게
		줄일 수 있습니다 (얼마나 줄일까요?).
	\item	단정문을 감싸는 이 어토믹 블록은 단정문에 더해 \co{sum} 과
		\co{j} 의 초기화를 포함하도록 확장되어야 합니다.
		이 또한 상태 공간을 줄입니다 (다시 말하지만, 얼마나 줄일까요?).
	\end{enumerate}

	\iffalse

	There are several:
	\begin{enumerate}
	\item	The declaration of \co{sum} should be moved to within
		the init block, since it is not used anywhere else.
	\item	The assertion code should be moved outside of the
		initialization loop.  The initialization loop can
		then be placed in an atomic block, greatly reducing
		the state space (by how much?).
	\item	The atomic block covering the assertion code should
		be extended to include the initialization of \co{sum}
		and \co{j}, and also to cover the assertion.
		This also reduces the state space (again, by how
		much?).
	\end{enumerate}

	\fi

}\QuickQuizEndE
}

\subsection{Promela Example: QRCU}
\label{sec:formal:Promela Example: QRCU}

이 마지막 예는 Oleg Nesterov 의
QRCU~\cite{OlegNesterov2006QRCU,OlegNesterov2006aQRCU} 의 실제 사용된 것이지만
\co{synchronize_qrcu()} 의 빠른경로를 더 빠르게 하기 위해 수정된 버전을
선보입니다.

그런데 먼저, QRCU 란 무엇일까요?

QRCU 는 SRCU~\cite{PaulEMcKenney2006c} 의 변종으로 더 높은 읽기 오버헤드를
(전혁 변수에 대한 원자적 값 증가와 감소) 극단적으로 낮은 grace-period
응답시간을 위해 포기합니다.
읽기 쓰레드가 없다면 grace period 는 1 마이크로세컨드 미만의 시간에 파악되는데,
대부분의 다른 RCU 구현의 수 밀리세컨드 grace period 응답시간에 비해 무척
빠릅니다.

\iffalse

This final example demonstrates a real-world use of Promela on Oleg
Nesterov's
QRCU~\cite{OlegNesterov2006QRCU,OlegNesterov2006aQRCU},
but modified to speed up the \co{synchronize_qrcu()}
fastpath.

But first, what is QRCU?

QRCU is a variant of SRCU~\cite{PaulEMcKenney2006c}
that trades somewhat higher read overhead
(atomic increment and decrement on a global variable) for extremely
low grace-period latencies.
If there are no readers, the grace period will be detected in less
than a microsecond, compared to the multi-millisecond grace-period
latencies of most other RCU implementations.

\fi

\begin{enumerate}
\item	QRCU 도메인을 정의하는 \co{qrcu_struct} 가 있습니다.
	SRCU 처럼 (그리고 다른 RCU 변종들과는 달리) QRCU 의 행동은 전역적이지
	않으며 특정 \co{qrcu_struct} 에 명시적입니다.
\item	QRCU read-side 크리티컬 섹션을 정의하는 \co{qrcu_read_lock()} 과
	\co{qrcu_read_unlock()} 이 있습니다.
	연관된 \co{qrcu_struct} 가 이 함수들에 넘겨져야만 하며,
	\co{qrcu_read_lock()} 의 리턴값은 \co{qrcu_read_unlock()} 에 넘겨져야
	합니다.

	예를 들면:

\iffalse

\item	There is a \co{qrcu_struct} that defines a QRCU domain.
	Like SRCU (and unlike other variants of RCU) QRCU's action
	is not global, but instead focused on the specified
	\co{qrcu_struct}.
\item	There are \co{qrcu_read_lock()} and \co{qrcu_read_unlock()}
	primitives that delimit QRCU read-side critical sections.
	The corresponding \co{qrcu_struct} must be passed into
	these primitives, and the return value from \co{qrcu_read_lock()}
	must be passed to \co{qrcu_read_unlock()}.

	For example:

\fi

\begin{VerbatimU}
idx = qrcu_read_lock(&my_qrcu_struct);
/* read-side critical section. */
qrcu_read_unlock(&my_qrcu_struct, idx);
\end{VerbatimU}

\item	모든 앞서서부터 존재한 QRCU read-side 크리티컬 섹션이 완료될 때까지
	기다리는 \co{synchronize_qrcu()} 함수가 있지만, SRCU 의
	\co{synchronize_srcu()} 와 같이, QRCU 의 \co{synchronize_qrcu()} 는
	같은 \co{qrcu_struct} 를 사용하는 read-side 크리티컬 섹션만 기다립니다.

	예를 들어, \co{synchronize_qrcu(&your_qrcu_struct)} 는 앞의 QRCU
	read-side 크리티컬 섹션을 기다리지 \emph{않을} 겁니다.
	대조적으로 \co{synchronize_qrcu(&my_qrcu_struct)} 는 같은
	\co{qrcu_struct} 를 공유하므로 기다려야 \emph{할겁니다}.

\iffalse

\item	There is a \co{synchronize_qrcu()} primitive that blocks until
	all pre-existing QRCU read-side critical sections complete,
	but, like SRCU's \co{synchronize_srcu()}, QRCU's
	\co{synchronize_qrcu()} need wait only for those read-side
	critical sections that are using the same \co{qrcu_struct}.

	For example, \co{synchronize_qrcu(&your_qrcu_struct)}
	would \emph{not} need to wait on the earlier QRCU read-side
	critical section.
	In contrast, \co{synchronize_qrcu(&my_qrcu_struct)}
	\emph{would} need to wait, since it shares the same
	\co{qrcu_struct}.

\fi

\end{enumerate}

리눅스 커널을 위한 QRCU 패치가
만들어졌습니다만~\cite{PaulMcKenney2007QRCUpatch}, 리눅스 커널에 받아들여지지는
못할 것 같습니다.

\iffalse

A Linux-kernel patch for QRCU has been
produced~\cite{PaulMcKenney2007QRCUpatch},
but is unlikely to ever be included in the Linux kernel.

\fi

\begin{listing}[htbp]
\input{CodeSamples/formal/promela/qrcu@gvar.fcv}
\caption{QRCU Global Variables}
\label{lst:formal:QRCU Global Variables}
\end{listing}

QRCU 를 위한 Promela 코드로 돌아와서, 전역 변수들은
Listing~\ref{lst:formal:QRCU Global Variables} 에 보인 것과 같습니다.
이 예는 락킹을 사용하고 \path{lock.h} 를 포함합니다.
읽기 쓰레드와 쓰기 쓰레드의 수 모두 두개의 \co{#define} 문을 통해 변화될 수
있어서 우리에게 조합적 상태 수 폭발을 일으킬 수 있는 두개의 방법을 제공합니다.
\co{idx} 변수는 \co{ctr} 변수의 두 원소 중 무엇이 읽기 쓰레드에 의해 사용될지
통제하며, \co{readerprogress} 변수는 언제 모든 읽기 쓰레드가 끝날지 정하는
단정문을 가능하게 합니다 (QRCU 업데이트는 모든 앞서서부터 존재한 읽기 쓰레드가
그들의 QRCU read-side 크리티컬 섹션을 완료하기 전까지 허용되지 않으므로).
\co{readerprogress} 배열 원소들은 다음과 같은 값을 가져서 연관된 읽기 쓰레드의
상태를 표시합니다:

\iffalse

Returning to the Promela code for QRCU, the global variables are as shown in
Listing~\ref{lst:formal:QRCU Global Variables}.
This example uses locking and includes \path{lock.h}.
Both the number of readers and writers can be varied using the
two \co{#define} statements, giving us not one but two ways to create
combinatorial explosion.
The \co{idx} variable controls which of the two elements of the \co{ctr}
array will be used by readers, and the \co{readerprogress} variable
allows an assertion to determine when all the readers are finished
(since a QRCU update cannot be permitted to complete until all
pre-existing readers have completed their QRCU read-side critical
sections).
The \co{readerprogress} array elements have values as follows,
indicating the state of the corresponding reader:

\fi

\begin{enumerate}[label={\arabic*}:,start=0,itemsep=0pt]
\item	not yet started.
\item	within QRCU read-side critical section.
\item	finished with QRCU read-side critical section.
\end{enumerate}

마지막으로, \co{mutex} 변수는 업데이트 쓰레드의 느린경로를 순차화 시키기 위해
사용됩니다.

\iffalse

Finally, the \co{mutex} variable is used to serialize updaters' slowpaths.

\fi

\begin{listing}[htbp]
\input{CodeSamples/formal/promela/qrcu@reader.fcv}
\caption{QRCU Reader Process}
\label{lst:formal:QRCU Reader Process}
\end{listing}

\begin{fcvref}[ln:formal:promela:qrcu:reader]
QRCU 읽기 쓰레드는
Listing~\ref{lst:formal:QRCU Reader Process} 에 보인 \co{qrcu_reader()}
프로세스로 모델링 되었습니다.
\Clnrefrange{do}{od} 의 \co{do-od} 반복문은 라인~\lnref{one} 의 ``1'' 보호를
가져서 무한 반복문이 됩니다.
라인~\lnref{curidx} 는 전역 인덱스의 현재 값을 가져오며
\clnrefrange{atm:b}{atm:e} 는 그 값이 0이 아니었다면 원자적으로 이를
증가시키고 (\co{atomic_inc_not_zero()}) 반복문을 종료합니다.
라인~\lnref{cs:entry} 는 RCU read-side 크리티컬 섹션 내로의 진입을 표시하고
라인~\lnref{cs:exit} 는 이 크리티컬 섹션으로부터 빠져나옴을 표시하는데, 둘 다
우리가 뒤에서 마주할 \co{assert()} 문의 이익을 위함입니다.
라인~\lnref{atm:dec} 는 우리가 값 증가시킨 카운터의 값을 감소시키고 RCU
read-side 크리티컬 섹션을 빠져나옵니다.
\end{fcvref}

\iffalse

\begin{fcvref}[ln:formal:promela:qrcu:reader]
QRCU readers are modeled by the \co{qrcu_reader()} process shown in
Listing~\ref{lst:formal:QRCU Reader Process}.
A \co{do-od} loop spans \clnrefrange{do}{od},
with a single guard of ``1''
on line~\lnref{one} that makes it an infinite loop.
Line~\lnref{curidx} captures the current value of the global index,
and \clnrefrange{atm:b}{atm:e}
atomically increment it (and break from the infinite loop)
if its value was non-zero (\co{atomic_inc_not_zero()}).
Line~\lnref{cs:entry} marks entry into the RCU read-side critical section, and
line~\lnref{cs:exit} marks exit from this critical section,
both lines for the benefit of
the \co{assert()} statement that we shall encounter later.
Line~\lnref{atm:dec} atomically decrements the same counter that we incremented,
thereby exiting the RCU read-side critical section.
\end{fcvref}

\fi

\begin{listing}[htbp]
\input{CodeSamples/formal/promela/qrcu@sum_unordered.fcv}
\caption{QRCU Unordered Summation}
\label{lst:formal:QRCU Unordered Summation}
\end{listing}

\begin{fcvref}[ln:formal:promela:qrcu:sum_unordered]
Listing~\ref{lst:formal:QRCU Unordered Summation}
에 보인 C 전처리기 매크로는 한쌍의 카운터의 값을 더해서 완화된 메모리 순서
규칙을 에뮬레이션 합니다.
\Clnrefrange{fetch:b}{fetch:e} 는 카운터들 중 하나를 읽어오고
라인~\lnref{sum_other} 는 이 쌍의 다른 하나를 읽어와 그것들을 더합니다.
어토믹 블록은 하나의 \co{do-od} 문으로 구성됩니다.
이 \co{do-od} 문은 (\clnrefrange{do}{od}) Promela 가 둘 중 하나를 비결정적으로
선택하게 하는 라인~\lnref{g1} 과~\lnref{g2} 에서의 보호를 갖는 무조건적인
두개의 분기를 갖는다는 점에서 일반적이지 않습니다 (그러나 다시 말하지만, 전체
상태 공간 탐색은 Promela 가 결국은 모든 가능한 상황의 선택을 하게 합니다).
첫번째 분기는 0번째 카운터를 읽어오고 \co{i} 를 1 로 설정하며
(라인~\lnref{sum_other} 가 첫번째 카운터를 읽어오도록), 두번째 분기는 그 반대를
행해서 첫번째 카운터를 읽어오고 \co{i} 를 0 으로 설정합니다
(라인~\lnref{sum_other} 가 두번째 카운터를 읽어오도록).
\end{fcvref}

\iffalse

\begin{fcvref}[ln:formal:promela:qrcu:sum_unordered]
The C-preprocessor macro shown in
Listing~\ref{lst:formal:QRCU Unordered Summation}
sums the pair of counters so as to emulate weak memory ordering.
\Clnrefrange{fetch:b}{fetch:e} fetch one of the counters,
and line~\lnref{sum_other} fetches the other
of the pair and sums them.
The atomic block consists of a single \co{do-od} statement.
This \co{do-od} statement (spanning \clnrefrange{do}{od}) is unusual in that
it contains two unconditional
branches with guards on lines~\lnref{g1} and~\lnref{g2}, which causes Promela to
non-deterministically choose one of the two (but again, the full
state-space search causes Promela to eventually make all possible
choices in each applicable situation).
The first branch fetches the zero-th counter and sets \co{i} to 1 (so
that line~\lnref{sum_other} will fetch the first counter), while the second
branch does the opposite, fetching the first counter and setting \co{i}
to 0 (so that line~\lnref{sum_other} will fetch the second counter).
\end{fcvref}

\fi

\QuickQuiz{
	이 \co{do-od} 문을 더 간단하게 짜는 방법이 있을까요?

	\iffalse

	Is there a more straightforward way to code the \co{do-od} statement?

	\fi

}\QuickQuizAnswer{
	그렇습니다.
	그걸 \co{if-fi} 로 대체하고 두개의 \co{break} 문을 없애세요.

	\iffalse

	Yes.
	Replace it with \co{if-fi} and remove the two \co{break} statements.

	\fi

}\QuickQuizEnd

\begin{listing}[htbp]
\input{CodeSamples/formal/promela/qrcu@updater.fcv}
\caption{QRCU Updater Process}
\label{lst:formal:QRCU Updater Process}
\end{listing}

\begin{fcvref}[ln:formal:promela:qrcu:updater]
\co{sum_unordered} 매크로를 봤으니 이제
Listing~\ref{lst:formal:QRCU Updater Process}
에 보인 업데이트 쪽 프로세스로 넘어가 봅시다.
이 업데이트 쪽 프로세스는 \clnrefrange{do}{od} 의 연관된 \co{do-od} 문을 무한정
반복합니다.
반복문의 각 패스는 먼저 전역 \co{readerprogress} 배열을 지역 \co{readerstart}
배열로 \clnrefrange{atm1:b}{atm1:e} 에서 스냅샷 찍어둡니다.
이 스냅샷은 라인~\lnref{assert} 에서의 단정을 위해 사용될 겁니다.
라인~\lnref{sum_unord} 는 \co{sum_unordered} 를 수행하며, 이어서
\clnrefrange{reinvoke:b}{reinvoke:e} 는 이 빠른 수행경로가 잠재적으로 사용
가능할 경우 \co{sum_unordered} 를 재수행 합니다.

\iffalse

\begin{fcvref}[ln:formal:promela:qrcu:updater]
With the \co{sum_unordered} macro in place, we can now proceed
to the update-side process shown in
Listing~\ref{lst:formal:QRCU Updater Process}.
The update-side process repeats indefinitely, with the corresponding
\co{do-od} loop ranging over \clnrefrange{do}{od}.
Each pass through the loop first snapshots the global \co{readerprogress}
array into the local \co{readerstart} array on
\clnrefrange{atm1:b}{atm1:e}.
This snapshot will be used for the assertion on line~\lnref{assert}.
Line~\lnref{sum_unord} invokes \co{sum_unordered}, and then
\clnrefrange{reinvoke:b}{reinvoke:e}
re-invoke \co{sum_unordered} if the fastpath is potentially
usable.

\fi

\Clnrefrange{slow:b}{slow:e} 는 필요하면 느린 수행경로를 수행하는데,
라인~\lnref{acq} 와~\lnref{rel} 은 업데이트 쪽 락을 잡고 내려놓으며,
\clnrefrange{flip_idx:b}{flip_idx:e} 는 인덱스를 뒤집고,
\clnrefrange{wait:b}{wait:e} 는 모든 앞서서부터 존재한 읽기 쓰레드들이
완료되기를 기다립니다.

\Clnrefrange{atm2:b}{atm2:e} 는 이어서 \co{readerprogress} 배열의 현재 값들을
\co{readerstart} 배열에 모아진 그것들과 비교하여, 이 업데이트 전에 시작된 모든
읽기 쓰레드는 여전히 진행중이지 않게 강제합니다.
\end{fcvref}

\iffalse

\Clnrefrange{slow:b}{slow:e} execute the slowpath code if need be, with
lines~\lnref{acq} and~\lnref{rel} acquiring and releasing the update-side lock,
\clnrefrange{flip_idx:b}{flip_idx:e} flipping the index, and
\clnrefrange{wait:b}{wait:e} waiting for
all pre-existing readers to complete.

\Clnrefrange{atm2:b}{atm2:e} then compare the current values
in the \co{readerprogress}
array to those collected in the \co{readerstart} array,
forcing an assertion failure should any readers that started before
this update still be in progress.
\end{fcvref}

\fi

\QuickQuizSeries{%
\QuickQuizB{
	\begin{fcvref}[ln:formal:promela:qrcu:updater]
	\Clnrefrange{atm1:b}{atm1:e} 와 \clnrefrange{atm2:b}{atm2:e}
	내의 오퍼레이션들은 현재의 모든 제품 마이크로프로세서에 원자적 구현이
	없음에도 왜 어토믹 블록이 사용되었나요?
	\end{fcvref}

	\iffalse

	\begin{fcvref}[ln:formal:promela:qrcu:updater]
	Why are there atomic blocks at \clnrefrange{atm1:b}{atm1:e}
	and \clnrefrange{atm2:b}{atm2:e}, when the operations
	within those atomic
	blocks have no atomic implementation on any current
	production microprocessor?
	\end{fcvref}

	\fi

}\QuickQuizAnswerB{
	그 오퍼레이션들은 단정문의 이익만을 위한 것들이기 때문입니다.
	그것들은 알고리즘 자체의 부분이 아닙니다.
	따라서 그것들을 어토믹으로 만드는데 문제가 없으며, 그렇게 표시하는게
	Promela 모델에 의해 탐색되어야 하는 상태 공간을 크게 줄여줍니다.

	\iffalse

	Because those operations are for the benefit of the
	assertion only.  They are not part of the algorithm itself.
	There is therefore no harm in marking them atomic, and
	so marking them greatly reduces the state space that must
	be searched by the Promela model.

	\fi

}\QuickQuizEndB
%
\QuickQuizE{
	\begin{fcvref}[ln:formal:promela:qrcu:updater]
        \Clnrefrange{reinvoke:b}{reinvoke:e}
	에서의 카운터들의 재 합산이 \emph{정말로} 필요한가요?
        \end{fcvref}

	\iffalse

	\begin{fcvref}[ln:formal:promela:qrcu:updater]
	Is the re-summing of the counters on
        \clnrefrange{reinvoke:b}{reinvoke:e}
	\emph{really} necessary?
        \end{fcvref}

	\fi

}\QuickQuizAnswerE{
	그렇습니다.  이를 보기 위해, 이 줄들을 지우고 모델을 돌려보세요.

	대안적으로, 다음 단계들을 생각해 보십시오:

	\iffalse

	Yes.  To see this, delete these lines and run the model.

	Alternatively, consider the following sequence of steps:

	\fi

	\begin{enumerate}
	\item	한 프로세스가 RCU read-side 크리티컬 섹션 내에 있어서
		\co{ctr[0]} 는 0이고 \co{ctr[1]} 은 2의 값을 갖습니다.
	\item	한 업데이트 쓰레드가 수행을 시작하고 이 카운터들의 합이 2임을
		보게 되어 빠른 수행경로를 수행하지 못하게 됩니다.  따라서 락을
		잡습니다.
	\item	두번째 업데이트 쓰레드가 수행을 시작해 \co{ctr[0]} 의 값을
		읽어오는데 0입니다.
	\item	첫번재 업데이트 스레드는 \co{ctr[0]} 에 1을 더하고 인덱스를
		뒤집고 (이제 0이 됩니다), \co{ctr[1]} 에서 1을 뺍니다 (이제 1이
		됩니다).
	\item	두번째 업데이트 쓰레드가 이제 1인 \co{ctr[1]} 의 값을 읽습니다.
	\item	두번째 업데이트 쓰레드는 원래 읽기 쓰레드가 여전히 완료되지
		못했음에도 이제 빠른 수행경로를 수행해도 안전하다고 잘못된
		결론을 내립니다.

	\iffalse

	\item	One process is within its RCU read-side critical
		section, so that the value of \co{ctr[0]} is zero and
		the value of \co{ctr[1]} is two.
	\item	An updater starts executing, and sees that the sum of
		the counters is two so that the fastpath cannot be
		executed.  It therefore acquires the lock.
	\item	A second updater starts executing, and fetches the value
		of \co{ctr[0]}, which is zero.
	\item	The first updater adds one to \co{ctr[0]}, flips
		the index (which now becomes zero), then subtracts
		one from \co{ctr[1]} (which now becomes one).
	\item	The second updater fetches the value of \co{ctr[1]},
		which is now one.
	\item	The second updater now incorrectly concludes that it
		is safe to proceed on the fastpath, despite the fact
		that the original reader has not yet completed.

	\fi

	\end{enumerate}
}\QuickQuizEndE
}

\begin{listing}[htbp]
\input{CodeSamples/formal/promela/qrcu@init.fcv}
\caption{QRCU Initialization Process}
\label{lst:formal:QRCU Initialization Process}
\end{listing}

\begin{fcvref}[ln:formal:promela:qrcu:init]
이제 남은건
Listing~\ref{lst:formal:QRCU Initialization Process}
의 초기화 블록 뿐입니다.
이 블록은 단순히 \clnrefrange{i_ctr:b}{i_ctr:e} 의 카운터 쌍을 초기화 하고,
\clnrefrange{spn_r:b}{spn_r:e} 에서 읽기 프로세스들을 시작시키며,
\clnrefrange{spn_u:b}{spn_u:e} 에서 업데이트 프로세스들을 시작시킵니다.
이 모든 일은 상태 공간을 줄이기 위해 어토믹 블록에서 행해집니다.
\end{fcvref}

\iffalse

\begin{fcvref}[ln:formal:promela:qrcu:init]
All that remains is the initialization block shown in
Listing~\ref{lst:formal:QRCU Initialization Process}.
This block simply initializes the counter pair on
\clnrefrange{i_ctr:b}{i_ctr:e},
spawns the reader processes on
\clnrefrange{spn_r:b}{spn_r:e}, and spawns the updater
processes on \clnrefrange{spn_u:b}{spn_u:e}.
This is all done within an atomic block to reduce state space.
\end{fcvref}

\fi

\subsubsection{Running the QRCU Example}
\label{sec:formal:Running the QRCU Example}

이 QRCU 예제를 수행하기 위해, 앞 섹션의 코드 조각들을 \path{qrcu.spin} 이라는
이름의 파일에 결합시키고 \co{spin_lock()} 과 \co{spin_unlock()} 의 정의들을
\path{lock.h} 라는 이름의 파일에 넣으십시오.
이어서 이 QRCU 모델을 빌드하고 수행하기 위해 다음 커맨드를 사용하세요:

\iffalse

To run the QRCU example, combine the code fragments in the previous
section into a single file named \path{qrcu.spin}, and place the definitions
for \co{spin_lock()} and \co{spin_unlock()} into a file named
\path{lock.h}.
Then use the following commands to build and run the QRCU model:

\fi

\begin{VerbatimU}
spin -a qrcu.spin
cc -DSAFETY [-DCOLLAPSE] -o pan pan.c
./pan [-mN]
\end{VerbatimU}

\begin{table}
\centering
\begin{threeparttable}
\rowcolors{1}{}{lightgray}
\renewcommand*{\arraystretch}{1.2}
\footnotesize
\begin{tabular}{S[table-format = 1.0]S[table-format = 1.0]S[table-format = 9.0]
		S[table-format = 6.0]S[table-format = 5.1]}
	\toprule
	\multicolumn{1}{r}{updaters} &
	    \multicolumn{1}{r}{readers} &
		\multicolumn{1}{r}{\# states} &
		    \multicolumn{1}{r}{depth} &
			\multicolumn{1}{r}{memory (MB)\tnote{a}} \\
	\midrule
	1 & 1 &         376 &      95 &    128.7 \\
	1 & 2 &       6 177 &     218 &    128.9 \\
	1 & 3 &      99 728 &     385 &    132.6 \\
	2 & 1 &      29 399 &     859 &    129.8 \\
	2 & 2 &   1 071 181 &   2 352 &    169.6 \\
	2 & 3 &  33 866 736 &  12 857 &  1 540.8 \\
	3 & 1 &   2 749 453 &  53 809 &    236.6 \\
	3 & 2 & 186 202 860 & 328 014 & 10 483.7 \\
	\bottomrule
\end{tabular}
\begin{tablenotes}
	\item [a] Obtained with the compiler flag \co{-DCOLLAPSE}
		specified.
\end{tablenotes}
\end{threeparttable}
\caption{Memory Usage of QRCU Model}
\label{tab:advsync:Memory Usage of QRCU Model}
\end{table}

이 출력물은 이 모델이
Table~\ref{tab:advsync:Memory Usage of QRCU Model}
에 보인 모든 경우를 통과함을 보입니다.
세개의 읽기 쓰레드와 세개의 업데이트 쓰레드를 수행하는게 나을 수도 있겠으나,
간단한 외삽은 이게 대략 0.5 테라바이트 메모리를 요할 것을 보입니다.
어떡하시겠어요?

\co{./pan} 은 메모리 부족 시에 조언을 주는데, 예를 들어 세개의 읽기 쓰레드와
세개의 업데이트 쓰레드를 시도하면:

\iffalse

The output shows that this model passes all of the cases shown in
Table~\ref{tab:advsync:Memory Usage of QRCU Model}.
It would be nice to run three readers and three
updaters, however, simple extrapolation indicates that this will
require about half a terabyte of memory.
What to do?

It turns out that \co{./pan} gives advice when it runs out of memory,
for example, when attempting to run three readers and three updaters:

\fi

\begin{VerbatimU}
hint: to reduce memory, recompile with
  -DCOLLAPSE # good, fast compression, or
  -DMA=96   # better/slower compression, or
  -DHC # hash-compaction, approximation
  -DBITSTATE # supertrace, approximation
\end{VerbatimU}

크게 증가된 탐색 오버헤드의 비용으로 상태 공간을 강하게 압축하는 코드를
생성하는 \co{-DMA=N} 컴파일러 플래그를 제안받은대로 시도해봅시다:

\iffalse

Let's try the suggested compiler flag \co{-DMA=N},
which generates code for aggressive compression of the
state space at the cost of greatly increased search overhead.
The required commands are as follows:

\fi

\begin{VerbatimU}
spin -a qrcu.spin
cc -DSAFETY -DMA=96 -O2 -o pan pan.c
./pan -m20000000
\end{VerbatimU}

여기서, 깊이 한계 20,000,000 은 간단한 외삽을 통해 추론된 예상 깊이보다 열배
이상 큽니다.
이게 메모리 사용량을 늘리지만, 너무 빡빡한 깊이 제한으로부터 기인하는 불완전한
탐색으로 인한 긴 수행 낭비를 막습니다.
이 수행은 \Power{9} 서버에서 3일보다 조금 더 걸렸습니다.
그 결과는
Listing~\ref{lst:formal:spinhint:3 Readers 3 Updaters QRCU Spin Output with -DMA=96}
에 보여져 있습니다.
이 Spin 수행은 오직 6.5\,GB 전체 메모리 사용만 가지고 성공적으로 완료되었는데,
이는 대략 0.5 테라바이트 사용량을 보이는 \co{-DCOLLAPSE} 사용보다 거의 백배
가량 낮은 수치입니다.

\iffalse

Here, the depth limit of 20,000,000 is an order of magnitude
larger than the expected depth deduced from simple extrapolation.
Although this increases up-front memory usage, it avoids wasting
a long run due to incomplete search resulting from a too-tight
depth limit.
This run took a little more than 3~days on a \Power{9} server.
The result is shown in
Listing~\ref{lst:formal:spinhint:3 Readers 3 Updaters QRCU Spin Output with -DMA=96}.
This Spin run completed successfully with a total memory
usage of only 6.5\,GB, which is almost two orders of magnitude
lower than the \co{-DCOLLAPSE} usage of about half a terabyte.

\fi

\begin{listing}
\VerbatimInput[numbers=none,fontsize=\scriptsize]{CodeSamples/formal/promela/qrcu.spin.33ma.lst}
\vspace*{-9pt}
\caption{3 Readers 3 Updaters QRCU Spin Output with \co{-DMA=96}}
\label{lst:formal:spinhint:3 Readers 3 Updaters QRCU Spin Output with -DMA=96}
\end{listing}

\QuickQuiz{
	상태들로 인해 점유되는 메모리에서 200대1 감소에 연관되는 0.48\,\%
	압축률!
	이 상태 공간 탐색은 \emph{정말로} 완벽한가요???

	\iffalse

	A compression rate of 0.48\,\% corresponds to a 200-to-1 decrease
	in memory occupied by the states!
	Is the state-space search \emph{really} exhaustive???

	\fi

}\QuickQuizAnswer{
	Spin 의 문서에 따르면, 네, 그렇습니다.

	\iffalse

	According to Spin's documentation, yes, it is.

	\fi

\begin{listing}
\VerbatimInput[numbers=none,fontsize=\scriptsize]{CodeSamples/formal/promela/qrcu.spin.col-ma.diff.lst}
\vspace*{-9pt}
\caption{Spin Output Diff of \co{-DCOLLAPSE} and \co{-DMA=88}}
\label{lst:formal:promela:Spin Output Diff of -DCOLLAPSE and -DMA=88}
\end{listing}

	간접적인 증거로, \co{-DCOLLAPSE} 와 \co{-DMA=88} (두개의 읽기 쓰레드와
	세개의 업데이트 쓰레드) 에 의한 수행 결과를 비교해 봅시다.
	이 수행들에서의 결과물의 차이점이
	Listing~\ref{lst:formal:promela:Spin Output Diff of -DCOLLAPSE and -DMA=88}
	에 보여져 있습니다.
	볼 수 있듯, 상태의 수에 (저장된 것과 매치된 것) 둘 다 동의합니다.

	\iffalse

	As an indirect evidence, let's compare the results of
	runs with \co{-DCOLLAPSE} and with \co{-DMA=88}
	(two readers and three updaters).
	The diff of outputs from those runs is shown in
	Listing~\ref{lst:formal:promela:Spin Output Diff of -DCOLLAPSE and -DMA=88}.
	As you can see, they agree on the numbers of states
	(stored and matched).

	\fi

}\QuickQuizEnd

\begin{table*}[tbp]
\rowcolors{6}{}{lightgray}
\renewcommand*{\arraystretch}{1.2}
\footnotesize
\centering
\OneColumnHSpace{-0.7in}%
\begin{tabular}{S[table-format = 1.0]S[table-format = 1.0]S[table-format = 9.0]
		S[table-format = 9.0]S[table-format = 2.0]S[table-format = 5.2]
		S[table-format = 4.2]S[table-format = 2.0]S[table-format = 4.2]
		S[table-format = 6.2]}
	\toprule
	\multicolumn{4}{r}{} & \multicolumn{3}{c}{\tco{-DCOLLAPSE}} &
					\multicolumn{3}{c}{\tco{-DMA=N}} \\
	\cmidrule(l){5-7} \cmidrule(l){8-10}
	\multicolumn{1}{r}{updaters} &
	    \multicolumn{1}{r}{readers} &
		\multicolumn{1}{r}{\# states} &
		    \multicolumn{1}{r}{depth reached} &
			\multicolumn{1}{r}{\tco{-wN}} &
			    \multicolumn{1}{r}{memory (MB)} &
				\multicolumn{1}{r}{runtime (s)} &
				    \multicolumn{1}{r}{\tco{N}} &
					\multicolumn{1}{r}{memory (MB)} &
					    \multicolumn{1}{r}{runtime (s)} \\
	\cmidrule{1-4} \cmidrule(l){5-7} \cmidrule(l){8-10}
	1 & 1 &           376 &         95 & 12 &     0.10 & 0.00 &
		40 &    0.29 &      0.00 \\
	1 & 2 &         6 177 &        218 & 12 &     0.39 & 0.01 &
		47 &    0.59 &      0.02 \\
	1 & 3 &        99 728 &        385 & 16 &     4.60 & 0.14 &
		54 &    3.04 &      0.45 \\
        2 & 1 &        29 399 &        859 & 16 &     2.30 & 0.03 &
		55 &    0.70 &      0.13 \\
        2 & 2 &     1 071 181 &      2 352 & 20 &    49.24 & 1.45 &
		62 &    7.77 &      5.76 \\
        2 & 3 &    33 866 736 &     12 857 & 24 & 1 540.70 & 62.5 &
		69 &  111.66 &    326    \\
        3 & 1 &     2 749 453 &     53 809 & 21 &   125.25 & 4.01 &
		70 &   11.41 &     19.5  \\
        3 & 2 &   186 202 860 &    328 014 & 28 & 10 482.51 & 390 &
		77 &  222.26 &   2560    \\
	3 & 3 & 9 664 707 100 &  2 055 621 &    &          &      &
		84 & 5557.02 & 266000    \\
	\bottomrule
\end{tabular}
\caption{QRCU Spin Result Summary}
\label{tab:formal:promela:QRCU Spin Result Summary}
\end{table*}

참고를 위해, Table~\ref{tab:formal:promela:QRCU Spin Result Summary}
은 \co{-DCOLLAPSE} 와 \co{-DMA=N} 컴파일러 플래그가 사용되었을 때의 Spin 결과를
요약합니다.
메모리 사용량은 최소의 충분한 탐색 깊이를 통해 얻어졌고 \co{-DMA=N} 패러미터는
표에 보여져 있습니다.
\co{-DCOLLAPSE} 수행을 위한 해쉬테이블의 크기는 작은 상태 공간을 해슁하는데
너무 많은 메모리가 사용되는 것을 막기 위해 \co{./pan} 의 \co{-wN} 옵션을 통해
조정되었습니다.
따라서 메모리 사용량은
Table~\ref{tab:advsync:Memory Usage of QRCU Model} 에 보인 것보다 작은데,
해쉬테이블 크기는 \co{-w24} 의 기본값부터 시작합니다.
수행시간은 \Power{9} 서버에서의 것으로 \co{-DCOLAAPSE} 보다 \co{-DMA=N} 이 대략
열배 가까이 높은 CPU 오버헤드로 고생함을 보이지만, 다른 한편 메모리 오버헤드는
열배 이상 줄입니다.

여기까진 좋습니다.
하지만 업데이트 쓰레드와 읽기 스레드를 조금 더 늘리는 것은 \co{-DMA=N} 을
사용하더라도 메모리를 소모시킬 겁니다.\footnote{
	대안적으로, CPU 소모량은 지나쳐질 거니다.}
그러니 어떡하죠?
여기 몇가지 가능한 접근법이 있습니다:

\iffalse

For reference, Table~\ref{tab:formal:promela:QRCU Spin Result Summary}
summarizes the Spin results with \co{-DCOLLAPSE} and \co{-DMA=N}
compiler flags.
The memory usage is obtained with minimal sufficient
search depths and \co{-DMA=N} parameters shown in the table.
Hashtable sizes for \co{-DCOLLAPSE} runs are tweaked by
the \co{-wN} option of \co{./pan} to avoid using too much
memory hashing small state spaces.
Hence the memory usage is smaller than what is shown in
Table~\ref{tab:advsync:Memory Usage of QRCU Model}, where the
hashtable size starts from the default of \co{-w24}.
The runtime is from a \Power{9} server, which shows that \co{-DMA=N}
suffers up to about an order of magnitude higher CPU overhead
than does \co{-DCOLLAPSE}, but on the other hand reduces memory overhead
by well over an order of magnitude.

So far so good.
But adding a few more updaters or readers would exhaust memory, even
with \co{-DMA=N}.\footnote{
	Alternatively, the CPU consumption would become excessive.}
So what to do?
Here are some possible approaches:

\fi

\begin{enumerate}
\item	더 작은 수의 읽기 쓰레드와 업데이트 쓰레드가 일반적인 경우를 검증하기
	충분한지 봅니다.
\item	수작업으로 정확성을 증명합니다.
\item	더 적합한 도구를 사용합니다.
\item	분할해 정복합니다.

\iffalse

\item	See whether a smaller number of readers and updaters suffice
	to prove the general case.
\item	Manually construct a proof of correctness.
\item	Use a more capable tool.
\item	Divide and conquer.

\fi

\end{enumerate}

다음 섹션은 이 방법들 각각을 이야기 합니다.

\iffalse

The following sections discuss each of these approaches.

\fi

\subsubsection{How Many Readers and Updaters Are Really Needed?}
\label{sec:formal:How Many Readers and Updaters Are Really Needed?}

한가지 방법은 \co{qrcu_updater()} 의 Promela 코드를 주의깊게 들여다보고 유일한
전역적 상태 변화는 락 아래에서만 일어남을 깨닫는 겁니다.
따라서, 한번에 단 하나의 업데이트 쓰레드만 읽기 쓰레드나 다른 업데이트
쓰레드에게 보이는 상태를 고칠 수 있습니다.
이는 Promela 는 전체 상태 공간 탐색을 하므로 단일 업데이트 쓰레드에 의해
순차적으로 변화들이 일어남을 의미합니다.
따라서, 최대 두개의 업데이트 쓰레드가 필요합니다: 하나는 상태를 변화시키고 다른
하나는 혼란스러워지기 위해.

읽기 쓰레드와의 상황은 덜 분명한데, 각 읽기 쓰레드는 하나의 read-side 크리티컬
섹션을 수행하고는 종료되기 때문입니다.
빠른 수행 경로는 카운터에서 0 또는 1까지만 볼 수 있으므로 일반적인 읽기
쓰레드의 수는 제한되어 있다고 말할 수도 있겠습니다.
이는 조사해 볼 가치 있는 부분일텐데, 실제로 다음 섹션에서 이야기 될 완전한
정확성 증명을 이끕니다.

\iffalse

One approach is to look carefully at the Promela code for
\co{qrcu_updater()} and notice that the only global state
change is happening under the lock.
Therefore, only one updater at a time can possibly be modifying
state visible to either readers or other updaters.
This means that any sequences of state changes can be carried
out serially by a single updater due to the fact that Promela does a full
state-space search.
Therefore, at most two updaters are required: one to change state
and a second to become confused.

The situation with the readers is less clear-cut, as each reader
does only a single read-side critical section then terminates.
It is possible to argue that the useful number of readers is limited,
due to the fact that the fastpath must see at most a zero and a one
in the counters.
This is a fruitful avenue of investigation, in fact, it leads to
the full proof of correctness described in the next section.

\fi

\subsubsection{Alternative Approach: Proof of Correctness}
\label{sec:formal:Alternative Approach: Proof of Correctness}

비정형적인 증명은~\cite{PaulMcKenney2007QRCUpatch} 다음과 같습니다:

\iffalse

An informal proof~\cite{PaulMcKenney2007QRCUpatch}
follows:

\fi

\begin{enumerate}
\item	\co{synchronize_qrcu()} 가 너무 빨리 끝나려면 정의에 의해
	\co{synchronize_qrcu()} 의 전체 수행 중간에 최소 하나의 읽기 쓰레드가
	존재했어야만 합니다.
\item	이 읽기 쓰레드에 연관된 카운터는 이 시간 동안 최소 1 이었을 겁니다.
\item	\co{synchronize_qrcu()} 코드는 전체 시간 동안 이 카운터 중 하나는 최소
	1이었을 것을 강제합니다.
\item	따라서, 어느 시점에서든, 카운터들 중 하나는 최소 2이거나 두 카운터 모두
	최소 1이 됩니다.
\item	그러나, \co{synchronize_qrcu()} 빠른 수행경로 코드는 한번에 카운터들 중
	하나만 읽을 수 있습니다.
	따라서 빠른 수행경로 코드가 첫번째 카운터를 0인 동안 읽어오지만 카운터
	뒤집기와 경주를 해서 두번째 카운터는 1로 보일 수 있습니다.
\item	그런 경주 상황 동안엔 최대 하나의 읽기 쓰레드가 존재할 수 있는데,
	그렇지 않으면 합산은 2 이상이 되어 업데이트 쓰레드가 느린 수행경로를
	취하게 할 것이기 때문입니다.

\iffalse

\item	For \co{synchronize_qrcu()} to exit too early, then
	by definition there must have been at least one reader
	present during \co{synchronize_qrcu()}'s full
	execution.
\item	The counter corresponding to this reader will have been
	at least 1 during this time interval.
\item	The \co{synchronize_qrcu()} code forces at least one
	of the counters to be at least 1 at all times.
\item	Therefore, at any given point in time, either one of the
	counters will be at least 2, or both of the counters will
	be at least one.
\item	However, the \co{synchronize_qrcu()} fastpath code
	can read only one of the counters at a given time.
	It is therefore possible for the fastpath code to fetch
	the first counter while zero, but to race with a counter
	flip so that the second counter is seen as one.
\item	There can be at most one reader persisting through such
	a race condition, as otherwise the sum would be two or
	greater, which would cause the updater to take the slowpath.

\fi

\item	그러나 그 경주가 빠른 수행경로의 카운터들에 대한 첫번째 읽기에서,
	그리고 또다시 두번째 읽기에서 일어났다면, 두 카운터 뒤집기가 있었어야
	합니다.
\item	이 업데이트 쓰레드는 카운터를 한번만 뒤집으므로, 그리고 업데이트 쪽
	락이 한쌍의 업데이트 쓰레드가 동시에 카운터를 뒤집는 것을 방지하므로,
	빠른 수행경로 코드가 뒤집기와 두번 경주할 수 있는 유일한 방법은 첫번째
	업데이트 쓰레드가 완료되었을 경우 뿐입니다.
\item	하지만 첫번째 업데이트 쓰레드는 앞서서부터 존재해온 읽기 쓰레드들이
	완료되기 전까지는 완료되지 못합니다.
\item	다라서, 빠른 수행 경로가 카운터 뒤집기와 두번 경주하려면, 모든
	앞서서부터 존재해온 읽기 쓰레드들이 완료되었어야 해서 빠른 수행경로를
	취하는게 안전합니다.

\iffalse

\item	But if the race occurs on the fastpath's first read of the
	counters, and then again on its second read, there have
	to have been two counter flips.
\item	Because a given updater flips the counter only once, and
	because the update-side lock prevents a pair of updaters
	from concurrently flipping the counters, the only way that
	the fastpath code can race with a flip twice is if the
	first updater completes.
\item	But the first updater will not complete until after all
	pre-existing readers have completed.
\item	Therefore, if the fastpath races with a counter flip
	twice in succession, all pre-existing readers must have
	completed, so that it is safe to take the fastpath.

\fi

\end{enumerate}

물론, 모든 병렬 알고리즘이 이렇게 간단한 증명을 갖지는 않습니다.
그런 경우에는 더 많은 적합한 도구들을 모아야 합니다.

\iffalse

Of course, not all parallel algorithms have such simple proofs.
In such cases, it may be necessary to enlist more capable tools.

\fi

\subsubsection{Alternative Approach: More Capable Tools}
\label{sec:formal:Alternative Approach: More Capable Tools}

Promlea 와 Spin 이 상당히 유용하지만 더 적합한 많은 도구들이 사용가능한데, 특히
하드웨어 검증에 있어 그렇습니다.
이는 저수준 병렬 알고리즘들에서는 종종 그렇듯 여러분의 알고리즘을 하드웨어
설계용 VHDL 언어로 변환할 수 있다면 이 도구들을 여러분의 코드에 적용할 수
있음을 의미합니다 (예를 들어, 이는 첫번째 realtime RCU 알고리즘을 위해
행해졌습니다).
그러나, 그런 도구들은 상당히 비용이 높을 수 있습니다.

상용 멀티프로세싱의 발전이 결국은 멋진 상태 공간 축소 기능을 갖춘 강력한 자유
소프트웨어 모델 검사기의 출현을 가능하게 할수도 있겠으나, 지금 당장은 도움이
되지 않습니다.

이와는 별개로, 고정된 양의 메모리를 필요로 하는 대략적 탐색을 지원하는 Spin
기능이 있습니다만, 저는 병렬 알고리즘을 검증하는데 있어 근사치를 믿지는
못했습니다.

또다른 방법은 분할해 정복하기일 수 있습니다.

\iffalse

Although Promela and Spin are quite useful,
much more capable tools are available, particularly for verifying
hardware.
This means that if it is possible to translate your algorithm
to the hardware-design VHDL language, as it often will be for
low-level parallel algorithms, then it is possible to apply these
tools to your code (for example, this was done for the first
realtime RCU algorithm).
However, such tools can be quite expensive.

Although the advent of commodity multiprocessing
might eventually result in powerful free-software model-checkers
featuring fancy state-space-reduction capabilities,
this does not help much in the here and now.

As an aside, there are Spin features that support approximate searches
that require fixed amounts of memory, however, I have never been able
to bring myself to trust approximations when verifying parallel
algorithms.

Another approach might be to divide and conquer.

\fi

\subsubsection{Alternative Approach: Divide and Conquer}
\label{sec:formal:Alternative Approach: Divide and Conquer}

커다른 병렬 알고리즘을 개별적으로 증명 가능한 작은 조각들로 쪼개는 게 종종
가능합니다.
예를 들어, 100억개의 상태를 갖는 모델은 두개의 100,000개 상태 모델로 쪼개질
수도 있습니다.
이 방법을 취하는 것은 Promela 같은 도구가 여러분의 알고리즘을 검증하기 쉽게
할뿐만 아니라 여러분의 알고리즘을 이해하기 쉽게 해주기도 합니다.

\iffalse

It is often possible to break down a larger parallel algorithm into
smaller pieces, which can then be proven separately.
For example, a 10-billion-state model might be broken into a pair
of 100,000-state models.
Taking this approach not only makes it easier for tools such as
Promela to verify your algorithms, it can also make your algorithms
easier to understand.

\fi

\subsubsection{Is QRCU Really Correct?}
\label{sec:formal:Is QRCU Really Correct?}

QRCU 는 정말로 정확할까요?
우리는 그렇다고 말하는 Promela 기반의 기계적 증명과 손을 통한 증명을
보았습니다.
하지만, \pplsur{Jade}{Alglave} et al.~\cite{JadeAlglave2013-cav} 의 논문은
다르게 말합니다 (page~12 아래쪽의 Section~5.1 을 보세요).
뭐가 맞을까요?

둘 다 맞는 것으로 드러났습니다!
QRCU 가 정형 검증 벤치마크에 추가되었을 때, 그것의 메모리 배리어는 생략되었고,
따라서 버그 있는 버전의 QRCU 가 되었습니다.
따라서 여기서의 진짜 뉴스는 여러 정형 검증 도구들이 이 버그있는 QRCU 를
잘못되게도 올바르다고 검증했습니다.
그리고 이게 정형 검증 도구들 그 자체들도 버그가 투입된 버전의 코드를 통해
검증되어야 하는 이유입니다.
특정 도구가 투입된 버그를 찾지 못한다면, 그 도구는 분명 믿지 못할 것입니다.

\iffalse

Is QRCU really correct?
We have a Promela-based mechanical proof and a by-hand proof that both
say that it is.
However, a paper by \pplsur{Jade}{Alglave} et al.~\cite{JadeAlglave2013-cav}
says otherwise (see Section~5.1 of the paper at the bottom of page~12).
Which is it?

It turns out that both are correct!
When QRCU was added to a suite of formal-verification benchmarks,
its memory barriers were omitted, thus resulting in a buggy version
of QRCU\@.
So the real news here is that a number of formal-verification tools
incorrectly proved this buggy QRCU correct.
And this is why formal-verification tools themselves should be tested
using bug-injected versions of the code being verified.
If a given tool cannot find the injected bugs, then that tool is
clearly untrustworthy.

\fi

\QuickQuiz{
	하지만 다른 정형 검증 도구들은 종종 특정 종류의 버그를 찾기 위해
	설계됩니다.
	예를 들어, 매우 적은 정형 검증 도구들은 명세서 내의 오류를 찾을 겁니다.
	그러니 이 ``분명 믿지 못할만 하다'' 는 좀 센 말 아닌가요?

	\iffalse

	But different formal-verification tools are often designed to
	locate particular classes of bugs.
	For example, very few formal-verification tools will find
	an error in the specification.
	So isn't this ``clearly untrustworthy'' judgment a bit harsh?

	\fi

}\QuickQuizAnswer{
	많은 정형 검증 도구들이 어떤 방법으로 특수화 되었음은 분명한
	사실입니다.
	예를 들어, Promela 는 실제 메모리 모델을 다루지 않고 (그게 Promela 에
	프로그램이 될 수 있긴
	합니다~\cite{Desnoyers:2013:MSM:2506164.2506174}),
	CMBC~\cite{EdmundClarke2004CBMC} 는 확률적 멈춤과 데드락을 탐지하지
	못하며,
	Nidhugg~\cite{CarlLeonardsson2014Nidhugg} 는 데이터 비결정성에 관한
	버그를 탐지하지 못합니다.
	하지만 이는 이 도구들이 찾게끔 설계되지 않은 버그를 찾을 수 있다고는
	믿어질 수 없음을 의미합니다.

	그리고 따라서 정형 검증 도구를 작성하는 사람들은 ``라벨에 진실을
	말해야'' 하고, 그들의 도구들이 어떤 종류의 버그를 찾고 못찾는지 분명히
	밝혀야 합니다.
	그러지 않는다면, 사용자는 그 도구가 어떤 버그를 탐지하지 못함을 처음
	발견했을 때 그 도구에 대한 매우 거칠고 공개적인 비난을 할 겁니다.
	그래요, 그래요, 최선을 다했다고 할 수 있는 뭔가가 있지만 적절한
	면책조항 없이 무언가를 지나치게 이야기 하는 것은 여러분의 도구가
	그로부터 회복될 수도 그러지 못할수도 있는 부정적 반응을 쉽게 야기할 수
	있습니다.

	경고했어요!

	\iffalse

	It is certainly true that many formal-verification tools are
	specialized in some way.
	For example, Promela does not handle realistic memory models
	(though they can be programmed into
	Promela~\cite{Desnoyers:2013:MSM:2506164.2506174}),
	CBMC~\cite{EdmundClarke2004CBMC} does not detect probabilistic
	hangs and deadlocks, and
	Nidhugg~\cite{CarlLeonardsson2014Nidhugg} does not detect
	bugs involving data nondeterminism.
	But this means that these tools cannot be trusted to find
	bugs that they are not designed to locate.

	And therefore people creating formal-verification tools should
	``tell the truth on the label'', clearly calling out what
	classes of bugs their tools can and cannot detect.
	Otherwise, the first time a practitioner finds a tool
	failing to detect a bug, that practitioner is likely to
	make extremely harsh and extremely public denunciations
	of that tool.
	Yes, yes, there is something to be said for putting your
	best foot forward, but putting it too far forward without
	appropriate disclaimers can easily trigger a land mine of
	negative reaction that your tool might or might not be able
	to recover from.

	You have been warned!

	\fi

}\QuickQuizEnd

따라서, 여러분이 QRCU 를 사용하려 한다면, 주의하세요.
그것의 정확성 증명 그 자체는 정확할수도 아닐수도 있습니다.
이는 Donal Knuth 가 무척 오래전 이야기 했듯 정형 검증이 테스트를 완전히 대체할
수 없는 이유 중 하나입니다.

\iffalse

Therefore, if you do intend to use QRCU, please take care.
Its proofs of correctness might or might not themselves be correct.
Which is one reason why formal verification is unlikely to
completely replace testing, as Donald Knuth pointed out so long ago.

\fi

\QuickQuiz{
	여기 설명된 QRCU 알고리즘을 위한 두개의 개별적 정확성 증명이 있고
	부정확성 증명은 다른 알고리즘에 대한 것이었는데 왜 의심의 여지가 있죠?

	\iffalse

	Given that we have two independent proofs of correctness for
	the QRCU algorithm described herein, and given that the
	proof of incorrectness covers what is known to be a different
	algorithm, why is there any room for doubt?

	\fi

}\QuickQuizAnswer{
	언제나 의심의 여지는 있습니다.
	이 경우, 그 두개의 정확성 증명은 실제 세계의 메모리 순서 모델의
	정형화를 사용했으므로 이 두개의 증명들은 올바르지 않은 메모리 순서 규칙
	가정에 기반했을 가능성을 높입니다.
	더 나아가서, 두 증명 모두 같은 사람에 의해 구축되었으므로 동일한 오류를
	가지고 있을 가능성이 꽤 있습니다.
	다시 말하지만, 항상 의심의 여지가 있습니다.

	\iffalse

	There is always room for doubt.
	In this case, it is important to keep in mind that the two proofs
	of correctness preceded the formalization of real-world memory
	models, raising the possibility that these two proofs are based
	on incorrect memory-ordering assumptions.
	Furthermore, since both proofs were constructed by the same person,
	it is quite possible that they contain a common error.
	Again, there is always room for doubt.

	\fi

}\QuickQuizEnd
