% together/count.tex
% SPDX-License-Identifier: CC-BY-SA-3.0

\section{Counter Conundrums}
\label{sec:together:Counter Conundrums}

이 섹션은 일부 카운터 문제에 대한 해결책들을 개략적으로 소개합니다.
\iffalse

This section outlines possible solutions to some counter conundrums.
\fi

\subsection{Counting Updates}
\label{sec:together:Counting Updates}

Schr\"odinger (
Section~\ref{sec:datastruct:Motivating Application} 를 참고하세요) 가 각 동물의
업데이트 횟수를 카운트 하고자 하고, 이 업데이트들은 데이터 원소별 락을 통해
동기화 된다고 해봅시다.
이 카운팅은 어떻게 해야 최선일까요?

물론,
Chapter~\ref{chp:Counting}
에서의 카운팅 알고리즘들 중 무엇이든 고려해 볼 수 있을 겁니다만, 이경우에
최적의 방법은 훨씬 간단합니다.
카운터를 각각의 데이터 원소에 위치시키고, 그 원소의 락으로 보호한 채로 카운터를
증가시키세요!
\iffalse

Suppose that Schr\"odinger (see
Section~\ref{sec:datastruct:Motivating Application})
wants to count the number of updates for each animal,
and that these updates are synchronized using a per-data-element lock.
How can this counting best be done?

Of course, any number of counting algorithms from
Chapter~\ref{chp:Counting}
might be considered, but the optimal approach is much simpler in this case.
Just place a counter in each data element, and increment it under the
protection of that element's lock!
\fi

\subsection{Counting Lookups}
\label{sec:together:Counting Lookups}

Schr\"odinger 가 각 동물에의 검색의 횟수 역시 세고 싶어하며, 검색은 RCU 를 통해
보호되고 있다고 생각해 봅시다.
이런 카운팅은 어떻게 하는게 최선일까요?

한가지 방법은
Section~\ref{sec:together:Counting Updates}
에서 이야기한 원소별 락을 사용해서 검색 카운터를 보호하는 것일 겁니다.
불행히도, 이는 모든 검색이 락을 잡을 것을 필요로 해서, 커다란 시스템에서는
상당한 병목지점이 될겁니다.

또다른 방법은 카운팅이 ``안된다고 하기'' 로, \co{noatime} 마운트 옵션의 예를
따릅니다.
이 방법이 사용 가능한 경우라면, 이게 최고의 방법임이 분명합니다: 무엇보다,
아무것도 하지 않는것보다 빠른 것은 없습니다.
검색 카운트가 면제될 수 없다면, 마저 읽어주세요!
\iffalse

Suppose that Schr\"odinger also wants to count the number of lookups for
each animal, where lookups are protected by RCU.
How can this counting best be done?

One approach would be to protect a lookup counter with the per-element
lock, as discussed in
Section~\ref{sec:together:Counting Updates}.
Unfortunately, this would require all lookups to acquire this lock,
which would be a severe bottleneck on large systems.

Another approach is to ``just say no'' to counting, following the example
of the \co{noatime} mount option.
If this approach is feasible, it is clearly the best:  After all, nothing
is faster than doing nothing.
If the lookup count cannot be dispensed with, read on!
\fi

Chapter~\ref{chp:Counting} 에 나온 모든 카운터들이 사용될 수 있는데,
Section~\ref{sec:count:Statistical Counters}
에서 설명한 통계적 카운터가 아마도 가장 흔한 선택이 될겁니다.
하지만, 이는 커다란 메모리 사용량을 초래합니다: 요구되는 카운터들의 갯수는
데이터 원소들의 갯수에 쓰레드의 갯수를 곱한 값이 됩니다.

이런 메모리 오버헤드가 지나치다면, CPU 별 카운터 대신 소켓별 카운터를 유지하고
Figure~\ref{fig:datastruct:Read-Only Hash-Table Performance For Schroedinger's Zoo, 60 CPUs}.
에 보여진 것처럼 해시 테이블 성능 결과를 주목하는 방법이 있습니다.
이는 카운터 증가는 어토믹 오퍼레이션이 될 것을 필요로 할 것인데, 특히 특정
쓰레드가 언제든 다른 CPU 로 옮겨갈 수 있는 사용자 모드 수행에서는 특히
그렇습니다.

만약 일부 원소들이 매우 자주 검색된다면, 쓰레드별 로그를 유지하며 특정 원소를
위한 여러 로그 항목들이 병합될 수 있는 식으로 업데이트들을 몰아서 처리하는
방법들이 있습니다.
특정 로그 항목이 충분히 많은 횟수 증가되었거나 충분히 많은 시간이 지났다면,
해당 로그 항목은 연관된 데이터 항목에 적용될 수 있을 겁니다.
Silas Boyd-Wickizer 는 이런 노선을 정형화
시켰습니다~\cite{SilasBoydWickizerPhD}.
\iffalse

Any of the counters from
Chapter~\ref{chp:Counting}
could be pressed into service, with the statistical counters described in
Section~\ref{sec:count:Statistical Counters}
being perhaps the most common choice.
However, this results in a large memory footprint: The number of counters
required is the number of data elements multiplied by the number of
threads.

If this memory overhead is excessive, then one approach is to keep
per-socket counters rather than per-CPU counters,
with an eye to the hash-table performance results depicted in
Figure~\ref{fig:datastruct:Read-Only Hash-Table Performance For Schroedinger's Zoo, 60 CPUs}.
This will require that the counter increments be atomic operations,
especially for user-mode execution where a given thread could migrate
to another CPU at any time.

If some elements are looked up very frequently, there are a number
of approaches that batch updates by maintaining a per-thread log,
where multiple log entries for a given element can be merged.
After a given log entry has a sufficiently large increment or after
sufficient time has passed, the log entries may be applied to the
corresponding data elements.
Silas Boyd-Wickizer has done some work formalizing this
notion~\cite{SilasBoydWickizerPhD}.
\fi
