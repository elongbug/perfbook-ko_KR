% together/together.tex
% mainfile: ../perfbook.tex
% SPDX-License-Identifier: CC-BY-SA-3.0

\QuickQuizChapter{chp:Putting It All Together}{Putting It All Together}{qqztogether}
%
\Epigraph{You don't learn how to shoot and then learn how to launch
	  and then learn to do a controlled spin---you learn to
	  launch-shoot-spin.}{\emph{``Ender's Shadow'', Orson Scott Card}}

% And the paragraph preceding this is also instructive:
% ``I may be pissed off, but that doesn't mean I can't learn.''

이 \lcnamecref{chp:Putting It All Together} 는 동시성 프로그래밍 퍼즐들에 대한
힌트를 약간 제공합니다.
\Cref{sec:together:Counter Conundrums}
는 카운터 수수께끼를 생각해 보고,
\cref{sec:together:Refurbish Reference Counting}
는 레퍼런스 카운팅을 다시 들여다보며,
\cref{sec:together:Hazard-Pointer Helpers}
는 해저드 포인터를 돕고,
\cref{sec:together:Sequence-Locking Specials}
는 시퀀스 락킹의 특수한 경우들을 요약하며, 마지막으로
\cref{sec:together:RCU Rescues}
는 RCU 의 구조를 알아봅니다.

\iffalse

This \lcnamecref{chp:Putting It All Together}
gives some hints on concurrent-programming puzzles.
\Cref{sec:together:Counter Conundrums}
considers counter conundrums,
\cref{sec:together:Refurbish Reference Counting}
refurbishes reference counting,
\cref{sec:together:Hazard-Pointer Helpers}
helps with hazard pointers,
\cref{sec:together:Sequence-Locking Specials}
surmises on sequence-locking specials,
and finally
\cref{sec:together:RCU Rescues}
reflects on RCU rescues.

\fi

% together/count.tex
% mainfile: ../perfbook.tex
% SPDX-License-Identifier: CC-BY-SA-3.0

\section{Counter Conundrums}
\label{sec:together:Counter Conundrums}
%
\epigraph{Ford carried on counting quietly.
	  This is about the most aggressive thing you can do to a
	  computer, the equivalent of going up to a human being and saying
	  ``Blood \dots blood \dots blood \dots blood \dots''}
	 {\emph{Douglas Adams}}

이 \lcnamecref{sec:together:Counter Conundrums} 는 카운터 수수께끼들에 대한
해결책들을 정리해 봅니다.

\iffalse

This \lcnamecref{sec:together:Counter Conundrums}
outlines solutions to counter conundrums.

\fi

\subsection{Counting Updates}
\label{sec:together:Counting Updates}

Schr\"odinger
(\cref{sec:datastruct:Motivating Application} 참고) 가 각 동물의 업데이트
횟수를 세고 싶어하며, 이 업데이트는 데이터 원소별 락을 이용해 동기화 된다고
해봅시다.
이 카운팅은 어떻게 해야 가장 잘 할 수 있을까요?

물론, \cref{chp:Counting} 에서의 카운팅 알고리즘들이 얼마든지 충분할 수도
있지만, 최적의 방법은 상당히 간단합니다.
각 데이터 원소에 카운터를 놓고, 그 원소의 락의 보호 아래 그 카운터를 증가시키는
겁니다!

읽기 쓰레드가 락 없이 이 숫자에 접근하면, 업데이트 쓰레드는 이 카운터를
업데이트 하는데에 \co{WRITE_ONCE()} 를 사용하고 락 없는 읽기 쓰레드는 이를 읽기
위해 \co{READ_ONCE()} 를 사용해야 합니다.

\iffalse

Suppose that Schr\"odinger (see
\cref{sec:datastruct:Motivating Application})
wants to count the number of updates for each animal,
and that these updates are synchronized using a per-data-element lock.
How can this counting best be done?

Of course, any number of counting algorithms from \cref{chp:Counting}
might qualify, but the optimal approach is quite simple.
Just place a counter in each data element, and increment it under the
protection of that element's lock!

If readers access the count locklessly, then updaters should use
\co{WRITE_ONCE()} to update the counter and lockless readers should
use \co{READ_ONCE()} to load it.

\fi

\subsection{Counting Lookups}
\label{sec:together:Counting Lookups}

Schr\"odinger 는 각 동물을 위한 탐색 횟수도 세고 싶어하며, 탐색은 RCU 로
보호된다고 해봅시다.
이 경우에 무엇이 최선의 카운팅 방법일까요?

한가지 방법은
\cref{sec:together:Counting Updates} 에서 이야기 되었듯 이 탐색 카운터를 원소별
락을 이용해 보호하는 것입니다.
불행히도, 이는 각 탐색이 락을 획득할 것을 필요로 해서 큰 시스템에서는 상당한
병목지점이 될 겁니다.

또다른 방법은 카운팅이 ``안된다고 말하기'' 로, \co{noatime} mount 옵션의 예를
따릅니다.
이 방법이 적용 가능하다면, 이게 분명 최선입니다:  어쨌건, 아무것도 안하는
것보다 빠른건 없습니다.
이 탐색 카운트가 없어질 수 없다면, 계속 읽으세요!

\iffalse

Suppose that Schr\"odinger also wants to count the number of lookups for
each animal, where lookups are protected by RCU\@.
How can this counting best be done?

One approach would be to protect a lookup counter with the per-element
lock, as discussed in \cref{sec:together:Counting Updates}.
Unfortunately, this would require all lookups to acquire this lock,
which would be a severe bottleneck on large systems.

Another approach is to ``just say no'' to counting, following the example
of the \co{noatime} mount option.
If this approach is feasible, it is clearly the best:  After all, nothing
is faster than doing nothing.
If the lookup count cannot be dispensed with, read on!

\fi

\Cref{chp:Counting} 에서의 카운터들이 무엇이든 사용될 수 있겠는데,
\cref{sec:count:Statistical Counters} 의 통계적 카운터가 아마 가장 흔한 선택일
겁니다.
하지만, 이는 큰 메모리 사용량을 초래합니다: 필요한 카운터의 수는 데이터 원소의
수 곱하기 쓰레드의 수입니다.

이 메모리 오버헤드가 지나치다면,
\cref{fig:datastruct:Read-Only Hash-Table Performance For Schroedinger's Zoo; 448 CPUs}
에 보인 해쉬 테이블 성능을 참고해 CPU 별 카운터 대신 코어별 또는 소켓별
카운터를 두는 것이 한 방편이 되겠습니다.
이는 이 카운터 값 증가가 어토믹 오퍼레이션이 될 것을 필요로 하는데, 특정
쓰레드가 언제든 다른 CPU 로 옮겨질 수 있는 사용자 모드 수행에서는 특히
그렇습니다.

어떤 원소들이 매우 빈번하게 탐색된다면, 쓰레드별로 특정 원소를 위한 로그
항목들이 병합될 수 있는 로그를 두어서 업데이트를 몰아서 하는 방법이 여럿
있습니다.
특정 로그 항목이 충분히 큰 값 증가를 가지거나 충분한 시간이 흐른 후에는 이 로그
항목들이 연관된 데이터 원소에 적용될 수 있습니다.
Silas Boyd-Wickizer 는 이 방법들을 정형화
시켰습니다~\cite{SilasBoydWickizerPhD}.

\iffalse

Any of the counters from \cref{chp:Counting}
could be pressed into service, with the statistical counters described in
\cref{sec:count:Statistical Counters} being perhaps the most common choice.
However, this results in a large memory footprint: The number of counters
required is the number of data elements multiplied by the number of
threads.

If this memory overhead is excessive, then one approach is to keep
per-core or even per-socket counters rather than per-CPU counters,
with an eye to the hash-table performance results depicted in
\cref{fig:datastruct:Read-Only Hash-Table Performance For Schroedinger's Zoo; 448 CPUs}.
This will require that the counter increments be atomic operations,
especially for user-mode execution where a given thread could migrate
to another CPU at any time.

If some elements are looked up very frequently, there are a number
of approaches that batch updates by maintaining a per-thread log,
where multiple log entries for a given element can be merged.
After a given log entry has a sufficiently large increment or after
sufficient time has passed, the log entries may be applied to the
corresponding data elements.
Silas Boyd-Wickizer has done some work formalizing this
notion~\cite{SilasBoydWickizerPhD}.

\fi

% together/refcnt.tex

\section{Refurbish Reference Counting}
\label{sec:together:Refurbish Reference Counting}

Although reference counting is a conceptually simple technique,
many devils hide in the details when it is applied to concurrent
software.
After all, if the object was not subject to premature disposal,
there would be no need for the reference counter in the first place.
But if the object can be disposed of, what prevents disposal during
the reference-acquisition process itself?

There are a number of ways to refurbish reference counters for
use in concurrent software, including:

\begin{enumerate}
\item	A lock residing outside of the object must be held while
	manipulating the reference count.
\item	The object is created with a non-zero reference count, and new
	references may be acquired only when the current value of
	the reference counter is non-zero.
	If a thread does not have a reference to a given object,
	it may obtain one with the help of another thread that
	already has a reference.
\item	An existence guarantee is provided for the object, preventing
	it from being freed while some other
	entity might be attempting to acquire a reference.
	Existence guarantees are often provided by automatic
	garbage collectors, and, as will be seen in
	Section~\ref{sec:defer:Read-Copy Update (RCU)}, by RCU.
\item	A type-safety guarantee is provided for the object.
	An additional identity check must be performed once
	the reference is acquired.
	Type-safety guarantees can be provided by special-purpose
	memory allocators, for example, by the
	\co{SLAB_DESTROY_BY_RCU} feature within the Linux kernel,
	as will be seen in Section~\ref{sec:defer:Read-Copy Update (RCU)}.
\end{enumerate}

Of course, any mechanism that provides existence guarantees
by definition also provides type-safety guarantees.
This section will therefore group the last two answers together under the
rubric of RCU, leaving us with three general categories of
reference-acquisition protection: Reference counting, sequence
locking, and RCU.

\QuickQuiz{}
	Why not implement reference-acquisition using
	a simple compare-and-swap operation that only
	acquires a reference if the reference counter is
	non-zero?
\QuickQuizAnswer{
	Although this can resolve the race between the release of
	the last reference and acquisition of a new reference,
	it does absolutely nothing to prevent the data structure
	from being freed and reallocated, possibly as some completely
	different type of structure.
	It is quite likely that the ``simple compare-and-swap
	operation'' would give undefined results if applied to the
	differently typed structure.

	In short, use of atomic operations such as compare-and-swap
	absolutely requires either type-safety or existence guarantees.
} \QuickQuizEnd

\begin{table}[tb]
\footnotesize
\centering
\begin{tabular}{l||c|c|c}
	& \multicolumn{3}{c}{Release Synchronization} \\
	\cline{2-4}
	Acquisition     &         & Reference &     \\
	Synchronization & Locking & Counting  & RCU \\
	\hline
	\hline
	Locking		& -	  & CAM	      & CA  \\
	\hline
	Reference	& A	  & AM	      & A   \\
	Counting	&  	  &   	      &     \\
	\hline
	RCU		& CA	  & MCA	      & CA  \\
\end{tabular}
\caption{Reference Counting and Synchronization Mechanisms}
\label{tab:together:Reference Counting and Synchronization Mechanisms}
\end{table}

Given that the key reference-counting issue
is synchronization between acquisition
of a reference and freeing of the object, we have nine possible
combinations of mechanisms, as shown in
Table~\ref{tab:together:Reference Counting and Synchronization Mechanisms}.
This table
divides reference-counting mechanisms into the following broad categories:
\begin{enumerate}
\item	Simple counting with neither atomic operations, memory
	barriers, nor alignment constraints \makebox{(``-'')}.
\item	Atomic counting without memory barriers (``A'').
\item	Atomic counting, with memory barriers required only on release
	(``AM'').
\item	Atomic counting with a check combined with the atomic acquisition
	operation, and with memory barriers required only on release
	(``CAM'').
\item	Atomic counting with a check combined with the atomic acquisition
	operation (``CA'').
\item	Atomic counting with a check combined with the atomic acquisition
	operation, and with memory barriers also required on acquisition
	(``MCA'').
\end{enumerate}
However, because all Linux-kernel atomic operations that return a
value are defined to contain memory barriers,\footnote{
	With \co{atomic_read()} and \co{ATOMIC_INIT()} being the
	exceptions that prove the rule.}
all release operations
contain memory barriers, and all checked acquisition operations also
contain memory barriers.
Therefore, cases ``CA'' and ``MCA'' are equivalent to ``CAM'', so that
there are sections below for only the first four cases:
\makebox{``-''}, ``A'', ``AM'', and ``CAM''.
The Linux primitives that support reference counting are presented in
Section~\ref{sec:together:Linux Primitives Supporting Reference Counting}.
Later sections cite optimizations that can improve performance
if reference acquisition and release is very frequent, and the
reference count need be checked for zero only very rarely.

\subsection{Implementation of Reference-Counting Categories}
\label{sec:together:Implementation of Reference-Counting Categories}

Simple counting protected by locking (\makebox{``-''}) is described in
Section~\ref{sec:together:Simple Counting},
atomic counting with no memory barriers (``A'') is described in
Section~\ref{sec:together:Atomic Counting},
atomic counting with acquisition memory barrier (``AM'') is described in
Section~\ref{sec:together:Atomic Counting With Release Memory Barrier},
and
atomic counting with check and release memory barrier (``CAM'') is described in
Section~\ref{sec:together:Atomic Counting With Check and Release Memory Barrier}.

\subsubsection{Simple Counting}
\label{sec:together:Simple Counting}

Simple counting, with neither atomic operations nor memory barriers,
can be used when the reference-counter acquisition and release are
both protected by the same lock.
In this case, it should be clear that the reference count itself
may be manipulated non-atomically, because the lock provides any
necessary exclusion, memory barriers, atomic instructions, and disabling
of compiler optimizations.
This is the method of choice when the lock is required to protect
other operations in addition to the reference count, but where
a reference to the object must be held after the lock is released.
Figure~\ref{fig:together:Simple Reference-Count API} shows a simple
API that might be used to implement simple non-atomic reference
counting---although simple reference counting is almost always
open-coded instead.

\begin{figure}[tbp]
{ \scriptsize
\begin{verbbox}
  1 struct sref {
  2   int refcount;
  3 };
  4
  5 void sref_init(struct sref *sref)
  6 {
  7   sref->refcount = 1;
  8 }
  9
 10 void sref_get(struct sref *sref)
 11 {
 12   sref->refcount++;
 13 }
 14
 15 int sref_put(struct sref *sref,
 16              void (*release)(struct sref *sref))
 17 {
 18   WARN_ON(release == NULL);
 19   WARN_ON(release == (void (*)(struct sref *))kfree);
 20
 21   if (--sref->refcount == 0) {
 22     release(sref);
 23     return 1;
 24   }
 25   return 0;
 26 }
\end{verbbox}
}
\centering
\theverbbox
\caption{Simple Reference-Count API}
\label{fig:together:Simple Reference-Count API}
\end{figure}

\subsubsection{Atomic Counting}
\label{sec:together:Atomic Counting}

Simple atomic counting may be used in cases where any CPU acquiring
a reference must already hold a reference.
This style is used when a single CPU creates an object for its
own private use, but must allow other CPU, tasks, timer handlers,
or I/O completion handlers that it later spawns to also access this object.
Any CPU that hands the object off must first acquire a new reference
on behalf of the recipient object.
In the Linux kernel, the \co{kref} primitives are used to implement
this style of reference counting, as shown in
Figure~\ref{fig:together:Linux Kernel kref API}.

Atomic counting is required
because locking is not used to protect all reference-count operations,
which means that it is possible for two different CPUs to concurrently
manipulate the reference count.
If normal increment and decrement were used, a pair of CPUs might both
fetch the reference count concurrently, perhaps both obtaining
the value ``3''.
If both of them increment their value, they will both obtain ``4'',
and both will store this value back into the counter.
Since the new value of the counter should instead be ``5'', one
of the two increments has been lost.
Therefore, atomic operations must be used both for counter increments
and for counter decrements.

If releases are guarded by locking or RCU,
memory barriers are \emph{not} required, but for different reasons.
In the case of locking, the locks provide any needed memory barriers
(and disabling of compiler optimizations), and the locks also
prevent a pair of releases from running concurrently.
In the case of RCU, cleanup must be deferred until all currently
executing RCU read-side critical sections have completed, and
any needed memory barriers or disabling of compiler optimizations
will be provided by the RCU infrastructure.
Therefore, if two CPUs release the final two references concurrently,
the actual cleanup will be deferred until both CPUs exit their
RCU read-side critical sections.

\QuickQuiz{}
	Why isn't it necessary to guard against cases where one CPU
	acquires a reference just after another CPU releases the last
	reference?
\QuickQuizAnswer{
	Because a CPU must already hold a reference in order
	to legally acquire another reference.
	Therefore, if one CPU releases the last reference,
	there cannot possibly be any CPU that is permitted
	to acquire a new reference.
	This same fact allows the non-atomic check in line~22
	of Figure~\ref{fig:together:Linux Kernel kref API}.
} \QuickQuizEnd

\begin{figure}[tbp]
{ \scriptsize
\begin{verbbox}
  1 struct kref {
  2   atomic_t refcount;
  3 };
  4 
  5 void kref_init(struct kref *kref)
  6 {
  7   atomic_set(&kref->refcount, 1);
  8 }
  9 
 10 void kref_get(struct kref *kref)
 11 {
 12   WARN_ON(!atomic_read(&kref->refcount));
 13   atomic_inc(&kref->refcount);
 14 }
 15 
 16 static inline int
 17 kref_sub(struct kref *kref, unsigned int count,
 18          void (*release)(struct kref *kref))
 19 {
 20   WARN_ON(release == NULL);
 21 
 22   if (atomic_sub_and_test((int) count,
 23                           &kref->refcount)) {
 24     release(kref);
 25     return 1;
 26   }
 27   return 0;
 28 }
\end{verbbox}
}
\centering
\theverbbox
\caption{Linux Kernel kref API}
\label{fig:together:Linux Kernel kref API}
\end{figure}

The \co{kref} structure itself, consisting of a single atomic
data item, is shown in lines~1-3 of
Figure~\ref{fig:together:Linux Kernel kref API}.
The \co{kref_init()} function on lines~5-8 initializes the counter
to the value ``1''.
Note that the \co{atomic_set()} primitive is a simple
assignment, the name stems from the data type of \co{atomic_t}
rather than from the operation.
The \co{kref_init()} function must be invoked during object creation,
before the object has been made available to any other CPU.

The \co{kref_get()} function on lines~10-14 unconditionally atomically
increments the counter.
The \co{atomic_inc()} primitive does not necessarily explicitly
disable compiler
optimizations on all platforms, but the fact that the \co{kref}
primitives are in a separate module and that the Linux kernel build
process does no cross-module optimizations has the same effect.

The \co{kref_sub()} function on lines~16-28 atomically decrements the
counter, and if the result is zero, line~24 invokes the specified
\co{release()} function and line~25 returns, informing the caller
that \co{release()} was invoked.
Otherwise, \co{kref_sub()} returns zero, informing the caller that
\co{release()} was not called.

\QuickQuiz{}
	Suppose that just after the \co{atomic_sub_and_test()}
	on line~22 of
	Figure~\ref{fig:together:Linux Kernel kref API} is invoked,
	that some other CPU invokes \co{kref_get()}.
	Doesn't this result in that other CPU now having an illegal
	reference to a released object?
\QuickQuizAnswer{
	This cannot happen if these functions are used correctly.
	It is illegal to invoke \co{kref_get()} unless you already
	hold a reference, in which case the \co{kref_sub()} could
	not possibly have decremented the counter to zero.
} \QuickQuizEnd

\QuickQuiz{}
	Suppose that \co{kref_sub()} returns zero, indicating that
	the \co{release()} function was not invoked.
	Under what conditions can the caller rely on the continued
	existence of the enclosing object?
\QuickQuizAnswer{
	The caller cannot rely on the continued existence of the
	object unless it knows that at least one reference will
	continue to exist.
	Normally, the caller will have no way of knowing this, and
	must therefore carefullly avoid referencing the object after
	the call to \co{kref_sub()}.
} \QuickQuizEnd

\QuickQuiz{}
	Why not just pass \co{kfree()} as the release function?
\QuickQuizAnswer{
	Because the \co{kref} structure normally is embedded in
	a larger structure, and it is necessary to free the entire
	structure, not just the \co{kref} field.
	This is normally accomplished by defining a wrapper function
	that does a \co{container_of()} and then a \co{kfree()}.
} \QuickQuizEnd

\subsubsection{Atomic Counting With Release Memory Barrier}
\label{sec:together:Atomic Counting With Release Memory Barrier}

This style of reference is used in the Linux kernel's networking
layer to track the destination caches that are used in packet routing.
The actual implementation is quite a bit more involved; this section
focuses on the aspects of \co{struct dst_entry} reference-count
handling that matches this use case,
shown in Figure~\ref{fig:together:Linux Kernel dst-clone API}.

\begin{figure}[tbp]
{ \scriptsize
\begin{verbbox}
  1 static inline
  2 struct dst_entry * dst_clone(struct dst_entry * dst)
  3 {
  4   if (dst)
  5     atomic_inc(&dst->__refcnt);
  6   return dst;
  7 }
  8
  9 static inline
 10 void dst_release(struct dst_entry * dst)
 11 {
 12   if (dst) {
 13     WARN_ON(atomic_read(&dst->__refcnt) < 1);
 14     smp_mb__before_atomic_dec();
 15     atomic_dec(&dst->__refcnt);
 16   }
 17 }
\end{verbbox}
}
\centering
\theverbbox
\caption{Linux Kernel dst\_clone API}
\label{fig:together:Linux Kernel dst-clone API}
\end{figure}

The \co{dst_clone()} primitive may be used if the caller
already has a reference to the specified \co{dst_entry},
in which case it obtains another reference that may be handed off
to some other entity within the kernel.
Because a reference is already held by the caller, \co{dst_clone()}
need not execute any memory barriers.
The act of handing the \co{dst_entry} to some other entity might
or might not require a memory barrier, but if such a memory barrier
is required, it will be embedded in the mechanism used to hand the
\co{dst_entry} off.

The \co{dst_release()} primitive may be invoked from any environment,
and the caller might well reference elements of the \co{dst_entry}
structure immediately prior to the call to \co{dst_release()}.
The \co{dst_release()} primitive therefore contains a memory
barrier on line~14 preventing both the compiler and the CPU
from misordering accesses.

Please note that the programmer making use of \co{dst_clone()} and
\co{dst_release()} need not be aware of the memory barriers, only
of the rules for using these two primitives.

\subsubsection{Atomic Counting With Check and Release Memory Barrier}
\label{sec:together:Atomic Counting With Check and Release Memory Barrier}

Consider a situation where the caller must be able to acquire a new
reference to an object to which it does not already hold a reference.
The fact that initial reference-count acquisition can now run concurrently
with reference-count release adds further complications.
Suppose that a reference-count release finds that the new
value of the reference count is zero, signalling that it is
now safe to clean up the reference-counted object.
We clearly cannot allow a reference-count acquisition to
start after such clean-up has commenced, so the acquisition
must include a check for a zero reference count.
This check must be part of the atomic increment operation,
as shown below.

\QuickQuiz{}
	Why can't the check for a zero reference count be
	made in a simple ``if'' statement with an atomic
	increment in its ``then'' clause?
\QuickQuizAnswer{
	Suppose that the ``if'' condition completed, finding
	the reference counter value equal to one.
	Suppose that a release operation executes, decrementing
	the reference counter to zero and therefore starting
	cleanup operations.
	But now the ``then'' clause can increment the counter
	back to a value of one, allowing the object to be
	used after it has been cleaned up.
} \QuickQuizEnd

The Linux kernel's \co{fget()} and \co{fput()} primitives
use this style of reference counting.
Simplified versions of these functions are shown in
Figure~\ref{fig:together:Linux Kernel fget/fput API}.

\begin{figure}[tbp]
{ \fontsize{6.5pt}{7.5pt}\selectfont
\begin{verbbox}
  1 struct file *fget(unsigned int fd)
  2 {
  3   struct file *file;
  4   struct files_struct *files = current->files;
  5
  6   rcu_read_lock();
  7   file = fcheck_files(files, fd);
  8   if (file) {
  9     if (!atomic_inc_not_zero(&file->f_count)) {
 10       rcu_read_unlock();
 11       return NULL;
 12     }
 13   }
 14   rcu_read_unlock();
 15   return file;
 16 }
 17
 18 struct file *
 19 fcheck_files(struct files_struct *files, unsigned int fd)
 20 {
 21   struct file * file = NULL;
 22   struct fdtable *fdt = rcu_dereference((files)->fdt);
 23
 24   if (fd < fdt->max_fds)
 25     file = rcu_dereference(fdt->fd[fd]);
 26   return file;
 27 }
 28
 29 void fput(struct file *file)
 30 {
 31   if (atomic_dec_and_test(&file->f_count))
 32     call_rcu(&file->f_u.fu_rcuhead, file_free_rcu);
 33 }
 34
 35 static void file_free_rcu(struct rcu_head *head)
 36 {
 37   struct file *f;
 38
 39   f = container_of(head, struct file, f_u.fu_rcuhead);
 40   kmem_cache_free(filp_cachep, f);
 41 }
\end{verbbox}
}
\centering
\theverbbox
\caption{Linux Kernel fget/fput API}
\label{fig:together:Linux Kernel fget/fput API}
\end{figure}

Line~4 of \co{fget()} fetches the pointer to the current
process's file-descriptor table, which might well be shared
with other processes.
Line~6 invokes \co{rcu_read_lock()}, which
enters an RCU read-side critical section.
The callback function from any subsequent \co{call_rcu()} primitive
will be deferred until a matching \co{rcu_read_unlock()} is reached
(line~10 or 14 in this example).
Line~7 looks up the file structure corresponding to the file
descriptor specified by the \co{fd} argument, as will be
described later.
If there is an open file corresponding to the specified file descriptor,
then line~9 attempts to atomically acquire a reference count.
If it fails to do so, lines~10-11 exit the RCU read-side critical
section and report failure.
Otherwise, if the attempt is successful, lines~14-15 exit the read-side
critical section and return a pointer to the file structure.

The \co{fcheck_files()} primitive is a helper function for
\co{fget()}.
It uses the \co{rcu_dereference()} primitive to safely fetch an
RCU-protected pointer for later dereferencing (this emits a
memory barrier on CPUs such as DEC Alpha in which data dependencies
do not enforce memory ordering).
Line~22 uses \co{rcu_dereference()} to fetch a pointer to this
task's current file-descriptor table,
and line~24 checks to see if the specified file descriptor is in range.
If so, line~25 fetches the pointer to the file structure, again using
the \co{rcu_dereference()} primitive.
Line~26 then returns a pointer to the file structure or \co{NULL}
in case of failure.

The \co{fput()} primitive releases a reference to a file structure.
Line~31 atomically decrements the reference count, and, if the result
was zero, line~32 invokes the \co{call_rcu()} primitives in order to
free up the file structure (via the \co{file_free_rcu()} function
specified in \co{call_rcu()}'s second argument),
but only after all currently-executing
RCU read-side critical sections complete.
The time period required for all currently-executing RCU read-side
critical sections to complete is termed a ``grace period''.
Note that the \co{atomic_dec_and_test()} primitive contains
a memory barrier.
This memory barrier is not necessary in this example, since the structure
cannot be destroyed until the RCU read-side critical section completes,
but in Linux, all atomic operations that return a result must
by definition contain memory barriers.

Once the grace period completes, the \co{file_free_rcu()} function
obtains a pointer to the file structure on line~39, and frees it
on line~40.

This approach is also used by Linux's virtual-memory system,
see \co{get_page_unless_zero()} and \co{put_page_testzero()} for
page structures as well as \co{try_to_unuse()} and \co{mmput()}
for memory-map structures.

\subsection{Linux Primitives Supporting Reference Counting}
\label{sec:together:Linux Primitives Supporting Reference Counting}

The Linux-kernel primitives used in the above examples are
summarized in the following list.
\IfInBook{}{The RCU primitives may be unfamiliar to some readers,
	    so a brief overview with citations is included in
	    Section~\ref{sec:together:Background on RCU}.}

\begin{itemize}
\item	\co{atomic_t}
	Type definition for 32-bit quantity to be manipulated atomically.
\item	\co{void atomic_dec(atomic_t *var);}
	Atomically decrements the referenced variable without necessarily
	issuing a memory barrier or disabling compiler optimizations.
\item	\co{int atomic_dec_and_test(atomic_t *var);}
	Atomically decrements the referenced variable, returning
	\co{true} (non-zero) if the result is zero.
	Issues a memory barrier and disables compiler optimizations that
	might otherwise move memory references across this primitive.
\item	\co{void atomic_inc(atomic_t *var);}
	Atomically increments the referenced variable without necessarily
	issuing a memory barrier or disabling compiler optimizations.
\item	\co{int atomic_inc_not_zero(atomic_t *var);}
	Atomically increments the referenced variable, but only if the
	value is non-zero, and returning \co{true} (non-zero) if the
	increment occurred.
	Issues a memory barrier and disables compiler optimizations that
	might otherwise move memory references across this primitive.
\item	\co{int atomic_read(atomic_t *var);}
	Returns the integer value of the referenced variable.
	This need not be an atomic operation, and it need not issue any
	memory-barrier instructions.
	Instead of thinking of as ``an atomic read'', think of it as
	``a normal read from an atomic variable''.
\item	\co{void atomic_set(atomic_t *var, int val);}
	Sets the value of the referenced atomic variable to ``val''.
	This need not be an atomic operation, and it is not required
	to either issue memory
	barriers or disable compiler optimizations.
	Instead of thinking of as ``an atomic set'', think of it as
	``a normal set of an atomic variable''.
\item	\co{void call_rcu(struct rcu_head *head, void (*func)(struct rcu_head *head));}
	Invokes \co{func(head)} some time after all currently executing RCU
	read-side critical sections complete, however, the \co{call_rcu()}
	primitive returns immediately.
	Note that \co{head} is normally a field within an RCU-protected
	data structure, and that \co{func} is normally a function that
	frees up this data structure.
	The time interval between the invocation of \co{call_rcu()} and
	the invocation of \co{func} is termed a ``grace period''.
	Any interval of time containing a grace period is itself a
	grace period.
\item	\co{type *container_of(p, type, f);}
	Given a pointer \co{p} to a field \co{f} within a structure
	of the specified type, return a pointer to the structure.
\item	\co{void rcu_read_lock(void);}
	Marks the beginning of an RCU read-side critical section.
\item	\co{void rcu_read_unlock(void);}
	Marks the end of an RCU read-side critical section.
	RCU read-side critical sections may be nested.
\item	\co{void smp_mb__before_atomic_dec(void);}
	Issues a memory barrier and disables code-motion compiler
	optimizations only if the platform's \co{atomic_dec()}
	primitive does not already do so.
\item	\co{struct rcu_head}
	A data structure used by the RCU infrastructure to track
	objects awaiting a grace period.
	This is normally included as a field within an RCU-protected
	data structure.
\end{itemize}

\QuickQuiz{}
	An \co{atomic_read()} and an \co{atomic_set()} that are
	non-atomic?
	Is this some kind of bad joke???
\QuickQuizAnswer{
	It might well seem that way, but in situations where no other
	CPU has access to the atomic variable in question, the overhead
	of an actual atomic instruction would be wasteful.
	Two examples where no other CPU has access are
	during initialization and cleanup.
} \QuickQuizEnd

\subsection{Counter Optimizations}
\label{sec:together:Counter Optimizations}

In some cases where increments and decrements are common, but checks
for zero are rare, it makes sense to maintain per-CPU or per-task
counters, as was discussed in Chapter~\ref{chp:Counting}.
See the paper on sleepable read-copy update
(SRCU) for an example of this technique applied to
RCU~\cite{PaulEMcKenney2006c}.
This approach eliminates the need for atomic instructions or memory
barriers on the increment and decrement primitives, but still requires
that code-motion compiler optimizations be disabled.
In addition, the primitives such as \co{synchronize_srcu()}
that check for the aggregate reference
count reaching zero can be quite slow.
This underscores the fact that these techniques are designed
for situations where the references are frequently acquired and
released, but where it is rarely necessary to check for a zero
reference count.

% @@@ Difficulty of managing reference counts: leaks, premature freeing.

However, it is usually the case that use of reference counts requires
writing (often atomically) to a data structure that is otherwise
read only.
In this case, reference counts are imposing expensive cache misses
on readers.

It is therefore worthwhile to look into synchronization mechanisms
that do not require readers to write to the data structure being
traversed.
One possibility is the hazard pointers covered in
Section~\ref{sec:defer:Hazard Pointers}
and another is RCU, which is covered in
Section~\ref{sec:defer:Read-Copy Update (RCU)}.

% together/hazptr.tex
% mainfile: ../perfbook.tex
% SPDX-License-Identifier: CC-BY-SA-3.0

\section{Hazard-Pointer Helpers}
\label{sec:together:Hazard-Pointer Helpers}
%
\epigraph{It's the little things that count, hundreds of them.}
	 {\emph{Cliff Shaw}}

이 섹션은 해쉬 테이블을 다룰 때 생길 수 있는 문제들을 알아봅니다.
이 문제들은 많은 다른 탐색 구조체들에서도 발생할 수 있음을 유의하시기 바랍니다.

\iffalse

This section looks at some issues that can arise when dealing with
hash tables.
Please note that these issues also apply to many other search structures.

\fi

\subsection{Scalable Reference Count}
\label{sec:together:Scalable Reference Count}

Suppose a reference count is becoming a performance or scalability
bottleneck.
What can you do?

One approach is to instead use hazard pointers.

There are some differences, perhaps most notably that with
hazard pointers it is extremely expensive to determine when
the corresponding reference count has reached zero.

One way to work around this problem is to split the load between
reference counters and hazard pointers.
Each data element has a reference counter that tracks the number
of other data elements referencing this element on the one hand,
and readers use hazard pointers on the other.

Making this arrangement work both efficiently and correctly can be
quite challenging, and so interested readers are invited to examine
the UnboundedQueue and ConcurrentHashMap data structures implemented in
Folly open-source library.\footnote{
	\url{https://github.com/facebook/folly}}

% @@@ papers to maybe cite: OrcGC, ThreadScan, Fast and Robust Memory...

% @@@ Generalized hazard-pointer link counts, if and when.

% @@@ Representative hazard pointer for list, so that nothing
% @@@ in list gets freed until list's hazard pointer is released.
% @@@ Midpoint between hazard pointers and RCU, in fact, you
% @@@ could argue that Tasks Trace RCU with read-side memory
% @@@ barriers is sort of a per-CPU hazard pointers implementing RCU.
% @@@ Except no re-checking because CPUs cannot be freed.

% together/seqlock.tex
% mainfile: ../perfbook.tex
% SPDX-License-Identifier: CC-BY-SA-3.0

\section{Sequence-Locking Specials}
\label{sec:together:Sequence-Locking Specials}
%
\epigraph{The girl who can't dance says the band can't play.}
	 {\emph{Yiddish proverb}}

이 섹션은 시퀀스 락의 특별한 사용처들을 알아봅니다.

\iffalse

This section looks at some special uses of sequence locks.

\fi

\subsection{Correlated Data Elements}
\label{sec:together:Correlated Data Elements}

두개 이상의 원소들을 연관지어 봐야 하는 해쉬 테이블을 가지고 있다고 해봅시다.
이 원소들은 함께 업데이트 되며, 첫번째 원소의 기존 버전을 다른 원소의 새 버전과
함께 보고 싶지 않습니다.
예를 들어, Schr\"odinger 는 그의 in-memory 데이터베이스에 그의 동물들에 더해
확장된 가족을 넣고 싶습니다.
Schr\"odinger 는 결혼과 이혼이 급작스럽게 일어나지는 않음을 알지만, 그는 또한
전통주의자이기도 합니다.
따라서, 그는 그의 데이터베이스가 신부는 결혼했는데 신랑은 그렇지 않은, 또는 그
반대의 경우를 보이기를 원치 않습니다.
또한, Schr\"odinger 가 전통주의자라 생각한다면, 여러분은 그의 가족 구성원들 중
일부와 대화해 보세요!
달리 말하자면, Schr\"odinger 는 그의 데이터베이스가 결혼에 있어 일관적이길
원합니다.

\iffalse

Suppose we have a hash table where we need correlated views of two or
more of the elements.
These elements are updated together, and we do not want to see an old
version of the first element along with new versions of the other
elements.
For example, Schr\"odinger decided to add his extended family to his
in-memory database along with all his animals.
Although Schr\"odinger understands that marriages and divorces do not
happen instantaneously, he is also a traditionalist.
As such, he absolutely does not want his database ever to show that the
bride is now married, but the groom is not, and vice versa.
Plus, if you think Schr\"odinger is a traditionalist, you just
try conversing with some of his family members!
In other words, Schr\"odinger wants to be able to carry out a
wedlock-consistent traversal of his database.

\fi

한가지 방법은 시퀀스 락을 사용해서
(\cref{sec:defer:Sequence Locks} 를 참고하세요),
결혼에 관련된 업데이트가 \co{write_seqlock()} 의 보호 아래 진행되고 결혼
일관성이 피료한 읽기는 \co{read_seqbegin()} / \co{read_seqretry()} 반복문
아래에서 진행되게 하는 겁니다.
시퀀스 락은 RCU 보호의 대체제가 아님에 유의하세요:
시퀀스 락은 동시의 수정으로부터 보호를 해줍니다만, 동시의 삭제로부터의 보호를
위해선 여전히 RCU 가 필요합니다.

이 방법은 연관된 원소의 수가 작고 원소들에의 읽기 시간이 짧으며, 업데이트
비율이 낮을 때 상당히 잘 작동합니다.
그렇지 않다면, 읽기 쓰레드가 영원히 완료되지 못할 수도 있게끔 업데이트가
빈번하게 이러날 수도 있습니다.
Schr\"odinger 는 이런 문제가 일어날 만큼 그의 가장 덜 정상적인 지인들이
결혼하고 곧바로 이혼할 거라 생각하지 않지만, 그는 이 문제가 다른 환경에서는
일어날 수 있음을 알고 있습니다.
이 읽기 쓰레드 starvation 문제를 해결하는 한가지 방법은 읽기 쓰레드가 너무 많은
재시도를 했다면 업데이트 쪽 기능을 사용하게 하는 것입니다만, 이는 성능과
확장성을 모두 악화시킬 수 있습니다.
Starvation 을 막는 다른 방법은 여러 시퀀스 락을 상요하는 것으로, 이
Schr\"odinger 의 경우엔 종당 하나가 될 수 있겠습니다.

\iffalse

One approach is to use sequence locks
(see \cref{sec:defer:Sequence Locks}),
so that wedlock-related updates are carried out under the
protection of \co{write_seqlock()}, while reads requiring
wedlock consistency are carried out within
a \co{read_seqbegin()} / \co{read_seqretry()} loop.
Note that sequence locks are not a replacement for RCU protection:
Sequence locks protect against concurrent modifications, but RCU
is still needed to protect against concurrent deletions.

This approach works quite well when the number of correlated elements is
small, the time to read these elements is short, and the update rate is
low.
Otherwise, updates might happen so quickly that readers might never complete.
Although Schr\"odinger does not expect that even his least-sane relatives
will marry and divorce quickly enough for this to be a problem,
he does realize that this problem could well arise in other situations.
One way to avoid this reader-starvation problem is to have the readers
use the update-side primitives if there have been too many retries,
but this can degrade both performance and scalability.
Another way to avoid starvation is to have multiple sequence locks,
in Schr\"odinger's case, perhaps one per species.

\fi

또한, 만약 업데이트 쪽 기능이 너무 자주 사용된다면, 락 컨텐션으로 인해 낮은
성능과 확장성이 초래될 겁니다.
이를 막는 한가지 방법은 원소별 시퀀스 락을 두고 부부의 결혼 상태를 업데이트 할
때 부부 두명의 락을 모두 잡는 겁니다.
읽기 쓰레드는 이 한쌍의 멤버들의 결혼 상태에 대한 모든 변화에 대해 안정적인
읽기를 위해 이 부부의 락들 중 하나만 가지고 재시도 반복을 할 수 있습니다.
이는 높은 결혼과 이혼율에 의한 컨텐션을 막을 수 있으나, 데이터베이스의 한번의
스캔 동안의 모든 결혼 상태에 대한 일관적 시각을 얻기를 복잡하게 만듭니다.

결혼 상태가 그러길 바라듯이 원소 그룹 짓기가 잘 정의되었고 영구적이라면, 가능한
한가지 방법은 데이터 원소로의 포인터들을 더해서 특정 그룹의 멤버들을 함께
연결짓는 겁니다.
그러면 읽기 쓰레드는 첫번째 원소가 발견되면 같은 그룹의 데이터 원소들을
접근하기 위해 이 포인터들을 따라갈 수 있습니다.

이 기법은 리눅스 커널에서 널리 사용되는데, dcache subsystem
에서~\cite{NeilBrown2015RCUwalk} 특히 그렇습니다.
비슷한 방법이 해저드 포인터를 통해서도 동작할 수 있음을 알아 두시기 바랍니다.

\iffalse

In addition, if the update-side primitives are used too frequently,
poor performance and scalability will result due to lock contention.
One way to avoid this is to maintain a per-element sequence lock,
and to hold both spouses' locks when updating their marital status.
Readers can do their retry looping on either of the spouses' locks
to gain a stable view of any change in marital status involving both
members of the pair.
This avoids contention due to high marriage and divorce rates, but
complicates gaining a stable view of all marital statuses during a
single scan of the database.

If the element groupings are well-defined and persistent, which marital
status is hoped to be,
then one approach is to add pointers to the data elements to link
together the members of a given group.
Readers can then traverse these pointers to access all the data elements
in the same group as the first one located.

This technique is used heavily in the Linux kernel, perhaps most
notably in the dcache subsystem~\cite{NeilBrown2015RCUwalk}.
Note that it is likely that similar schemes also work with hazard
pointers.

\fi

또다른 방법은 데이터 원소들을 파편화 하고 각 업데이트가 그 업데이트로 영향 받는
모든 데이터 원소를 위한 모든 시퀀스 락에 대해 쓰기 권한 획득을 하게 하는
겁니다.
물론, 이 쓰기 권한 획득은 데드락을 막기 위해 신중히 가해져야 합니다.
읽기 쓰레드는 여러 시퀀스 락에 대한 읽기 권한 획득을 필요로 하겠습니다만 읽기
쓰레드가 하나의 데이터 원소만 읽어야 하는 놀랍도록 흔한 상황에서는 하나의
시퀀스 락에 대해서만 읽기 권한 획득이 필요합니다.

이 방법은 성공한 읽기 쓰레드에게 sequential consistency 를 제공하여, 각자는
특정 업데이트의 효과를 보거나 보지 못하며, 모든 중간의 업데이트는 읽기 쪽
재시도를 하게 합니다.
Sequential consistency 는 극단적으로 강력한 보장으로, 동일하게 강한 제한과 높은
오버헤드를 수반합니다.
이 경우, 우린 읽기 쓰레드가 starvation 에 빠질 수 있거나 업데이트 쪽 락을
획득해야 할수도 있음을 보았습니다.
업데이트가 흔하지 않은 경우에 이는 잘 동작하지만, 이는 업데이트가 재시도되는
읽기 쓰레드가 액세스하는 데이터에는 여향을 주지도 않은 업데이트에조차 재시도를
불필요하게 강제합니다.
따ㅓ라서 \cref{sec:together:Correlated Fields} 는 읽기 쓰레드의 starvation 을
막을 뿐 아니라 모든 형태의 읽기 쪽 재시도를 막는 완화된 형태의 일관성을
다룹니다.

\iffalse

Another approach is to shard the data elements, and then have each update
write-acquire all the sequence locks needed to cover the data elements
affected by that update.
Of course, these write acquisitions must be done carefully in order to
avoid deadlock.
Readers would also need to read-acquire multiple sequence locks, but
in the surprisingly common case where readers only look up one data
element, only one sequence lock need be read-acquired.

This approach provides sequential consistency to successful readers,
each of which will either see the effects of a given update or not,
with any partial updates resulting in a read-side retry.
Sequential consistency is an extremely strong guarantee, incurring equally
strong restrictions and equally high overheads.
In this case, we saw that readers might be starved on the one hand, or
might need to acquire the update-side lock on the other.
Although this works very well in cases where updates are infrequent,
it unnecessarily forces read-side retries even when the update does not
affect any of the data that a retried reader accesses.
\Cref{sec:together:Correlated Fields} therefore covers a much weaker form
of consistency that not only avoids reader starvation, but also avoids
any form of read-side retry.

\fi

\subsection{Upgrade to Writer}
\label{sec:together:Upgrade to Writer}

As discussed in
\cref{sec:defer:RCU is a Reader-Writer Lock Replacement},
RCU permits readers to upgrade to writers.
This capability can be quite useful when a reader scanning an
RCU-protected data structure notices that the current element
needs to be updated.
What happens when you try this trick with sequence locking?

It turns out that this sequence-locking trick is actually used in
the Linux kernel, for example, by the \co{sdma_flush()} function in
\path{drivers/infiniband/hw/hfi1/sdma.c}.
The effect is to doom the enclosing reader to retry.
This trick is therefore used when the reader detects some condition
that requires a retry.

% together/applyrcu.tex

\section{RCU Rescues}
\label{sec:together:RCU Rescues}

이 섹션은 이 책의 앞부분에서 이야기한 몇가지 예제들에 RCU 를 어떻게
적용하는지를 보입니다.
일부 경우들에 있어서는 RCU 는 간단한 코드를 제공하고, 어떤 경우에는 더 나은
성능과 확장성을 제공하며, 또다른 경우에는 두가지를 모두 제공합니다.
\iffalse

This section shows how to apply RCU to some examples discussed earlier
in this book.
In some cases, RCU provides simpler code, in other cases better
performance and scalability, and in still other cases, both.
\fi

\subsection{RCU and Per-Thread-Variable-Based Statistical Counters}
\label{sec:together:RCU and Per-Thread-Variable-Based Statistical Counters}

Section~\ref{sec:count:Per-Thread-Variable-Based Implementation}
는 대략적으로 평범한 값 증가 연산 (C \co{++} 오퍼레이터) 과 같은---하지만
\co{inc_count()}를 통해서만 값을 증가시키는---훌륭한 성능과 선형적 확장성을
보이는 통계적 카운터들의 구현을 설명했습니다.
불행히도, \co{read_count()} 를 통해 값을 읽어와야 하는 쓰레드들은 글로벌 락을 잡아야만 했고, 따라서 높은 오버헤드를 일으키고 낮은 확장성으로 고통받아야 했습니다.
락 기반의 구현 코드는
Page~\pageref{fig:count:Per-Thread Statistical Counters} 의
Figure~\ref{fig:count:Per-Thread Statistical Counters} 에 보여져 있습니다.
\iffalse

Section~\ref{sec:count:Per-Thread-Variable-Based Implementation}
described an implementation of statistical counters that provided
excellent
performance, roughly that of simple increment (as in the C \co{++}
operator), and linear scalability---but only for incrementing
via \co{inc_count()}.
Unfortunately, threads needing to read out the value via \co{read_count()}
were required to acquire a global
lock, and thus incurred high overhead and suffered poor scalability.
The code for the lock-based implementation is shown in
Figure~\ref{fig:count:Per-Thread Statistical Counters} on
Page~\pageref{fig:count:Per-Thread Statistical Counters}.
\fi

\QuickQuiz{}
	대체 왜 그런 글로벌 락이 필요했던 거지요?
	\iffalse

	Why on earth did we need that global lock in the first place?
	\fi
\QuickQuizAnswer{
	특정 쓰레드의 \co{__thread} 변수들은 그 쓰레드가 종료될 때 없어집니다.
	따라서 다른 쓰레드의 \co{__thread} 변수들을 접근하는 모든 오퍼레이션은
	쓰레드 종료와 동기화 되어야 할 필요가 있습니다.
	그런 동기화가 없다면, 방금 종료된 쓰레드의 \co{__thread} 변수로의
	접근은 segmentation fault 를 초래할 겁니다.
	\iffalse

	A given thread's \co{__thread} variables vanish when that
	thread exits.
	It is therefore necessary to synchronize any operation that
	accesses other threads' \co{__thread} variables with
	thread exit.
	Without such synchronization, accesses to \co{__thread} variable
	of a just-exited thread will result in segmentation faults.
	\fi
} \QuickQuizEnd

\subsubsection{Design}

원하는건 \co{inc_count()} 만이 아니라 \co{read_count()} 에서도 훌륭한 성능과
확장성을 얻기 위해 \co{read_count()} 의 쓰레드 횡단을 보호하는 데에
\co{final_mutex} 대신에 RCU 를 사용하는 것입니다.
하지만, 계산된 합계의 정확성을 포기하지도 않고 싶습니다.
자세히 말하자면, 특정 스레드가 종료될 때에, 우린 종료되는 쓰레드의 카운트를
잃어버릴 수도, 그걸 두번씩 세서도 안됩니다.
그런 에러는 결과의 전체 정확성과 동일한 정도의 비정확성을 초래할 수 있는데,
달리 말하자면 그런 에러는 결과값이 완전히 쓸모없게 만들 수 있습니다.
그리고 사실, \co{final_mutex} 의 목적들 중 하나는 쓰레드들이 \co{read_count()}
의 실행 사이에 들어왔다 나갔다 하지 않음을 분명히 하는 것입니다.
\iffalse

The hope is to use RCU rather than \co{final_mutex} to protect the
thread traversal in \co{read_count()} in order to obtain excellent
performance and scalability from \co{read_count()}, rather than just
from \co{inc_count()}.
However, we do not want to give up any accuracy in the computed sum.
In particular, when a given thread exits, we absolutely cannot
lose the exiting thread's count, nor can we double-count it.
Such an error could result in inaccuracies equal to the full
precision of the result, in other words, such an error would
make the result completely useless.
And in fact, one of the purposes of \co{final_mutex} is to
ensure that threads do not come and go in the middle of \co{read_count()}
execution.
\fi

\QuickQuiz{}
	어쨌든, \co{read_count()} 의 정확성을 대체 뭔가요?
	\iffalse

	Just what is the accuracy of \co{read_count()}, anyway?
	\fi
\QuickQuizAnswer{
	Page~\pageref{fig:count:Per-Thread Statistical Counters} 의
	Figure~\ref{fig:count:Per-Thread Statistical Counters} 를 참고하세요.
	동시적인 \co{inc_count()} 의 실행이 존재하지 않는다면,
	\co{read_count()} 는 분명한 결과를 내놓을 것은 분명합니다.
	하지만, \co{inc_count()} 의 동시적인 실행이 \emph{존재한다면}, 그
	합계값은 \co{read_count()} 가 그 합산을 진행함에 따라 실제로 달라질
	것입니다.
	그렇다곤 하나, 쓰레드의 생성과 종료는 \co{final_mutex} 에 의해
	배제되므로, \co{counterp} 안의 포인터들은 상수로 유지될 것입니다.
	\iffalse

	Refer to
	Figure~\ref{fig:count:Per-Thread Statistical Counters} on
	Page~\pageref{fig:count:Per-Thread Statistical Counters}.
	Clearly, if there are no concurrent invocations of \co{inc_count()},
	\co{read_count()} will return an exact result.
	However, if there \emph{are} concurrent invocations of
	\co{inc_count()}, then the sum is in fact changing as
	\co{read_count()} performs its summation.
	That said, because thread creation and exit are excluded by
	\co{final_mutex}, the pointers in \co{counterp} remain constant.
	\fi

	메모리의 즉석 스냅샷을 얻어올 수 있는 가상의 기계를 상상해 봅시다.
	이 기계가 \co{read_count()} 의 실행 시작지점의 스냅샷과
	\co{read_count()} 의 실행 종료 시점의 스냅샷을 만들어낸다고 생각해
	봅시다.
	그렇다면 \co{read_count()} 는 이 두 스냡샷들 사이의 어떤 시점에서의 각
	쓰레드의 카운터에 접근을 할 것이고, 따라서 이 두개의 스냡샷들에 의해
	값이 포괄적으로 한정지어지는 결과를 얻어오게 될 것입니다.
	따라서, 전체 합계는 이 두개의 스냡샷으로부터 각각 얻어와질 수 있는
	합계들의 쌍에 의해 그 값이 한정지어질 것입니다 (다시 말하지만,
	포괄적으로).

	따라서 예상되는 에러는 이 두개의 스냅샷으로부터 얻어올 수 있는 두개의
	합계의 쌍들 사이의 차이의 절반일 것이고, 이는 \co{read_count()} 의 실행
	시간에 단위 시간당 \co{inc_count()} 의 예상되는 호출 횟수를 곱한 값의
	절반입니다.
	\iffalse

	Let's imagine a mythical machine that is able to take an
	instantaneous snapshot of its memory.
	Suppose that this machine takes such a snapshot at the
	beginning of \co{read_count()}'s execution, and another
	snapshot at the end of \co{read_count()}'s execution.
	Then \co{read_count()} will access each thread's counter
	at some time between these two snapshots, and will therefore
	obtain a result that is bounded by those of the two snapshots,
	inclusive.
	The overall sum will therefore be bounded by the pair of sums that
	would have been obtained from each of the two snapshots (again,
	inclusive).

	The expected error is therefore half of the difference between
	the pair of sums that would have been obtained from each of the
	two snapshots, that is to say, half of the execution time of
	\co{read_count()} multiplied by the number of expected calls to
	\co{inc_count()} per unit time.
	\fi

	또는, 수식을 선호하는 분들을 위해 표시하면:
	\begin{equation}
	\epsilon = \frac{T_r R_i}{2}
	\end{equation}
	로, $\epsilon$ 는 \co{read_count()} 의 리턴 값에 예측되는 에러이고,
	$T_r$ 은 \co{read_count()} 가 실행되는데 걸리는 시간이고, $R_i$ 는 단위
	시간당 \co{inc_count()} 호출 횟수의 비율입니다.
	(그리고 당연하지만, $T_r$ 과 $R_i$ 는 같은 단위 시간을 사용해야 합니다:
	마이크로세컨드와 마이크로세컨드당 호출 횟수, 초와 초당 호출 횟수, 뭐가
	됐든, 같은 단위를 사용하기만 한다면.)
	\iffalse

	Or, for those who prefer equations:
	\begin{equation}
	\epsilon = \frac{T_r R_i}{2}
	\end{equation}
	where $\epsilon$ is the expected error in \co{read_count()}'s
	return value,
	$T_r$ is the time that \co{read_count()} takes to execute,
	and $R_i$ is the rate of \co{inc_count()} calls per unit time.
	(And of course, $T_r$ and $R_i$ should use the same units of
	time: microseconds and calls per microsecond, seconds and calls
	per second, or whatever, as long as they are the same units.)
	\fi
} \QuickQuizEnd

따라서, 우리가 \co{final_mutex} 를 없애려 한다면, 우리는 일관성을 보장하기 위한
어떤 다른 방법을 사용해야 합니다.
한가지 방법은 앞서 종료된 쓰레드들 전체를 위한 전체 카운트와 쓰레드별
카운터로의 포인터들의 배열을 하나의 구조체에 넣는 것입니다.
\co{read_count()} 에 의해 접근될 수 있는 그런 구조체는 상수가 되므로,
\co{read_count()} 가 일관적인 데이터를 보게 될 것을 보장합니다.
\iffalse

Therefore, if we are to dispense with \co{final_mutex}, we will need
to come up with some other method for ensuring consistency.
One approach is to place the total count for all previously exited
threads and the array of pointers to the per-thread counters into a single
structure.
Such a structure, once made available to \co{read_count()}, is
held constant, ensuring that \co{read_count()} sees consistent data.
\fi

\subsubsection{Implementation}

\begin{figure}[bp]
{ \scriptsize
\begin{verbbox}
  1 struct countarray {
  2   unsigned long total;
  3   unsigned long *counterp[NR_THREADS];
  4 };
  5 
  6 long __thread counter = 0;
  7 struct countarray *countarrayp = NULL;
  8 DEFINE_SPINLOCK(final_mutex);
  9 
 10 void inc_count(void)
 11 {
 12   counter++;
 13 }
 14 
 15 long read_count(void)
 16 {
 17   struct countarray *cap;
 18   unsigned long sum;
 19   int t;
 20 
 21   rcu_read_lock();
 22   cap = rcu_dereference(countarrayp);
 23   sum = cap->total;
 24   for_each_thread(t)
 25     if (cap->counterp[t] != NULL)
 26       sum += *cap->counterp[t];
 27   rcu_read_unlock();
 28   return sum;
 29 }
 30 
 31 void count_init(void)
 32 {
 33   countarrayp = malloc(sizeof(*countarrayp));
 34   if (countarrayp == NULL) {
 35     fprintf(stderr, "Out of memory\n");
 36     exit(-1);
 37   }
 38   memset(countarrayp, '\0', sizeof(*countarrayp));
 39 }
 40 
 41 void count_register_thread(void)
 42 {
 43   int idx = smp_thread_id();
 44 
 45   spin_lock(&final_mutex);
 46   countarrayp->counterp[idx] = &counter;
 47   spin_unlock(&final_mutex);
 48 }
 49 
 50 void count_unregister_thread(int nthreadsexpected)
 51 {
 52   struct countarray *cap;
 53   struct countarray *capold;
 54   int idx = smp_thread_id();
 55 
 56   cap = malloc(sizeof(*countarrayp));
 57   if (cap == NULL) {
 58     fprintf(stderr, "Out of memory\n");
 59     exit(-1);
 60   }
 61   spin_lock(&final_mutex);
 62   *cap = *countarrayp;
 63   cap->total += counter;
 64   cap->counterp[idx] = NULL;
 65   capold = countarrayp;
 66   rcu_assign_pointer(countarrayp, cap);
 67   spin_unlock(&final_mutex);
 68   synchronize_rcu();
 69   free(capold);
 70 }
\end{verbbox}
}
\centering
\theverbbox
\caption{RCU and Per-Thread Statistical Counters}
\label{fig:together:RCU and Per-Thread Statistical Counters}
\end{figure}

Lines~1-4 of
Figure~\ref{fig:together:RCU and Per-Thread Statistical Counters}
show the \co{countarray} structure, which contains a
\co{->total} field for the count from previously exited threads,
and a \co{counterp[]} array of pointers to the per-thread
\co{counter} for each currently running thread.
This structure allows a given execution of \co{read_count()}
to see a total that is consistent with the indicated set of running
threads.

Lines~6-8 contain the definition of the per-thread \co{counter}
variable, the global pointer \co{countarrayp} referencing
the current \co{countarray} structure, and
the \co{final_mutex} spinlock.

Lines~10-13 show \co{inc_count()}, which is unchanged from
Figure~\ref{fig:count:Per-Thread Statistical Counters}.

Lines~15-29 show \co{read_count()}, which has changed significantly.
Lines~21 and~27 substitute \co{rcu_read_lock()} and
\co{rcu_read_unlock()} for acquisition and release of \co{final_mutex}.
Line~22 uses \co{rcu_dereference()} to snapshot the
current \co{countarray} structure into local variable \co{cap}.
Proper use of RCU will guarantee that this \co{countarray} structure
will remain with us through at least the end of the current RCU
read-side critical section at line~27.
Line~23 initializes \co{sum} to \co{cap->total}, which is the
sum of the counts of threads that have previously exited.
Lines~24-26 add up the per-thread counters corresponding to currently
running threads, and, finally, line 28 returns the sum.

The initial value for \co{countarrayp} is
provided by \co{count_init()} on lines~31-39.
This function runs before the first thread is created, and its job
is to allocate
and zero the initial structure, and then assign it to \co{countarrayp}.

Lines~41-48 show the \co{count_register_thread()} function, which
is invoked by each newly created thread.
Line~43 picks up the current thread's index, line~45 acquires
\co{final_mutex}, line~46 installs a pointer to this thread's
\co{counter}, and line~47 releases \co{final_mutex}.

\QuickQuiz{}
	Hey!!!
	Line~46 of
	Figure~\ref{fig:together:RCU and Per-Thread Statistical Counters}
	modifies a value in a pre-existing \co{countarray} structure!
	Didn't you say that this structure, once made available to
	\co{read_count()}, remained constant???
\QuickQuizAnswer{
	Indeed I did say that.
	And it would be possible to make \co{count_register_thread()}
	allocate a new structure, much as \co{count_unregister_thread()}
	currently does.

	But this is unnecessary.
	Recall the derivation of the error bounds of \co{read_count()}
	that was based on the snapshots of memory.
	Because new threads start with initial \co{counter} values of
	zero, the derivation holds even if we add a new thread partway
	through \co{read_count()}'s execution.
	So, interestingly enough, when adding a new thread, this
	implementation gets the effect of allocating a new structure,
	but without actually having to do the allocation.
} \QuickQuizEnd

Lines~50-70 shows \co{count_unregister_thread()}, which is invoked
by each thread just before it exits.
Lines~56-60 allocate a new \co{countarray} structure,
line~61 acquires \co{final_mutex} and line~67 releases it.
Line~62 copies the contents of the current \co{countarray} into
the newly allocated version, line~63 adds the exiting thread's \co{counter}
to new structure's total, and line~64 \co{NULL}s the exiting thread's
\co{counterp[]} array element.
Line~65 then retains a pointer to the current (soon to be old)
\co{countarray} structure, and line~66 uses \co{rcu_assign_pointer()}
to install the new version of the \co{countarray} structure.
Line~68 waits for a grace period to elapse, so that any threads that
might be concurrently executing in \co{read_count}, and thus might
have references to the old \co{countarray} structure, will be allowed
to exit their RCU read-side critical sections, thus dropping any such
references.
Line~69 can then safely free the old \co{countarray} structure.

\subsubsection{Discussion}

\QuickQuiz{}
	Wow!
	Figure~\ref{fig:together:RCU and Per-Thread Statistical Counters}
	contains 69 lines of code, compared to only 42 in
	Figure~\ref{fig:count:Per-Thread Statistical Counters}.
	Is this extra complexity really worth it?
\QuickQuizAnswer{
	This of course needs to be decided on a case-by-case basis.
	If you need an implementation of \co{read_count()} that
	scales linearly, then the lock-based implementation shown in
	Figure~\ref{fig:count:Per-Thread Statistical Counters}
	simply will not work for you.
	On the other hand, if calls to \co{count_read()} are sufficiently
	rare, then the lock-based version is simpler and might thus be
	better, although much of the size difference is due
	to the structure definition, memory allocation, and \co{NULL}
	return checking.

	Of course, a better question is ``Why doesn't the language
	implement cross-thread access to \co{__thread} variables?''
	After all, such an implementation would make both the locking
	and the use of RCU unnecessary.
	This would in turn enable an implementation that
	was even simpler than the one shown in
	Figure~\ref{fig:count:Per-Thread Statistical Counters}, but
	with all the scalability and performance benefits of the
	implementation shown in
	Figure~\ref{fig:together:RCU and Per-Thread Statistical Counters}!
} \QuickQuizEnd

Use of RCU enables exiting threads to wait until other threads are
guaranteed to be done using the exiting threads' \co{__thread} variables.
This allows the \co{read_count()} function to dispense with locking,
thereby providing
excellent performance and scalability for both the \co{inc_count()}
and \co{read_count()} functions.
However, this performance and scalability come at the cost of some increase
in code complexity.
It is hoped that compiler and library writers employ user-level
RCU~\cite{MathieuDesnoyers2009URCU} to provide safe cross-thread
access to \co{__thread} variables, greatly reducing the
complexity seen by users of \co{__thread} variables.

\subsection{RCU and Counters for Removable I/O Devices}
\label{sec:together:RCU and Counters for Removable I/O Devices}

Section~\ref{sec:count:Applying Specialized Parallel Counters}
showed a fanciful pair of code fragments for dealing with counting
I/O accesses to removable devices.
These code fragments suffered from high overhead on the fastpath
(starting an I/O) due to the need to acquire a reader-writer
lock.

This section shows how RCU may be used to avoid this overhead.

The code for performing an I/O is quite similar to the original, with
a RCU read-side critical section being substituted for the reader-writer
lock read-side critical section in the original:

\vspace{5pt}
\begin{minipage}[t]{\columnwidth}
\small
\begin{verbatim}
  1 rcu_read_lock();
  2 if (removing) {
  3   rcu_read_unlock();
  4   cancel_io();
  5 } else {
  6   add_count(1);
  7   rcu_read_unlock();
  8   do_io();
  9   sub_count(1);
 10 }
\end{verbatim}
\end{minipage}
\vspace{5pt}

The RCU read-side primitives have minimal overhead, thus speeding up
the fastpath, as desired.

The updated code fragment removing a device is as follows:

\vspace{5pt}
\begin{minipage}[t]{\columnwidth}
\small
\begin{verbatim}
  1 spin_lock(&mylock);
  2 removing = 1;
  3 sub_count(mybias);
  4 spin_unlock(&mylock);
  5 synchronize_rcu();
  6 while (read_count() != 0) {
  7   poll(NULL, 0, 1);
  8 }
  9 remove_device();
\end{verbatim}
\end{minipage}
\vspace{5pt}

Here we replace the reader-writer lock with an exclusive spinlock and
add a \co{synchronize_rcu()} to wait for all of the RCU read-side
critical sections to complete.
Because of the \co{synchronize_rcu()},
once we reach line~6, we know that all remaining I/Os have been accounted
for.

Of course, the overhead of \co{synchronize_rcu()} can be large,
but given that device removal is quite rare, this is usually a good
tradeoff.

\subsection{Array and Length}
\label{sec:together:Array and Length}

\begin{figure}[tbp]
{ \scriptsize
\begin{verbbox}
 1 struct foo {
 2   int length;
 3   char *a;
 4 };
\end{verbbox}
}
\centering
\theverbbox
\caption{RCU-Protected Variable-Length Array}
\label{fig:together:RCU-Protected Variable-Length Array}
\end{figure}

Suppose we have an RCU-protected variable-length array, as shown in
Figure~\ref{fig:together:RCU-Protected Variable-Length Array}.
The length of the array \co{->a[]} can change dynamically, and at any
given time, its length is given by the field \co{->length}.
Of course, this introduces the following race condition:

\begin{enumerate}
\item	The array is initially 16 characters long, and thus \co{->length}
	is equal to 16.
\item	CPU~0 loads the value of \co{->length}, obtaining the value 16.
\item	CPU~1 shrinks the array to be of length 8, and assigns a pointer
	to a new 8-character block of memory into \co{->a[]}.
\item	CPU~0 picks up the new pointer from \co{->a[]}, and stores a
	new value into element 12.
	Because the array has only 8 characters, this results in
	a SEGV or (worse yet) memory corruption.
\end{enumerate}

How can we prevent this?

One approach is to make careful use of memory barriers, which are
covered in Section~\ref{sec:advsync:Memory Barriers}.
This works, but incurs read-side overhead and, perhaps worse, requires
use of explicit memory barriers.

\begin{figure}[tbp]
{ \scriptsize
\begin{verbbox}
 1 struct foo_a {
 2   int length;
 3   char a[0];
 4 };
 5 
 6 struct foo {
 7   struct foo_a *fa;
 8 };
\end{verbbox}
}
\centering
\theverbbox
\caption{Improved RCU-Protected Variable-Length Array}
\label{fig:together:Improved RCU-Protected Variable-Length Array}
\end{figure}

A better approach is to put the value and the array into the same structure,
as shown in
Figure~\ref{fig:together:Improved RCU-Protected Variable-Length Array}.
Allocating a new array (\co{foo_a} structure) then automatically provides
a new place for the array length.
This means that if any CPU picks up a reference to \co{->fa}, it is
guaranteed that the \co{->length} will match the \co{->a[]}
length~\cite{Arcangeli03}.

\begin{enumerate}
\item	The array is initially 16 characters long, and thus \co{->length}
	is equal to 16.
\item	CPU~0 loads the value of \co{->fa}, obtaining a pointer to
	the structure containing the value 16 and the 16-byte array.
\item	CPU~0 loads the value of \co{->fa->length}, obtaining the value 16.
\item	CPU~1 shrinks the array to be of length 8, and assigns a pointer
	to a new \co{foo_a} structure containing an 8-character block
	of memory into \co{->a[]}.
\item	CPU~0 picks up the new pointer from \co{->a[]}, and stores a
	new value into element 12.
	But because CPU~0 is still referencing the old \co{foo_a}
	structure that contains the 16-byte array, all is well.
\end{enumerate}

Of course, in both cases, CPU~1 must wait for a grace period before
freeing the old array.

A more general version of this approach is presented in the next section.

\subsection{Correlated Fields}
\label{sec:together:Correlated Fields}

\begin{figure}[tbp]
{ \scriptsize
\begin{verbbox}
 1 struct animal {
 2   char name[40];
 3   double age;
 4   double meas_1;
 5   double meas_2;
 6   double meas_3;
 7   char photo[0]; /* large bitmap. */
 8 };
\end{verbbox}
}
\centering
\theverbbox
\caption{Uncorrelated Measurement Fields}
\label{fig:together:Uncorrelated Measurement Fields}
\end{figure}

Suppose that each of Sch\"odinger's animals is represented by the
data element shown in
Figure~\ref{fig:together:Uncorrelated Measurement Fields}.
The \co{meas_1}, \co{meas_2}, and \co{meas_3} fields are a set
of correlated measurements that are updated periodically.
It is critically important that readers see these three values from
a single measurement update: If a reader sees an old value of
\co{meas_1} but new values of \co{meas_2} and \co{meas_3}, that
reader will become fatally confused.
How can we guarantee that readers will see coordinated sets of these
three values?

One approach would be to allocate a new \co{animal} structure,
copy the old structure into the new structure, update the new
structure's \co{meas_1}, \co{meas_2}, and \co{meas_3} fields,
and then replace the old structure with a new one by updating
the pointer.
This does guarantee that all readers see coordinated sets of
measurement values, but it requires copying a large structure due
to the \co{->photo[]} field.
This copying might incur unacceptably large overhead.

\begin{figure}[tbp]
{ \scriptsize
\begin{verbbox}
 1 struct measurement {
 2   double meas_1;
 3   double meas_2;
 4   double meas_3;
 5 };
 6 
 7 struct animal {
 8   char name[40];
 9   double age;
10   struct measurement *mp;
11   char photo[0]; /* large bitmap. */
12 };
\end{verbbox}
}
\centering
\theverbbox
\caption{Correlated Measurement Fields}
\label{fig:together:Correlated Measurement Fields}
\end{figure}

Another approach is to insert a level of indirection, as shown in
Figure~\ref{fig:together:Correlated Measurement Fields}.
When a new measurement is taken, a new \co{measurement} structure
is allocated, filled in with the measurements, and the \co{animal}
structure's \co{->mp} field is updated to point to this new
\co{measurement} structure using \co{rcu_assign_pointer()}.
After a grace period elapses, the old \co{measurement} structure
can be freed.

\QuickQuiz{}
	But cant't the approach shown in
	Figure~\ref{fig:together:Correlated Measurement Fields}
	result in extra cache misses, in turn resulting in additional
	read-side overhead?
\QuickQuizAnswer{
	Indeed it can.

\begin{figure}[tbp]
{ \scriptsize
\begin{verbbox}
 1 struct measurement {
 2   double meas_1;
 3   double meas_2;
 4   double meas_3;
 5 };
 6 
 7 struct animal {
 8   char name[40];
 9   double age;
10   struct measurement *mp;
11   struct measurement meas;
12   char photo[0]; /* large bitmap. */
13 };
\end{verbbox}
}
\centering
\theverbbox
\caption{Localized Correlated Measurement Fields}
\label{fig:together:Localized Correlated Measurement Fields}
\end{figure}

	One way to avoid this cache-miss overhead is shown in
	Figure~\ref{fig:together:Localized Correlated Measurement Fields}:
	Simply embed an instance of a \co{measurement} structure
	named \co{meas}
	into the \co{animal} structure, and point the \co{->mp}
	field at this \co{->meas} field.

	Measurement updates can then be carried out as follows:

	\begin{enumerate}
	\item	Allocate a new \co{measurement} structure and place
		the new measurements into it.
	\item	Use \co{rcu_assign_pointer()} to point \co{->mp} to
		this new structure.
	\item	Wait for a grace period to elapse, for example using
		either \co{synchronize_rcu()} or \co{call_rcu()}.
	\item	Copy the measurements from the new \co{measurement}
		structure into the embedded \co{->meas} field.
	\item	Use \co{rcu_assign_pointer()} to point \co{->mp}
		back to the old embedded \co{->meas} field.
	\item	After another grace period elapses, free up the
		new \co{measurement} field.
	\end{enumerate}

	This approach uses a heavier weight update procedure to eliminate
	the extra cache miss in the common case.
	The extra cache miss will be incurred only while an update is
	actually in progress.
} \QuickQuizEnd

This approach enables readers to see correlated values for selected
fields with minimal read-side overhead.

% Birthstone/tombstone for moving records when readers cannot be permitted
% to see extraneous records.

% Flag for deletion (if not already covered in the defer chapter).

% @@@ Later add section on updates: hashed arrays of locks, fifos/streaming,
% batching to trade off latency for perf/scale.

\QuickQuizAnswersChp{qqztogether}
