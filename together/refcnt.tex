% together/refcnt.tex

\section{Refurbish Reference Counting}
\label{sec:together:Refurbish Reference Counting}
%
\epigraph{Counting is the religion of this generation.  It is its
	  hope and its salvation.}
	 {\emph{Gertrude Stein}}

레퍼런스 카운팅이 개념적으로는 간단한 테크닉이지만, 동시성 소프트웨어에
적용되면 그 디테일 안에 많은 악마들이 숨어있습니다.
무엇보다도, 오브젝트가 지나치게 일찍 폐기되는 일이 없는 오브젝트라면, 애초에
레퍼런스 카운터가 필요하지도 않았을 겁니다.
하지만 오브젝트가 폐기될 수 있다면, 레퍼런스를 얻어오는 프로세스 그 자체가
진행되는 동안 폐기되는 것을 무엇으로 막을 수 있을까요?

동시성 소프트웨어에서 사용되기 위해 레퍼런스 카운터를 재정비하는 몇가지
방법들이 있는데, 다음과 같은 것들이 포함됩니다:
\iffalse

Although reference counting is a conceptually simple technique,
many devils hide in the details when it is applied to concurrent
software.
After all, if the object was not subject to premature disposal,
there would be no need for the reference counter in the first place.
But if the object can be disposed of, what prevents disposal during
the reference-acquisition process itself?

There are a number of ways to refurbish reference counters for
use in concurrent software, including:
\fi

\begin{enumerate}
\item	레퍼런스 카운트를 조정하는 동안에는 바깥에 존재하는 오브젝트의 락을
	잡아야만 하기.
\item	오브젝트는 0이 아닌 레퍼런스 카운트와 함께 생성되고, 새로운
	레퍼런스들은 레퍼런스 카운터의 현재 값이 0이 아닐 때에만 획득될 수 있게
	하기.
	어떤 레퍼런스가 특정 오브젝트로의 레퍼런스를 가지고 있지 않다면, 이미
	레퍼런스를 가지고 있는 다른 쓰레드의 도움으로 해당 오브젝트로의
	레퍼런스를 얻을 수 있음.
\item	해당 오브젝트를 위한 존재 보장이 제공되어서, 해당 오브젝트가 다른
	무언가가 레퍼런스를 얻으려 시도하는 동안은 해제되지 않도록 하기.
	존재 보장은 자동화된 garbage collector 를 통해서, 또는
	Section~\ref{sec:defer:Read-Copy Update (RCU)} 에서 보여지듯이 RCU 를
	통해서 제공되곤 합니다.
\item	오브젝트를 위한 타입 안정성을 제공하기.
	레퍼런스가 일단 획득되면 추가적인 아이덴티티 체크가 이뤄져야만 합니다.
	타입 안정성 보장은 예를 들어,
	Section~\ref{sec:defer:Read-Copy Update (RCU)} 에서 보여진 것과 같이
	리눅스 커널 안에서 \co{SLAB_TYPESAFE_BY_RCU} 기능과 같이 특수 목적
	메모리 할당자를 통해 제공될 수 있습니다.
\iffalse

\item	A lock residing outside of the object must be held while
	manipulating the reference count.
\item	The object is created with a non-zero reference count, and new
	references may be acquired only when the current value of
	the reference counter is non-zero.
	If a thread does not have a reference to a given object,
	it may obtain one with the help of another thread that
	already has a reference.
\item	An existence guarantee is provided for the object, preventing
	it from being freed while some other
	entity might be attempting to acquire a reference.
	Existence guarantees are often provided by automatic
	garbage collectors, and, as will be seen in
	Section~\ref{sec:defer:Read-Copy Update (RCU)}, by RCU.
\item	A type-safety guarantee is provided for the object.
	An additional identity check must be performed once
	the reference is acquired.
	Type-safety guarantees can be provided by special-purpose
	memory allocators, for example, by the
	\co{SLAB_TYPESAFE_BY_RCU} feature within the Linux kernel,
	as will be seen in Section~\ref{sec:defer:Read-Copy Update (RCU)}.
\fi
\end{enumerate}

물론, 존재 보장을 제공하는 모든 메커니즘은 그 정의에 의해 타입 안정성 보장도
제공합니다.
따라서 이 섹션은 마지막의 두 답들을 RCU 의 전례 하에 레퍼런스 획득 보호에
일반적으로 사용되는 세개의 카테고리로 그룹지어 보겠습니다: 레퍼런스 카운팅,
시퀀스 락킹, 그리고 RCU.
\iffalse

Of course, any mechanism that provides existence guarantees
by definition also provides type-safety guarantees.
This section will therefore group the last two answers together under the
rubric of RCU, leaving us with three general categories of
reference-acquisition protection: Reference counting, sequence
locking, and RCU.
\fi

\QuickQuiz{}
	레퍼런스 획득을 단순히 레퍼런스 카운터의 값이 0이 아닌 경우에만
	레퍼런스를 획득하는 compare-and-swap 오퍼레이션으로 간단하게 만들지는
	않는거죠?
	\iffalse

	Why not implement reference-acquisition using
	a simple compare-and-swap operation that only
	acquires a reference if the reference counter is
	non-zero?
	\fi
\QuickQuizAnswer{
	이 방법이 마지막 레퍼런스의 해제와 새로운 레퍼런스 획득 사이의 레이스를
	해결할 수 있겠지만, 데이터 구조체가 해제되고 전혀 다른 종류의 구조체로
	재할당 되는 것을 방지하는데에 있어서는 어떤 일도 해주지 않습니다.
	``간단한 compare-and-swap 오퍼레이션'' 이 다른 종류의 구조체에 적용되면
	정의되지 않은 결과를 낼 가능성이 상당히 큽니다.

	짧게 말해서, compare-and-swap 과 같은 어토믹 오퍼레이션의 사용은 타입
	안정성이나 존재 보장을 필요로 합니다.
	\iffalse

	Although this can resolve the race between the release of
	the last reference and acquisition of a new reference,
	it does absolutely nothing to prevent the data structure
	from being freed and reallocated, possibly as some completely
	different type of structure.
	It is quite likely that the ``simple compare-and-swap
	operation'' would give undefined results if applied to the
	differently typed structure.

	In short, use of atomic operations such as compare-and-swap
	absolutely requires either type-safety or existence guarantees.
	\fi
} \QuickQuizEnd

\begin{table}[tb]
\renewcommand*{\arraystretch}{1.25}
\rowcolors{3}{}{lightgray}
\small
\centering
\begin{tabular}{lccc}
	\toprule
	& \multicolumn{3}{c}{Release Synchronization} \\
	\cmidrule(l){2-4}
	\parbox[c]{.8in}{Acquisition\\Synchronization}
			& Locking
				& \parbox[c]{.5in}{Reference\\Counting}
				        & RCU \\
	\cmidrule{1-1} \cmidrule(l){2-4}
	Locking		& $-$	  & CAM	      & CA  \\
	\parbox[c][6ex]{.8in}{Reference\\Counting}
			& A	  & AM        & A   \\
	RCU		& CA	  & MCA	      & CA  \\
	\bottomrule
\end{tabular}
\caption{Reference Counting and Synchronization Mechanisms}
\label{tab:together:Reference Counting and Synchronization Mechanisms}
\end{table}

레퍼런스 카운팅의 핵심 이슈는 레퍼런스의 획득과 오브젝트의 해제 사이의 동기화란
점을 놓고 보면,
Table~\ref{tab:together:Reference Counting and Synchronization Mechanisms} 에
보인 것과 같은 아홉개의 메커니즘 조합이 존재할 수 있습니다.
이 표는 레퍼런스 카운팅 메커니즘들을 다음과 같이 넓은 카테고리들로 나눕니다:
\iffalse

Given that the key reference-counting issue
is synchronization between acquisition
of a reference and freeing of the object, we have nine possible
combinations of mechanisms, as shown in
Table~\ref{tab:together:Reference Counting and Synchronization Mechanisms}.
This table
divides reference-counting mechanisms into the following broad categories:
\fi
\begin{enumerate}
\item	어토믹 오퍼레이션, 메모리 베리어, 정렬 제약도 없는 간단한 카운팅
	(``$-$'').
\item	메모리 배리어 없이 진행되는 어토믹 카운팅 (``A'').
\item	릴리즈 시에만 메모리 배리어를 사용하는 어토믹 카운팅 (``AM'').
\item	어토믹 획득 오퍼레이션과 릴리즈 시에만 필요시 되는 메모리 배리어들과
	조합된 체크를 하는 어토믹 카운팅 (``CAM'').
\item	어토믹 획득 오퍼레이션과 조합된 체크를 하는 어토믹 카운팅 (``CA'').
\item	어토믹 획득 오퍼레이션과 레퍼런스 획득 때에도 필요시 되는 메모리
	배리어들과 조합된 체크를 사용하는 어토믹 카운팅 (``MCA'').
\iffalse

\item	Simple counting with neither atomic operations, memory
	barriers, nor alignment constraints (``$-$'').
\item	Atomic counting without memory barriers (``A'').
\item	Atomic counting, with memory barriers required only on release
	(``AM'').
\item	Atomic counting with a check combined with the atomic acquisition
	operation, and with memory barriers required only on release
	(``CAM'').
\item	Atomic counting with a check combined with the atomic acquisition
	operation (``CA'').
\item	Atomic counting with a check combined with the atomic acquisition
	operation, and with memory barriers also required on acquisition
	(``MCA'').
\fi
\end{enumerate}
하지만, 값을 리턴하는 모든 리눅스 커널 어토믹 오퍼레이션들은 메모리 배리어를
포함하도록 정의되어 있으므로,\footnote{
	\co{atomic_read()} 와 \co{ATOMIC_INIT()} 는 이 규칙에 예외가 됩니다.}
모든 릴리즈 오퍼레이션은 메모리 배리어를 포함하고, 모든 체크되는 레퍼런스
획득 오퍼레이션 또한 메모리 배리어를 포함하게 됩니다.
따라서, ``CA'' 와 ``MCA'' 는 ``CAM'' 과 동일해서, 앞의 네개의 경우들을 위한
섹션들만이 남게 됩니다:
``$-$``, ``A'', ``AM'', 그리고 ``CAM''.
레퍼런스 카운팅을 지원하는 리눅스의 기능들은
Section~\ref{sec:together:Linux Primitives Supporting Reference Counting} 에
소개되어 있습니다.
뒤의 섹션들은 레퍼런스 획득과 해제가 매우 빈번할 때, 그리고 레퍼런스 카운트가
매우 가끔씩만 0인지 여부를 체크해야 하는 경우에 대해 성능을 개선할 수 있는
최적화 방법들을 인용합니다.
\iffalse

However, because all Linux-kernel atomic operations that return a
value are defined to contain memory barriers,\footnote{
	With \co{atomic_read()} and \co{ATOMIC_INIT()} being the
	exceptions that prove the rule.}
all release operations
contain memory barriers, and all checked acquisition operations also
contain memory barriers.
Therefore, cases ``CA'' and ``MCA'' are equivalent to ``CAM'', so that
there are sections below for only the first four cases:
``$-$'', ``A'', ``AM'', and ``CAM''.
The Linux primitives that support reference counting are presented in
Section~\ref{sec:together:Linux Primitives Supporting Reference Counting}.
Later sections cite optimizations that can improve performance
if reference acquisition and release is very frequent, and the
reference count need be checked for zero only very rarely.
\fi

\subsection{Implementation of Reference-Counting Categories}
\label{sec:together:Implementation of Reference-Counting Categories}

락킹으로 보호되는 간단한 카운팅 (\makebox{``$-$''}) 이
Section~\ref{sec:together:Simple Counting} 에 설명되고,
메모리 배리어 없이 수행되는 어토믹한 카운팅 (``A'') 이
Section~\ref{sec:together:Atomic Counting} 에 설명되고,
레퍼런스 획득 시의 메모리 배리어와 함께 수행되는 어토믹한 카운팅 (``AM'') 이
Section~\ref{sec:together:Atomic Counting With Release Memory Barrier}
에 설명되며, 레퍼런스 획득시의 메모리 배리어와 체크와 함께 수행되는 어토믹한
카운팅 (``CAM'') 이
Section~\ref{sec:together:Atomic Counting With Check and Release Memory Barrier}
에서 설명됩니다.
\iffalse

Simple counting protected by locking (``$-$'') is described in
Section~\ref{sec:together:Simple Counting},
atomic counting with no memory barriers (``A'') is described in
Section~\ref{sec:together:Atomic Counting},
atomic counting with acquisition memory barrier (``AM'') is described in
Section~\ref{sec:together:Atomic Counting With Release Memory Barrier},
and
atomic counting with check and release memory barrier (``CAM'') is described in
Section~\ref{sec:together:Atomic Counting With Check and Release Memory Barrier}.
\fi

\subsubsection{Simple Counting}
\label{sec:together:Simple Counting}

어토믹 오퍼레이션도 메모리 배리어도 사용하지 않는 간단한 카운팅은 레퍼런스
카운터 획득과 해제가 모두 같은 락으로 보호될 때에 사용될 수 있습니다.
이 경우에 레퍼런스 카운터 자체는 어토믹하지 않게 조정될 것이란 게 분명한데,
락은 모든 필요한 배타성, 메모리 배리어, 어토믹 인스트럭션, 그리고 컴파일러
최적화에 대한 방지를 제공하기 때문입니다.
이는 해당 락이 레퍼런스 카운트 만이 아니라 다른 오퍼레이션들도 보호해야 하지만
해당 오브젝트로의 레퍼런스가 해당 락을 해제한 후에 잡아야만 할 때에 선택될 수
있는 방법입니다.
Listing~\ref{lst:together:Simple Reference-Count API} 는 어토믹하지 않은
레퍼런스 카운팅을 구현하는데 사용될 수 있는 간단한 API 를 보입니다---간단한
레퍼런스 카운팅은 거의 항상 인터페이스 없이 곧바로 코딩되곤 하지만요.
\iffalse

Simple counting, with neither atomic operations nor memory barriers,
can be used when the reference-counter acquisition and release are
both protected by the same lock.
In this case, it should be clear that the reference count itself
may be manipulated non-atomically, because the lock provides any
necessary exclusion, memory barriers, atomic instructions, and disabling
of compiler optimizations.
This is the method of choice when the lock is required to protect
other operations in addition to the reference count, but where
a reference to the object must be held after the lock is released.
Listing~\ref{lst:together:Simple Reference-Count API} shows a simple
API that might be used to implement simple non-atomic reference
counting---although simple reference counting is almost always
open-coded instead.
\fi

\begin{listing}[tbp]
{ \scriptsize
\begin{verbbox}
  1 struct sref {
  2   int refcount;
  3 };
  4
  5 void sref_init(struct sref *sref)
  6 {
  7   sref->refcount = 1;
  8 }
  9
 10 void sref_get(struct sref *sref)
 11 {
 12   sref->refcount++;
 13 }
 14
 15 int sref_put(struct sref *sref,
 16              void (*release)(struct sref *sref))
 17 {
 18   WARN_ON(release == NULL);
 19   WARN_ON(release == (void (*)(struct sref *))kfree);
 20
 21   if (--sref->refcount == 0) {
 22     release(sref);
 23     return 1;
 24   }
 25   return 0;
 26 }
\end{verbbox}
}
\centering
\theverbbox
\caption{Simple Reference-Count API}
\label{lst:together:Simple Reference-Count API}
\end{listing}

\subsubsection{Atomic Counting}
\label{sec:together:Atomic Counting}

간단한 어토믹 카운팅은 레퍼런스를 획득하는 모든 CPU 가 레퍼런스를 이미 잡고
있어야 하는 경우에 사용될 수 있습니다.
이 스타일은 하나의 CPU 가 자신의 사용을 위해 오브젝트를 생성하지만 나중에
시작된 다른 CPU, 태스크, 타이머 핸들러, 또는 I/O 완료 핸들러가 그 오브젝트에
접근할 수 있도록 해야만 할 때 사용됩니다.
이 오브젝트를 넘겨주는 CPU 는 받게되는 오브젝트를 위해 새로운 레퍼런스를 먼저
획득해야만 합니다.
리눅스 커널에서는 \co{kref} 기능이 이런 스타일의 레퍼런스 카운팅을 구현하기
위해 사용되는데,
Listing~\ref{lst:together:Linux Kernel kref API} 에 보여져 있습니다.
\iffalse

Simple atomic counting may be used in cases where any CPU acquiring
a reference must already hold a reference.
This style is used when a single CPU creates an object for its
own private use, but must allow other CPU, tasks, timer handlers,
or I/O completion handlers that it later spawns to also access this object.
Any CPU that hands the object off must first acquire a new reference
on behalf of the recipient object.
In the Linux kernel, the \co{kref} primitives are used to implement
this style of reference counting, as shown in
Listing~\ref{lst:together:Linux Kernel kref API}.
\fi

모든 레퍼런스 카운팅 오퍼레이션이 락킹으로 보호되지는 않는데, 이는 두개의 다른
CPU 가 동시에 레퍼런스 카운트를 조정할 수 있음을 의미하기 때문에 어토믹
카운팅이 필요해집니다.
평범한 값 증가와 감소 오퍼레이션이 사용된다면, 한쌍의 CPU 가 둘 다 레퍼런스
카운트를 동시에 가져와서, ``3'' 이라는 값을 읽을 수 있을 겁니다.
둘 다 그 값을 증가시키려 한다면, 이들은 모두 ``4'' 를 얻게 될테고, 둘 다 이
값을 카운터에 다시 써넣을 겁니다.
카운터의 새로운 값은 ``5'' 가 되었어야 하므로, 두개의 값 증가 오퍼레이션 중
하나는 증발한 셈입니다.
따라서, 카운터의 값 증가에도 값 감소에도 어토믹 오퍼레이션이 사용되어야만
합니다.
\iffalse

Atomic counting is required
because locking is not used to protect all reference-count operations,
which means that it is possible for two different CPUs to concurrently
manipulate the reference count.
If normal increment and decrement were used, a pair of CPUs might both
fetch the reference count concurrently, perhaps both obtaining
the value ``3''.
If both of them increment their value, they will both obtain ``4'',
and both will store this value back into the counter.
Since the new value of the counter should instead be ``5'', one
of the two increments has been lost.
Therefore, atomic operations must be used both for counter increments
and for counter decrements.
\fi

레퍼런스 해제가 락킹이나 RCU 로 보호된다면, 메모리 배리어는 필요하지
\emph{않을테지만}, 다른 이유로 인해서입니다.
락킹의 경우, 락은 모든 필요한 메모리 배리어들을 제공하고 (그리고 컴파일러
최적화를 불능화 시킵니다), 락은 또한 두개의 레퍼런스 해제가 동시에 수행되는
것을 막습니다.
RCU 의 경우, 현재 수행중인 RCU read-side 크리티컬 섹션들이 모두 완료되기
전까지는 정리 작업이 유예되어야만 하고, 모든 필요한 메모리 배리어나 컴파일러
최적화의 불능화가 RCU 기능으로 제공될 겁니다.
따라서, 두개의 CPU 가 두개의 레퍼런스를 동시에 해제시킨다면, 실제 정리작업은
두개의 CPU 가 모두 RCU read-side 크리티컬 섹션들을 빠져나갈 때까지 유예될
겁니다.
\iffalse

If releases are guarded by locking or RCU,
memory barriers are \emph{not} required, but for different reasons.
In the case of locking, the locks provide any needed memory barriers
(and disabling of compiler optimizations), and the locks also
prevent a pair of releases from running concurrently.
In the case of RCU, cleanup must be deferred until all currently
executing RCU read-side critical sections have completed, and
any needed memory barriers or disabling of compiler optimizations
will be provided by the RCU infrastructure.
Therefore, if two CPUs release the final two references concurrently,
the actual cleanup will be deferred until both CPUs exit their
RCU read-side critical sections.
\fi

\QuickQuiz{}
	한 CPU 가 마지막 레퍼런스를 해제한 직후에 다른 CPU 가 레퍼런스를
	획득하는 경우에 대해서도 보호를 해줘야 하지 않나요?
	\iffalse

	Why isn't it necessary to guard against cases where one CPU
	acquires a reference just after another CPU releases the last
	reference?
	\fi
\QuickQuizAnswer{
	CPU 는 합법적으로 다른 레퍼런스를 얻기 위해서는 이미 레퍼런스를 가지고
	있어야 하기 때문입니다.
	따라서, 한 CPU 가 마지막 레퍼런스를 해제했다면, 새로운 레퍼런스를
	획득할 수 있는 CPU 는 존재할 수가 없습니다.
	이와 같은 사실이
	Listing~\ref{lst:together:Linux Kernel kref API} 의 line~22 의
	어토믹하지 않은 검사를 가능하게 합니다.
	\iffalse

	Because a CPU must already hold a reference in order
	to legally acquire another reference.
	Therefore, if one CPU releases the last reference,
	there cannot possibly be any CPU that is permitted
	to acquire a new reference.
	This same fact allows the non-atomic check in line~22
	of Listing~\ref{lst:together:Linux Kernel kref API}.
	\fi
} \QuickQuizEnd

\begin{listing}[tbp]
{ \scriptsize
\begin{verbbox}
  1 struct kref {
  2   atomic_t refcount;
  3 };
  4 
  5 void kref_init(struct kref *kref)
  6 {
  7   atomic_set(&kref->refcount, 1);
  8 }
  9 
 10 void kref_get(struct kref *kref)
 11 {
 12   WARN_ON(!atomic_read(&kref->refcount));
 13   atomic_inc(&kref->refcount);
 14 }
 15 
 16 static inline int
 17 kref_sub(struct kref *kref, unsigned int count,
 18          void (*release)(struct kref *kref))
 19 {
 20   WARN_ON(release == NULL);
 21 
 22   if (atomic_sub_and_test((int) count,
 23                           &kref->refcount)) {
 24     release(kref);
 25     return 1;
 26   }
 27   return 0;
 28 }
\end{verbbox}
}
\centering
\theverbbox
\caption{Linux Kernel \tco{kref} API}
\label{lst:together:Linux Kernel kref API}
\end{listing}

\co{kref} 구조체 자체는 하나의 어토믹 데이터 아이템으로 구성되며,
Listing~\ref{lst:together:Linux Kernel kref API} 의 line~1-3 에 보여져 있습니다.
Line~5-8 의 \co{kref_init()} 함수는 이 카운터를 값 ``1'' 로 초기화 시킵니다.
\co{atomic_set()} 기능은 단순한 값 할당으로, 그 이름은 \co{atomic_t} 의 데이터
타입에서 왔지, 그 동작에서 온것이 아님을 알아두시기 바랍니다.
\co{kref_init()} 함수는 오브젝트 생성 때에, 해당 오브젝트가 어떤 다른 CPU 들에
의해 접근될 수 있게 되기 전에 실행되어야만 합니다.
\iffalse

The \co{kref} structure itself, consisting of a single atomic
data item, is shown in lines~1-3 of
Listing~\ref{lst:together:Linux Kernel kref API}.
The \co{kref_init()} function on lines~5-8 initializes the counter
to the value ``1''.
Note that the \co{atomic_set()} primitive is a simple
assignment, the name stems from the data type of \co{atomic_t}
rather than from the operation.
The \co{kref_init()} function must be invoked during object creation,
before the object has been made available to any other CPU.
\fi

Line~10-14 의 \co{kref_get()} 함수는 무조건적으로 카운터를 어토믹하게
증가시킵니다.
\co{atomic_inc()} 기능은 모든 플랫폼에서 컴파일러 최적화를 명시적으로 불능화
시켜야만 하지는 않습니다만, \co{kref} 이 별개의 모듈에 존재하고 리눅스 커널
빌드 프로세스는 모듈간 최적화를 하지 않는다는 사실이 똑같은 효과를 냅니다.

Line~16-28 의 \co{kref_sub()} 함수는 어토믹하게 카운터를 감소시키고, 만약 그
결과가 0이라면, line~24 에서 명시된 \co{release()} 함수를 호출하고 line~25 에서
호출자에게 \co{release()} 가 호출되었음을 알리면서 리턴합니다.
그렇지 않다면, \co{kref_sub()} 은 0을 리턴해서 호출자에게 \co{release()} 가
호출되지 않았음을 알립니다.
\iffalse

The \co{kref_get()} function on lines~10-14 unconditionally atomically
increments the counter.
The \co{atomic_inc()} primitive does not necessarily explicitly
disable compiler
optimizations on all platforms, but the fact that the \co{kref}
primitives are in a separate module and that the Linux kernel build
process does no cross-module optimizations has the same effect.

The \co{kref_sub()} function on lines~16-28 atomically decrements the
counter, and if the result is zero, line~24 invokes the specified
\co{release()} function and line~25 returns, informing the caller
that \co{release()} was invoked.
Otherwise, \co{kref_sub()} returns zero, informing the caller that
\co{release()} was not called.
\fi

\QuickQuiz{}
	Listing~\ref{lst:together:Linux Kernel kref API} 의 line~22 에서
	\co{atomic_sub_and_test()} 가 호출된 직후에, 어떤 다른 CPU 가
	\co{kref_get()} 을 호출했다고 생각해 봅시다.
	이는 이 다른 CPU 가 이제 비합법적으로 해제된 오브젝트로의 레퍼런스를
	갖게 된 거 아닌가요?
	\iffalse

	Suppose that just after the \co{atomic_sub_and_test()}
	on line~22 of
	Listing~\ref{lst:together:Linux Kernel kref API} is invoked,
	that some other CPU invokes \co{kref_get()}.
	Doesn't this result in that other CPU now having an illegal
	reference to a released object?
	\fi
\QuickQuizAnswer{
	그런 일은 이 함수가 올바르게 사용된다면 일어날 수 없는 일입니다.
	이미 레퍼런스를 가지고 있지 않다면 \co{kref_get()} 을 호출하는 것은
	비합법적인 일이어서 \co{kref_sub()} 는 카운터의 값을 0으로 감소시킬 수
	없었을 겁니다.
	\iffalse

	This cannot happen if these functions are used correctly.
	It is illegal to invoke \co{kref_get()} unless you already
	hold a reference, in which case the \co{kref_sub()} could
	not possibly have decremented the counter to zero.
	\fi
} \QuickQuizEnd

\QuickQuiz{}
	\co{kref_sub()} 가 0을 리턴해서 \co{release()} 함수가 호출되지 않았음을
	알렸다고 생각해 봅시다.
	어떤 조건에서 호출자는 이 오브젝트의 존재의 지속에 의존할 수 있을까요?
	\iffalse

	Suppose that \co{kref_sub()} returns zero, indicating that
	the \co{release()} function was not invoked.
	Under what conditions can the caller rely on the continued
	existence of the enclosing object?
	\fi
\QuickQuizAnswer{
	호출자는 최소 하나의 레퍼런스가 계속해서 존재할 것임을 알지 못한다면
	오브젝트의 존재 지속에 의존할 수 없습니다.
	일반적으로, 호출자는 이를 알 방법이 없을 것이고, 따라서 \co{kref_sub()}
	후에 오브젝트를 참조하는 것을 주의깊게 피해야만 합니다.
	\iffalse

	The caller cannot rely on the continued existence of the
	object unless it knows that at least one reference will
	continue to exist.
	Normally, the caller will have no way of knowing this, and
	must therefore carefullly avoid referencing the object after
	the call to \co{kref_sub()}.
	\fi
} \QuickQuizEnd

\QuickQuiz{}
	왜 그냥 해제 함수로 \co{kfree()} 를 넘기지 않는거죠?
	\iffalse

	Why not just pass \co{kfree()} as the release function?
	\fi
\QuickQuizAnswer{
	일반적으로 \co{kref} 구조체는 더 커다란 구조체에 포함되어 있게 되므로,
	\co{kref} 필드만이 아니라 전체 구조체를 해제시킬 필요가 있습니다.
	이는 일반적으로 \co{container_of()} 와 \co{kfree()} 를 호출하는 wrapper
	함수를 정의하는 것으로 이뤄질 수 있습니다.
	\iffalse

	Because the \co{kref} structure normally is embedded in
	a larger structure, and it is necessary to free the entire
	structure, not just the \co{kref} field.
	This is normally accomplished by defining a wrapper function
	that does a \co{container_of()} and then a \co{kfree()}.
	\fi
} \QuickQuizEnd

\subsubsection{Atomic Counting With Release Memory Barrier}
\label{sec:together:Atomic Counting With Release Memory Barrier}

이런 스타일의 레퍼런스는 리눅스 커널의 네트워킹 레이어에서 패킷 라우팅에
사용되는 목적지 캐시를 추적하는데에 사용됩니다.
실제 구현은 훨씬 더 관련되어 있습니다; 이 섹션은 이 사용예에 적합하며
Listing~\ref{lst:together:Linux Kernel dst-clone API} 에 보인
\co{struct dst_entry} 레퍼런스 카운트 핸들링 부분에 집중하고 있습니다.
\iffalse

This style of reference is used in the Linux kernel's networking
layer to track the destination caches that are used in packet routing.
The actual implementation is quite a bit more involved; this section
focuses on the aspects of \co{struct dst_entry} reference-count
handling that matches this use case,
shown in Listing~\ref{lst:together:Linux Kernel dst-clone API}.
\fi

\begin{listing}[tbp]
{ \scriptsize
\begin{verbbox}
  1 static inline
  2 struct dst_entry * dst_clone(struct dst_entry * dst)
  3 {
  4   if (dst)
  5     atomic_inc(&dst->__refcnt);
  6   return dst;
  7 }
  8
  9 static inline
 10 void dst_release(struct dst_entry * dst)
 11 {
 12   if (dst) {
 13     WARN_ON(atomic_read(&dst->__refcnt) < 1);
 14     smp_mb__before_atomic_dec();
 15     atomic_dec(&dst->__refcnt);
 16   }
 17 }
\end{verbbox}
}
\centering
\theverbbox
\caption{Linux Kernel \tco{dst_clone} API}
\label{lst:together:Linux Kernel dst-clone API}
\end{listing}

\co{dst_clone()} 함수는 호출자가 이미 명시된 \co{dst_entry} 에 대한 레퍼런스를
가지고 있을 때 사용될 수 있는데, 이는 커널 내의 다른 존재에게 넘겨줄 수도 있는
또다른 레퍼런스를 획득하는 경우입니다.
호출자에 의해 이미 레퍼런스가 하나 잡혀 있기 때문에, \co{dst_clone()} 은 어떤
메모리 배리어도 실행할 필요가 없습니다.
어떤 다른 존재에게 \co{dst_entry} 를 넘겨주는 행위는 메모리 배리어를 필요로
할수도, 필요로 하지 않을 수도 있습니다만, 그런 메모리 배리어가 필요하다면, 그
메모리 배리어는 \co{dst_entry} 를 넘기는 메커니즘 내에 내장되어야 합니다.
\iffalse

The \co{dst_clone()} primitive may be used if the caller
already has a reference to the specified \co{dst_entry},
in which case it obtains another reference that may be handed off
to some other entity within the kernel.
Because a reference is already held by the caller, \co{dst_clone()}
need not execute any memory barriers.
The act of handing the \co{dst_entry} to some other entity might
or might not require a memory barrier, but if such a memory barrier
is required, it will be embedded in the mechanism used to hand the
\co{dst_entry} off.
\fi

\co{dst_release()} 함수는 어떤 환경에서도 호출될 수 있고, 호출자는
\co{dst_release()} 호출에 앞서 \co{dst_entry} 구조체의 원소들을 직접 참조할
수도 있습니다.
따라서 \co{dst_release()} 함수는 line~14 에서 메모리 배리어를 포함해서
컴파일러나 CPU 가 접근 순서를 잘못 재배치하지 않도록 합니다.

\co{dst_clone()} 과 \co{dst_release()} 를 사용하는 프로그래머는 메모리 배리어에
대해서 신경쓸 필요가 없고, 단지 이 두개의 함수들의 사용규칙만 알면 된다는
점을 알아 두시기 바랍니다.
\iffalse

The \co{dst_release()} primitive may be invoked from any environment,
and the caller might well reference elements of the \co{dst_entry}
structure immediately prior to the call to \co{dst_release()}.
The \co{dst_release()} primitive therefore contains a memory
barrier on line~14 preventing both the compiler and the CPU
from misordering accesses.

Please note that the programmer making use of \co{dst_clone()} and
\co{dst_release()} need not be aware of the memory barriers, only
of the rules for using these two primitives.
\fi

\subsubsection{Atomic Counting With Check and Release Memory Barrier}
\label{sec:together:Atomic Counting With Check and Release Memory Barrier}

호출자가 현재는 레퍼런스를 가지고 있지 않은 오브젝트로의 레퍼런스를 획득해야만
하는 경우를 생각해 봅시다.
최초의 레퍼런스 카운트 획득은 레퍼런스 카운트 해제와 동시에 이뤄질 수 있다는
사실이 복잡도를 더욱 증가시킵니다.
레퍼런스 카운트 해제가 레퍼런스 카운트의 해제 후 값이 0이 될 것을 발견했고 그
레퍼런스에 연관된 오브젝트가 해제되어도 안전하다는 신호를 날렸다고 생각해
봅시다.
그런 해제작업이 개시된 후에 레퍼런스 카운트 획득을 허가할 수 없을 것은
분명하므로, 레퍼런스 획득은 레퍼런스 카운트가 0인지에 대한 검사를 포함해야만
합니다.
이런 검사는 다음에 보인 것과 같이 어토믹한 값 증가 오퍼레이션이 한 부분이
되어야만 합니다.
\iffalse

Consider a situation where the caller must be able to acquire a new
reference to an object to which it does not already hold a reference.
The fact that initial reference-count acquisition can now run concurrently
with reference-count release adds further complications.
Suppose that a reference-count release finds that the new
value of the reference count is zero, signalling that it is
now safe to clean up the reference-counted object.
We clearly cannot allow a reference-count acquisition to
start after such clean-up has commenced, so the acquisition
must include a check for a zero reference count.
This check must be part of the atomic increment operation,
as shown below.
\fi

\QuickQuiz{}
	레퍼런스 카운트의 값이 0인지에 대한 검사는 왜 간단히 어토믹 값 증가
	오퍼레이션을 갖는 ``if'' 문의 ``then'' 절에 들어갈 수 없는거죠?
	\iffalse

	Why can't the check for a zero reference count be
	made in a simple ``if'' statement with an atomic
	increment in its ``then'' clause?
	\fi
\QuickQuizAnswer{
	``if'' 조건이 완료되었고 레퍼런스 카운터의 값이 1이라는 걸 알게
	되었다고 해봅시다.
	이어서 레퍼런스 해제 오퍼레이션이 실행되어서 이 레퍼런스 카운터의 값을
	0으로 바꾸었고 오브젝트 해제 오퍼레이션을 시작시킵니다.
	하지만 이제 ``then'' 절은 이 카운터의 값을 다시 1로 증가시킬 수가
	있어서 오브젝트가 해제된 후에 사용되는 상황을 가능하게 해버리고 맙니다.
	\iffalse

	Suppose that the ``if'' condition completed, finding
	the reference counter value equal to one.
	Suppose that a release operation executes, decrementing
	the reference counter to zero and therefore starting
	cleanup operations.
	But now the ``then'' clause can increment the counter
	back to a value of one, allowing the object to be
	used after it has been cleaned up.
	\fi
} \QuickQuizEnd

리눅스 커널의 \co{fget()} 과 \co{fput()} 기능이 이런 스타일의 레퍼런스
카운팅을 사용합니다.
이 함수들의 간료화된 버전이
Listing~\ref{lst:together:Linux Kernel fget/fput API} 에 보여져 있습니다.
\iffalse

The Linux kernel's \co{fget()} and \co{fput()} primitives
use this style of reference counting.
Simplified versions of these functions are shown in
Listing~\ref{lst:together:Linux Kernel fget/fput API}.
\fi

\begin{listing}[tbp]
{ \fontsize{6.5pt}{7.5pt}\selectfont
\begin{verbbox}
  1 struct file *fget(unsigned int fd)
  2 {
  3   struct file *file;
  4   struct files_struct *files = current->files;
  5
  6   rcu_read_lock();
  7   file = fcheck_files(files, fd);
  8   if (file) {
  9     if (!atomic_inc_not_zero(&file->f_count)) {
 10       rcu_read_unlock();
 11       return NULL;
 12     }
 13   }
 14   rcu_read_unlock();
 15   return file;
 16 }
 17
 18 struct file *
 19 fcheck_files(struct files_struct *files, unsigned int fd)
 20 {
 21   struct file * file = NULL;
 22   struct fdtable *fdt = rcu_dereference((files)->fdt);
 23
 24   if (fd < fdt->max_fds)
 25     file = rcu_dereference(fdt->fd[fd]);
 26   return file;
 27 }
 28
 29 void fput(struct file *file)
 30 {
 31   if (atomic_dec_and_test(&file->f_count))
 32     call_rcu(&file->f_u.fu_rcuhead, file_free_rcu);
 33 }
 34
 35 static void file_free_rcu(struct rcu_head *head)
 36 {
 37   struct file *f;
 38
 39   f = container_of(head, struct file, f_u.fu_rcuhead);
 40   kmem_cache_free(filp_cachep, f);
 41 }
\end{verbbox}
}
\centering
\theverbbox
\caption{Linux Kernel \tco{fget}/\tco{fput} API}
\label{lst:together:Linux Kernel fget/fput API}
\end{listing}

\co{fget()} 의 Line~4 는, 다른 프로세스들과 공유될 수도 있는 현재 프로세스의
file-descriptor 테이블을 가져옵니다.
Line~6 는 RCU read-side 크리티컬 섹션을 들어가는 \co{rcu_read_lock()} 을
호출합니다.
뒤따르는 \co{call_rcu()} 에 전달되는 콜백 함수의 수행은 매칭되는
\co{rcu_read_unlock()} 이 호출될 때까지 (이 경우 line~10 또는~14) 연기됩니다.
Line~7 은 \co{fd} 인자에 명시된 파일 디스크립터에 연관되는 file 구조체를
탐색하는데, 이에 대해서는 뒤에서 설명하겠습니다.
명시된 파일 디스크립터에 연관된 열린 파일이 존재한다면 line~9 에서는 어토믹하게
레퍼런스 카운트를 획득하려 시도합니다.
만약 그렇게 하는데 실패한다면, line~10-11 은 RCU read-side 크리티컬 섹션을
빠져나오고 실패했음을 알립니다.
그렇지 않고 시도가 성공했다면, line~14-15 는 이 read-side 크리티컬 섹션을
빠져나오고 해당 file 구조체로의 포인터를 리턴합니다.
\iffalse

Line~4 of \co{fget()} fetches the pointer to the current
process's file-descriptor table, which might well be shared
with other processes.
Line~6 invokes \co{rcu_read_lock()}, which
enters an RCU read-side critical section.
The callback function from any subsequent \co{call_rcu()} primitive
will be deferred until a matching \co{rcu_read_unlock()} is reached
(line~10 or~14 in this example).
Line~7 looks up the file structure corresponding to the file
descriptor specified by the \co{fd} argument, as will be
described later.
If there is an open file corresponding to the specified file descriptor,
then line~9 attempts to atomically acquire a reference count.
If it fails to do so, lines~10-11 exit the RCU read-side critical
section and report failure.
Otherwise, if the attempt is successful, lines~14-15 exit the read-side
critical section and return a pointer to the file structure.
\fi

\co{fcheck_files()} 기능은 \co{fget()} 을 위한 도움을 주는 함수입니다.
해당 함수는 나중의 디레퍼런싱을 위해 (이는 데이터 종속성이 메모리 순서 규칙을
강제하지 않는 DEC Alpha 와 같은 CPU 들에서는 메모리 배리어를 실행합니다)
RCU 로 보호되는 포인터를 안전히 가져오기 위해 사용됩니다.
Line~22 는 이 태스크의 현재 file-descriptor 테이블로의 포인터를 가져오기 위해
\co{rcu_dereference()} 를 사용하고, line~24 는 명시된 파일 디스크립터가 범위
안에 있는지 체크합니다.
만약 그렇다면 line~25 는 해당 file 구조체로의 포인터를 가져오는데, 여기서도
\co{rcu_dereference()} 기능이 사용됩니다.
그러고나서 Line~26 은 성공했다면 해당 file 구조체로의 포인터를 리턴하고,
실패했다면 \co{NULL} 을 리턴하게 됩니다.
\iffalse

The \co{fcheck_files()} primitive is a helper function for
\co{fget()}.
It uses the \co{rcu_dereference()} primitive to safely fetch an
RCU-protected pointer for later dereferencing (this emits a
memory barrier on CPUs such as DEC Alpha in which data dependencies
do not enforce memory ordering).
Line~22 uses \co{rcu_dereference()} to fetch a pointer to this
task's current file-descriptor table,
and line~24 checks to see if the specified file descriptor is in range.
If so, line~25 fetches the pointer to the file structure, again using
the \co{rcu_dereference()} primitive.
Line~26 then returns a pointer to the file structure or \co{NULL}
in case of failure.
\fi

\co{fput()} 기능은 file 구조체로의 레퍼런스를 해제합니다.
Line~31 은 어토믹하게 레퍼런스 카운트를 감소시키고, 만약 그 감소 결과가 0이라면
line~32 에서 해당 file 구조체를 모든 현재 수행중인 RCU read-side 크리티컬
섹션들이 완료된 후에 (\co{call_rcu()} 의 두번째 인자로 전달되는
\co{file_free_rcu()} 를 통해) 해제하기 위해 \co{call_rcu()} 기능을
실행시킵니다.
모든 현재 수행중인 RCU read-side 크리티컬 섹션들이 완료되기까지 필요한 시간은
``grace period'' 라고 불립니다.
\co{atomic_dec_and_test()} 기능은 메모리 배리어를 포함하고 있음을 알아두시기
바랍니다.
이 메모리 배리어는 이 예제에서는 필요치 않은데, 이 구조체는 RCU read-side
크리티컬 섹션이 완료되기 전까지는 해제될 수 없기 때문입니다만 리눅스에서는 그
결과를 리턴하는 모든 어토믹 오퍼레이션들은 정의에 의해 메모리 배리어를
포함해야만 합니다.
\iffalse

The \co{fput()} primitive releases a reference to a file structure.
Line~31 atomically decrements the reference count, and, if the result
was zero, line~32 invokes the \co{call_rcu()} primitives in order to
free up the file structure (via the \co{file_free_rcu()} function
specified in \co{call_rcu()}'s second argument),
but only after all currently-executing
RCU read-side critical sections complete.
The time period required for all currently-executing RCU read-side
critical sections to complete is termed a ``grace period''.
Note that the \co{atomic_dec_and_test()} primitive contains
a memory barrier.
This memory barrier is not necessary in this example, since the structure
cannot be destroyed until the RCU read-side critical section completes,
but in Linux, all atomic operations that return a result must
by definition contain memory barriers.
\fi

일단 grace period 가 완료되면, \co{file_free_rcu()} 함수가 line~39 에서 file
구조체로의 포인터를 가져오고 line~40 에서 해제시킵니다.

이런 방법은 리눅스의 가상 메모리 시스템에서도 사용되는데, 페이지 구조체를
위해선 \co{get_page_unless_zero()} 와 \co{put_page_testzero()} 를, 메모리 맵
구조체를 위해서는 \co{try_to_unuse()} 와 \co{mmput()} 을 참고하시기 바랍니다.
\iffalse

Once the grace period completes, the \co{file_free_rcu()} function
obtains a pointer to the file structure on line~39, and frees it
on line~40.

This approach is also used by Linux's virtual-memory system,
see \co{get_page_unless_zero()} and \co{put_page_testzero()} for
page structures as well as \co{try_to_unuse()} and \co{mmput()}
for memory-map structures.
\fi

\subsection{Linux Primitives Supporting Reference Counting}
\label{sec:together:Linux Primitives Supporting Reference Counting}

앞의 예제들에서 사용된 리눅스 커널의 기능들이 다음의 리스트에 요약되어
있습니다.
\IfInBook{}{RCU 기능들은 일부 독자들에게는 친숙하지 않을 수도 있으므로,
	    Section~\ref{sec:together:Background on RCU} 에 인용과 간략한
	    개요가 있습니다.}
\iffalse

The Linux-kernel primitives used in the above examples are
summarized in the following list.
\IfInBook{}{The RCU primitives may be unfamiliar to some readers,
	    so a brief overview with citations is included in
	    Section~\ref{sec:together:Background on RCU}.}
\fi

\begin{description}[style=nextline]
\item	[\tco{atomic_t}]
	어토믹하게 조정되는 32-bit 크기의 것을 위한 타입 정의.
\item	[\tco{void atomic_dec(atomic_t *var);}]
	메모리 배리어를 요청하거나 컴파일러 최적화를 불능화시키지 않으며
	참조된 변수의 값을 어토믹하게 감소.
\item	[\tco{int atomic_dec_and_test(atomic_t *var);}]
	어토믹하게 참조된 변수의 값을 감소시키고 그 감소 결과값이 0이라면
	\co{true} (0 이 아닌 값) 을 리턴.
	이 기능의 앞뒤로 메모리 참조가 옮겨가지 못하도록 메모리 배리어와
	컴파일러 최적화 불능화를 수행.
\item	[\tco{void atomic_inc(atomic_t *var);}]
	메모리 배리어를 요청하거나 컴파일러 최적화를 불능화시키지 않으며 참조된
	변수의 값을 어토믹하게 증가.
\item	[\tco{int atomic_inc_not_zero(atomic_t *var);}]
	참조된 변수의 값이 0이 아니라면 그 값을 어토믹하게 증가시키고 값 증가가
	수행되었다면 \co{true} (0이 아닌 값) 을 리턴.
	이 기능의 앞뒤로 메모리 참조가 옮겨가지 못하도록 메모리 배리어와
	컴파일러 최적화 불능화를 수행.
\iffalse

\item	[\tco{atomic_t}]
	Type definition for 32-bit quantity to be manipulated atomically.
\item	[\tco{void atomic_dec(atomic_t *var);}]
	Atomically decrements the referenced variable without necessarily
	issuing a memory barrier or disabling compiler optimizations.
\item	[\tco{int atomic_dec_and_test(atomic_t *var);}]
	Atomically decrements the referenced variable, returning
	\co{true} (non-zero) if the result is zero.
	Issues a memory barrier and disables compiler optimizations that
	might otherwise move memory references across this primitive.
\item	[\tco{void atomic_inc(atomic_t *var);}]
	Atomically increments the referenced variable without necessarily
	issuing a memory barrier or disabling compiler optimizations.
\item	[\tco{int atomic_inc_not_zero(atomic_t *var);}]
	Atomically increments the referenced variable, but only if the
	value is non-zero, and returning \co{true} (non-zero) if the
	increment occurred.
	Issues a memory barrier and disables compiler optimizations that
	might otherwise move memory references across this primitive.
\fi
\item	[\tco{int atomic_read(atomic_t *var);}]
	참조된 변수의 정수 값을 리턴.
	이는 어토믹 오퍼레이션이어야 할 필요가 없고, 메모리 배리어 명령어를
	요청할 필요도 없음.
	``어토믹한 읽기'' 라고 생각하기 보다는 ``어토믹 변수로부터의 평범한
	읽기'' 라고 생각할 것.
\item	[\tco{void atomic_set(atomic_t *var, int val);}]
	참조된 어토믹 변수의 값을 ``val'' 로 설정.
	이는 어토믹 오퍼레이션이어야 할 필요가 없으며, 메모리 배리어나 컴파일러
	최적화의 불능화를 수행할 필요도 없음.
	이를 ``어토믹한 값 설정'' 으로 생각하기 보다는 ``어토믹 변수에의 평범한
	값 설정'' 으로 생각할 것.
\iffalse

\item	[\tco{int atomic_read(atomic_t *var);}]
	Returns the integer value of the referenced variable.
	This need not be an atomic operation, and it need not issue any
	memory-barrier instructions.
	Instead of thinking of as ``an atomic read'', think of it as
	``a normal read from an atomic variable''.
\item	[\tco{void atomic_set(atomic_t *var, int val);}]
	Sets the value of the referenced atomic variable to ``val''.
	This need not be an atomic operation, and it is not required
	to either issue memory
	barriers or disable compiler optimizations.
	Instead of thinking of as ``an atomic set'', think of it as
	``a normal set of an atomic variable''.
\fi
\item	[\tco{void call_rcu(struct rcu_head *head, void (*func)(struct rcu_head *head));}]
	일부 시간이 지나서 현재 수행중인 RCU read-side 크리티컬 섹션들이 완료된
	후에 \co{func(head)} 를 수행하되 \co{call_rcu()} 기능 자체는 즉시
	리턴함.
	\co{head} 는 일반적으로 RCU 로 보호되는 데이터 구조체이고, \co{func} 는
	일반적으로 이 데이터 구조체를 메모리에서 해제시키는 함수임을 알아둘 것.
	\co{call_rcu()} 의 호출과 \co{func} 호출 사이의 시간 간격은 ``grace
	period'' 라 불리움.
	Grace period 를 포함하는 모든 시간 간격은 그 자체로 grace period 임.
\item	[\tco{type *container_of(p, type, f);}]
	명시된 타입의 구조체의 필드 \co{f} 로의 포인터 \co{p} 를 받고 해당
	구조체로의 포인터를 리턴.
\item	[\tco{void rcu_read_lock(void);}]
	RCU read-side 크리티컬 섹션의 시작을 표시.
\item	[\tco{void rcu_read_unlock(void);}]
	RCU read-side 크리티컬 섹션의 종료를 표시.
	RCU read-side 크리티컬 섹션은 중첩될 수 있음.
\item	[\tco{void smp_mb__before_atomic_dec(void);}]
	해당 플랫폼의 \co{atomic_dec()} 기능이 그러지 않는다면 코드를 움직일 수
	있는 컴파일러 최적화의 불능화와 메모리 배리어를 요청.
\item	[\tco{struct rcu_head}]
	Grace period 를 기다리는 오브젝트들을 추적하기 위해 RCU 설비에서
	사용되는 데이터 구조체.
	일반적으로 RCU 로 보호되는 데이터 구조체 내에 하나의 필드로 포함되어
	있음.
\iffalse

\item	[\tco{void call_rcu(struct rcu_head *head, void (*func)(struct rcu_head *head));}]
	Invokes \co{func(head)} some time after all currently executing RCU
	read-side critical sections complete, however, the \co{call_rcu()}
	primitive returns immediately.
	Note that \co{head} is normally a field within an RCU-protected
	data structure, and that \co{func} is normally a function that
	frees up this data structure.
	The time interval between the invocation of \co{call_rcu()} and
	the invocation of \co{func} is termed a ``grace period''.
	Any interval of time containing a grace period is itself a
	grace period.
\item	[\tco{type *container_of(p, type, f);}]
	Given a pointer \co{p} to a field \co{f} within a structure
	of the specified type, return a pointer to the structure.
\item	[\tco{void rcu_read_lock(void);}]
	Marks the beginning of an RCU read-side critical section.
\item	[\tco{void rcu_read_unlock(void);}]
	Marks the end of an RCU read-side critical section.
	RCU read-side critical sections may be nested.
\item	[\tco{void smp_mb__before_atomic_dec(void);}]
	Issues a memory barrier and disables code-motion compiler
	optimizations only if the platform's \co{atomic_dec()}
	primitive does not already do so.
\item	[\tco{struct rcu_head}]
	A data structure used by the RCU infrastructure to track
	objects awaiting a grace period.
	This is normally included as a field within an RCU-protected
	data structure.
\fi
\end{description}

\QuickQuiz{}
	어토믹하지 않은 \co{atomic_read()} 와 \co{atomic_set()} 이라구요?
	이건 또 무슨 농담이죠???
	\iffalse

	An \co{atomic_read()} and an \co{atomic_set()} that are
	non-atomic?
	Is this some kind of bad joke???
	\fi
\QuickQuizAnswer{
	그렇게 보일수도 있겠습니다만, 현재 문제시 되는 어토믹 변수에 어떤 다른
	CPU 도 접근을 하지 않는 상황에서는 실제 어토믹 인스트럭션의 오버헤드는
	낭비가 될 수 있습니다.
	다른 CPU 가 접근하지 안는 경우의 두가지 예는 초기화와 정리 작업 입니다.
	\iffalse

	It might well seem that way, but in situations where no other
	CPU has access to the atomic variable in question, the overhead
	of an actual atomic instruction would be wasteful.
	Two examples where no other CPU has access are
	during initialization and cleanup.
	\fi
} \QuickQuizEnd

\subsection{Counter Optimizations}
\label{sec:together:Counter Optimizations}

카운터 증가와 감소가 흔하게 일어나지만 카운터 값이 0인지에 대한 검사는 가끔만
이뤄지는 일부 경우들에 있어서는,
Chapter~\ref{chp:Counting} 에서 이야기한 것처럼 per-CPU 또는 per-task
카운터들을 두는게 말이 될겁니다.
이런 테크닉이 RCU 에 적용된 예를 보기 위해 sleepable read-copy update (SRCU) 에
대한 논문~\cite{PaulEMcKenney2006c} 을 참고하세요.
이 방법은 카운터 값 증가와 감소 기능에서의 어토믹 명령어나 메모리 배리어의
사용의 필요를 없애줍니다만, 코드의 위치를 움직이는 컴파일러 최적화의 제거는
여전히 필요합니다.
또한, 합산된 레퍼런스 카운트가 0이 되었는지를 체크하는 \co{synchronize_srcu()}
와 같은 기능은 상당히 느려집니다.
이는 이런 테크닉은 레퍼런스가 빈번히 획득되고 해제되지만 레퍼런스 카운트가 0이
되었는지에 대한 검사는 가끔만 일어나는 상황을 위해서 설계된 것이란 점을
강조합니다.
\iffalse

In some cases where increments and decrements are common, but checks
for zero are rare, it makes sense to maintain per-CPU or per-task
counters, as was discussed in Chapter~\ref{chp:Counting}.
See the paper on sleepable read-copy update
(SRCU) for an example of this technique applied to
RCU~\cite{PaulEMcKenney2006c}.
This approach eliminates the need for atomic instructions or memory
barriers on the increment and decrement primitives, but still requires
that code-motion compiler optimizations be disabled.
In addition, the primitives such as \co{synchronize_srcu()}
that check for the aggregate reference
count reaching zero can be quite slow.
This underscores the fact that these techniques are designed
for situations where the references are frequently acquired and
released, but where it is rarely necessary to check for a zero
reference count.
\fi

% @@@ Difficulty of managing reference counts: leaks, premature freeing.

하지만, 레퍼런스 카운트의 사용은 레퍼런스 카운트가 사용되지 않았다면 읽기
전용이었을 수 있는 데이터 구조에 (대부분의 경우 어토믹한) 쓰기를 필요로 한다는
점이 문제인 경우가 많습니다.
이런 경우, 레퍼런스 카운트는 읽기를 하는 쓰레드에게 비싼 캐시 미스를
부과합니다.

따라서 읽기를 하는 쓰레드가 횡단하게 되는 데이터 구조체에 쓰기를 하지 않아도
되게 하는 동기화 메커니즘을 찾아볼 가치가 있습니다.
그런 것들 중 하나는
Section~\ref{sec:defer:Hazard Pointers} 에서 다룬 해저드 포인터이고, 또 다른
하나는
Section~\ref{sec:defer:Read-Copy Update (RCU)} 에서 다룬 RCU 입니다.
\iffalse

However, it is usually the case that use of reference counts requires
writing (often atomically) to a data structure that is otherwise
read only.
In this case, reference counts are imposing expensive cache misses
on readers.

It is therefore worthwhile to look into synchronization mechanisms
that do not require readers to write to the data structure being
traversed.
One possibility is the hazard pointers covered in
Section~\ref{sec:defer:Hazard Pointers}
and another is RCU, which is covered in
Section~\ref{sec:defer:Read-Copy Update (RCU)}.
\fi
