% defer/rcuAPI.tex

\subsection{RCU Linux-Kernel API}
\label{sec:defer:RCU Linux-Kernel API}
\OriginallyPublished{Section}{sec:defer:RCU Linux-Kernel API}{RCU Linux-Kernel API}{Linux Weekly News}{PaulEMcKenney2008WhatIsRCUAPI}

이 섹션은 RCU를 리눅스 커널 API 관점에서 봅니다.
Section~\ref{sec:defer:RCU has a Family of Wait-to-Finish APIs}
은 RCU 의 wait-to-finish API들을, 그리고
Section~\ref{sec:defer:RCU has Publish-Subscribe and Version-Maintenance APIs}
에서는 RCU 의 publish-subscribe 와 버전 관리 API 를 소개합니다.
마지막으로,
Section~\ref{sec:defer:So, What is RCU Really?}
에서는 결론을 정리합니다.
\iffalse

This section looks at RCU from the viewpoint of its Linux-kernel API.
Section~\ref{sec:defer:RCU has a Family of Wait-to-Finish APIs}
presents RCU's wait-to-finish APIs, and
Section~\ref{sec:defer:RCU has Publish-Subscribe and Version-Maintenance APIs}
presents RCU's publish-subscribe and version-maintenance APIs.
Finally,
Section~\ref{sec:defer:So, What is RCU Really?}
presents concluding remarks.
\fi

\subsubsection{RCU has a Family of Wait-to-Finish APIs}
\label{sec:defer:RCU has a Family of Wait-to-Finish APIs}

\begin{sidewaystable*}[htbp]
\rowcolors{1}{}{lightgray}
\renewcommand*{\arraystretch}{1.3}
\centering
\caption{RCU Wait-to-Finish APIs}
\label{tab:defer:RCU Wait-to-Finish APIs}
\footnotesize
\begin{tabularx}{7.9in}{>{\raggedright\arraybackslash}p{1.08in}
    >{\raggedright\arraybackslash}X
    >{\raggedright\arraybackslash}X
    >{\raggedright\arraybackslash}X
    >{\raggedright\arraybackslash}X
    >{\raggedright\arraybackslash}p{1.22in}}
\toprule
Attribute &
    RCU Classic &
	RCU BH &
	    RCU Sched &
		Realtime RCU &
		    SRCU \\
\midrule
Purpose &
    Original &
	Prevent DDoS attacks &
	    Wait for preempt-disable regions, hardirqs, \& NMIs &
	        Realtime response &
		    Sleeping readers \\
Availability &
    2.5.43 &
	2.6.9 &
	    2.6.12 &
		2.6.26 &
		    2.6.19 \\
Read-side primitives &
    \tco{rcu_read_lock()}~! \tco{rcu_read_unlock()}~! &
	\tco{rcu_read_lock_bh()} \tco{rcu_read_unlock_bh()} &
	    \tco{preempt_disable()} \tco{preempt_enable()} (and friends) &
	        \tco{rcu_read_lock()} \tco{rcu_read_unlock()} &
		    \tco{srcu_read_lock()} \tco{srcu_read_unlock()} \\
{ Update-side primitives (synchronous) } &
    { \tco{synchronize_rcu()} \tco{synchronize_net()} } &
	\tco{synchronize_rcu_bh()} &
	    \tco{synchronize_sched()} &
	        { \tco{synchronize_rcu()} \tco{synchronize_net()} } &
		    \tco{synchronize_srcu()} \\
{ Update-side primitives (asynchronous/callback) } &
    \tco{call_rcu()} ! &
	\tco{call_rcu_bh()} &
	    \tco{call_rcu_sched()} &
	        \tco{call_rcu()} &
		    \tco{call_srcu()} \\
{ Update-side primitives (wait for callbacks) } &
    \tco{rcu_barrier()} &
	\tco{rcu_barrier_bh()} &
	    \tco{rcu_barrier_sched()} &
	        \tco{rcu_barrier()} &
		    N/A \\
Type-safe memory &
    \tco{SLAB_DESTROY_BY_RCU} &
	&
	    &
	        \tco{SLAB_DESTROY_BY_RCU} &
		    \\
Read side constraints &
    No blocking &
	No bottom-half (BH) enabling &
	    No blocking &
	        Only preemption and lock acquisition &
		    No \tco{synchronize_srcu()} with same \tco{srcu_struct} \\
Read side overhead &
    Preempt disable/enable (free on non-\tco{PREEMPT}) &
	BH disable/enable &
	    Preempt disable/enable (free on non-\tco{PREEMPT}) &
	        Simple instructions, \IRQ\ disable/enable &
		    Simple instructions, preempt disable/enable, memory barriers \\
Asynchronous update-side overhead &
    sub-microsecond &
	sub-microsecond &
	    sub-microsecond &
	        sub-microsecond &
		    N/A \\
Grace-period latency &
    10s of milliseconds &
	10s of milliseconds &
	    10s of milliseconds &
	        10s of milliseconds &
		    10s of milliseconds \\
Non-\tco{PREEMPT_RT} implementation &
    RCU Classic &
	RCU BH &
	    RCU Classic &
	        Preemptible RCU &
		    SRCU \\
\tco{PREEMPT_RT} implementation &
    Preemptible RCU &
	Realtime RCU &
	    Forced Schedule on all CPUs &
	        Realtime RCU &
		    SRCU \\
\bottomrule
\end{tabularx}
\end{sidewaystable*}

``RCU 는 무엇인가'' 에 대한 가장 단순한 답변은 RCU 는 리눅스 커널에서
사용되는 API 라는 것으로, 각각 잠들 수 없는 버전과 잠들 수 있는 버전 API 의
RCU 읽기 쓰레드 기다리기 부분을 보이는
Table~\ref{tab:defer:RCU Wait-to-Finish APIs},
그리고 그 API 의 publish-subscribe 부분을 보이는
Table~\ref{tab:defer:RCU Publish-Subscribe and Version Maintenance APIs} 에
요약되어 있습니다.
\iffalse

The most straightforward answer to ``what is RCU'' is that RCU is
an API used in the Linux kernel, as summarized by
Table~\ref{tab:defer:RCU Wait-to-Finish APIs},
which shows the wait-for-RCU-readers portions of the non-sleepable and
sleepable APIs, respectively,
and by
Table~\ref{tab:defer:RCU Publish-Subscribe and Version Maintenance APIs},
which shows the publish-subscribe portions of the API.
\fi

RCU 가 처음이라면
Table~\ref{tab:defer:RCU Wait-to-Finish APIs} 의 행들 중 하나에만 집중해 볼
것을 고려해 볼만 한데, 각각의 행은 리눅스 커널의 RCU API 패밀리 중 하나의
멤버를 요약하고 있습니다.
예를 들어, 리눅스 커널에서 RCU 가 어떻게 사용되는지를 이해하고자 하는게 주된
목표라면, ``RCU Classic'' 이 가장 자주 사용되므로 여기서부터 시작하는게 좋을
것입니다.
반면에, 자신의 이익을 위해 RCU 를 이해하고자 한다면 ``SRCU'' 가 가장 간단한 API
를 제공합니다.
나중에도 언제든 다른 행을 볼 수 있습니다.

이미 RCU 에 친숙하다면, 이 표는 유용한 레퍼런스로 사용될 수 있을 겁니다.
\iffalse

If you are new to RCU, you might consider focusing on just one
of the columns in
Table~\ref{tab:defer:RCU Wait-to-Finish APIs},
each of which summarizes one member of the Linux kernel's RCU API family.
For example, if you are primarily interested in understanding how RCU
is used in the Linux kernel, ``RCU Classic'' would be the place to start,
as it is used most frequently.
On the other hand, if you want to understand RCU for its own sake,
``SRCU'' has the simplest API.
You can always come back for the other columns later.

If you are already familiar with RCU, these tables can
serve as a useful reference.
\fi

\QuickQuiz{}
	Table~\ref{tab:defer:RCU Wait-to-Finish APIs} 의 일부 셀들은 왜 느낌표
	(``!'') 를 가지고 있나요?
	\iffalse

	Why do some of the cells in
	Table~\ref{tab:defer:RCU Wait-to-Finish APIs}
	have exclamation marks (``!'')?
	\fi
\QuickQuizAnswer{
	느낌표와 함께 표시된 API 멤버 (\co{rcu_read_lock()},
	\co{rcu_read_unlock()}, 그리고 \co{call_rcu()}) 만이 과거 90년대 중반에
	Paul E. McKenney 가 신경썼던 리눅스 RCU API 의 멤버입니다.
	이 시절에, 그는 그가 RCU 에 대해 알아야 할 것들을 모두 알고 있다는
	잘못된 인상을 가지고 있었습니다.
	\iffalse

	The API members with exclamation marks (\co{rcu_read_lock()},
	\co{rcu_read_unlock()}, and \co{call_rcu()}) were the
	only members of the Linux RCU API that Paul E. McKenney was aware
	of back in the mid-90s.
	During this timeframe, he was under the mistaken impression that
	he knew all that there is to know about RCU.
	\fi
} \QuickQuizEnd

이 ``RCU Classic'' 행은 RCU read-side 크리티컬 섹션들은 \co{rcu_read_lock()} 과
\co{rcu_read_unlock()} 으로 구분지어지고 중첩될수도 있는, 최초의 RCU 구현에
해당합니다.
여기에 연관되는 동기적인 업데이트 쪽 기능들인 \co{synchronize_rcu()} 와 그것과
같은 의미인 \co{synchronize_net()} 은 동시에 실행중인 RCU read-side 크리티컬
섹션들이 모두 완료되기를 기다립니다.
이 기다림의 길이는 ``grace period'' 라고 알려져 있습니다.
비동기적 업데이트 쪽 기능인 \co{call_rcu()} 는 뒤따르는 grace period 후에 특정
함수를 특정 인자와 함께 호출해 줍니다.
예를 들어, \co{call_rcu(p,f);} 는 다음의 grace period 후에 ``RCU callback''
\co{f(p)} 의 호출이 이뤄지게 합니다.
\co{call_rcu()} 를 사용하는 리눅스 커널 모듈을 언로딩 한다던가 해서 모든 RCU
callback 들이 완료되기를 기다려야만 하는 상황도
존재합니다~\cite{PaulEMcKenney2007rcubarrier}.
\co{rcu_barrier()} 기능이 그 일을 합니다.
더 최신의 계층적 RCU~\cite{PaulEMcKenney2008HierarchicalRCU} 구현 또한 ``RCU
Classic'' 시맨틱을 고수함을 알아두세요.
\iffalse

The ``RCU Classic'' column corresponds to the original RCU implementation,
in which RCU read-side critical sections are delimited by
\co{rcu_read_lock()} and \co{rcu_read_unlock()}, which
may be nested.
The corresponding synchronous update-side primitives,
\co{synchronize_rcu()}, along with its synonym
\co{synchronize_net()}, wait for any currently executing
RCU read-side critical sections to complete.
The length of this wait is known as a ``grace period''.
The asynchronous update-side primitive, \co{call_rcu()},
invokes a specified function with a specified argument after a
subsequent grace period.
For example, \co{call_rcu(p,f);} will result in
the ``RCU callback'' \co{f(p)}
being invoked after a subsequent grace period.
There are situations,
such as when unloading a Linux-kernel module that uses \co{call_rcu()},
when it is necessary to wait for all
outstanding RCU callbacks to complete~\cite{PaulEMcKenney2007rcubarrier}.
The \co{rcu_barrier()} primitive does this job.
Note that the more recent hierarchical
RCU~\cite{PaulEMcKenney2008HierarchicalRCU}
implementation also adheres to ``RCU Classic'' semantics.
\fi

마지막으로, RCU 는
Section~\ref{sec:defer:RCU is a Way of Providing Type-Safe Memory} 에서 설명한
것처럼 type-safe 메모리~\cite{Cheriton96a} 를 제공하는데 사용될 수도 있습니다.
RCU 의 문맥에서, type-safe 메모리는 주어진 데이터 원소가 그것에 접근하는 모든
RCU read-side 크리티컬 섹션 사이에서 그 타입이 바뀌지 않는다는 것을 보장합니다.
RCU 기반의 type-safe 메모리를 사용하기 위해서는 \co{SLAB_DESTROY_BY_RCU} 를
\co{kmem_cache_create()} 에 넘겨야 합니다.
\co{SLAB_DESTROY_BY_RCU} 는 \co{kmem_cache_alloc()} 이 \co{kmem_cache_free()}
로 자유가 된 메모리를 즉시 재할당 하는 것을 막는 일은 \emph{결코} 하지 않음을
알아두는 게 중요합니다!
사실, \co{rcu_dereference} 로 리턴된, \co{SLAB_DESTROY_RCU} 로 보호되는 데이터
구조체는 상당히 여러번 메모리 해제되고 재할당 될 수 있는데, 심지어
\co{rcu_read_lock()} 으로 보호되고 있을 때도 그러합니다.
대신, \co{SLAB_DESTROY_BY_RCU} 는 RCU grace period 가 끝나기 전까지는
\co{kmem_cache_free()} 가 완전히 해제된 데이터 구조체들의 slab 을 시스템에
반납하는 것을 방지해 줍니다.
요약해서, 비록 데이터 원소가 굉장히 자주 해제되고 재할당될 수 있지만, 최소한
그것의 타입은 똑같이 남아있을 겁니다.
\iffalse

Finally, RCU may be used to provide
type-safe memory~\cite{Cheriton96a}, as described in
Section~\ref{sec:defer:RCU is a Way of Providing Type-Safe Memory}.
In the context of RCU, type-safe memory guarantees that a given
data element will not change type during any RCU read-side critical section
that accesses it.
To make use of RCU-based type-safe memory, pass
\co{SLAB_DESTROY_BY_RCU} to
\co{kmem_cache_create()}.
It is important to note that \co{SLAB_DESTROY_BY_RCU} will
\emph{in no way}
prevent \co{kmem_cache_alloc()} from immediately reallocating
memory that was just now freed via \co{kmem_cache_free()}!
In fact, the \co{SLAB_DESTROY_BY_RCU}-protected data structure
just returned by \co{rcu_dereference} might be freed and reallocated
an arbitrarily large number of times, even when under the protection
of \co{rcu_read_lock()}.
Instead, \co{SLAB_DESTROY_BY_RCU} operates by preventing
\co{kmem_cache_free()}
from returning a completely freed-up slab of data structures
to the system until after an RCU grace period elapses.
In short, although the data element might be freed and reallocated arbitrarily
often, at least its type will remain the same.
\fi

\QuickQuiz{}
	많은 수의 RCU read-side 크리티컬 섹션들이 \co{synchronize_rcu()} 실행을
	무기한 블록시키는 걸 어떻게 막을 수 있나요?
	\iffalse

	How do you prevent a huge number of RCU read-side critical
	sections from indefinitely blocking a \co{synchronize_rcu()}
	invocation?
	\fi
\QuickQuizAnswer{
	RCU read-side 크리티컬 섹션들이 \co{synchronize_rcu()} 실행을 무기한
	블록하는걸 막을 필요가 전혀 없는데, \co{synchronize_rcu()} 실행은
	\emph{전부터 존재한} RCU read-side 크리티컬 섹션들만 기다리면 되기
	때문입니다.
	따라서 각 RCU read-side 크리티컬 섹션이 한정된 길이를 갖는다면, 아무
	문제가 없습니다.
	\iffalse

	There is no need to do anything to prevent RCU read-side
	critical sections from indefinitely blocking a
	\co{synchronize_rcu()} invocation, because the
	\co{synchronize_rcu()} invocation need wait only for
	\emph{pre-existing} RCU read-side critical sections.
	So as long as each RCU read-side critical section is
	of finite duration, there should be no problem.
	\fi
} \QuickQuizEnd

\QuickQuiz{}
	\co{synchronize_rcu()} API 는 전부터 존재한 인터럽트 핸들러들이 모두
	완료되길 기다리죠, 맞죠?
	\iffalse

	The \co{synchronize_rcu()} API waits for all pre-existing
	interrupt handlers to complete, right?
	\fi
\QuickQuizAnswer{
	전혀 그렇지 않아요!
	그리고 preemption 가능한 RCU 를 사용하고 있다면 특히나 그렇지 않습니다!
	그러고 싶다면 \co{synchronize_irq()} 를 사용해야 합니다.
	또는, 대안적으로 \co{synchronize_rcu()} 가 기다리기를 바라는 인터럽트
	핸들러 안에 \co{rcu_read_lock()} 과 \co{rcu_read_unlock()} 를위치시킬
	수 있겠습니다.
	\iffalse

	Absolutely not!
	And especially not when using preemptible RCU!
	You instead want \co{synchronize_irq()}.
	Alternatively, you can place calls to \co{rcu_read_lock()}
	and \co{rcu_read_unlock()} in the specific interrupt handlers that
	you want \co{synchronize_rcu()} to wait for.
	\fi
} \QuickQuizEnd

``RCU BH'' 행에서, \co{rcu_read_lock_bh()} 와 \co{rcu_read_unlock_bh()} 는
크리티컬 섹션을 구분짓고, \co{synchronize_rcu_bh()} 는 하나의 grace period 를
기다리며, \co{call_rcu_bh()} 는 다음 grace period 후에 특정 함수를 특정 인자와
함께 호출해 줍니다.
\iffalse

In the ``RCU BH'' column, \co{rcu_read_lock_bh()} and
\co{rcu_read_unlock_bh()} delimit RCU read-side critical
sections, \co{synchronize_rcu_bh()} waits for a grace period,
and \co{call_rcu_bh()} invokes the specified
function and argument after a later grace period.
\fi

\QuickQuiz{}
	이것들을 섞어서 활용하면 어떻게 되나요?
	예를 들어, \co{rcu_read_lock()} 과 \co{rcu_read_unlock()} 을 RCU
	read-side 크리티컬 섹션을 구분하는데 사용하지만 \co{call_rcu_bh()} 를
	RCU callback 을 위해 사용한다고 하면요?
	\iffalse

	What happens if you mix and match?
	For example, suppose you use \co{rcu_read_lock()} and
	\co{rcu_read_unlock()} to delimit RCU read-side critical
	sections, but then use \co{call_rcu_bh()} to post an
	RCU callback?
	\fi
\QuickQuizAnswer{
	\co{call_rcu_bh()} 가 실행되는 시점에 \co{rcu_read_lock_bh()} 와
	\co{rcu_read_unlock_bh()} 로 구분된 RCU read-side 크리티컬 섹션들이
	존재하지 않는다면 RCU 는 이 콜백을 곧바로 수행해도 되는 권한을 갖게
	되어서, 해당 RCU read-side 크리티컬 섹션이 사용중인 데이터 구조체를
	메모리에서 해제해버릴 수도 있습니다!
	이건 단순히 이론적인 가능성이 아닙니다: \co{rcu_read_lock()} 과
	\co{rcu_read_unlock()} 으로 구분지어졌고 오랫동안 동작중인 RCU
	read-side 크리티컬 섹션은 이런 실패에 취약합니다.

	하지만, \co{rcu_dereference()} 함수들은 모든 RCU 변종들에 적용됩니다.
	(변종별 \co{rcu_dereference()} 를 만들려는 시도도 있었지만, 그건 너무
	혼란스러웠습니다.)
	\iffalse

	If there happened to be no RCU read-side critical
	sections delimited by \co{rcu_read_lock_bh()} and
	\co{rcu_read_unlock_bh()} at the time \co{call_rcu_bh()}
	was invoked, RCU would be within its rights to invoke the callback
	immediately, possibly freeing a data structure still being used by
	the RCU read-side critical section!
	This is not merely a theoretical possibility: a long-running RCU
	read-side critical section delimited by \co{rcu_read_lock()}
	and \co{rcu_read_unlock()} is vulnerable to this failure mode.

	However, the \co{rcu_dereference()} family of functions apply
	to all flavors of RCU.
	(There was an attempt to have per-flavor variants of
	\co{rcu_dereference()}, but it was just too messy.)
	\fi
} \QuickQuizEnd

\QuickQuiz{}
	하드웨어 인터럽트 핸들러들은 묵시적인 \co{rcu_read_lock_bh()} 의 보호
	아래 있다고 생각되어도 되겠죠?
	\iffalse

	Hardware interrupt handlers can be thought of as being
	under the protection of an implicit \co{rcu_read_lock_bh()},
	right?
	\fi
\QuickQuizAnswer{
	전혀 그렇지 않아요!
	그리고 preemption 가능한 RCU 를 사용중일 때에는 특히나 그렇지 않습니다!
	``rcu\_bh'' 로 보호되는 데이터 구조체를 인터럽트 핸들러 내에서 접근하려
	한다면, 명시적으로 \co{rcu_read_lock_bh()} 와 \co{rcu_read_unlock_bh()}
	를 사용해야 합니다.
	\iffalse

	Absolutely not!
	And especially not when using preemptible RCU!
	If you need to access ``rcu\_bh''-protected data structures
	in an interrupt handler, you need to provide explicit calls to
	\co{rcu_read_lock_bh()} and \co{rcu_read_unlock_bh()}.
	\fi
} \QuickQuizEnd

``RCU Sched'' 행에서, preemption 을 불가능하게 하는 모든 동작은 RCU read-side
크리티컬 섹션처럼 동작되고, \co{synchronize_sched()} 는 연관된 RCU grace period
를 기다립니다.
이 RCU API 패밀리는 2.6.12 커널에서 들어왔는데, 이것이 과거의
\co{synchronize_kernel()} API 를 (RCU Classic 을 위한) 지금의
\co{synchronize_rcu()} 와 (RCU Sched 를 위한) \co{synchronize_sched()} 로
나눴습니다.
RCU Sched 는 처음부터 비동기적인 \co{call_rcu_sched()} 인터페이스를 가지고 있진
않았다가 2.6.26 에서 추가되었음을 알아두세요.
리눅스 커뮤니티의 어떤 의미에서의 minimalist 철학에 의해, API 들은 필요한
경우에 기반해서 추가됩니다.
\iffalse

In the ``RCU Sched'' column, anything that disables preemption
acts as an RCU read-side critical section, and \co{synchronize_sched()}
waits for the corresponding RCU grace period.
This RCU API family was added in the 2.6.12 kernel, which split the
old \co{synchronize_kernel()} API into the current
\co{synchronize_rcu()} (for RCU Classic) and
\co{synchronize_sched()} (for RCU Sched).
Note that RCU Sched did not originally have an asynchronous
\co{call_rcu_sched()} interface, but one was added in 2.6.26.
In accordance with the quasi-minimalist philosophy of the Linux
community, APIs are added on an as-needed basis.
\fi

\QuickQuiz{}
	RCU Classic 과 RCU Sched 를 섞어서 사용하면 어떻게 되나요?
	\iffalse

	What happens if you mix and match RCU Classic and RCU Sched?
	\fi
\QuickQuizAnswer{
	Non-\co{PREEMPT} 나 \co{PREEMPT} 커널에서는 이 두개의 일은 우연히
	섞이게 되는데, 그런 커널 빌드에서 RCU Classic 과 RCU Sched 는 같은
	구현으로 매핑되기 때문입니다.
	하지만, 이런 조합은 -rt 패치셋을 사용하는 \co{PREEMPT_RT} 빌드에서는
	치명적인데 Realtime RCU 의 read-side 크리티컬 섹션들은 preemption 당할
	수 있어서 \co{synchronize_sched()} 가 RCU read-side 크리티컬 섹션이
	\co{rcu_read_unlock()} 호출을 하기 전에 리턴할 수 있기 때문입니다.
	이는 데이터 구조체가 그 구조체를 사용하는 read-side 크리티컬 섹션이
	끝나기 전에 메모리 해제될 수 있게 해서 커널의 보험 통계적 리스크를 무척
	증가시킬 수 있습니다.

	실제로, RCU Classic 과 RCU Sched 의 분리는 preemption 가능해야 하는 RCU
	read-side 크리티컬 섹션들로부터 영감을 받았습니다.
	\iffalse

	In a non-\co{PREEMPT} or a \co{PREEMPT} kernel, mixing these
	two works ``by accident'' because in those kernel builds, RCU Classic
	and RCU Sched map to the same implementation.
	However, this mixture is fatal in \co{PREEMPT_RT} builds using the -rt
	patchset, due to the fact that Realtime RCU's read-side critical
	sections can be preempted, which would permit
	\co{synchronize_sched()} to return before the
	RCU read-side critical section reached its \co{rcu_read_unlock()}
	call.
	This could in turn result in a data structure being freed before the
	read-side critical section was finished with it,
	which could in turn greatly increase the actuarial risk experienced
	by your kernel.

	In fact, the split between RCU Classic and RCU Sched was inspired
	by the need for preemptible RCU read-side critical sections.
	\fi
} \QuickQuizEnd

\QuickQuiz{}
	일반적으로, 모든 전부터 존재한 인터럽트 핸들러들을 기다리는데에
	\co{synchronize_sched()} 에 의존해선 안됩니다, 맞죠?
	\iffalse

	In general, you cannot rely on \co{synchronize_sched()} to
	wait for all pre-existing interrupt handlers,
	right?
	\fi
\QuickQuizAnswer{
	맞습니다!
	-rt 리눅스는 쓰레드로 도는 인터럽트 핸들러를 사용하기 때문에, 인터럽트
	핸들러 내부에서의 컨텍스트 스위치가 있을 수 있습니다.
	\co{synchronize_sched()} 는 각 CPU 가 컨텍스트 스위치를 할 때까지만
	기다리기 때문에, 특정 인터럽트 핸들러가 완료되기 전에 리턴을 할수도
	있습니다.

	특정 인터럽트 핸들러가 완료되기까지 기다려야 한다면, 그대신
	\co{synchronize_irq()} 를 사용하거나 기다리기 원하는 인터럽트 핸들러
	안에 명시적으로 RCU read-side 크리티컬 섹션을 넣어야 합니다.
	\iffalse

	That is correct!
	Because -rt Linux uses threaded interrupt handlers, there can
	be context switches in the middle of an interrupt handler.
	Because \co{synchronize_sched()} waits only until each
	CPU has passed through a context switch, it can return
	before a given interrupt handler completes.

	If you need to wait for a given interrupt handler to complete,
	you should instead use \co{synchronize_irq()} or place
	explicit RCU read-side critical sections in the interrupt
	handlers that you wish to wait on.
	\fi
} \QuickQuizEnd

``Realtime RCU'' 행은 RCU Classic 과 똑같은 API 를 가지고 있는데, 차이점은 RCU
read-side 크리티컬 섹션들이 preemption 당할 수 있고 spinlock 을 획득하는 사이
블락될 수 있다는 점 뿐입니다.
Realtime RCU 의 설계는 다른곳에도 설명되어
있습니다~\cite{PaulEMcKenney2007PreemptibleRCU}.
\iffalse

The ``Realtime RCU'' column has the same API as does
RCU Classic, the only difference being that RCU read-side critical
sections may be preempted and may block while acquiring spinlocks.
The design of Realtime RCU is described
elsewhere~\cite{PaulEMcKenney2007PreemptibleRCU}.
\fi

Table~\ref{tab:defer:RCU Wait-to-Finish APIs} 의 ``SRCU'' 행은 RCU
read-side 크리티컬 섹션들 안에서 일반적인 잠들기를 허용하는 특수한 RCU API 를
보입니다~\cite{PaulEMcKenney2006c}.
물론, SRCU read-side 크리티컬 섹션 안에서의 \co{synchronize_srcu()} 사용은
스스로의 deadlock 을 유발할 수 있으므로, 반드시 피해져야 합니다.
앞의 RCU 구현들과 SRCU 의 차이점은 각각의 별개의 SRCU 사용처마다 호출자가
\co{srcu_struct} 를 할당해야 한다는 겁니다.
이 방법은 SRCU read-side 크리티컬 섹션이 연관되지 않은 \co{synchronize_srcu()}
실행을 블록하는 것을 방지합니다.
또한, 이 RCU 변종에서는 \co{srcu_read_lock()} 이 연관된 \co{srcu_read_unlock()}
에 전달되어야 하는 값을 리턴합니다.
\iffalse

The ``SRCU'' column in
Table~\ref{tab:defer:RCU Wait-to-Finish APIs}
displays a specialized RCU API that permits
general sleeping in RCU read-side critical
sections~\cite{PaulEMcKenney2006c}.
Of course,
use of \co{synchronize_srcu()} in an SRCU read-side
critical section can result in
self-deadlock, so should be avoided.
SRCU differs from earlier RCU implementations in that the caller
allocates an \co{srcu_struct} for each distinct SRCU
usage.
This approach prevents SRCU read-side critical sections from blocking
unrelated \co{synchronize_srcu()} invocations.
In addition, in this variant of RCU, \co{srcu_read_lock()}
returns a value that must be passed into the corresponding
\co{srcu_read_unlock()}.
\fi

\QuickQuiz{}
	\co{call_srcu()} 사용을 조심해야 하는 이유는 무엇일까요?
	\iffalse

	Why should you be careful with \co{call_srcu()}?
	\fi
\QuickQuizAnswer{
	하나의 태스크는 SRCU 콜백들을 매우 빠르게 등록할 수 있습니다.
	SRCU 가 읽기 쓰레드들이 임의의 기간동안 블록될 수 있도록 허용함을
	생각하면, 이는 임의의 커다란 양의 메모리를 소모할 것을 알 수 있습니다.
	반대로, 동기적인 \co{synchronize_srcu()} 인터페이스에서는 주어진
	태스크는 다음 grace period 를 기다리는 걸 시작하기 전에 주어진 grace
	period 를 기다리는 것을 마무리 해야만 합니다.
	\iffalse

	A single task could register SRCU callbacks very quickly.
	Given that SRCU allows readers to block for arbitrary periods of
	time, this could consume an arbitrarily large quantity of memory.
	In contrast, given the synchronous \co{synchronize_srcu()}
	interface, a given task must finish waiting for a given grace period
	before it can start waiting for the next one.
	\fi
} \QuickQuizEnd

\QuickQuiz{}
	어떤 조건에서 \co{synchronize_srcu()} 가 SRCU read-side 크리티컬 섹션
	내에서 안전하게 사용될 수 있을까요?
	\iffalse

	Under what conditions can \co{synchronize_srcu()} be safely
	used within an SRCU read-side critical section?
	\fi
\QuickQuizAnswer{
	원론적으로, 특정 \co{srcu_struct} 와 함께, \co{synchronize_srcu()} 는
	어떤 다른 \co{srcu_struct} 를 사용하는 SRCU read-side 크리티컬 섹션
	안에서 사용될 수 있습니다.
	하지만, 실제에서는 이런 짓을 하는건 거의 분명하게 나쁜 생각입니다.
	세부적으로는
	Figure~\ref{fig:defer:Multistage SRCU Deadlocks} 에 보인 코드가 여전히
	데드락을 일으킬 수 있을 것입니다.
	\iffalse

	In principle, you can use
	\co{synchronize_srcu()} with a given \co{srcu_struct}
	within an SRCU read-side critical section that uses some other
	\co{srcu_struct}.
	In practice, however, doing this is almost certainly a bad idea.
	In particular, the code shown in
	Listing~\ref{lst:defer:Multistage SRCU Deadlocks}
	could still result in deadlock.
	\fi
%
\begin{listing}[htbp]
\scriptsize
{
\begin{verbbox}
  1 idx = srcu_read_lock(&ssa);
  2 synchronize_srcu(&ssb);
  3 srcu_read_unlock(&ssa, idx);
  4
  5 /* . . . */
  6
  7 idx = srcu_read_lock(&ssb);
  8 synchronize_srcu(&ssa);
  9 srcu_read_unlock(&ssb, idx);
\end{verbbox}
}
\centering
\theverbbox
\caption{Multistage SRCU Deadlocks}
\label{lst:defer:Multistage SRCU Deadlocks}
\end{listing}
} \QuickQuizEnd

리눅스 커널은 분명히 놀랍도록 많은 RCU API 와 구현을 가지고 있습니다.
이 수를 줄였으면 하는 바람이 있는데, 리눅스 커널의 특정 빌드는 현재 세개의
API 뒤에 최대 네개의 구현을 가지고 있다는 사실 (RCU Classic 과 Realtime
RCU 는 같은 API 를 공유합니다) 이 그 증거입니다.
하지만,많은 락킹 API 가운데 하나를 제거하려면 필요한 것처럼 충분한 조사와
분석이 필요할 겁니다.

이 다양한 RCU API 들은 RCU read-side 크리티컬 섹션들이 반드시 제공해야 하는
forward-progress 보장사항과 그 범위로 차별화되는데, 다음과 같습니다:
\iffalse

The Linux kernel currently has a surprising number of RCU APIs and
implementations.
There is some hope of reducing this number, evidenced by the fact
that a given build of the Linux kernel currently has at most
four implementations behind three APIs (given that RCU Classic
and Realtime RCU share the same API).
However, careful inspection and analysis will be required, just as
would be required in order to eliminate one of the many locking APIs.

The various RCU APIs are distinguished by the forward-progress
guarantees that their RCU read-side critical sections must provide,
and also by their scope, as follows:
\fi

\begin{enumerate}
\item	RCU BH: read-side 크리티컬 섹션들은 NMI 와 인터럽트 핸들러를 제외한
	모든 것에 대해 forward progress 를 보장해야만 합니다만
	software-interrupt (\co{softirq}) 핸들러는 제외입니다.
	RCU BH 는 범위내에서 글로벌 합니다.
\item	RCU Sched: read-side 크리티컬 섹션들은 NMI 와 \co{softirq} 핸들러를
	포함한 \IRQ\ 핸들러들을 제외한 모든것들에 forward progress 를 보장해야
	합니다.
	RCU Sched 는 범위내에서 글로벌합니다.
\item	RCU (Classic 과 Real-time 둘 다): read-side 크리티컬 섹션들은 NMI
	핸들러, \IRQ\ 핸들러, \co{softirq} 핸들러, 그리고 (real-time 의 경우) 더
	높은 우선순위의 real-time task 를 제외한 모든 것에 forward progress 를
	보장해야 합니다.
	RCU 는 범위내에서 글로벌합니다.
\item	SRCU: read-side 크리티컬 섹션은 다른 태스크가 연관된 grace period 가
	완료되기를 기다리고 있지 않다면 forward progress 를 보장하지 않아도
	됩니다.
	이런 상황에서 이 read-side 크리티컬 섹션들은 수초 이내에는 완료되어야
	합니다 (그리고 더 빠르면 더 좋습니다).\footnote{
		단순히 forward-progress guarantee 가 없다고 말하는 대신 이런
		명확한 설명을 하도록 재촉해준 James Bottomley 에게 감사의
		말씀을 드립니다.}
	SRCU 의 범위는 각각 연관된 \co{srcu_struct} 의 사용에 따라 정의됩니다.
\iffalse

\item	RCU BH: read-side critical sections
	must guarantee forward progress against everything except for
	NMI and interrupt handlers, but not including software-interrupt
	(\co{softirq}) handlers.
	RCU BH is global in scope.
\item	RCU Sched: read-side critical sections must guarantee forward
	progress against everything except for NMI and \IRQ\ handlers,
	including \co{softirq} handlers.
	RCU Sched is global in scope.
\item	RCU (both classic and real-time): read-side critical sections
	must guarantee forward progress against everything except for
	NMI handlers, \IRQ\ handlers, \co{softirq} handlers, and (in the
	real-time case) higher-priority real-time tasks.
	RCU is global in scope.
\item	SRCU: read-side critical sections need not guarantee
	forward progress unless some other task is waiting for the
	corresponding grace period to complete, in which case these
	read-side critical sections should complete in no more than
	a few seconds (and preferably much more quickly).\footnote{
		Thanks to James Bottomley for urging me to this
		formulation, as opposed to simply saying that
		there are no forward-progress guarantees.}
	SRCU's scope is defined by the use of the corresponding
	\co{srcu_struct}.
\fi
\end{enumerate}

달리 말하자면, SRCU 는 개발자가 그 범위를 제한할 수 있도록 하는 것으로
극단적으로 약한 forward-progress 보장사항의 문제를 보완합니다.
\iffalse

In other words, SRCU compensate for their extremely weak
forward-progress guarantees by permitting the developer to restrict
their scope.
\fi

\subsubsection{RCU has Publish-Subscribe and Version-Maintenance APIs}
\label{sec:defer:RCU has Publish-Subscribe and Version-Maintenance APIs}

다행히도, 다음의 표에 보여진 RCU publish-subscribe 와 version-maintenance
기능들은 앞서 언급된 RCU 의 변종들 모두에 적용됩니다.
이 공통성은 어떤 경우들에는 더 많은 코드가 공유될 수 있게 해서, 그렇지 않다면
일어날 수 있는 API 증식을 분명히 줄여줍니다.
RCU publish-subscribe API 들의 원래 목적은 메모리 배리어들을 이 API 안에
묻어버려서 리눅스 커널 프로그래머들이 리눅스가 지원하는 각각의 20 종류가 넘는
CPU 패밀리들~\cite{Spraul01} 의 메모리 순서 모델의 전문가가 되지 않더라도 RCU
를 사용할 수 있게 하려는 것이었습니다.
\iffalse

Fortunately, the RCU publish-subscribe and version-maintenance
primitives shown in the following
table apply to all of the variants of RCU discussed above.
This commonality can in some cases allow more code to be shared,
which certainly reduces the API proliferation that would otherwise
occur.
The original purpose of the RCU publish-subscribe APIs was to
bury memory barriers into these APIs, so that Linux kernel
programmers could use RCU without needing to become expert on
the memory-ordering models of each of the 20+ CPU families
that Linux supports~\cite{Spraul01}.
\fi

\begin{table*}[tb]
\renewcommand*{\arraystretch}{1.15}
\footnotesize
\centering
\begin{tabular}{lllp{1.2in}}
\toprule
Category &
	Primitives &
		Availability &
			Overhead \\
\midrule
List traversal &
	\tco{list_for_each_entry_rcu()} &
		2.5.59 &
			Simple instructions (memory barrier on Alpha) \\
\midrule
List update &
	\tco{list_add_rcu()} &
		2.5.44 &
			Memory barrier \\
&
	\tco{list_add_tail_rcu()} &
		2.5.44 &
			Memory barrier \\
&
	\tco{list_del_rcu()} &
		2.5.44 &
			Simple instructions \\
&
	\tco{list_replace_rcu()} &
		2.6.9 &
			Memory barrier \\
&
	\tco{list_splice_init_rcu()} &
		2.6.21 &
			Grace-period latency \\
\midrule
Hlist traversal &
	\tco{hlist_for_each_entry_rcu()} &
		2.6.8 &
			Simple instructions (memory barrier on Alpha) \\
&
	\tco{hlist_add_after_rcu()} &
		2.6.14 &
			Memory barrier \\
&
	\tco{hlist_add_before_rcu()} &
		2.6.14 &
			Memory barrier \\
&
	\tco{hlist_add_head_rcu()} &
		2.5.64 &
			Memory barrier \\
&
	\tco{hlist_del_rcu()} &
		2.5.64 &
			Simple instructions \\
&
	\tco{hlist_replace_rcu()} &
		2.6.15 &
			Memory barrier \\
\midrule
Pointer traversal &
	\tco{rcu_dereference()} &
		2.6.9 &
			Simple instructions (memory barrier on Alpha) \\
\midrule
Pointer update &
	\tco{rcu_assign_pointer()} &
		2.6.10 &
			Memory barrier \\
\bottomrule
\end{tabular}
\caption{RCU Publish-Subscribe and Version Maintenance APIs}
\label{tab:defer:RCU Publish-Subscribe and Version Maintenance APIs}
\end{table*}

카테고리들 중 처음 두개의 카테고리들은 순환형의 이중 링크드 리스트인 리눅스
\co{struct list_head} 리스트들에 동작합니다.
\co{list_for_each_entry_rcu()} 함수는 RCU 로 보호되는 리스트를 type-safe 하게
횡단하는데 새로운 리스트 원소가 횡단과 동시에 삽입되는 상황을 위한 메모리
순서의 강제 역시 합니다.
Alpha 외의 플랫폼들에서, 이 함수는 \co{list_for_each_entry()} 에 비해 성능
하락을 주긴 하지만 그 정도는 적거나 아예 없습니다.
\co{list_add_rcu()}, \co{list_add_tail_rcu()}, 그리고 \co{list_replace_rcu()}
함수들은 RCU 아닌 비슷한 것들과 유사합니다만 완화된 순서의 기계들에서는
추가적인 메모리 배리어로의 오버헤드를 갖습니다.
\co{list_del_rcu()} 함수는 또한 RCU 아닌 비슷한 것과 유사합니다만,
\co{list_del()} 이 그래야 하는 것처럼 \co{prev} 와 \co{next} 포인터들을 모두
파괴하는게 아니라 \co{prev} 포인터만 파괴하기 때문에 신기하게도 매우 조금
빠릅니다.
마지막으로, \co{list_splice_init_rcu()} 함수는 역시 RCU 아닌 비슷한 것들과
유사합니다만 grace-period 대기시간을 갖습니다.
이 grace period 의 목적은 RCU 읽기 쓰레드들이 원본 리스트의 횡단을 그것이
리스트 헤더로부터 분리되기를 완료하기 전까지 안전하게 마치도록 하는 것입니다 --
이에 실패하는 것은 그런 읽기 쓰레드들이 그 횡단을 마무리하지 못하게 할수도
있습니다.
\iffalse

The first pair of categories operate on Linux
\co{struct list_head} lists, which are circular, doubly-linked
lists.
The \co{list_for_each_entry_rcu()} primitive traverses an
RCU-protected list in a type-safe manner, while also enforcing
memory ordering for situations where a new list element is inserted
into the list concurrently with traversal.
On non-Alpha platforms, this primitive incurs little or no performance
penalty compared to \co{list_for_each_entry()}.
The \co{list_add_rcu()}, \co{list_add_tail_rcu()},
and \co{list_replace_rcu()} primitives are analogous to
their non-RCU counterparts, but incur the overhead of an additional
memory barrier on weakly-ordered machines.
The \co{list_del_rcu()} primitive is also analogous to its
non-RCU counterpart, but oddly enough is very slightly faster due to the
fact that it poisons only the \co{prev} pointer rather than
both the \co{prev} and \co{next} pointers as
\co{list_del()} must do.
Finally, the \co{list_splice_init_rcu()} primitive is similar
to its non-RCU counterpart, but incurs a full grace-period latency.
The purpose of this grace period is to allow RCU readers to finish
their traversal of the source list before completely disconnecting
it from the list header---failure to do this could prevent such
readers from ever terminating their traversal.
\fi

\QuickQuiz{}
	\co{list_del_rcu()} 는 왜 \co{next} 와 \co{prev} 두 포인터를 모두
	파괴하지 않는거죠?
	\iffalse

	Why doesn't \co{list_del_rcu()} poison both the \co{next}
	and \co{prev} pointers?
	\fi
\QuickQuizAnswer{
	\co{next} 포인터를 파괴하는 행위는 이 포인터를 사용해야 하는 동시의 RCU
	읽기 쓰레드들과 간섭될 수 있습니다.
	하지만, RCU 읽기 쓰레드들은 \co{prev} 포인터의 사용으로부터 숨겨져
	있으므로 이건 안전하게 파괴될 수 있습니다.
	\iffalse

	Poisoning the \co{next} pointer would interfere
	with concurrent RCU readers, who must use this pointer.
	However, RCU readers are forbidden from using the \co{prev}
	pointer, so it may safely be poisoned.
	\fi
} \QuickQuizEnd

다음의 두 카테고리들은 리눅스의 선형 링크드 리스트인 \co{struct hlist_head} 에
대해 동작합니다.
\co{struct list_head} 에 비해 \co{struct hlist_head} 의 장점은 하나의 포인터를
갖는 리스트 헤더만이 필요해서 커다란 해시 테이블에서는 상당한 양의 메모리를
아낄 수 있다는 점입니다.
이 표에서의 \co{struct hlist_head} 의 기능들과 RCU 를 사용하지 않는 비슷한
것들과의 관계는 \co{struct list_head} 기능들이 그러한 것과 거의 같습니다.
\iffalse

The second pair of categories operate on Linux's
\co{struct hlist_head}, which is a linear linked list.
One advantage of \co{struct hlist_head} over
\co{struct list_head} is that the former requires only
a single-pointer list header, which can save significant memory in
large hash tables.
The \co{struct hlist_head} primitives in the table
relate to their non-RCU counterparts in much the same way as do the
\co{struct list_head} primitives.
\fi

마지막의 두 카테고리들은 포인터에 직접적으로 동작하는데, RCU 로 보호되는
배열들이나 tree 들과 같은, RCU 로 보호되지만 리스트가 아닌 데이터 구조체들에
사용하기에 유용합니다.
\co{rcu_assign_pointer()} 함수는 어떤 앞의 초기화는 완화된 순서 규칙의
기계들에서도 이 포인터로의 할당 전으로 순서지어지게 합니다.
유사하게, \co{rcu_dereference()} 함수는 Alpha CPU 에서는, 뒤따르는 해당
포인터를 디레퍼런스 하는 코드가 연관된 \co{rcu_assign_pointer()} 앞의 초기화
코드의 효과를 볼 수 있도록 합니다.
Alpha 외의 CPU 에서는 \co{rcu_dereference()} 는 어떤 포인터 디레퍼런스들이 RCU
로 보호되고 있는지를 표합니다.
\iffalse

The final pair of categories operate directly on pointers, and
are useful for creating RCU-protected non-list data structures,
such as RCU-protected arrays and trees.
The \co{rcu_assign_pointer()} primitive ensures that any
prior initialization remains ordered before the assignment to the
pointer on weakly ordered machines.
Similarly, the \co{rcu_dereference()} primitive ensures that subsequent
code dereferencing the pointer will see the effects of initialization code
prior to the corresponding \co{rcu_assign_pointer()} on
Alpha CPUs.
On non-Alpha CPUs, \co{rcu_dereference()} documents which pointer
dereferences are protected by RCU.
\fi

\QuickQuiz{}
	일반적으로, \co{rcu_dereference()} 에 사용되는 모든 포인터는
	\emph{반드시} 항상 \co{rcu_assign_pointer()} 로 업데이트
	되어야만 합니다.
	이 규칙에 예외는 뭐가 있을까요?
	\iffalse

	Normally, any pointer subject to \co{rcu_dereference()} \emph{must}
	always be updated using \co{rcu_assign_pointer()}.
	What is an exception to this rule?
	\fi
\QuickQuizAnswer{
	그런 예외 가운데 하나는 여러 원소가 연결된 데이터 구조체가 다른 CPU
	들에서 접근할 수 없는 상태에서 하나의 유닛으로써 초기화되었고 하나의
	\co{rcu_assign_pointer()} 가 이 데이터 구조체로의 글로벌 포인터를
	설정하는데 사용된 경우입니다.
	비록 그 구조체가 글로벌하게 보여진 후에 일어나는 그런 할당들은
	\emph{반드시} \co{rcu_assign_pointer()} 를 사용해야 하지만, 초기화
	시점의 포인터 할당은 \co{rcu_assign_pointer()} 를 사용할 필요가
	없습니다.
	\iffalse

	One such exception is when a multi-element linked
	data structure is initialized as a unit while inaccessible to other
	CPUs, and then a single \co{rcu_assign_pointer()} is used
	to plant a global pointer to this data structure.
	The initialization-time pointer assignments need not use
	\co{rcu_assign_pointer()}, though any such assignments that
	happen after the structure is globally visible \emph{must} use
	\co{rcu_assign_pointer()}.
	\fi

	하지만, 이 초기화 코드가 매우 자주 실행되는 코드 수행경로에 있는게
	아니라면, 이론적으로는 불필요할지라도 \co{rcu_assign_pointer()} 를
	사용하는게 현명할 것입니다.
	어떤 ``사소한'' 변경이 그 초기화는 혼자서 행하게 된다는 소중한 가정을
	무효화 시키기는 매우 쉽습니다.
	\iffalse

	However, unless this initialization code is on an impressively hot
	code-path, it is probably wise to use \co{rcu_assign_pointer()}
	anyway, even though it is in theory unnecessary.
	It is all too easy for a ``minor'' change to invalidate your cherished
	assumptions about the initialization happening privately.
	\fi
} \QuickQuizEnd

\QuickQuiz{}
	이런 횡단과 업데이트 기능들이 어떤 RCU API 패밀리 멤버들과 함께
	사용된다 하더라도 어떤 문제는 없나요?
	\iffalse

	Are there any downsides to the fact that these traversal and update
	primitives can be used with any of the RCU API family members?
	\fi
\QuickQuizAnswer{
	어떤 경우에는 ``sparse'' 와 같은 자동화된 코드 검사기 (또는 사람이) 가
	어떤 종류의 RCU read-side 크리티컬 섹션이 어떤 RCU 횡단 기능들과
	연관되어 있는지 알기 어려울 수 있습니다.
	예를 들어,
	Listing~\ref{lst:defer:Diverse RCU Read-Side Nesting} 에 보여진 것과
	같은 코드를 생각해 봅시다.
	\iffalse

	It can sometimes be difficult for automated
	code checkers such as ``sparse'' (or indeed for human beings) to
	work out which type of RCU read-side critical section a given
	RCU traversal primitive corresponds to.
	For example, consider the code shown in
	Listing~\ref{lst:defer:Diverse RCU Read-Side Nesting}.
	\fi

\begin{listing}[htbp]
{ \scriptsize
\begin{verbbox}
  1 rcu_read_lock();
  2 preempt_disable();
  3 p = rcu_dereference(global_pointer);
  4
  5 /* . . . */
  6
  7 preempt_enable();
  8 rcu_read_unlock();
\end{verbbox}
}
\centering
\theverbbox
\caption{Diverse RCU Read-Side Nesting}
\label{lst:defer:Diverse RCU Read-Side Nesting}
\end{listing}

	\co{rcu_deference()} 기능은 RCU Classic 크리티컬 섹션 안에 있을까요 RCU
	Sched 크리티컬 섹션 안에 있을까요?
	이걸 알아내려면 어떻게 해야할까요?
	\iffalse

	Is the \co{rcu_dereference()} primitive in an RCU Classic
	or an RCU Sched critical section?
	What would you have to do to figure this out?
	\fi
} \QuickQuizEnd

\subsubsection{Where Can RCU's APIs Be Used?}
\label{sec:defer:Where Can RCU's APIs Be Used?}

\begin{figure}[tb]
\begin{center}
\resizebox{3in}{!}{\includegraphics{defer/RCUenvAPI}}
\end{center}
\caption{RCU API Usage Constraints}
\label{fig:defer:RCU API Usage Constraints}
\end{figure}

Figure~\ref{fig:defer:RCU API Usage Constraints}
는 커널 내의 어떤 환경에서 어떤 API 들이 사용될 수 있는지 보입니다.
RCU read-side 기능들은 NMI를 포함해 어떤 환경에서도 사용될 수 있고, RCU 의
변경 기능들과 비동기적인 grace-period 기능들은 NMI 를 제외한 모든
환경에서 사용될 수 있으며, RCU 의 동기적인 grace-period 기능들은
프로세스 컨텍스트에서만 사용될 수 있습니다.
RCU 리스트 횡단 기능들은 \co{list_for_each_entry_rcu()},
\co{hlist_for_each_entry_rcu()} 등을 포함합니다.
비슷하게, RCU list 변경 기능들은 \co{list_add_rcu()}, \co{hlist_del_rcu()} 등이
포함됩니다.

다른 종류의 RCU 로부터의 기능들은 대체될 수도 있는데, 예를 들어
\co{srcu_read_lock()} 은 \co{rcu_read_lock()} 이 사용될 수 있는 모든
컨텍스트에서 사용될 수 있습니다.
\iffalse

Figure~\ref{fig:defer:RCU API Usage Constraints}
shows which APIs may be used in which in-kernel environments.
The RCU read-side primitives may be used in any environment, including NMI,
the RCU mutation and asynchronous grace-period primitives may be used in any
environment other than NMI, and, finally, the RCU synchronous grace-period
primitives may be used only in process context.
The RCU list-traversal primitives include \co{list_for_each_entry_rcu()},
\co{hlist_for_each_entry_rcu()}, etc.
Similarly, the RCU list-mutation primitives include
\co{list_add_rcu()}, \co{hlist_del_rcu()}, etc.

Note that primitives from other families of RCU may be substituted,
for example, \co{srcu_read_lock()} may be used in any context
in which \co{rcu_read_lock()} may be used.
\fi

\subsubsection{So, What \emph{is} RCU Really?}
\label{sec:defer:So, What is RCU Really?}

그 핵심에 있어서, RCU 는 더도 덜도 아니고 삽입의 공개와 구독을 지원하고 모든
RCU 읽기 쓰레드들이 완료되길 기다리며, 여러 버전들을 관리하는 API 입니다.
그렇다곤 하나, RCU 위에 reader-writer 락킹, 레퍼런스 카운팅, 그리고
Section~\ref{sec:defer:RCU Usage} 에서 열거한 존재 보장과 같은 고차원의 것들을
만드는 것이 가능합니다.
더 나아가서, 리눅스 커뮤니티가 RCU 의 흥미롭고 새로운 사용처를 찾을 것이라 믿어
의심치 않습니다, 커널의 여러 동기화 기능들에 대해서 그랬듯이요.

물론, 더 완벽한 RCU 에 대한 이해는 이 API 들을 가지고 할 수 있는 일에 대한
것들도 포함할 것입니다.

하지만, 많은 경우에 RCU 에 대한 완벽한 이해는 RCU 구현의 예를 필요로 할겁니다.
따라서 다음 섹션은 복잡도와 기능성을 늘려가는 일련의 ``장난감'' RCU 구현들을
소개합니다.
\iffalse

At its core, RCU is nothing more nor less than an API that supports
publication and subscription for insertions, waiting for all RCU readers
to complete, and maintenance of multiple versions.
That said, it is possible to build higher-level constructs
on top of RCU, including the reader-writer-locking, reference-counting,
and existence-guarantee constructs listed in
Section~\ref{sec:defer:RCU Usage}.
Furthermore, I have no doubt that the Linux community will continue to
find interesting new uses for RCU,
just as they do for any of a number of synchronization
primitives throughout the kernel.

Of course, a more-complete view of RCU would also include
all of the things you can do with these APIs.

However, for many people, a complete view of RCU must include sample
RCU implementations.
The next section therefore presents a series of ``toy'' RCU implementations
of increasing complexity and capability.
\fi
