% defer/updates.tex
% mainfile: ../perfbook.tex
% SPDX-License-Identifier: CC-BY-SA-3.0

\section{What About Updates?}
\label{sec:defer:What About Updates?}
%
\epigraph{The only thing constant in life is change.}
	 {\emph{Fran\c{c}ois de la Rochefoucauld}}

이 챕터에서 이야기된 미뤄서 처리하기 기법들은 읽기가 대부분인 상황에 거의
직접적으로 적용이 가능합니다만, ``하지만 업데이트는요?'' 라는 질문을 떠올리게
합니다.
어쨌건, 읽기 쓰레드의 성능과 확장성을 증가시키는 건 모두 좋지만, 쓰기
쓰레드에게도 훌륭한 성능과 확장성을 원하는 것은 자연스러운 일입니다.

우린 이미 쓰기 쓰레드를 위한 높은 성능과 확장성을 제공하는 상황을 하나 봤는데,
Chapter~\ref{chp:Counting} 에서 알아본 카운팅 알고리즘들입니다.
이 알고리즘들은 부분적으로 분리된 데이터 구조들을 사용하여 업데이트가
지역적으로 일어날 수 있는 동안 더 비싼 읽기들은 전체 데이터 구조를 가로지르며
덧셈을 해야만 했습니다.
Silas Boyd-Wickhizer 는 이 방향을 OpLog 를 생성하게끔 범용화 시켰는데, 이를
그는 리눅스 커널 pathname 탐색, VM 역 매핑, 그리고 \co{stat()} 시스템콜에
적용했습니다~\cite{SilasBoydWickizerPhD}.

\iffalse

The deferred-processing techniques called out in this chapter are most
directly applicable to read-mostly situations, which begs the question
``But what about updates?''
After all, increasing the performance and scalability of readers is all
well and good, but it is only natural to also want great performance and
scalability for writers.

We have already seen one situation featuring high performance and
scalability for writers, namely the counting algorithms surveyed in
Chapter~\ref{chp:Counting}.
These algorithms featured partially partitioned data structures so
that updates can operate locally, while the more-expensive reads
must sum across the entire data structure.
Silas Boyd-Wickhizer has generalized this notion to produce
OpLog, which he has applied to
Linux-kernel pathname lookup, VM reverse mappings, and the \co{stat()} system
call~\cite{SilasBoydWickizerPhD}.

\fi

``Disruptor'' 라 불리는 또다른 방법이 많은 양의 입력 데이터 스트림을 처리하는
어플리케이션들을 위해 설계되었습니다.
이 방법은 single-producer-single-consumer FIFO queue 에 기반하여 동기화의
필요성을 최소화 시키기 위함입니다~\cite{AdrianSutton2013LCA:Disruptor}.
Java 어플리케이션에서 Disruptor 는 또한 가비지 콜렉터 사용을 최소화 시키는
이점이 있습니다.

그리고 물론, 가능한 곳에서는 완전히 조각난 또는 ``shard'' 된 시스템은
Chapter~\ref{cha:Partitioning and Synchronization Design} 에서 이냐기 되었듯
훌륭한 성능과 확장성을 제공합니다.

다음 챕터는 여러 타입의 데이터 구조의 맥락에서 업데이트를 살펴봅니다.

\iffalse

Another approach, called ``Disruptor'', is designed for applications
that process high-volume streams of input data.
The approach is to rely on single-producer-single-consumer FIFO queues,
minimizing the need for synchronization~\cite{AdrianSutton2013LCA:Disruptor}.
For Java applications, Disruptor also has the virtue of minimizing use
of the garbage collector.

And of course, where feasible, fully partitioned or ``sharded'' systems
provide excellent performance and scalability, as noted in
Chapter~\ref{cha:Partitioning and Synchronization Design}.

The next chapter will look at updates in the context of several types
of data structures.

\fi
