% defer/rcuexercises.tex

\subsection{RCU Exercises}
\label{sec:defer:RCU Exercises}

이 섹션은 RCU 를 이 책에서 앞서 보인 여러 예제들에 적용하는 내용에 대한
여러개의 Quick Quizz 들로 구성되어 있습니다.
각 Quick Quiz 에의 답은 힌트를 일부 제공하고, 또한 그 해법이 자세히 설명되어
있는 뒤의 섹션으로의 포인터를 담고 있습니다.
\co{rcu_read_lock()}, \co{rcu_read_unlock()}, \co{rcu_dereference()},
\co{rcu_assign_pointer()}, 그리고 \co{synchronize_rcu()} 기능들만으로도 이
문제들에 충분해야 할 겁니다.
\iffalse

This section is organized as a series of Quick Quizzes that invite you
to apply RCU to a number of examples earlier in this book.
The answer to each Quick Quiz gives some hints, and also contains a
pointer to a later section where the solution is explained at length.
The \co{rcu_read_lock()}, \co{rcu_read_unlock()}, \co{rcu_dereference()},
\co{rcu_assign_pointer()}, and \co{synchronize_rcu()} primitives should
suffice for most of these exercises.
\fi

\QuickQuiz{}
	Figure~\ref{fig:count:Per-Thread Statistical Counters}
	(\url{count_end.c})
	에 보인 통계적 카운터의 구현은 \co{read_count()} 안에서 합계를 구하는
	것을 보호하기 위해 글로벌 락을 사용했는데 이는 부족한 성능과 음의
	확장성을 일으켰습니다.
	\co{read_count()} 가 훌륭한 성능과 좋은 확장성을 제공할 수 있게 하기
	위해 RCU 를 어떻게 적용해 볼 수 있을까요.
	(\co{read_count()} 의 확장성은 모든 쓰레드의 카운터들을 스캔해야 한다는
	필요성으로 인해 제한되어진다는 점을 명심하세요.)
	\iffalse

	The statistical-counter implementation shown in
	Figure~\ref{fig:count:Per-Thread Statistical Counters}
	(\url{count_end.c})
	used a global lock to guard the summation in \co{read_count()},
	which resulted in poor performance and negative scalability.
	How could you use RCU to provide \co{read_count()} with
	excellent performance and good scalability.
	(Keep in mind that \co{read_count()}'s scalability will
	necessarily be limited by its need to scan all threads'
	counters.)
	\fi
\QuickQuizAnswer{
	힌트: 글로벌 변수 \co{finalcount} 와 배열 \co{counterp[]} 를 하나의 RCU
	로 보호되는 구조체 안에 넣으세요.
	초기화 때, 이 구조체는 할당되고 모두 0과 \co{NULL} 로 설정될 겁니다.

	\co{inc_count()} 함수는 바뀔 필요가 없을 겁니다.

	\co{read_count()} 함수는 \co{final_mutex} 를 잡는 대신에
	\co{rcu_read_lock()} 를 사용할 것이고, 현재 구조체로의 레퍼런스를
	얻어오는데에 \co{rcu_dereference()} 를 사용해야 할겁니다.
	\iffalse

	Hint: place the global variable \co{finalcount} and the
	array \co{counterp[]} into a single RCU-protected struct.
	At initialization time, this structure would be allocated
	and set to all zero and \co{NULL}.

	The \co{inc_count()} function would be unchanged.

	The \co{read_count()} function would use \co{rcu_read_lock()}
	instead of acquiring \co{final_mutex}, and would need to
	use \co{rcu_dereference()} to acquire a reference to the
	current structure.
	\fi

	\co{count_register_thread()} 함수는 이 새로 생성된 쓰레드와 연관된 배열
	원소를 그 쓰레드의 쓰레드별 \co{counter} 변수로의 레퍼런스로 설정할
	겁니다.

	\co{count_unregister_thread()} 함수는 새로운 구조체를 할당받고,
	\co{final_mutex} 를 잡은 후, 기존의 구조체를 새로운 것으로 복사하고,
	나가는 쓰레드의 \co{counter} 변수를 전체 값에 더하고, 이 \co{counter}
	변수로의 포인터를 \co{NULL} 로 만든 후, 이 새로운 구조체를 과거의 것의
	자리에 설치하기 위해 \co{rcu_assign_pointer()} 를 사용한 후,
	\co{final_mutex} 를 풀고, grace period 를 기다린 후, 마지막으로 기존의
	구조체를 메모리 해제시켜야 할 것입니다.
	\iffalse

	The \co{count_register_thread()} function would set the
	array element corresponding to the newly created thread
	to reference that thread's per-thread \co{counter} variable.

	The \co{count_unregister_thread()} function would need to
	allocate a new structure, acquire \co{final_mutex},
	copy the old structure to the new one, add the outgoing
	thread's \co{counter} variable to the total, \co{NULL}
	the pointer to this same \co{counter} variable,
	use \co{rcu_assign_pointer()} to install the new structure
	in place of the old one, release \co{final_mutex},
	wait for a grace period, and finally free the old structure.
	\fi

	이게 정말 동작할까요?
	그 이유는 무엇이죠?

	더 자세한 내용을 위해서는
	page~\pageref{sec:together:RCU and Per-Thread-Variable-Based Statistical Counters}
	의
	Section~\ref{sec:together:RCU and Per-Thread-Variable-Based Statistical Counters}
	를 참고하세요.
	\iffalse

	Does this really work?
	Why or why not?

	See
	Section~\ref{sec:together:RCU and Per-Thread-Variable-Based Statistical Counters}
	on
	page~\pageref{sec:together:RCU and Per-Thread-Variable-Based Statistical Counters}
	for more details.
	\fi
} \QuickQuizEnd


\QuickQuiz{}
	Section~\ref{sec:count:Applying Specialized Parallel Counters}
	는 디바이스들을 제거하기 위해 I/O 액세스들을 카운트하는 일을 처리하는
	한쌍의 코드 조각들을 보였습니다.
	이 코드 조각들은 reader-writer 락을 잡아야 하는 이유로 (I/O 를
	시작하는) 빠른 수행 경로의 높은 오버헤드 문제를 겪었습니다.
	여기에 훌륭한 성능과 확장성을 가져오기 위해 RCU 를 어떻게 사용할 수
	있을까요?
	(I/O 액세스를 하는 일반적인 경우의 첫번째 코드 조각의 성능이 디바이스
	제거 코드 조각의 것보다 훨씬 더 중요함을 명심하세요.)
	\iffalse

	Section~\ref{sec:count:Applying Specialized Parallel Counters}
	showed a fanciful pair of code fragments that dealt with counting
	I/O accesses to removable devices.
	These code fragments suffered from high overhead on the fastpath
	(starting an I/O) due to the need to acquire a reader-writer
	lock.
	How would you use RCU to provide excellent performance and
	scalability?
	(Keep in mind that the performance of the common-case first
	code fragment that does I/O accesses is much more important
	than that of the device-removal code fragment.)
	\fi
\QuickQuizAnswer{
	힌트: 이 reader-writer 락의 read 권한 획득을 RCU read-side 크리티컬
	섹션들로 바꾸고, device 제거 코드를 이에 맞게 조정하세요.
	이 문제에 대한 하나의 해결책을 위해선
	Page~\pageref{sec:together:RCU and Counters for Removable I/O Devices}
	의
	Section~\ref{sec:together:RCU and Counters for Removable I/O Devices}
	를 참고하세요.
	\iffalse

	Hint: replace the read-acquisitions of the reader-writer lock
	with RCU read-side critical sections, then adjust the
	device-removal code fragment to suit.

	See
	Section~\ref{sec:together:RCU and Counters for Removable I/O Devices}
	on
	Page~\pageref{sec:together:RCU and Counters for Removable I/O Devices}
	for one solution to this problem.
	\fi
} \QuickQuizEnd
