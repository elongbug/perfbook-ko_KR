% glossary.tex
% SPDX-License-Identifier: CC-BY-SA-3.0

\chapter{Glossary and Bibliography}
%
\Epigraph{Dictionaries are inherently circular in nature.}
	 {\emph{``Self Reference in word definitions'',
	        David~Levary~et~al.}}

\begin{description}
\item[Associativity:]
	The number of cache lines that can be held simultaneously in
	a given cache, when all of these cache lines hash identically
	in that cache.
	A cache that could hold four cache lines for each possible
	hash value would be termed a ``four-way set-associative'' cache,
	while a cache that could hold only one cache line for each
	possible hash value would be termed a ``direct-mapped'' cache.
	A cache whose associativity was equal to its capacity would
	be termed a ``fully associative'' cache.
	Fully associative caches have the advantage of eliminating
	associativity misses, but, due to hardware limitations,
	fully associative caches are normally quite limited in size.
	The associativity of the large caches found on modern microprocessors
	typically range from two-way to eight-way.
\item[Associativity Miss:]
	A cache miss incurred because the corresponding CPU has recently
	accessed more data hashing to a given set of the cache than will
	fit in that set.
	Fully associative caches are not subject to associativity misses
	(or, equivalently, in fully associative caches, associativity
	and capacity misses are identical).
\item[Atomic:]
	An operation is considered ``atomic'' if it is not possible to
	observe any intermediate state.
	For example, on most CPUs, a store to a properly aligned pointer
	is atomic, because other CPUs will see either the old value or
	the new value, but are guaranteed not to see some mixed value
	containing some pieces of the new and old values.
\item[Cache:]
	In modern computer systems, CPUs have caches in which to hold
	frequently used data.
	These caches can be thought of as hardware hash tables with very
	simple hash functions,
	but in which each hash bucket (termed a ``set'' by hardware types)
	can hold only a limited number of data items.
	The number of data items that can be held by each of a cache's hash
	buckets is termed the cache's ``associativity''.
	These data items are normally called ``cache lines'', which
	can be thought of a fixed-length blocks of data that circulate
	among the CPUs and memory.
\item[Cache Coherence:]
	A property of most modern SMP machines where all CPUs will
	observe a sequence of values for a given variable that is
	consistent with at least one global order of values for
	that variable.
	Cache coherence also guarantees that at the end of a group
	of stores to a given variable, all CPUs will agree
	on the final value for that variable.
	Note that cache coherence applies only to the series of values
	taken on by a single variable.
	In contrast, the memory consistency model for a given machine
	describes the order in which loads and stores to groups of
	variables will appear to occur.
	See Section~\ref{sec:advsync:Single-Variable Memory Consistency}
	for more information.
\item[Cache Coherence Protocol:]
	A communications protocol, normally implemented in hardware,
	that enforces memory consistency and ordering, preventing
	different CPUs from seeing inconsistent views of data held
	in their caches.
\item[Cache Geometry:]
	The size and associativity of a cache is termed its geometry.
	Each cache may be thought of as a two-dimensional array,
	with rows of cache lines (``sets'') that have the same hash
	value, and columns of cache lines (``ways'') in which every
	cache line has a different hash value.
	The associativity of a given cache is its number of
	columns (hence the name ``way''---a two-way set-associative
	cache has two ``ways''), and the size of the cache is its
	number of rows multiplied by its number of columns.
\item[Cache Line:]
	(1) The unit of data that circulates among the CPUs and memory,
	usually a moderate power of two in size.
	Typical cache-line sizes range from 16 to 256 bytes. \\
	(2) A physical location in a CPU cache capable of holding
	one cache-line unit of data. \\
	(3) A physical location in memory capable of holding one
	cache-line unit of data, but that it also aligned
	on a cache-line boundary.
	For example, the address of the first word of a cache line
	in memory will end in 0x00 on systems with 256-byte cache lines.
\item[Cache Miss:]
	A cache miss occurs when data needed by the CPU is not in
	that CPU's cache.
	The data might be missing because of a number of reasons,
	including:
	(1) this CPU has never accessed the data before
	(``startup'' or ``warmup'' miss),
	(2) this CPU has recently accessed more
	data than would fit in its cache, so that some of the older
	data had to be removed (``capacity'' miss),
	(3) this CPU
	has recently accessed more data in a given set\footnote{
		In hardware-cache terminology, the word ``set''
		is used in the same way that the word ``bucket''
		is used when discussing software caches.}
	than that set could hold (``associativity'' miss),
	(4) some other CPU has written to the data (or some other
	data in the same cache line) since this CPU has accessed it
	(``communication miss''), or
	(5) this CPU attempted to write to a cache line that is
	currently read-only, possibly due to that line being replicated
	in other CPUs' caches.
\item[Capacity Miss:]
	A cache miss incurred because the corresponding CPU has recently
	accessed more data than will fit into the cache.
\item[Code Locking:]
	A simple locking design in which a ``global lock'' is used to protect
	a set of critical sections, so that access by a given thread
	to that set is
	granted or denied based only on the set of threads currently
	occupying the set of critical sections, not based on what
	data the thread intends to access.
	The scalability of a code-locked program is limited by the code;
	increasing the size of the data set will normally not increase
	scalability (in fact, will typically \emph{decrease} scalability
	by increasing ``lock contention'').
	Contrast with ``data locking''.
\item[Communication Miss:]
	A cache miss incurred because the some other CPU has written to
	the cache line since the last time this CPU accessed it.
\item[Critical Section:]
	A section of code guarded by some synchronization mechanism,
	so that its execution constrained by that primitive.
	For example, if a set of critical sections are guarded by
	the same global lock, then only one of those critical sections
	may be executing at a given time.
	If a thread is executing in one such critical section,
	any other threads must wait until the first thread completes
	before executing any of the critical sections in the set.
\item[Data Locking:]
	A scalable locking design in which each instance of a given
	data structure has its own lock.
	If each thread is using a different instance of the
	data structure, then all of the threads may be executing in
	the set of critical sections simultaneously.
	Data locking has the advantage of automatically scaling to
	increasing numbers of CPUs as the number of instances of
	data grows.
	Contrast with ``code locking''.
\item[Direct-Mapped Cache:]
	A cache with only one way, so that it may hold only one cache
	line with a given hash value.
\item[Embarrassingly Parallel:]
	A problem or algorithm where adding threads does not significantly
	increase the overall cost of the computation, resulting in
	linear speedups as threads are added (assuming sufficient
	CPUs are available).
\item[Exclusive Lock:]
	An exclusive lock is a mutual-exclusion mechanism that
	permits only one thread at a time into the
	set of critical sections guarded by that lock.
\item[False Sharing:]
	If two CPUs each frequently write to one of a pair of data items,
	but the pair of data items are located in the same cache line,
	this cache line will be repeatedly invalidated, ``ping-ponging''
	back and forth between the two CPUs' caches.
	This is a common cause of ``cache thrashing'', also called
	``cacheline bouncing'' (the latter most commonly in the Linux
	community).
	False sharing can dramatically reduce both performance and
	scalability.
\item[Fragmentation:]
	A memory pool that has a large amount of unused memory, but
	not laid out to permit satisfying a relatively small request
	is said to be fragmented.
	External fragmentation occurs when the space is divided up
	into small fragments lying between allocated blocks of memory,
	while internal fragmentation occurs when specific requests or
	types of requests have been allotted more memory than they
	actually requested.
\item[Fully Associative Cache:]
	A fully associative cache contains only
	one set, so that it can hold any subset of
	memory that fits within its capacity.
\item[Grace Period:]
	A grace period is any contiguous time interval such that
	any RCU read-side critical section that began before the
	start of that interval has
	completed before the end of that same interval.
	Many RCU implementations define a grace period to be a
	time interval during which each thread has passed through at
	least one quiescent state.
	Since RCU read-side critical sections by definition cannot
	contain quiescent states, these two definitions are almost
	always interchangeable.
\item[Heisenbug:]
	A timing-sensitive bug that disappears from sight when you
	add print statements or tracing in an attempt to track it
	down.
\item[Hot Spot:]
	Data structure that is very heavily used, resulting in high
	levels of contention on the corresponding lock.
	One example of this situation would be a hash table with
	a poorly chosen hash function.
\item[Humiliatingly Parallel:]
	A problem or algorithm where adding threads significantly
	\emph{decreases} the overall cost of the computation, resulting in
	large superlinear speedups as threads are added (assuming sufficient
	CPUs are available).
\item[Invalidation:]
	When a CPU wishes to write to a data item, it must first ensure
	that this data item is not present in any other CPUs' cache.
	If necessary, the item is removed from the other CPUs' caches
	via ``invalidation'' messages from the writing CPUs to any
	CPUs having a copy in their caches.
\item[IPI:]
	Inter-processor interrupt, which is an
	interrupt sent from one CPU to another.
	IPIs are used heavily in the Linux kernel, for example, within
	the scheduler to alert CPUs that a high-priority process is now
	runnable.
\item[IRQ:]
	Interrupt request, often used as an abbreviation for ``interrupt''
	within the Linux kernel community, as in ``irq handler''.
\item[Linearizable:]
	A sequence of operations is ``linearizable'' if there is at
	least one global ordering of the sequence that is consistent
	with the observations of all CPUs and/or threads.
	Linearizability is much prized by many researchers, but less
	useful in practice than one might
	expect~\cite{AndreasHaas2012FIFOisnt}.
\item[Lock:]
	A software abstraction that can be used to guard critical sections,
	as such, an example of a ''mutual exclusion mechanism''.
	An ``exclusive lock'' permits only one thread at a time into the
	set of critical sections guarded by that lock, while a
	``reader-writer lock'' permits any number of reading
	threads, or but one writing thread, into the set of critical
	sections guarded by that lock.  (Just to be clear, the presence
	of a writer thread in any of a given reader-writer lock's
	critical sections will prevent any reader from entering
	any of that lock's critical sections and vice versa.)
\item[Lock Contention:]
	A lock is said to be suffering contention when it is being
	used so heavily that there is often a CPU waiting on it.
	Reducing lock contention is often a concern when designing
	parallel algorithms and when implementing parallel programs.
\item[Memory Consistency:]
	A set of properties that impose constraints on the order in
	which accesses to groups of variables appear to occur.
	Memory consistency models range from sequential consistency,
	a very constraining model popular in academic circles, through
	process consistency, release consistency, and weak consistency.
\item[MESI Protocol:]
	The
	cache-coherence protocol featuring
	modified, exclusive, shared, and invalid (MESI) states,
	so that this protocol is named after the states that the
	cache lines in a given cache can take on.
	A modified line has been recently written to by this CPU,
	and is the sole representative of the current value of
	the corresponding memory location.
	An exclusive cache line has not been written to, but this
	CPU has the right to write to it at any time, as the line
	is guaranteed not to be replicated into any other CPU's cache
	(though the corresponding location in main memory is up to date).
	A shared cache line is (or might be) replicated in some other
	CPUs' cache, meaning that this CPU must interact with those other
	CPUs before writing to this cache line.
	An invalid cache line contains no value, instead representing
	``empty space'' in the cache into which data from memory might
	be loaded.
\item[Mutual-Exclusion Mechanism:]
	A software abstraction that regulates threads' access to
	``critical sections'' and corresponding data.
\item[NMI:]
	Non-maskable interrupt.
	As the name indicates, this is an extremely high-priority
	interrupt that cannot be masked.
	These are used for hardware-specific purposes such as profiling.
	The advantage of using NMIs for profiling is that it allows you
	to profile code that runs with interrupts disabled.
\item[NUCA:]
	Non-uniform cache architecture, where groups of CPUs share
	caches.
	CPUs in a group can therefore exchange cache lines with each
	other much more quickly than they can with CPUs in other groups.
	Systems comprised of CPUs with hardware threads will generally
	have a NUCA architecture.
\item[NUMA:]
	Non-uniform memory architecture, where memory is split into
	banks and each such bank is ``close'' to a group of CPUs,
	the group being termed a ``NUMA node''.
	An example NUMA machine is Sequent's NUMA-Q system, where
	each group of four CPUs had a bank of memory near by.
	The CPUs in a given group can access their memory much
	more quickly than another group's memory.
\item[NUMA Node:]
	A group of closely placed CPUs and associated memory within
	a larger NUMA machines.
	Note that a NUMA node might well have a NUCA architecture.
\item[Pipelined CPU:]
	A CPU with a pipeline, which is
	an internal flow of instructions internal to the CPU that
	is in some way similar to an assembly line, with many of
	the same advantages and disadvantages.
	In the 1960s through the early 1980s, pipelined CPUs were the
	province of supercomputers, but started appearing in microprocessors
	(such as the 80486) in the late 1980s.
\item[Process Consistency:]
	A memory-consistency model in which each CPU's stores appear to
	occur in program order, but in which different CPUs might see
	accesses from more than one CPU as occurring in different orders.
\item[Program Order:]
	The order in which a given thread's instructions
	would be executed by a now-mythical ``in-order'' CPU that
	completely executed each instruction before proceeding to
	the next instruction.
	(The reason such CPUs are now the stuff of ancient myths
	and legends is that they were extremely slow.
	These dinosaurs were one of the many victims of
	Moore's-Law-driven increases in CPU clock frequency.
	Some claim that these beasts will roam the earth once again,
	others vehemently disagree.)
\item[Quiescent State:]
	In RCU, a point in the code where there can be no references held
	to RCU-protected data structures, which is normally any point
	outside of an RCU read-side critical section.
	Any interval of time during which all threads pass through at
	least one quiescent state each is termed a ``grace period''.
\item[Read-Copy Update (RCU):]
	A synchronization mechanism that can be thought of as a replacement
	for reader-writer locking or reference counting.
	RCU provides extremely low-overhead access for readers, while
	writers incur additional overhead maintaining old versions
	for the benefit of pre-existing readers.
	Readers neither block nor spin, and thus cannot participate in
	deadlocks, however, they also can see stale data and can
	run concurrently with updates.
	RCU is thus best-suited for read-mostly situations where
	stale data can either be tolerated (as in routing tables)
	or avoided (as in the Linux kernel's System V IPC implementation).
\item[Read-Side Critical Section:]
	A section of code guarded by read-acquisition of
	some reader-writer synchronization mechanism.
	For example, if one set of critical sections are guarded by
	read-acquisition of
	a given global reader-writer lock, while a second set of critical
	section are guarded by write-acquisition of that same reader-writer
	lock, then the first set of critical sections will be the
	read-side critical sections for that lock.
	Any number of threads may concurrently execute the read-side
	critical sections, but only if no thread is executing one of
	the write-side critical sections.
\item[Reader-Writer Lock:]
	A reader-writer lock is a mutual-exclusion mechanism that
	permits any number of reading
	threads, or but one writing thread, into the set of critical
	sections guarded by that lock.
	Threads attempting to write must wait until all pre-existing
	reading threads release the lock, and, similarly, if there
	is a pre-existing writer, any threads attempting to write must
	wait for the writer to release the lock.
	A key concern for reader-writer locks is ``fairness'':
	can an unending stream of readers starve a writer or vice versa.
\item[Sequential Consistency:]
	A memory-consistency model where all memory references appear to occur
	in an order consistent with
	a single global order, and where each CPU's memory references
	appear to all CPUs to occur in program order.
\item[Store Buffer:]
	A small set of internal registers used by a given CPU
	to record pending stores
	while the corresponding cache lines are making their
	way to that CPU.
	Also called ``store queue''.
\item[Store Forwarding:]
	An arrangement where a given CPU refers to its store buffer
	as well as its cache so as to ensure that the software sees
	the memory operations performed by this CPU as if they
	were carried out in program order.
\item[Super-Scalar CPU:]
	A scalar (non-vector) CPU capable of executing multiple instructions
	concurrently.
	This is a step up from a pipelined CPU that executes multiple
	instructions in an assembly-line fashion---in a super-scalar
	CPU, each stage of the pipeline would be capable of handling
	more than one instruction.
	For example, if the conditions were exactly right,
	the Intel Pentium Pro CPU from the mid-1990s could
	execute two (and sometimes three) instructions per clock cycle.
	Thus, a 200MHz Pentium Pro CPU could ``retire'', or complete the
	execution of, up to 400 million instructions per second.
\item[Teachable:]
	A topic, concept, method, or mechanism that the teacher understands
	completely and is therefore comfortable teaching.
\item[Transactional Memory (TM):]
	Shared-memory synchronization scheme featuring ``transactions'',
	each of which is an atomic sequence of operations
	that offers atomicity, consistency, isolation, but differ from
	classic transactions in that they do not offer
	durability.
	Transactional memory may be implemented either in hardware
	(hardwire transactional memory, or HTM), in software (software
	transactional memory, or STM), or in a combination of hardware
	and software (``unbounded'' transactional memory, or UTM).
\item[Unteachable:]
	A topic, concept, method, or mechanism that the teacher does
	not understand well is therefore uncomfortable teaching.
\item[Vector CPU:]
	A CPU that can apply a single instruction to multiple items of
	data concurrently.
	In the 1960s through the 1980s, only supercomputers had vector
	capabilities, but the advent of MMX in x86 CPUs and VMX in
	PowerPC CPUs brought vector processing to the masses.
\item[Write Miss:]
	A cache miss incurred because the corresponding CPU attempted
	to write to a cache line that is read-only, most likely due
	to its being replicated in other CPUs' caches.
\item[Write-Side Critical Section:]
	A section of code guarded by write-acquisition of
	some reader-writer synchronization mechanism.
	For example, if one set of critical sections are guarded by
	write-acquisition of
	a given global reader-writer lock, while a second set of critical
	section are guarded by read-acquisition of that same reader-writer
	lock, then the first set of critical sections will be the
	write-side critical sections for that lock.
	Only one thread may execute in the write-side critical section
	at a time, and even then only if there are no threads are
	executing concurrently in any of the corresponding read-side
	critical sections.
\end{description}
